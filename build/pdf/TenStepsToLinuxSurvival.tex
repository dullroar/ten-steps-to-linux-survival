\documentclass[10pt,american,]{book}
\usepackage{lmodern}
\usepackage{amssymb,amsmath}
\usepackage{ifxetex,ifluatex}
\usepackage{fixltx2e} % provides \textsubscript
\ifnum 0\ifxetex 1\fi\ifluatex 1\fi=0 % if pdftex
  \usepackage[T1]{fontenc}
  \usepackage[utf8]{inputenc}
\else % if luatex or xelatex
  \ifxetex
    \usepackage{mathspec}
  \else
    \usepackage{fontspec}
  \fi
  \defaultfontfeatures{Ligatures=TeX,Scale=MatchLowercase}
  \newcommand{\euro}{€}
    \setmainfont[]{DejaVu Serif}
    \setsansfont[]{DejaVu Sans}
    \setmonofont[Mapping=tex-ansi]{Ubuntu Mono}
\fi
% use upquote if available, for straight quotes in verbatim environments
\IfFileExists{upquote.sty}{\usepackage{upquote}}{}
% use microtype if available
\IfFileExists{microtype.sty}{%
\usepackage{microtype}
\UseMicrotypeSet[protrusion]{basicmath} % disable protrusion for tt fonts
}{}
\usepackage[margin=1in, paperwidth=7in, paperheight=9in]{geometry}
\usepackage{hyperref}
\PassOptionsToPackage{usenames,dvipsnames}{color} % color is loaded by hyperref
\hypersetup{unicode=true,
            pdftitle={Ten Steps to Linux Survival},
            pdfauthor={Jim Lehmer},
            pdfsubject={Bash for Windows People},
            colorlinks=true,
            linkcolor=black,
            citecolor=black,
            urlcolor=black,
            breaklinks=true}
\urlstyle{same}  % don't use monospace font for urls
\ifnum 0\ifxetex 1\fi\ifluatex 1\fi=0 % if pdftex
  \usepackage[shorthands=off,main=american]{babel}
\else
  \usepackage{polyglossia}
  \setmainlanguage[variant=american]{english}
\fi
\usepackage{color}
\usepackage{fancyvrb}
\newcommand{\VerbBar}{|}
\newcommand{\VERB}{\Verb[commandchars=\\\{\}]}
\DefineVerbatimEnvironment{Highlighting}{Verbatim}{commandchars=\\\{\}}
% Add ',fontsize=\small' for more characters per line
\usepackage{framed}
\definecolor{shadecolor}{RGB}{248,248,248}
\newenvironment{Shaded}{\begin{snugshade}}{\end{snugshade}}
\newcommand{\KeywordTok}[1]{\textcolor[rgb]{0.13,0.29,0.53}{\textbf{{#1}}}}
\newcommand{\DataTypeTok}[1]{\textcolor[rgb]{0.13,0.29,0.53}{{#1}}}
\newcommand{\DecValTok}[1]{\textcolor[rgb]{0.00,0.00,0.81}{{#1}}}
\newcommand{\BaseNTok}[1]{\textcolor[rgb]{0.00,0.00,0.81}{{#1}}}
\newcommand{\FloatTok}[1]{\textcolor[rgb]{0.00,0.00,0.81}{{#1}}}
\newcommand{\ConstantTok}[1]{\textcolor[rgb]{0.00,0.00,0.00}{{#1}}}
\newcommand{\CharTok}[1]{\textcolor[rgb]{0.31,0.60,0.02}{{#1}}}
\newcommand{\SpecialCharTok}[1]{\textcolor[rgb]{0.00,0.00,0.00}{{#1}}}
\newcommand{\StringTok}[1]{\textcolor[rgb]{0.31,0.60,0.02}{{#1}}}
\newcommand{\VerbatimStringTok}[1]{\textcolor[rgb]{0.31,0.60,0.02}{{#1}}}
\newcommand{\SpecialStringTok}[1]{\textcolor[rgb]{0.31,0.60,0.02}{{#1}}}
\newcommand{\ImportTok}[1]{{#1}}
\newcommand{\CommentTok}[1]{\textcolor[rgb]{0.56,0.35,0.01}{\textit{{#1}}}}
\newcommand{\DocumentationTok}[1]{\textcolor[rgb]{0.56,0.35,0.01}{\textbf{\textit{{#1}}}}}
\newcommand{\AnnotationTok}[1]{\textcolor[rgb]{0.56,0.35,0.01}{\textbf{\textit{{#1}}}}}
\newcommand{\CommentVarTok}[1]{\textcolor[rgb]{0.56,0.35,0.01}{\textbf{\textit{{#1}}}}}
\newcommand{\OtherTok}[1]{\textcolor[rgb]{0.56,0.35,0.01}{{#1}}}
\newcommand{\FunctionTok}[1]{\textcolor[rgb]{0.00,0.00,0.00}{{#1}}}
\newcommand{\VariableTok}[1]{\textcolor[rgb]{0.00,0.00,0.00}{{#1}}}
\newcommand{\ControlFlowTok}[1]{\textcolor[rgb]{0.13,0.29,0.53}{\textbf{{#1}}}}
\newcommand{\OperatorTok}[1]{\textcolor[rgb]{0.81,0.36,0.00}{\textbf{{#1}}}}
\newcommand{\BuiltInTok}[1]{{#1}}
\newcommand{\ExtensionTok}[1]{{#1}}
\newcommand{\PreprocessorTok}[1]{\textcolor[rgb]{0.56,0.35,0.01}{\textit{{#1}}}}
\newcommand{\AttributeTok}[1]{\textcolor[rgb]{0.77,0.63,0.00}{{#1}}}
\newcommand{\RegionMarkerTok}[1]{{#1}}
\newcommand{\InformationTok}[1]{\textcolor[rgb]{0.56,0.35,0.01}{\textbf{\textit{{#1}}}}}
\newcommand{\WarningTok}[1]{\textcolor[rgb]{0.56,0.35,0.01}{\textbf{\textit{{#1}}}}}
\newcommand{\AlertTok}[1]{\textcolor[rgb]{0.94,0.16,0.16}{{#1}}}
\newcommand{\ErrorTok}[1]{\textcolor[rgb]{0.64,0.00,0.00}{\textbf{{#1}}}}
\newcommand{\NormalTok}[1]{{#1}}
\usepackage{graphicx,grffile}
\makeatletter
\def\maxwidth{\ifdim\Gin@nat@width>\linewidth\linewidth\else\Gin@nat@width\fi}
\def\maxheight{\ifdim\Gin@nat@height>\textheight\textheight\else\Gin@nat@height\fi}
\makeatother
% Scale images if necessary, so that they will not overflow the page
% margins by default, and it is still possible to overwrite the defaults
% using explicit options in \includegraphics[width, height, ...]{}
\setkeys{Gin}{width=\maxwidth,height=\maxheight,keepaspectratio}
% Make links footnotes instead of hotlinks:
\renewcommand{\href}[2]{#2\footnote{\url{#1}}}
\usepackage[normalem]{ulem}
% avoid problems with \sout in headers with hyperref:
\pdfstringdefDisableCommands{\renewcommand{\sout}{}}
\setlength{\parindent}{0pt}
\setlength{\parskip}{6pt plus 2pt minus 1pt}
\setlength{\emergencystretch}{3em}  % prevent overfull lines
\providecommand{\tightlist}{%
  \setlength{\itemsep}{0pt}\setlength{\parskip}{0pt}}
\setcounter{secnumdepth}{0}

\title{Ten Steps to Linux Survival\\\vspace{0.5em}{\large Bash for Windows People}}
\author{Jim Lehmer}
\date{2016}
% Packages I added.
\usepackage{placeins}
\usepackage{caption}
\usepackage{float}
\usepackage{flafter}
\usepackage{makeidx}
\usepackage[nottoc]{tocbibind}
\usepackage{fancyhdr}
\usepackage{xcolor}
\usepackage{framed}
\usepackage{tocloft}
% End of packages I added.

% Overriding commands.
\makeindex
\restylefloat{figure}
\floatplacement{figure}{H}
% Originals from http://www-rohan.sdsu.edu/~aty/bibliog/latex/floats.html
% General parameters, for ALL pages:
\renewcommand{\topfraction}{0.7}	  % max fraction of floats at top
\renewcommand{\bottomfraction}{0.7}	% max fraction of floats at bottom
\renewcommand{\textfraction}{0.3}	  % allow minimal text w. figs
% Parameters for TEXT pages (not float pages):
\setcounter{topnumber}{2}
\setcounter{bottomnumber}{2}
\setcounter{totalnumber}{2}         % 2 may work better
% Parameters for FLOAT pages (not text pages):
% N.B.: floatpagefraction MUST be less than topfraction !!
\renewcommand{\floatpagefraction}{0.65}	% require fuller float pages
% Parameters for double-column pages:
\renewcommand{\dbltopfraction}{0.9}	% fit big float above 2-col. text
\renewcommand{\dblfloatpagefraction}{0.7}	% require fuller float pages
\setcounter{dbltopnumber}{2}        % for 2-column pages
\overfullrule=2cm   % Black bars to show overfull boxes
\numberwithin{figure}{chapter}
\DeclareRobustCommand{\drbsd}[1]{
    \index{BSD (operating system)}\index{#1 (operating system)}
    \index{operating systems!#1}
    \index{operating systems!BSD}
}
\DeclareRobustCommand{\drcap}[1]{\begin{figure}[H]\caption{#1}\end{figure}}
\DeclareRobustCommand{\drcmd}[2]{
    \index{#1@\texttt{#1} (#2 command)}
    \index{#2 (\texttt{#1} command)}
    \index{commands!#1@\texttt{#1} (#2)}
}
\DeclareRobustCommand{\drcom}[1]{
    \index{#1@\texttt{#1} (compression command)}
    \index{commands!#1@\texttt{#1} (compression)}
    \index{compression!#1@\texttt{#1} command}
}
\DeclareRobustCommand{\drdis}[1]{
    \index{#1 (Linux distro)}
    \index{Linux distros!#1}
}
\DeclareRobustCommand{\drdoc}[1]{
    \index{#1@\texttt{#1} (documentation command)}
    \index{commands!#1@\texttt{#1} (documentation)}
    \index{documentation!#1@\texttt{#1} command}
}
\DeclareRobustCommand{\dreds}[1]{
    \index{#1@\texttt{#1} (editor)}
    \index{commands!#1@\texttt{#1} (editor)}
    \index{editors!#1@\texttt{#1}}
}
\DeclareRobustCommand{\drenv}[2]{
    \index{environment variables!#1@\texttt{\$}\texttt{#1} (#2)}
}
\DeclareRobustCommand{\drfnd}[2]{
    \index{#1@\texttt{#1} (#2 command)}
    \index{commands!#1@\texttt{#1} (#2)}
    \index{files and directories!#2 (\texttt{#1} command)}
}
\DeclareRobustCommand{\drios}[1]{
    \index{#1 (operating system)}
    \index{operating systems!#1}
}
\DeclareRobustCommand{\drnet}[1]{
    \index{#1@\texttt{#1} (network command)}
    \index{commands!#1@\texttt{#1} (network)}
    \index{network commands!#1@\texttt{#1}}
}
\DeclareRobustCommand{\drpkg}[1]{
    \index{#1@\texttt{#1} (package management)}
    \index{commands!#1@\texttt{#1} (package management)}
    \index{package management!#1@\texttt{#1}}
}
\DeclareRobustCommand{\drprm}[1]{
    \index{#1@\texttt{#1} (file permissions)}
    \index{commands!#1@\texttt{#1} (file permissions)}
    \index{files and directories!permissions!\texttt{#1} command}
    \index{file permissions!#1@\texttt{#1}}
}
\DeclareRobustCommand{\drscr}[1]{
    \index{#1@\texttt{#1} (scripting)}
    \index{commands!#1@\texttt{#1} (scripting)}
    \index{scripting!#1@\texttt{#1}}
}
\DeclareRobustCommand{\drshl}[1]{
    \index{#1 (shell)}
    \index{shells!#1}
}
\DeclareRobustCommand{\drstd}[1]{
    \index{#1@\textit{#1} (stream)}
    \index{I/O!streams!#1@\textit{#1}}
    \index{streams!#1@\textit{#1}}
}
\DeclareRobustCommand{\drsys}[2]{
    \index{#1@\texttt{#1} (#2 command)}
    \index{commands!#1@\texttt{#1} (#2)}
    \index{system commands!#2 (\texttt{#1} command)}
}
\DeclareRobustCommand{\drtxt}[1]{
    \index{#1@\texttt{#1} (text processing)}
    \index{commands!#1@\texttt{#1} (text processing)}
    \index{text processing!#1@\texttt{#1}}
}
\DeclareRobustCommand{\drunx}[1]{
    \index{#1 (UNIX-like environment)}
    \index{UNIX-like environments!#1}
}
\DeclareRobustCommand{\drvic}[2]{
    \index{vicommands@\texttt{vi} commands!#1@\texttt{#1} (#2)}
}
\renewcommand{\KeywordTok}[1]{{#1}}
\renewcommand{\DataTypeTok}[1]{{#1}}
\renewcommand{\DecValTok}[1]{{#1}}
\renewcommand{\BaseNTok}[1]{{#1}}
\renewcommand{\FloatTok}[1]{{#1}}
\renewcommand{\CharTok}[1]{{#1}}
\renewcommand{\StringTok}[1]{{#1}}
\renewcommand{\CommentTok}[1]{{#1}}
\renewcommand{\OtherTok}[1]{{#1}}
\renewcommand{\AlertTok}[1]{{#1}}
\renewcommand{\FunctionTok}[1]{{#1}}
\renewcommand{\RegionMarkerTok}[1]{{#1}}
\renewcommand{\ErrorTok}[1]{{#1}}
\renewcommand{\NormalTok}[1]{{#1}}
\renewcommand{\ConstantTok}[1]{{#1}}
\renewcommand{\SpecialCharTok}[1]{{#1}}
\renewcommand{\VerbatimStringTok}[1]{{#1}}
\renewcommand{\SpecialStringTok}[1]{{#1}}
\renewcommand{\ImportTok}[1]{{#1}}
\renewcommand{\DocumentationTok}[1]{{#1}}
\renewcommand{\AnnotationTok}[1]{{#1}}
\renewcommand{\CommentVarTok}[1]{{#1}}
\renewcommand{\VariableTok}[1]{{#1}}
\renewcommand{\ControlFlowTok}[1]{{#1}}
\renewcommand{\OperatorTok}[1]{{#1}}
\renewcommand{\BuiltInTok}[1]{{#1}}
\renewcommand{\ExtensionTok}[1]{{#1}}
\renewcommand{\PreprocessorTok}[1]{{#1}}
\renewcommand{\AttributeTok}[1]{{#1}}
\renewcommand{\RegionMarkerTok}[1]{{#1}}
\renewcommand{\InformationTok}[1]{{#1}}
\renewcommand{\WarningTok}[1]{{#1}}
% End of overriding commands.

% Redefines (sub)paragraphs to behave more like sections
\ifx\paragraph\undefined\else
\let\oldparagraph\paragraph
\renewcommand{\paragraph}[1]{\oldparagraph{#1}\mbox{}}
\fi
\ifx\subparagraph\undefined\else
\let\oldsubparagraph\subparagraph
\renewcommand{\subparagraph}[1]{\oldsubparagraph{#1}\mbox{}}
\fi

\begin{document}
\maketitle

\setlength{\cftfignumwidth}{4em}
\pagestyle{empty}
\setcounter{chapter}{-2}
\renewcommand{\chaptername}{Step}
\renewcommand{\contentsname}{Steps}
% Fix TOC overflow.
%\renewcommand{\@pnumwidth}{3em} 
%\renewcommand{\@tocrmarg}{4em}
% End fix TOC overflow.

{
\hypersetup{linkcolor=black}
\setcounter{tocdepth}{2}
\tableofcontents
}
\listoffigures
\cleardoublepage
\includegraphics{./images/MervBW.jpg}~\\
 Merv sez, "Don't panic." \clearpage
~\\
\hspace*{0.333em}\\
 By James Lehmer ~\\
\hspace*{0.333em}\\
 v1.0 ~\\
\hspace*{0.333em}\\
 \includegraphics{./images/cc-by-sa.png}\\
\emph{Ten Steps to Linux Survival - Bash for Windows People} by James
Lehmer is licensed under a
\href{http://creativecommons.org/licenses/by-sa/4.0/}{Creative Commons
Attribution-ShareAlike 4.0 International License}. ~\\
\hspace*{0.333em}\\
 \textbf{Dedicated to my first three technical mentors}

\begin{itemize}
\item
  Jim Proffer, who taught me digging deeper was fun and let me do so,
  often in production!
\item
  Jerry Wood, who taught me to stop and think, and once called me an
  "inveterate toolmaker" in a review, a badge I still wear with pride.
\item
  Kim Manchak, who allowed me to be more than he hired me to be, and
  continues to be a great chess opponent.
\end{itemize}

Thank you, gentlemen. I've tried to pay it forward. This book is part of
that.

\pagestyle{fancy} \fancyhead{}
\fancyhead[LE]{\slshape TEN\ STEPS\ TO\ LINUX\ SURVIVAL}
\fancyhead[RO]{\slshape \leftmark} \renewcommand{\headrulewidth}{0.4pt}

\hypertarget{introduction}{\chapter{Introduction}\label{introduction}}

\begin{quote}
\emph{"And you may ask yourself, 'Well, how did I get here?'"} - Talking
Heads (\emph{Once in a Lifetime})
\end{quote}

This is my little "Linux and Bash in 10 steps" guide. It's based on what
I consider the essentials for \sout{floundering around} acting like I
know what I'm doing in Linux, BSD and "UNIX-flavored" systems and
looking impressive among people who have only worked on Windows in the
GUI. Your "10 steps" may be different than mine and that's fine, but
this list is mine.

I said ten things, but I lied, because history is really important, so
we will start at step \#0. And since this is before even that I guess
that means this is a 12-step program...

Here is what we'll cover in the rest of this book:

\begin{enumerate}
\def\labelenumi{\arabic{enumi}.}
\setcounter{enumi}{-1}
\item
  \protect\hyperlink{some-history}{\textbf{Some History}} – UNIX vs.
  BSD, System V vs. BSD, Linux vs. BSD, POSIX, “UNIX-like,” Cygwin, and
  why any of this matters now, “Why does this script off the internet
  work on this system and not on that one?”
\item
  \protect\hyperlink{come-out-of-your-shell}{\textbf{Come Out of Your
  Shell}} – \texttt{sh} vs. \texttt{ash} vs. \texttt{bash} vs.
  everything else, "REPL”, interactive vs. scripts, command history, tab
  expansion, environment variables and "A path! A path!"
\item
  \protect\hyperlink{file-under-directories}{\textbf{File Under
  "Directories"}} – \texttt{ls}, \texttt{mv}, \texttt{cp}, \texttt{rm}
  (\texttt{-rf\ *}), \texttt{cat},
  \texttt{chmod}/\texttt{chgrp}/\texttt{chown} and everyone's favorite,
  \texttt{touch}.
\item
  \protect\hyperlink{finding-meaning}{\textbf{Finding Meaning}} – the
  \texttt{find} command in all its glory. Probably the single most
  useful command in "UNIX" (I think).
\item
  \protect\hyperlink{grokking-grep}{\textbf{Grokking \texttt{grep}}} –
  and probably gawking at \texttt{awk} while we are at it, which means
  regular expressions, too. Now we have two problems.
\item
  \protect\hyperlink{just-a-series-of-pipes}{\textbf{“Just a Series of
  Pipes”}} – \emph{stdin}/\emph{stdout}/\emph{stderr}, redirects, piping
  between commands.
\item
  \protect\hyperlink{vi}{\textbf{\texttt{vi}}} (had to be \#6, if you
  think about it) – how to stay sane for 10 minutes in \texttt{vi}.
  Navigation, basic editing, find, change/change-all, cut and paste,
  undo, saving and canceling. Plus easier alternatives like
  \texttt{nano}, and why \texttt{vi} still matters.
\item
  \protect\hyperlink{the-whole-wide-world}{\textbf{The Whole Wide
  World}} – \texttt{curl}, \texttt{wget}, \texttt{ifconfig},
  \texttt{ping}, \texttt{ssh}, \texttt{telnet}, \texttt{/etc/hosts} and
  email before Outlook.
\item
  \protect\hyperlink{the-man-behind-the-curtain}{\textbf{The Man Behind
  the Curtain}} - \texttt{/proc}, \texttt{/dev}, \texttt{ps},
  \texttt{/var/log}, \texttt{/tmp} and other things under the covers.
\item
  \protect\hyperlink{how-do-you-know-what-you-dont-know-man}{\textbf{How
  Do You Know What You Don’t Know, \texttt{man}?}} – \texttt{man},
  \texttt{info}, \texttt{apropos}, Linux Documentation Project, Debian
  and Arch guides, StackOverflow and the dangers of searching for
  “\texttt{man\ find}” or “\texttt{man\ touch}” on the internet.
\item
  \protect\hyperlink{and-so-on}{\textbf{And So On}} - \texttt{/etc},
  starting and stopping services,
  \texttt{apt-get}/\texttt{rpm}/\texttt{yum}, and more.
\end{enumerate}

Plus \protect\hyperlink{appendices}{some stuff} at the end to tie the
whole room together.

The most current release of the book should always be available for
download in different formats on
\href{https://github.com/dullroar/ten-steps-to-linux-survival/releases}{GitHub}.

\section*{Batteries Not Included}\label{batteries-not-included}
\addcontentsline{toc}{section}{Batteries Not Included}

It should be obvious that there is \textbf{\emph{plenty}} that is not
covered:

\begin{itemize}
\item
  \textbf{System initialization} - besides, the whole "UNIX" world is in
  flux right now over system initialization architecture and the shift
  from \href{https://en.wikipedia.org/wiki/Init}{"init"} scripts to
  \href{https://en.wikipedia.org/wiki/Systemd}{\texttt{systemd}}.
\item
  \textbf{Scripting logic} - scripting, logic constructs
  (\texttt{if}/\texttt{fi}, \texttt{while}/\texttt{done}, and the like).
\item
  \textbf{Desktops} - X Windows and the plethora of desktop environments
  like GNOME, KDE, Cinnamon, Mate, Unity and on and on. This is where
  "UNIX" systems get the farthest apart in terms of interoperability,
  settings and customization.
\item
  \textbf{Servers} - setting up or configuring web servers like Apache
  or node, email servers like dovecot, Samba servers for file shares,
  and so on.
\item
  \textbf{Security} - other than the simple basics of the file system
  security model.
\end{itemize}

Plus so much more. Again, this is not meant to be exhaustive, but to
help someone whose system administration experience has been limited to
Windows.

\section*{Please, Give (Suggestions)
Generously}\label{please-give-suggestions-generously}
\addcontentsline{toc}{section}{Please, Give (Suggestions) Generously}

That said, if you find something amiss in here - a typo, a misconception
or mistake, or a command or parameter you \textbf{\emph{really, really,
really}} think should be in here even though I said I am not trying to
be exhaustive, feel free to
\href{https://github.com/dullroar/ten-steps-to-linux-survival}{clone it
from GitHub}, make your changes and send me a \texttt{git\ pull}
request. Or you can try to
\href{https://github.com/dullroar/ten-steps-to-linux-survival/issues}{file
it as an issue} and I'll see how I feel that day.

\section*{Why?}\label{why}
\addcontentsline{toc}{section}{Why?}

Because I work in a primarily Windows-oriented shop, and I seem to be
"the guy" that everyone comes to when they need help on a Linux or
related system. I don't count myself a Linux guru (\textbf{\emph{at
all}}), but I have been running it since 1996 (Slackware on a laptop
with 8MB of memory!), and have worked on or run at home various ports
and flavors and and versions and distros of "UNIX" over the years,
including:

\begin{itemize}
\item
  \textbf{AIX} \drios{AIX}
\item
  \textbf{FreeBSD} \drbsd{FreeBSD}
\item
  \textbf{HP/UX} \drios{HP/UX}
\item
  \textbf{Linux} - literally more distros than I can count or remember,
  but at least Debian, Fedora, Yellow Dog, Ubuntu/Kubuntu/Xubuntu, Mint,
  Raspbian, Gentoo, Red Hat and of course the venerable Slackware.
  \drdis{Debian} \drdis{Fedora} \drdis{Gentoo} \drdis{Kubuntu}
  \drdis{Mint} \drdis{Raspbian} \drdis{Red Hat} \drdis{Slackware}
  \drdis{Ubuntu} \drdis{Xubuntu} \drdis{Yellow Dog}
\item
  \textbf{Solaris} \drios{Solaris}
\item
  \textbf{SunOS} \drios{SunOS}
\end{itemize}

All on various machines and machine architectures from mighty Sun
servers to generic "Intel" VMs down to Raspberry Pis, plus an original
"wedge" iMac running as a kitchen kiosk long after its "Best by" date
and OS/9's demise, thanks to Yellow Dog Linux. \drdis{Yellow Dog}

All that while also working on MVS, VSE, OS/2, DOS since 3.x, Windows
since 1.x, etc., etc. I don't think I am special when I list all that -
there are lots of people with my level of experience \textbf{\emph{and
better}}, especially in commercial software engineering. I am just one
of them.

But for some reason there are many places, especially in small and
medium business (SMB) environments, where the "stack" tends to be more
purely Microsoft because it keeps things simpler and cheaper for the
smaller staff. I work in such a place. The technical staff is quite
competent, but when they bump up against systems whose primary "user
interface" for system administration is a \texttt{bash} command prompt
and some scripts, they panic.

This is my attempt to help my co-workers by saying:

\begin{quote}
\emph{"Don't panic."} - Douglas Adams (\emph{Hitchhiker's Guide to the
Galaxy})
\end{quote}

It started out as a proposal I made a while ago to develop a "lunch and
learn" session of about 60-90 minutes of what I considered to be "a
Linux survival guide." The list in the
\protect\hyperlink{introduction}{\emph{Introduction}} above is based on
my original email proposal. The audience is entirely technical,
primarily "IT" (Windows/Cisco/VMWare/Exchange/SAN admins).

My goal is not to get into scripting, or system setup and hardening, or
the thousand different ways to slice a file. Instead, the scenario I see
in my head is for one of the participants in that "lunch and learn,"
armed with that discussion and having glanced through this book, to be
better able to survive if dropped into the jungle with:

\begin{quote}
\emph{"The main www site is down, and all the people who know about it
are out. It's running on some sort of Linux, I think, and the
credentials and IP address are scrawled on this sticky note. Can you get
in and poke around and see if you can figure it out?"} - your boss (next
Tuesday morning)
\end{quote}

As I started to type out my notes of what I considered to be
"essential," they just kept growing and growing. Many nights, weekends
and lunch hours later, this is the result. The slides were much easier
to prepare now that I have the "notes"!

\textbf{Note:} - The slides are included in
\href{https://github.com/dullroar/ten-steps-to-linux-survival/releases}{the
same GitHub repository as this book}.

\section*{Caveat Administrator}\label{caveat-administrator}
\addcontentsline{toc}{section}{Caveat Administrator}

Even so, anything like this is incomplete. Anyone truly knowledgable of
Linux will splutter their coffee into their neckbeard\footnote{Stereotype
  intentional.} at least once a chapter because I don't mention a
parameter on a command or an entire subject at all! And that's right -
because this "survival guide" is already long enough.

This book is not meant to be an authoritative source, but instead a
\href{https://en.wikipedia.org/wiki/Fake_book}{"fake book"} for getting
up and running \textbf{\emph{quickly}} with the sheer basics, plus
knowing where to go for help. I modeled it explicitly after "short and
opinionated" tech books such as Douglas Crockford's
\href{http://shop.oreilly.com/product/9780596517748.do}{\emph{Javascript:
The Good Parts}} and especially those licensed under
\href{http://creativecommons.org/licenses/by-sa/4.0/}{Creative Commons},
such as the books from \href{http://greenteapress.com/}{Green Tea
Press}. If you like those big tech books that are priced by the
kilogram, this is not the book for you.

It is also not a replacement for reading the real documentation and
doing research and testing, especially in production! But hopefully it
will help get you through that "Can you get in and poke around and see
if you can figure it out?" scenario above. And if Linux should start
becoming more of your job, maybe this will help as a gentle push toward
"RTFM" along with thinking in "The UNIX Way."

\textbf{WARNING:} \textbf{\emph{Many of the commands in this book can
alter your system and possibly damage it.}}

Obvious candidates include the file system commands like \texttt{rm},
the \texttt{vi} editor (obviously), and some of the "system admin"
commands mentioned later, including system and service restarts. Use
your common sense plus the various resources for documentation mentioned
in this book to make sure you aren't doing anything destructive to your
system, especially in production. \dreds{vi} \drfnd{rm}{remove file}

\textbf{\emph{YOU HAVE BEEN WARNED!}}

\section*{Conventions}\label{conventions}
\addcontentsline{toc}{section}{Conventions}

If a command, file name or other "computer code" is shown in-line in a
sentence, it will appear in a fixed-width font, e.g.,
\texttt{ls\ -\/-recursive\ *.txt}.

If a command and its output, script code or something else is shown in a
block, it will appear like this:

\drcap{Sample command}

\begin{Shaded}
\begin{Highlighting}[]
\KeywordTok{~} \NormalTok{$ }\KeywordTok{ps} \NormalTok{-AH}
  \KeywordTok{PID} \NormalTok{TTY          TIME CMD}
    \KeywordTok{2} \NormalTok{?        00:00:00 kthreadd}
    \KeywordTok{3} \NormalTok{?        00:00:00   ksoftirqd/0}
    \KeywordTok{5} \NormalTok{?        00:00:00   kworker/0:0H}
    \KeywordTok{7} \NormalTok{?        00:00:06   rcu_sched}
    \KeywordTok{8} \NormalTok{?        00:00:02   rcuos/0}
    \KeywordTok{9} \NormalTok{?        00:00:01   rcuos/1}
   \KeywordTok{10} \NormalTok{?        00:00:03   rcuos/2}
   \KeywordTok{11} \NormalTok{?        00:00:01   rcuos/3}
   \KeywordTok{12} \NormalTok{?        00:00:00   rcuos/4}
   \KeywordTok{13} \NormalTok{?        00:00:00   rcuos/5}
   \KeywordTok{14} \NormalTok{?        00:00:00   rcuos/6}
   \KeywordTok{15} \NormalTok{?        00:00:00   rcuos/7}
   \KeywordTok{16} \NormalTok{?        00:00:00   rcu_bh}
   \KeywordTok{17} \NormalTok{?        00:00:00   rcuob/0}
   \KeywordTok{18} \NormalTok{?        00:00:00   rcuob/1}
   \KeywordTok{19} \NormalTok{?        00:00:00   rcuob/2}
   \KeywordTok{20} \NormalTok{?        00:00:00   rcuob/3}
   \KeywordTok{21} \NormalTok{?        00:00:00   rcuob/4}
   \KeywordTok{22} \NormalTok{?        00:00:00   rcuob/5}
   \KeywordTok{23} \NormalTok{?        00:00:00   rcuob/6}
   \KeywordTok{24} \NormalTok{?        00:00:00   rcuob/7}
\KeywordTok{...and} \NormalTok{so on...}
\end{Highlighting}
\end{Shaded}

All such blocks have been normalized to only show a maximum of 80x24
characters. This is intentional. While most modern "UNIX" systems and
terminal windows like \texttt{ssh} can handle any geometry, there are
still systems and situations where you get the same terminal size that
your grandfather would've used. It is best to learn how to deal with
these by using \texttt{less}, redirection and the like. \drtxt{less}
\drnet{ssh}
\index{files and directories!paginate!less@\texttt{less} command}
\index{pagination!\texttt{less} command}

The examples in this book typically show something like
\texttt{\textasciitilde{}\ \$} before the command, or
\texttt{\textasciitilde{}\ \#} (when logged in as root) or \texttt{\%}
(when running under \texttt{csh}). These "command prompts" are set in
\texttt{bash} via the
\href{https://www.linux.com/learn/docs/ldp/443-bash-prompt-howto}{\texttt{PS1}
environment variable} and are not meant to be typed in as part of the
command. \drenv{PS1}{command prompt string} \drshl{bash} \drshl{csh}

In the few places where a "UNIX" command is shown in comparison to a
"DOS" command run under \texttt{CMD.EXE}, the latter is shown in all
uppercase to help distinguish it from the "UNIX" equivalent, even though
\texttt{CMD.EXE} is case-insensitive. In other words, \texttt{set} will
be shown for \texttt{bash} and \texttt{SET} for \texttt{CMD.EXE}.
\drshl{CMD.EXE}

\section*{How to Get There from Here}\label{how-to-get-there-from-here}
\addcontentsline{toc}{section}{How to Get There from Here}

Before we get too far, let's talk about how to connect to a "UNIX"
system in the first place. If you have an actual physical machine you
can use the console. If you are running VMs you can use the VM
software's console mechanism as well. But most such systems run
\href{http://www.openssh.com/}{OpenSSH}, a "secure shell" service
(\href{http://linux.die.net/man/8/sshd}{\texttt{sshd}}), which creates
an encrypted terminal connection via TCP/IP over port 22 (so obviously,
if you are connecting off-premise the appropriate firewall holes will
have to be in place on both sides). This allows you to connect from
anywhere you want to work, like from your laptop sitting on the couch
watching TV. \drnet{sshd}

On Windows there are generally two ways to establish SSH sessions with
"UNIX" systems:

\begin{enumerate}
\def\labelenumi{\arabic{enumi}.}
\tightlist
\item
  \href{http://www.chiark.greenend.org.uk/~sgtatham/putty/}{\textbf{PuTTY}}
  - this is a Windows program and is pretty self-explanatory. Just give
  it the remote system's address and connect.
\end{enumerate}

\drnet{PuTTY}

\begin{enumerate}
\def\labelenumi{\arabic{enumi}.}
\setcounter{enumi}{1}
\tightlist
\item
  \href{http://linux.die.net/man/1/ssh}{\textbf{\texttt{ssh}}} - if you
  are running \href{http://cygwin.com/}{Cygwin} on Windows or own a Mac,
  you can simply use the \texttt{ssh} command from the Cygwin or Mac
  command prompt:
\end{enumerate}

\begin{Shaded}
\begin{Highlighting}[]
\KeywordTok{~} \NormalTok{$ }\KeywordTok{ssh} \NormalTok{remotehost}
\KeywordTok{~} \NormalTok{$ }\KeywordTok{ssh} \NormalTok{myuser@remotehost}
\KeywordTok{~} \NormalTok{$ }\KeywordTok{ssh} \NormalTok{myuser%mypassword@remotehost}
\end{Highlighting}
\end{Shaded}

\drnet{ssh} \drunx{Cygwin}

With either PuTTY or \texttt{ssh}, you typically must supply
credentials. On PuTTY you can specify them in advance in the Windows UI,
or answer the userid and password prompts when the terminal window
opens. With the \texttt{ssh} command you can answer the prompts or
specify them on the command line, although it is not recommended to pass
the password via the command line unless you have your \texttt{bash}
history file set to not record the \texttt{ssh} command (covered later).

\textbf{Note:} There are also ways to connect using public/private key
pairs, but that is beyond the scope of this book.

The first time you connect to a remote system via SSH you are going to
see a prompt similar to the following:

\drcap{First ssh connection}

\begin{Shaded}
\begin{Highlighting}[]
\KeywordTok{~} \NormalTok{$ }\KeywordTok{ssh} \NormalTok{myuser@remotehost}
\KeywordTok{The} \NormalTok{authenticity of host }\StringTok{'remotehost (192.168.123.1)'} \NormalTok{can}\StringTok{'t be established.}
\StringTok{ECDSA key fingerprint is 11:c4:c5:dd:75:b0:26:83:dc:94:34:ef:10:f5:d9:c7.}
\StringTok{Are you sure you want to continue connecting (yes/no)? }
\end{Highlighting}
\end{Shaded}

Simply answer \texttt{yes} and the remote host's key fingerprint will be
stored so you don't have to answer it again. However, if you've already
answered that prompt and you see it again \textbf{\emph{for the same
machine}}, that means the remote machine's IP address or other
configuration has changed. That is often OK if you know that happened -
changing the hosting provider for your public web server will trigger it
for sure. However, if you know of no such changes, it may be indication
of a compromise, and you should abort the login and ask around first.

\section*{Acknowledgments}\label{acknowledgments}
\addcontentsline{toc}{section}{Acknowledgments}

Thanks to Ken Astl for reading an early draft of this book, and to Jason
Koopmans for passing it around his work. I appreciate the detailed and
thoughtful discussions I had with Margaret Devere around designing good
indexes. I received excellent advice and promotion from Professor Allen
Downey. My boss, Bryan Henderson, was supportive of the original
"lunch-and-learn" concept and listened to me ramble about this book.
Thanks to my coworkers who attended those lunch-and-learn sessions,
asked questions and helped me refine this book - Aaron Vandegriff, Rob
Harvey, Jason Walters, Carmen Samson, John Wieland, Patrick Mistler and
our CIO, Rick Derks. And finally, I owe more than I can repay (as usual)
to my wife Leslie for putting up with me while I obsessed over this
project.

\hypertarget{some-history}{\chapter{Some History}\label{some-history}}

\textbf{\emph{UNIX vs. BSD, System V vs. BSD, Linux vs. BSD, POSIX,
“UNIX-like,” Cygwin, and why any of this matters now. “Why does this
script off the internet work on this system and not on that one?”}}

\begin{quote}
\emph{"That men do not learn very much from the lessons of history is
the most important of all the lessons of history."} - Aldous Huxley
\end{quote}

UNIX and its successors such as Linux have a long history reaching into
the depths of time:

\begin{itemize}
\item
  \textbf{Prehistory} - late 1960s, Nixon, Vietnam, Woodstock, Moon
  landing, \href{https://en.wikipedia.org/wiki/Multics}{Multics} at MIT,
  GE and Bell Labs. \drios{Multics}
\item
  \textbf{In the beginning} - early 1970s, Nixon drags on, Watergate,
  Bell Labs, \href{https://en.wikipedia.org/wiki/Ken_Thompson}{Thompson}
  \& \href{https://en.wikipedia.org/wiki/Dennis_Ritchie}{Ritchie},
  \href{https://en.wikipedia.org/wiki/History_of_Unix}{UNIX} is born.
  \drios{UNIX}
\item
  \textbf{More trouble from Berkeley} - late 1970s, Carter, disco, Iran
  hostages, UC Berkeley releases the
  \href{https://en.wikipedia.org/wiki/Berkeley_Software_Distribution}{Berkeley
  Software Distribution} (BSD), a port based on the Bell Labs UNIX. Let
  the forking begin! \drios{BSD}
\item
  \textbf{UNIX goes commercial} - 1980s, Reagan, Iran Contra,
  \emph{E.T.}, AT\&T releases
  \href{https://en.wikipedia.org/wiki/UNIX_System_V}{System V} as first
  commercial UNIX. From the same background as Bell Labs UNIX, System V
  evolved with subtle and not so subtle differences in approaches to
  command syntax, networking and much more. It is this release and
  AT\&T's copyrights that are the basis of all the SCO-vs-Linux lawsuits
  2-3 decades later. \drios{System V}
\item
  \textbf{Explosion of "UNIX"} - late 1980s/early 1990s, Bush I, Berlin
  Wall falls, Gulf War I, proliferation of proprietary (and different)
  "UNIX" platforms:

  \begin{itemize}
  \item
    \textbf{HP HP-UX} \drios{HP-UX}
  \item
    \textbf{Sun SunOS} - BSD flavor. \drios{SunOS}
  \item
    \textbf{Sun Solaris} - System V flavor. Now Oracle Solaris.
    \drios{Solaris}
  \item
    \textbf{IBM AIX} \drios{AIX}
  \item
    \textbf{SGI IRIX} \drios{IRIX}
  \item
    \textbf{...and many, many more!} - although mostly all that's left
    now is HP-UX, AIX and Solaris.
  \end{itemize}
\item
  \textbf{Linux} - 1991+, Clinton I, grunge, \emph{Titanic},
  \href{https://en.wikipedia.org/wiki/Linus_Torvalds}{Linus Torvalds}
  releases a project called
  \href{https://en.wikipedia.org/wiki/Linux}{Linux} based on
  \href{https://en.wikipedia.org/wiki/MINIX}{MINIX} (and hence why Linus
  says Linux is pronounced like "MINIX" and not like "Linus").
  \drios{Linux} \drios{MINIX}
\item
  \textbf{Proliferation of the BSDs} - mid-to-late 1990s, still Clinton
  I, Monicagate, Kosovo, various ports of BSD including
  \href{https://en.wikipedia.org/wiki/NetBSD}{NetBSD},
  \href{https://en.wikipedia.org/wiki/FreeBSD}{FreeBSD} and
  \href{https://en.wikipedia.org/wiki/OpenBSD}{OpenBSD}. All happen in
  the same time frame as Linux. Like Linux distros, each has its own
  focus and prejudices, some of which are distinctly "anti-Linux." The
  "big three" are all still in heavy use today, especially among ISPs.
  The perception is still out there among a generation of sysadmins that
  Linux is for the desktop and BSDs for servers, but that reality
  shifted a long time ago. \drbsd{FreeBSD} \drbsd{NetBSD}
  \drbsd{OpenBSD}
\item
  \textbf{Ports of call} - 2000+, Bush II \& Obama, Afghanistan \& Gulf
  War II, lots of cross-porting of everything open source. However,
  \href{https://en.wikipedia.org/wiki/Open-source_license}{licenses
  matter}, and
  \href{https://en.wikipedia.org/wiki/Comparison_of_free_and_open-source_software_licenses}{there
  sure are a lot of them}. While things have settled down some with the
  dismissal of the SCO lawsuit, intellectual property remains a problem
  area in open source, even as the use of open source software (OSS) has
  exploded.
\end{itemize}

\textbf{Q:} So, what's Linux? Or BSD? Or even UNIX?

\textbf{A:} Depends on who you're asking and in what context!

Hence, for the rest of this text I will tend to talk somewhat
interchangeably about "Linux" and "UNIX" and the like. When it matters,
I will mention which OS I am discussing by name, but often I will use
"UNIX" (in quotes) to mean anything in the "family tree" of the original
Bell Labs offspring, or that "acts like," well, UNIX.

To further muddy the waters, there have been multiple attempts to
"standardize" whatever it is this thing is called:

\begin{itemize}
\tightlist
\item
  \href{https://en.wikipedia.org/wiki/POSIX}{\textbf{POSIX}} - a de jure
  set of standards created in the 1980s and 1990s to try to bring order
  to the chaos that was commercial UNIX-flavored operating systems of
  the time. It worked. Sorta. Especially once the US government started
  wanting systems to be "POSIX-compliant." \drunx{POSIX}
\end{itemize}

\textbf{Note:} No system runs POSIX. All POSIX-compliant system are
"similar but different." Even Windows can claim to be POSIX-compliant in
some respects (and has an installable POSIX subsystem) but that doesn't
mean POSIX-compliant code will run there unchanged. \drios{Windows}

\begin{itemize}
\item
  \href{https://en.wikipedia.org/wiki/GNU_Project}{\textbf{GNU Project}}
  - \href{https://en.wikipedia.org/wiki/Richard_Stallman}{Richard
  Stallman} founded the
  \href{https://en.wikipedia.org/wiki/Free_Software_Foundation}{Free
  Software Foundation} (FSF) and GNU project in the mid-1980s,
  \textbf{\emph{long}} before Linux (GNU = "GNU's Not Unix"). The GNU
  project delivers \href{https://www.gnu.org/software/software.html}{a
  suite of programs and tools}, many of which are used in both Linux and
  BSD variants as de facto standards. \drunx{GNU}
\item
  \textbf{Various Linux Efforts} - there have also been various
  movements over the years, some more successful than others, to
  "standardize" Linux or some part of it, such as the file system
  layout, the \texttt{init} system, documentation, and now even what is
  part of the most basic "core OS" for things like better
  containerization.
\end{itemize}

\section*{Why Does This Matter?}\label{why-does-this-matter}
\addcontentsline{toc}{section}{Why Does This Matter?}

Because there are various "flavors" of commands and tools, based on
whether you're dealing with a System V (Linux) or BSD (Free/Net/Open)
descendant. Some of the OS versions are strong in security, or
networking, or as a desktop. Certain things are "built-in" to the
operating system but most are installed as packages, and depending on
the source of the package it may or may not work correctly on another
"UNIX" system without effort.

It is similar to the history and relationship between
\texttt{COMMAND.EXE} in DOS and \texttt{CMD.EXE} in Windows 10, where
this would work in both: \drshl{COMMAND.EXE} \drshl{CMD.EXE}

\begin{Shaded}
\begin{Highlighting}[]
\KeywordTok{COPY} \NormalTok{A.TXT B.TXT}
\end{Highlighting}
\end{Shaded}

But only the later, long file name and network-aware \texttt{CMD.EXE}
could handle:

\begin{Shaded}
\begin{Highlighting}[]
\KeywordTok{COPY} \StringTok{"My 2015 Tax Returns.pdf"} \DataTypeTok{\textbackslash{}\textbackslash{}}\NormalTok{MyServer\textbackslash{}Finances\textbackslash{}.}
\end{Highlighting}
\end{Shaded}

In UNIX-land over time these differences seem to be getting better, but
there are still "gotchas," often involving the differences in open
source licenses in the underlying code. There are fundamental
differences and assumptions between the "GNU" and "GPL" licenses on the
one side and "MIT" and "BSD" licenses on the other. I am not a lawyer,
but I would summarize:

\begin{itemize}
\item
  \textbf{FSF/GNU/GPL} - mostly concerned with keeping open source
  "open," that is sharable and modifiable by all.
\item
  \textbf{BSD \& MIT} - more focused on letting anyone do anything to
  the code as long as the original author is acknowledged and liability
  released.
\end{itemize}

The best thing is to be vaguely aware of this history and licenses and
if something isn't available on a certain platform or if a command isn't
taking a specific parameter to search for variants.

For example, note the difference in output between showing all processes
with the \href{http://linux.die.net/man/1/ps}{\texttt{ps}}
(\emph{process}) command on a Linux system, in this case Linux Mint
under \texttt{bash}: \drshl{bash} \drsys{ps}{list processes}

\drcap{\texttt{ps} on Linux in \texttt{bash}}

\begin{Shaded}
\begin{Highlighting}[]
\KeywordTok{~} \NormalTok{$ }\KeywordTok{ps} \NormalTok{-a}
  \KeywordTok{PID} \NormalTok{TTY          TIME CMD}
 \KeywordTok{4508} \NormalTok{pts/3    00:00:00 su}
 \KeywordTok{4516} \NormalTok{pts/3    00:00:00 bash}
 \KeywordTok{4594} \NormalTok{pts/3    00:00:00 ps}
\end{Highlighting}
\end{Shaded}

Versus the "same" command on a FreeBSD system at my ISP, where
\texttt{csh} is the default shell: \drbsd{FreeBSD} \drshl{csh}

\drcap{\texttt{ps} on FreeBSD in \texttt{csh}}

\begin{Shaded}
\begin{Highlighting}[]
\KeywordTok{%ps} \NormalTok{-a}
  \KeywordTok{PID}  \NormalTok{TT  STAT      TIME COMMAND}
 \KeywordTok{5073}  \NormalTok{p0  Ss     0:00.02 -csh (csh)}
 \KeywordTok{5115}  \NormalTok{p0  RN+    0:00.00 ps -a}
\end{Highlighting}
\end{Shaded}

To make things even more confusing, the Linux version of \texttt{ps} has
been written to understand the BSD-style syntax and flags, too!

\section*{Panic at the Distro}\label{panic-at-the-distro}
\addcontentsline{toc}{section}{Panic at the Distro}

Remember that "Linux," FreeBSD, OpenBSD and NetBSD are all really just
OS kernels, boot loaders, drivers and enough functionality to get a
computer up and running. Most functionality comes via other "packages."
From almost the beginning there have been alternative approaches to both
what packages should (and should not) be included, as well as how to
best manage the installing, updating and removal of those packages.
\drbsd{FreeBSD} \drbsd{NetBSD} \drbsd{OpenBSD} \drios{Linux}

In the BSD world each major port has its own approach. In the Linux
world the job of deciding all this and putting it all together falls to
distributions or "distros." These have evolved over time into a series
of
\href{https://en.wikipedia.org/wiki/Linux_distribution\#Popular_distributions}{"families"}
based in large part around the
\href{https://en.wikipedia.org/wiki/Package_manager}{package management
tool} predominantly used: \index{Linux distros}

\begin{itemize}
\item
  \textbf{\texttt{apt-get}, \texttt{dpkg} and \texttt{.deb} files} -
  \href{https://en.wikipedia.org/wiki/Debian}{Debian} flavors, such as
  \href{https://en.wikipedia.org/wiki/Ubuntu_\%28operating_system\%29}{Ubuntu}
  and \href{https://en.wikipedia.org/wiki/Linux_Mint}{Mint}. Mint is
  currently my desktop Linux of choice and Debian my preferred server
  OS, both based on familiarity. \drdis{Debian} \drdis{Mint}
  \drdis{Ubuntu} \drpkg{apt-get} \drpkg{dpkg}
  \index{files and directories!debpackagefiles@\texttt{.deb} package files}
\item
  \textbf{\texttt{pacman}} -
  \href{https://en.wikipedia.org/wiki/Arch_Linux}{Arch} flavors.
  \drdis{Arch} \drpkg{pacman}
\item
  \textbf{\texttt{rpm}, \texttt{yum} and \texttt{.rpm} files} - Red Hat
  flavors,such as
  \href{https://en.wikipedia.org/wiki/Fedora_\%28operating_system\%29}{Fedora},
  \href{https://en.wikipedia.org/wiki/Red_Hat_Enterprise_Linux}{Red Hat
  Enterprise} and \href{https://en.wikipedia.org/wiki/CentOS}{CentOS}.
  \drdis{CentOS} \drdis{Fedora} \drdis{Red Hat}
  \drdis{Red Hat Enterprise Linux (RHEL)} \drpkg{rpm} \drpkg{yum}
  \index{files and directories!debpackagefiles@\texttt{.rpm} package files}
\item
  \textbf{Source code} -
  \href{https://en.wikipedia.org/wiki/Gentoo_Linux}{Gentoo} tends to be
  a "compile from scratch" environment, much like
  \href{https://en.wikipedia.org/wiki/FreeBSD_Ports}{FreeBSD}.
  \drbsd{FreeBSD} \drdis{Gentoo}
\item
  \textbf{"Tar balls"} - source code or binaries delivered via archived
  and zipped directories. Common on
  \href{https://en.wikipedia.org/wiki/Slackware}{Slackware}, some
  others. \drdis{Slackware}
\end{itemize}

\section*{Get Embed With Me}\label{get-embed-with-me}
\addcontentsline{toc}{section}{Get Embed With Me}

A lot of firmware in embedded devices is based on some sort of "UNIX"
flavor. Networking gear at both the consumer and enterprise level,
storage devices and so on all tend to run something that "looks like"
UNIX at some level.
\href{https://en.wikipedia.org/wiki/BusyBox}{BusyBox} is a good example
of a "UNIX-like" shell (command prompt) used by many embedded systems.
Of course, as to what's actually available, who knows? If you can get
shell open, the best thing to do is see what works. \drshl{BusyBox}
\drunx{BusyBox}

\section*{Cygwin}\label{cygwin}
\addcontentsline{toc}{section}{Cygwin}

\href{http://cygwin.com/}{Cygwin} is an interesting beast. It is a DLL
for Windows that implements most of the POSIX and related UNIX-like
"system API calls" for programming, and then is also a series of ported
open source packages, including shells, utilities and even desktop
environments, all \textbf{\emph{recompiled}} to run on Windows as long
as the Cygwin DLL is accessible. Like a Linux distro it has an installer
that is a "package manager," and if a package isn't available, you can
usually recompile the source code using Cygwin. \drunx{Cygwin}

You cannot run Linux or BSD binaries on Cygwin without recompiling them
first. \textbf{However}, you can often run \textbf{\emph{scripts}} from
a Linux environment on Cygwin with little or no tweaking. Which means
you can then take advantage of a lot of excellent open source tools
simply by installing their packages in Cygwin and running scripts
against them.

Ultimately, though, Cygwin is of limited use, basically for getting to
some open source tools on Windows without having to set up a Linux box.
You can do a lot of amazing things with Cygwin with enough effort
(including running X and a desktop environment like GNOME!), but at some
point why not expend that effort in standing up a "real" Linux (virtual)
machine anyway?

\hypertarget{come-out-of-your-shell}{\chapter{Come Out of Your
Shell}\label{come-out-of-your-shell}}

\textbf{\emph{\texttt{sh} vs. \texttt{ash} vs. \texttt{bash} vs.
everything else, "REPL”, interactive vs. scripts, command history, tab
expansion, environment variables and "A path! A path!"}}

\begin{quote}
\emph{"If you hold a shell up to your ear, you can hear the OS."} - me
\end{quote}

To avoid getting all pedantic, I am just going to define a shell as an
environment in which you can execute commands. People tend to think of a
shell as a "command prompt," but you can run a shell without running a
command prompt, but not vice versa - an interactive command prompt is an
instance of a shell environment almost by definition.

Examples of shells:

\begin{itemize}
\item
  \href{https://technet.microsoft.com/en-us/library/cc754340.aspx}{\textbf{\texttt{CMD.EXE}}}
  - yes, Windows has a shell. \drios{Windows} \drshl{CMD.EXE}
\item
  \href{https://technet.microsoft.com/en-us/library/ms714469\%28v=VS.85\%29.aspx}{\textbf{\texttt{PowerShell.exe}}}
  - in fact, it has at least two! \drshl{Powershell}
\end{itemize}

In UNIX-land:

\begin{itemize}
\item
  \href{https://en.wikipedia.org/wiki/Bourne_shell}{\textbf{\texttt{sh}}}
  - the "original" Bourne shell in UNIX, which spawned: \drshl{sh}

  \begin{itemize}
  \item
    \href{https://en.wikipedia.org/wiki/Almquist_shell}{\textbf{\texttt{ash}}}
    - Almquist shell. \drshl{ash}

    \begin{itemize}
    \tightlist
    \item
      \textbf{\texttt{dash}} - Debian Almquist shell (replaced
      \texttt{ash} in Debian). \drshl{dash}
    \end{itemize}
  \item
    \href{https://en.wikipedia.org/wiki/Bash_\%28Unix_shell\%29}{\textbf{\texttt{bash}}}
    - Bourne-again shell (get it?), the "standard" Linux shell (as much
    as anything is standard across Linux distros). \drshl{bash}
  \item
    \href{https://en.wikipedia.org/wiki/Korn_shell}{\textbf{\texttt{ksh}}}
    - Korn shell. \drshl{ksh}
  \item
    \href{https://en.wikipedia.org/wiki/Z_shell}{\textbf{\texttt{zsh}}}
    - Z shell. \drshl{zsh}
  \end{itemize}
\item
  \href{https://en.wikipedia.org/wiki/C_shell}{\textbf{\texttt{csh}}} -
  C shell, historically it is the default shell on BSD systems (although
  there are arguments on why you should
  \href{http://www.faqs.org/faqs/unix-faq/shell/csh-whynot/}{\textbf{\emph{never
  use it}}}). \drshl{csh}
\item
  \textbf{...and many more!} -
  \href{https://en.wikipedia.org/wiki/Unix_shell\#Shell_categories}{tons,
  really}.
\end{itemize}

Most Linux distros use \texttt{bash}, but the BSDs are all over the
place. We're going to assume \texttt{bash} for the rest of this
tutorial. With few modifications, anything in the \texttt{sh} hierarchy
above can usually run in the other members of the same tree.

\section*{\texorpdfstring{\texttt{bash}
Built-Ins}{bash Built-Ins}}\label{bash-built-ins}
\addcontentsline{toc}{section}{\texttt{bash} Built-Ins}

Every shell has some "built-in" commands that are implemented as part of
the shell and not as an external command or program, and \texttt{bash}
has its share, as shown by running the
\href{http://linux.die.net/man/1/help}{\texttt{help}} command in a
\texttt{bash} terminal: \drdoc{help} \drshl{bash}

\drcap{Built-in commands in \texttt{bash}}

\begin{Shaded}
\begin{Highlighting}[]
\KeywordTok{~} \NormalTok{$ }\KeywordTok{help}
\KeywordTok{GNU} \NormalTok{bash, version 4.3.11(1)}\KeywordTok{-release} \NormalTok{(x86_64-pc-linux-gnu)}
\KeywordTok{These} \NormalTok{shell commands are defined internally.  Type }\KeywordTok{`help}\StringTok{' to see this list.}
\StringTok{Type `help name'} \NormalTok{to find out more about the function }\KeywordTok{`}\NormalTok{name}\StringTok{'.}
\StringTok{Use `info bash'} \NormalTok{to find out more about the shell in general.}
\KeywordTok{Use} \KeywordTok{`man} \NormalTok{-k}\StringTok{' or `info'} \NormalTok{to find out more about commands not in this list.}

\KeywordTok{A} \NormalTok{star (*) }\KeywordTok{next} \NormalTok{to a name means that the command is disabled.}

 \KeywordTok{job_spec} \NormalTok{[}\KeywordTok{&}\NormalTok{]                            }\KeywordTok{history} \NormalTok{[-c] [-d offset] [n] or hist}\KeywordTok{>}
 \KeywordTok{((} \NormalTok{expression }\KeywordTok{))}                        \KeywordTok{if} \KeywordTok{COMMANDS;} \KeywordTok{then} \KeywordTok{COMMANDS; [} \NormalTok{elif C>}
 \NormalTok{. filename [arguments]                  jobs [-lnprs] [jobspec ...] or jobs >}
 \NormalTok{:                                       kill [}\OtherTok{-s} \NormalTok{sigspec | }\OtherTok{-n} \NormalTok{signum | -sigs>}
 \NormalTok{[ arg...}\KeywordTok{ ]}                              \KeywordTok{let} \NormalTok{arg [arg ...]}
\KeywordTok{ [[} \NormalTok{expression}\KeywordTok{ ]]}                        \KeywordTok{local} \OtherTok{[option]} \OtherTok{name[=value]} \NormalTok{...}
 \KeywordTok{alias} \NormalTok{[-p] [name[=value] ... ]          logout [n]}
 \KeywordTok{bg} \NormalTok{[job_spec ...]                       mapfile [-n count] [-O origin] [-s c}\KeywordTok{>}
 \KeywordTok{bind} \NormalTok{[-lpsvPSVX] [-m keymap] [-f file}\KeywordTok{>}  \NormalTok{popd [-n] [+N }\KeywordTok{|} \KeywordTok{-N}\NormalTok{]}
 \KeywordTok{break} \NormalTok{[n]                               printf [-v var] format [arguments]}
 \KeywordTok{builtin} \NormalTok{[shell-builtin [arg ...]]       pushd [-n] [+N }\KeywordTok{|} \KeywordTok{-N} \KeywordTok{|} \KeywordTok{dir}\NormalTok{]}
 \KeywordTok{caller} \NormalTok{[expr]                           pwd [-LP]}
 \KeywordTok{case} \NormalTok{WORD}\KeywordTok{ in} \NormalTok{[PATTERN [}\KeywordTok{|} \NormalTok{PATTERN]...}\KeywordTok{)>}  \KeywordTok{read} \OtherTok{[-ers]} \OtherTok{[-a array]} \OtherTok{[-d delim]} \OtherTok{[->}
\OtherTok{ cd [-L|[-P [-e]}\NormalTok{] [}\KeywordTok{-@}\NormalTok{]] [dir]            readarray [-n count] [-O origin] [-s}\KeywordTok{>}
\KeywordTok{...and} \NormalTok{so on...}
\end{Highlighting}
\end{Shaded}

Why does this matter? Because if you are in an environment and something
as fundamental as \texttt{echo} isn't working, you may not be working in
a shell that is going to act like a "\texttt{sh}" shell.
\textbf{\emph{In general}}, \texttt{sh}, \texttt{ash}, \texttt{bash},
\texttt{dash} and \texttt{ksh} all act similarly enough that you don't
care, but sometimes you may have to care. Knowing if you are on a
\texttt{csh} variant or even something more esoteric can be key.
\drtxt{echo} \drshl{csh}

Pay attention to the first line in script files, which will typically
have a
\href{https://en.wikipedia.org/wiki/Shebang_\%28Unix\%29}{"shebang"}
line that looks like this: \index{*@\texttt{\#"!} (shebang)}

\drcap{\texttt{bash} "shebang"}

\begin{Shaded}
\begin{Highlighting}[]
\CommentTok{#!/bin/bash}
\end{Highlighting}
\end{Shaded}

In this case we know the script is expecting to be executed by
\texttt{bash}, and in fact should throw an error if \texttt{/bin/bash}
doesn't exist. For example, on the FreeBSD system I have access to,
\texttt{dash} is not installed. So consider the following
\texttt{hello.sh} script: \drbsd{FreeBSD}

\drcap{Script with `dash` "shebang"}

\begin{Shaded}
\begin{Highlighting}[]
\CommentTok{#!/bin/dash}
\KeywordTok{echo} \NormalTok{Hello, World!}
\end{Highlighting}
\end{Shaded}

When I try to run it on FreeBSD, I get:

\drcap{"Shebang" error}

\begin{Shaded}
\begin{Highlighting}[]
\KeywordTok \NormalTok{./hello.sh }
\KeywordTok{Hello}\NormalTok{, World!}
\end{Highlighting}
\end{Shaded}

Note that on some systems \texttt{\#!/bin/sh} points to an alias of
\texttt{bash}, and on some it is a different implementation of the
original \texttt{sh} command, such as \texttt{ash} or \texttt{dash}. Now
you know what to search for if you hit problems as simple as an expected
"built-in" command not being found.

\section*{Everything You Know is (Almost)
Wrong}\label{everything-you-know-is-almost-wrong}
\addcontentsline{toc}{section}{Everything You Know is (Almost) Wrong}

\texttt{CMD.EXE} has a lineage that is a mish-mash of CP/M and UNIX
excreted through three decades of backwards compatibility to that
devil's spawn we call DOS. It has gotten even muddier over the years as
Microsoft has added more commands, PowerShell, POSIX subsystems, etc.
\drios{CP/M} \drios{DOS} \drshl{CMD.EXE}

But even so, there are some similarities between \texttt{CMD.EXE} and a
Linux shell like \texttt{bash}. In both \texttt{bash} and
\texttt{CMD.EXE} the \href{http://linux.die.net/man/1/set}{\texttt{set}}
command shows you all environment variables that have been set. Here's
\texttt{bash}: \drcmd{set}{set shell options} \drshl{bash}
\drshl{CMD.EXE}
\index{environment variables!displaying!setcommand@\texttt{set} command}

\drcap{\texttt{set} command in \texttt{bash}}

\begin{Shaded}
\begin{Highlighting}[]
\KeywordTok{~} \NormalTok{$ }\KeywordTok{set}
\OtherTok{BASH=}\NormalTok{/bin/bash}
\OtherTok{BASHOPTS=}\NormalTok{checkwinsize:cmdhist:complete_fullquote:expand_aliases:extglob:extquote}
\NormalTok{:}\KeywordTok{force_fignore}\NormalTok{:histappend:interactive_comments:login_shell:progcomp:promptvars:s}
\KeywordTok{ourcepath}
\OtherTok{BASH_ALIASES=()}
\OtherTok{BASH_ARGC=()}
\OtherTok{BASH_ARGV=()}
\OtherTok{BASH_CMDS=()}
\OtherTok{BASH_COMPLETION_COMPAT_DIR=}\NormalTok{/etc/bash_completion.d}
\OtherTok{BASH_LINENO=()}
\OtherTok{BASH_SOURCE=()}
\OtherTok{BASH_VERSINFO=([0]=}\StringTok{"4"} \OtherTok{[1]=}\StringTok{"3"} \OtherTok{[2]=}\StringTok{"11"} \OtherTok{[3]=}\StringTok{"1"} \OtherTok{[4]=}\StringTok{"release"} \OtherTok{[5]=}\StringTok{"x86_64-pc-lin}
\StringTok{ux-gnu"}\OtherTok{)}
\OtherTok{BASH_VERSION=}\StringTok{'4.3.11(1)-release'}
\OtherTok{COLORTERM=}\NormalTok{gnome-terminal}
\OtherTok{COLUMNS=}\NormalTok{80}
\OtherTok{DIRSTACK=()}
\OtherTok{DISPLAY=}\NormalTok{:0}
\OtherTok{EUID=}\NormalTok{1003}
\OtherTok{GROUPS=()}
\OtherTok{HISTCONTROL=}\NormalTok{ignoreboth}
\OtherTok{HISTFILE=}\NormalTok{/home/myuser/.bash_history}
\KeywordTok{...and} \NormalTok{so on...}
\end{Highlighting}
\end{Shaded}

And \texttt{CMD.EXE}:

\drcap{\texttt{SET} command in \texttt{CMD.EXE}}

\begin{Shaded}
\begin{Highlighting}[]
\KeywordTok{C}\NormalTok{:\textbackslash{}Users\textbackslash{}myuser}\KeywordTok{>}\NormalTok{SET}
\OtherTok{ALLUSERSPROFILE=}\NormalTok{C:\textbackslash{}}\KeywordTok{ProgramData}
\OtherTok{APPDATA=}\NormalTok{C:\textbackslash{}}\KeywordTok{Users}\NormalTok{\textbackslash{}myuser\textbackslash{}AppData\textbackslash{}Roaming}
\OtherTok{CommonProgramFiles=}\NormalTok{C:\textbackslash{}}\KeywordTok{Program} \NormalTok{Files\textbackslash{}Common Files}
\KeywordTok{CommonProgramFiles}\NormalTok{(x86)=}\KeywordTok{C}\NormalTok{:\textbackslash{}Program Files (x86)\textbackslash{}}\KeywordTok{Common} \NormalTok{Files}
\OtherTok{CommonProgramW6432=}\NormalTok{C:\textbackslash{}}\KeywordTok{Program} \NormalTok{Files\textbackslash{}Common Files}
\OtherTok{COMPUTERNAME=}\NormalTok{JLEHMER650}
\OtherTok{ComSpec=}\NormalTok{C:\textbackslash{}}\KeywordTok{Windows}\NormalTok{\textbackslash{}system32\textbackslash{}cmd.exe}
\OtherTok{FP_NO_HOST_CHECK=}\NormalTok{NO}
\OtherTok{HOMEDRIVE=}\NormalTok{C:}
\OtherTok{HOMEPATH=}\NormalTok{\textbackslash{}}\KeywordTok{Users}\NormalTok{\textbackslash{}myuser}
\OtherTok{LOCALAPPDATA=}\NormalTok{C:\textbackslash{}}\KeywordTok{Users}\NormalTok{\textbackslash{}myuser\textbackslash{}AppData\textbackslash{}Local}
\OtherTok{LOGONSERVER=}\DataTypeTok{\textbackslash{}\textbackslash{}}\NormalTok{JLEHMER650}
\OtherTok{NUMBER_OF_PROCESSORS=}\NormalTok{4}
\OtherTok{OS=}\NormalTok{Windows_NT}
\OtherTok{Path=}\NormalTok{C:\textbackslash{}}\KeywordTok{Windows}\NormalTok{\textbackslash{}system32}\KeywordTok{;C}\NormalTok{:\textbackslash{}Windows}\KeywordTok{;C}\NormalTok{:\textbackslash{}Windows\textbackslash{}System32\textbackslash{}Wbem}\KeywordTok{;C}\NormalTok{:\textbackslash{}Windows\textbackslash{}system32}
\NormalTok{\textbackslash{}}\KeywordTok{config}\NormalTok{\textbackslash{}systemprofile\textbackslash{}.dnx\textbackslash{}bin}\KeywordTok{;C}\NormalTok{:\textbackslash{}Program Files\textbackslash{}Microsoft DNX\textbackslash{}Dnvm}\DataTypeTok{\textbackslash{};}\NormalTok{C:\textbackslash{}Program F}
\KeywordTok{iles} \NormalTok{(x86)\textbackslash{}}\KeywordTok{nodejs}\DataTypeTok{\textbackslash{};}\NormalTok{C:\textbackslash{}Program Files\textbackslash{}Microsoft\textbackslash{}Web Platform Installer}\DataTypeTok{\textbackslash{};}\NormalTok{C:\textbackslash{}Program}
 \KeywordTok{Files}\NormalTok{\textbackslash{}Microsoft SQL Server\textbackslash{}130\textbackslash{}Tools\textbackslash{}Binn}\DataTypeTok{\textbackslash{};}\NormalTok{C:\textbackslash{}Program Files (x86)\textbackslash{}}\KeywordTok{Microsoft} \NormalTok{SQL}
 \KeywordTok{Server}\NormalTok{\textbackslash{}130\textbackslash{}DTS\textbackslash{}Binn}\DataTypeTok{\textbackslash{};}\NormalTok{C:\textbackslash{}Program Files\textbackslash{}Microsoft SQL Server\textbackslash{}120\textbackslash{}Tools\textbackslash{}Binn}\DataTypeTok{\textbackslash{};}\NormalTok{C:\textbackslash{}P}
\KeywordTok{rogram} \NormalTok{Files (x86)\textbackslash{}}\KeywordTok{Microsoft} \NormalTok{SDKs\textbackslash{}Azure\textbackslash{}CLI\textbackslash{}wbin}\KeywordTok{;C}\NormalTok{:\textbackslash{}Windows\textbackslash{}System32\textbackslash{}WindowsPowe}
\KeywordTok{rShell}\NormalTok{\textbackslash{}v1.0\textbackslash{}}
\NormalTok{PATHEXT=.COM}\KeywordTok{;.EXE;.BAT;.CMD;.VBS;.VBE;.JS;.JSE;.WSF;.WSH;.MSC}
\KeywordTok{...and} \NormalTok{so on...}
\end{Highlighting}
\end{Shaded}

Similarly, the \href{http://linux.die.net/man/1/echo}{\texttt{echo}}
command can be used to show you the contents of an environment variable
like \texttt{HOME} on \texttt{bash}: \drtxt{echo}
\drenv{HOME}{current user's home directory}
\index{environment variables!displaying!echocommand@\texttt{echo} command}

\drcap{\texttt{echo} the \texttt{HOME} environment variable in \texttt{bash}}

\begin{Shaded}
\begin{Highlighting}[]
\KeywordTok{~} \NormalTok{$ }\KeywordTok{echo} \OtherTok{$HOME}
\KeywordTok{/home/myuser}
\end{Highlighting}
\end{Shaded}

Versus the \texttt{HOMEPATH} variable under \texttt{CMD.EXE}:

\drcap{\texttt{echo} the \texttt{HOMEPATH} environment variable in \texttt{CMD.EXE}}

\begin{Shaded}
\begin{Highlighting}[]
\KeywordTok{C}\NormalTok{:}\DataTypeTok{\textbackslash{}>} \NormalTok{ECHO %HOMEPATH%}
\NormalTok{\textbackslash{}}\KeywordTok{Users}\NormalTok{\textbackslash{}myuser}
\end{Highlighting}
\end{Shaded}

This example shows some valuable differences between shells. Even though
both have the concept of environment variables and displaying their
contents using the "same" \texttt{echo} command, note that:

\begin{enumerate}
\def\labelenumi{\arabic{enumi}.}
\item
  The syntax for accessing an environment variable is
  \texttt{\$variable} in \texttt{bash} and \texttt{\%variable\%} in
  \texttt{CMD.EXE}. \index{environment variables!syntax}
\item
  \texttt{bash} is case-sensitive and so \texttt{echo\ \$HOME} works but
  \texttt{echo\ \$home} does not. \texttt{CMD.EXE} is
  \textbf{\emph{not}} case-sensitive, so either
  \texttt{echo\ \%homedrive\%} or \texttt{echo\ \%HOMEDRIVE\%} (or
  \texttt{EcHo\ \%hOmEdRiVe\%}) would work.
\end{enumerate}

One final note of caution. You can set up command aliases in
\texttt{bash} and other shells that allow you to define a
\texttt{CMD.EXE}-style \texttt{dir} command as a substitute for the
\texttt{ls} command in \texttt{bash}, or \texttt{copy} for \texttt{cp},
\texttt{del} for \texttt{rm}, and so on. I recommend you don't do this
for at least two reasons: \drfnd{cp}{copy}

\begin{enumerate}
\def\labelenumi{\arabic{enumi}.}
\item
  It is difficult to get these right in terms of being able to map all
  the various parameters from the \texttt{bash} command to the
  appropriate parameters for a \texttt{CMD.EXE}-style command. Most
  people don't go that far, which means you then end up with a "toy"
  substitute for the \texttt{CMD.EXE} command, and have to fall back to
  the native commands anyway.
\item
  It simply delays you actually learning about the "UNIX" environment.
  You end up relying on a crutch that then must be replicated on every
  system you touch. In my opinion it is better to just learn the native
  commands, because then you are instantly productive at any shell
  window.
\end{enumerate}

\section*{You're a Product of Your Environment
(Variables)}\label{youre-a-product-of-your-environment-variables}
\addcontentsline{toc}{section}{You're a Product of Your Environment
(Variables)}

It is much more common to set up environment variables to control
run-time execution in Linux than in Windows. In fact, it is quite common
to assign a given environment variable for the single execution of a
program, to the point that \texttt{bash} has built-in "one-line" support
for it: \index{environment variables!assigning}

\drcap{Assign \texttt{FOO} environment variable before executing script}

\begin{Shaded}
\begin{Highlighting}[]
\KeywordTok{~} \NormalTok{$ }\OtherTok{FOO=}\NormalTok{myval }\KeywordTok{/home/myuser/myscript}
\end{Highlighting}
\end{Shaded}

This sets the environment variable \texttt{FOO} to "myval" but only for
the duration and scope of running \texttt{myscript}.

By convention, environment variables are named all uppercase, whereas
all scripts and programs tend to be named all lowercase. Remember,
almost without exception "UNIX" is case-sensitive and Windows is not.

You can assign multiple variables for a single command or script
execution simply by separating them with spaces:

\drcap{Set multiple environment variables at once}

\begin{Shaded}
\begin{Highlighting}[]
\KeywordTok{~} \NormalTok{$ }\OtherTok{FOO=}\NormalTok{myval }\OtherTok{BAR=}\NormalTok{yourval }\OtherTok{BAZ=}\NormalTok{ourvals }\KeywordTok{/home/myuser/myscript}
\end{Highlighting}
\end{Shaded}

Note that passing in values in this way does not safeguard sensitive
information from other users on the system who can see the values at
least while the script is running using the \texttt{ps\ -x} command. In
addition, the entire command will be written to your
\texttt{.bash\_history} file, too. Theoretically that should be safe,
but if you are using this to pass in a password to a command, for
example, and your id gets compromised, your \texttt{.bash\_history} will
be just as exposed as if you had the password saved in a script file.
\drsys{ps}{list processes}

You can also set the value of environment variables to the output of
another command by surrounding it with paired ` ("back ticks", or "grave
accents"):

\drcap{Set environment variable to output from a command}

\begin{Shaded}
\begin{Highlighting}[]
\KeywordTok{~} \NormalTok{$ }\OtherTok{FILETYPE=}\KeywordTok{`file} \NormalTok{--brief --mime-type header.tex}\KeywordTok{`}
\KeywordTok{~} \NormalTok{$ }\KeywordTok{echo} \OtherTok{$FILETYPE}
\KeywordTok{text/plain}
\end{Highlighting}
\end{Shaded}

Sometimes you want to keep certain sensitive commands from being records
in your \texttt{.bash\_history} file, since it is a simple text file and
if you ever got hacked, the attacker could peruse it. For example, some
commands take userids and passwords as parameters. To keep a command
like that from being recorded in your command history, export the
following, preferably in the \texttt{.profile} or \texttt{.bashrc}
scripts in your home directory:
\drenv{HISTIGNORE}{commands to ignore in command history}

\drcap{Hiding commands from command history}

\begin{Shaded}
\begin{Highlighting}[]
\KeywordTok{export} \OtherTok{HISTIGNORE=}\StringTok{"*smbclient*"}
\end{Highlighting}
\end{Shaded}

\subsection*{Who Am I?}\label{who-am-i}
\addcontentsline{toc}{subsection}{Who Am I?}

When writing scripts that can be run by any user, it may be helpful to
know their user name at run-time. There are at least two different ways
to determine that. The first is via the \texttt{USER} environment
variable: \drenv{USER}{current user}

\drcap{\texttt{USER} environment variable}

\begin{Shaded}
\begin{Highlighting}[]
\KeywordTok{~} \NormalTok{$ }\KeywordTok{echo} \OtherTok{$USER}
\KeywordTok{myuser}
\end{Highlighting}
\end{Shaded}

The second is with a command with one of the best names, ever -
\href{http://linux.die.net/man/1/whoami}{\texttt{whoami}}:
\drcmd{whoami}{existential question}

\drcap{\texttt{whoami} command}

\begin{Shaded}
\begin{Highlighting}[]
\KeywordTok{~} \NormalTok{$ }\KeywordTok{whoami}
\KeywordTok{myuser}
\end{Highlighting}
\end{Shaded}

Some environments set the \texttt{USER} environment variable, some set a
\texttt{USERNAME} variable, and some like Mint set both. I think it is
better to use \texttt{whoami}, which tends to be on almost all systems.
\drenv{USERNAME}{current user}

\section*{Paths (a Part of Any Balanced
Shrubbery)}\label{paths-a-part-of-any-balanced-shrubbery}
\addcontentsline{toc}{section}{Paths (a Part of Any Balanced Shrubbery)}

The concept of a "path" for finding executables is almost identical
between "UNIX" and Windows, and Windows lifted it from UNIX (or CP/M,
which lifted it from UNIX). Look at the output of the \texttt{PATH}
environment variable under \texttt{bash}:
\drenv{PATH}{execution search path} \drios{CP/M}

\drcap{\texttt{PATH} environment variable in \texttt{bash}}

\begin{Shaded}
\begin{Highlighting}[]
\KeywordTok{~} \NormalTok{$ }\KeywordTok{echo} \OtherTok{$PATH}
\KeywordTok{/usr/local}\NormalTok{/bin:}\KeywordTok{/usr}\NormalTok{/bin:/bin:}\KeywordTok{/usr/local}\NormalTok{/games:}\KeywordTok{/usr/games}
\end{Highlighting}
\end{Shaded}

Echoing the \texttt{PATH} environment variable under \texttt{CMD.EXE}
works, too: \drenv{PATH}{execution search path}

\drcap{\texttt{PATH} environment variable in \texttt{CMD.EXE}}

\begin{verbatim}
C:\Users\myuser>ECHO %PATH%
C:\Windows\system32;C:\Windows;C:\Windows\System32\Wbem;C:\Windows\system32\conf
ig\systemprofile\.dnx\bin;C:\Program Files\Microsoft DNX\Dnvm\;C:\Program Files
(x86)\nodejs\;C:\Program Files\Microsoft\Web Platform Installer\;C:\Program File
s\Microsoft SQL Server\130\Tools\Binn\;C:\Program Files (x86)\Microsoft SQL Serv
er\130\DTS\Binn\;C:\Program Files\Microsoft SQL Server\120\Tools\Binn\;C:\Progra
m Files (x86)\Microsoft SDKs\Azure\CLI\wbin;C:\Windows\System32\WindowsPowerShel
l\v1.0\
\end{verbatim}

Note the differences and similarities. Both the paths are evaluated left
to right. Both use separators between path components, a \texttt{;} for
DOS and Windows, a \texttt{:} for Linux. Both delimit their directory
names with slashes, with \texttt{\textbackslash{}} for DOS and Windows
and \texttt{/} for Linux. But Linux has no concept of a "drive letter"
like \texttt{C:} in Windows, and instead everything is mounted in a
single namespace hierarchy starting at the root \texttt{/}. We'll be
talking more about directories, paths and file systems in the next
chapter. \index{*@\texttt{/} (path separator)}

Just to muddy the waters further, notice how Cygwin under Windows shows
the \texttt{PATH} environment variable with \texttt{bash} syntax but a
combination of both Cygwin and Windows directories, and Windows drive
letters like \texttt{C:} mapped to \texttt{/cygdrive/c}:
\drenv{PATH}{execution search path} \drunx{Cygwin}

\drcap{\texttt{PATH} environment variable in Cygwin}

\begin{Shaded}
\begin{Highlighting}[]
\NormalTok{$ }\KeywordTok{echo} \OtherTok{$PATH}
\KeywordTok{/usr/local}\NormalTok{/bin:}\KeywordTok{/usr}\NormalTok{/bin:}\KeywordTok{/cygdrive/c/Windows}\NormalTok{/system32:}\KeywordTok{/cygdrive/c}\NormalTok{/Windows:}\KeywordTok{/cygdri}
\KeywordTok{ve/c/Windows/System32}\NormalTok{/Wbem:}\KeywordTok{/cygdrive/c/Windows/system32/config/systemprofile/.dn}
\KeywordTok{x}\NormalTok{/bin:}\KeywordTok{/cygdrive/c/Program} \NormalTok{Files/Microsoft DNX/Dnvm:/cygdrive/c/Program Files (x8}
\KeywordTok{6}\NormalTok{)/nodejs:}\KeywordTok{/cygdrive/c/Program} \NormalTok{Files/Microsoft/Web Platform Installer:/cygdrive/c}
\KeywordTok{/Program} \NormalTok{Files/Microsoft SQL Server/130/Tools/Binn:/cygdrive/c/Program Files (x8}
\KeywordTok{6}\NormalTok{)}\KeywordTok{/Microsoft} \NormalTok{SQL Server/130/DTS/Binn:/cygdrive/c/Program Files/Microsoft SQL Ser}
\KeywordTok{ver/120/Tools}\NormalTok{/Binn:}\KeywordTok{/cygdrive/c/Program} \NormalTok{Files (x86)}\KeywordTok{/Microsoft} \NormalTok{SDKs/Azure/CLI/wbin}
\NormalTok{:}\KeywordTok{/cygdrive/c/Windows/System32/WindowsPowerShell/v1.0}
\end{Highlighting}
\end{Shaded}

\section*{Open Your Shell and
Interact}\label{open-your-shell-and-interact}
\addcontentsline{toc}{section}{Open Your Shell and Interact}

The actual "command prompt" is when you run a shell in an "interactive
session" in a terminal window. This might be from logging into the
console of a Linux VM, or starting a terminal window in a X window
manager like GNOME or KDE, or \texttt{ssh}'ing into an interactive
session of a remote machine, or even running a Cygwin command prompt
under Windows. \drnet{ssh} \drunx{Cygwin}

Command prompts allow you to work in a so-called "REPL" environment
(Read, Evaluate, Print, Loop). You can run a series of commands once, or
keep refining a command or commands until you get them working the way
you want, then transfer their sequence to a script file to capture it.

Real wizards at using the shell can often show off their magic with an
incredible one-liner typed from memory with lots of obscure commands
piped together and invoked with cryptic options.

I am not a real shell wizard. See
\protect\hyperlink{how-do-you-know-what-you-dont-know-man}{chapter 9}
for how you can fake it like I do.

\section*{Getting Lazy}\label{getting-lazy}
\addcontentsline{toc}{section}{Getting Lazy}

Most modern interactive shells like \texttt{bash} and \texttt{CMD.EXE}
allow for tab expansion and command history, at least for the current
session of the shell. \drshl{CMD.EXE}

Tab expansion is "auto-complete" for the command prompt. Let's say you
have some files in a directory:

\drcap{List some files}

\begin{Shaded}
\begin{Highlighting}[]
\KeywordTok{~/Documents} \NormalTok{$ ls}
\KeywordTok{Disabled} \NormalTok{User Accounts.csv  elsewhere  LOLcatz.jpg  MyResume.md}
\end{Highlighting}
\end{Shaded}

Without tab expansion, typing out something like this is painful:

\drcap{Lots of typing and escape characters}

\begin{Shaded}
\begin{Highlighting}[]
\KeywordTok{~/Documents} \NormalTok{$ mv Disabled}\DataTypeTok{\textbackslash{} }\NormalTok{User}\DataTypeTok{\textbackslash{} }\NormalTok{Accounts.csv elsewhere/.}
\end{Highlighting}
\end{Shaded}

But with tab expansion, we can simply type \texttt{mv\ D\^{}t}, where
\texttt{\^{}t} represents hitting the \texttt{Tab} key, and since there
is only one file that starts with a "D", tab expansion will fill in the
rest of the file name for us:

\drcap{Tab expansion magic}

\begin{Shaded}
\begin{Highlighting}[]
\KeywordTok{~/Documents} \NormalTok{$ mv Disabled}\DataTypeTok{\textbackslash{} }\NormalTok{User}\DataTypeTok{\textbackslash{} }\NormalTok{Accounts.csv}
\end{Highlighting}
\end{Shaded}

Then we can go about our business of finishing our command.

One place tab completion in \texttt{bash} is different than
\texttt{CMD.EXE} is that in \texttt{bash} if you hit \texttt{Tab} and
there are multiple candidates, it will expand as far as it can and then
show you a list of files that match up to that point and allow you to
type in more characters and hit \texttt{Tab} again to complete it.
Whereas in \texttt{CMD.EXE} it will "cycle" between the multiple
candidates, showing you each one as the completion option in turn. Both
are useful, but each is subtly different and can give you fits when
moving between one environment and another. \drshl{CMD.EXE}

\textbf{Pro Tip:} Remember, UNIX was built by people on slow, klunky
teletypes and terminals, and they hated to type! Tab expansion is your
friend and you should use it as often as possible. It gives at least
three benefits:

\begin{enumerate}
\def\labelenumi{\arabic{enumi}.}
\item
  Saves you typing.
\item
  Helps eliminate misspellings in a long file or command name.
\item
  Acts as an error checker, because if the tab doesn't expand, chances
  are you are specifying something else (the beginning part of the file
  name) wrong.
\end{enumerate}

The other thing to remember about the interactive shell is command
history. Again, both \texttt{CMD.EXE} and \texttt{bash} give you command
history, but \texttt{CMD.EXE} only remembers it for the session, while
\texttt{bash} stores it in one of your hidden "profile" or "dot" files
in your home directory called \texttt{.bash\_history}, which you can
display with \texttt{ls\ -a}: \drshl{CMD.EXE}
\index{files and directories!special!bashhistory@\texttt{.bash\_history}}

\drcap{\texttt{ls} command showing hidden files}

\begin{Shaded}
\begin{Highlighting}[]
\KeywordTok{~} \NormalTok{$ }\KeywordTok{ls} \NormalTok{-a}
\KeywordTok{.}              \KeywordTok{.config}    \NormalTok{.gconf           .mozilla  Templates}
\KeywordTok{..}             \NormalTok{.dbus      .gnome2          Music     Videos}
\KeywordTok{.bash_history}  \NormalTok{Desktop    .gnome2_private  Pictures  .xsession-errors}
\KeywordTok{.bash_logout}   \NormalTok{.dmrc      .ICEauthority    .profile}
\KeywordTok{.cache}         \NormalTok{Documents  .linuxmint       Public}
\KeywordTok{.cinnamon}      \NormalTok{Downloads  .local           .ssh}
\end{Highlighting}
\end{Shaded}

Inside, \texttt{.bash\_history} is just a text file, with the most
recent commands at the bottom.

The \texttt{bash} shell supports a rich interactive environment for
searching for, editing and saving command history. However, the biggest
thing you need to remember to fake it is simply that the up and down
arrows work in the command prompt and bring back your recent commands so
you can update them and re-execute them.

\textbf{Note:} If you start multiple sessions under the same account,
the saved history will be of the last login to successfully write back
out \texttt{.bash\_history}.

\hypertarget{file-under-directories}{\chapter{File Under
"Directories"}\label{file-under-directories}}

\textbf{\emph{\texttt{ls}, \texttt{mv}, \texttt{cp}, \texttt{rm}
(\texttt{-rf\ *}), \texttt{cat},
\texttt{chmod}/\texttt{chgrp}/\texttt{chown} and everyone's favorite,
\texttt{touch}.}}

\begin{quote}
\emph{"I'm in the phone book! I'm somebody now!"} - Navin Johnson
(\emph{The Jerk})
\end{quote}

Typically in Linux we are scripting and otherwise moving around files.
The file system under the covers may be one of any number of supported
formats, including:

\begin{itemize}
\item
  \href{https://en.wikipedia.org/wiki/Btrfs}{\textbf{btrfs}}
\item
  \href{https://en.wikipedia.org/wiki/Ext2}{\textbf{ext2}}
\item
  \href{https://en.wikipedia.org/wiki/Ext3}{\textbf{ext3}}
\item
  \href{https://en.wikipedia.org/wiki/Ext4}{\textbf{ext4}},
\item
  \href{https://en.wikipedia.org/wiki/ReiserFS}{\textbf{ReiserFS}}
\item
  \href{https://en.wikipedia.org/wiki/ZFS}{\textbf{ZFS}}
\item
  \textbf{...and so many more!} - NTFS, FAT, CDFS, etc.
\end{itemize}

Each has its strengths and weaknesses. While Linux tends to treat the
ext* file systems as preferred, it can write to a lot of file systems
and can read even more.

As mentioned before, the biggest differences between Linux and Windows
is that the Linux environments do not have a concept of "drive letters."
Instead everything is "mounted" under a single hierarchy that starts at
the "root directory" or \texttt{/}: \drios{Windows}
\index{*@\texttt{/} (root directory)}\index{files and directories!root (\texttt{/})}

\drcap{Listing of the root directory}

\begin{Shaded}
\begin{Highlighting}[]
\KeywordTok{~} \NormalTok{$ }\KeywordTok{ls} \NormalTok{/}
\KeywordTok{bin}    \NormalTok{dev   home        lib64       mnt    Other  run   sys  var}
\KeywordTok{boot}   \NormalTok{Docs  initrd.img  lost+found  Music  proc   sbin  tmp  vmlinuz}
\KeywordTok{cdrom}  \NormalTok{etc   lib         media       opt    root   srv   usr}
\end{Highlighting}
\end{Shaded}

The root file system may be backed by a disk device, memory or even the
network. It will have one or more directories under it. Multiple
physical drives and network locations can be "mounted" virtually
anywhere, under any directory or subdirectory in the hierarchy.

\textbf{Note:} Dynamically mounted devices like USB drives and DVDs are
often mounted automatically under either a \texttt{/mnt} or
\texttt{/media} directory.

\section*{Looking at Files}\label{looking-at-files}
\addcontentsline{toc}{section}{Looking at Files}

As we've already seen, the command to \emph{list} the contents of a
directory is \href{http://linux.die.net/man/1/ls}{\texttt{ls}}:
\drfnd{ls}{list directory contents}

\drcap{Listing directory contents}

\begin{Shaded}
\begin{Highlighting}[]
\KeywordTok{~} \NormalTok{$ }\KeywordTok{ls}
\KeywordTok{Desktop}  \NormalTok{Documents  Downloads  Music  Pictures  Public  Templates  Videos}
\end{Highlighting}
\end{Shaded}

Remember, "UNIX" environments think of files that start with a
\texttt{.} as "hidden." If you want to see all these
\href{https://en.wikipedia.org/wiki/Hidden_file_and_hidden_directory\#Unix_and_Unix-like_environments}{"dotfiles"},
you can use \texttt{ls\ -a}, in this case on an average "home"
directory: \index{files and directories!hidden (dotfiles)}

\drcap{Listing a home directory showing hidden "dotfiles"}

\begin{Shaded}
\begin{Highlighting}[]
\KeywordTok{~} \NormalTok{$ }\KeywordTok{ls} \NormalTok{-a}
\KeywordTok{.}              \KeywordTok{.config}    \NormalTok{.gconf           .mozilla  Templates}
\KeywordTok{..}             \NormalTok{.dbus      .gnome2          Music     Videos}
\KeywordTok{.bash_history}  \NormalTok{Desktop    .gnome2_private  Pictures  .xsession-errors}
\KeywordTok{.bash_logout}   \NormalTok{.dmrc      .ICEauthority    .profile}
\KeywordTok{.cache}         \NormalTok{Documents  .linuxmint       Public}
\KeywordTok{.cinnamon}      \NormalTok{Downloads  .local           .ssh}
\end{Highlighting}
\end{Shaded}

Wow! That's a lot of dotfiles!

If you want to see some details for each file, use \texttt{ls\ -l}:

\drcap{Detailed listing of home directory}

\begin{Shaded}
\begin{Highlighting}[]
\KeywordTok{~} \NormalTok{$ }\KeywordTok{ls} \NormalTok{-l}
\KeywordTok{total} \NormalTok{32}
\KeywordTok{drwxr-xr-x} \NormalTok{2 myuser mygroup 4096 Dec 13 18:18 Desktop}
\KeywordTok{drwxr-xr-x} \NormalTok{3 myuser mygroup 4096 Dec 13 18:22 Documents}
\KeywordTok{drwxr-xr-x} \NormalTok{2 myuser mygroup 4096 Dec 13 18:18 Downloads}
\KeywordTok{drwxr-xr-x} \NormalTok{2 myuser mygroup 4096 Dec 13 18:18 Music}
\KeywordTok{drwxr-xr-x} \NormalTok{2 myuser mygroup 4096 Dec 13 18:18 Pictures}
\KeywordTok{drwxr-xr-x} \NormalTok{2 myuser mygroup 4096 Dec 13 18:18 Public}
\KeywordTok{drwxr-xr-x} \NormalTok{2 myuser mygroup 4096 Dec 13 18:18 Templates}
\KeywordTok{drwxr-xr-x} \NormalTok{2 myuser mygroup 4096 Dec 13 18:18 Videos}
\end{Highlighting}
\end{Shaded}

And of course parameters can be combined, as with the two above:

\drcap{Detailed listing of home directory with "dotfiles"}

\begin{Shaded}
\begin{Highlighting}[]
\KeywordTok{~} \NormalTok{$ }\KeywordTok{ls} \NormalTok{-al}
\KeywordTok{total} \NormalTok{112}
\KeywordTok{drwxr-xr-x} \NormalTok{21 myuser mygroup 4096 Dec 13 18:19 .}
\KeywordTok{drwxr-xr-x}  \NormalTok{6 root   root    4096 Dec 13 14:24 ..}
\KeywordTok{-rw-------}  \NormalTok{1 myuser mygroup  287 Dec 13 18:19 .bash_history}
\KeywordTok{-rw-r--r--}  \NormalTok{1 myuser mygroup  220 Dec 13 14:24 .bash_logout}
\KeywordTok{drwx------}  \NormalTok{5 myuser mygroup 4096 Dec 13 18:18 .cache}
\KeywordTok{drwxr-xr-x}  \NormalTok{3 myuser mygroup 4096 Dec 13 18:18 .cinnamon}
\KeywordTok{drwxr-xr-x} \NormalTok{12 myuser mygroup 4096 Dec 13 18:18 .config}
\KeywordTok{drwx------}  \NormalTok{3 myuser mygroup 4096 Dec 13 18:18 .dbus}
\KeywordTok{drwxr-xr-x}  \NormalTok{2 myuser mygroup 4096 Dec 13 18:18 Desktop}
\KeywordTok{-rw-------}  \NormalTok{1 myuser mygroup   29 Dec 13 18:18 .dmrc}
\KeywordTok{drwxr-xr-x}  \NormalTok{3 myuser mygroup 4096 Dec 13 18:22 Documents}
\KeywordTok{drwxr-xr-x}  \NormalTok{2 myuser mygroup 4096 Dec 13 18:18 Downloads}
\KeywordTok{drwx------}  \NormalTok{3 myuser mygroup 4096 Dec 13 18:18 .gconf}
\KeywordTok{drwx------}  \NormalTok{3 myuser mygroup 4096 Dec 13 18:18 .gnome2}
\KeywordTok{drwx------}  \NormalTok{2 myuser mygroup 4096 Dec 13 18:18 .gnome2_private}
\KeywordTok{-rw-------}  \NormalTok{1 myuser mygroup  668 Dec 13 18:18 .ICEauthority}
\KeywordTok{drwxr-xr-x}  \NormalTok{3 myuser mygroup 4096 Dec 13 18:18 .linuxmint}
\KeywordTok{drwxr-xr-x}  \NormalTok{3 myuser mygroup 4096 Dec 13 18:18 .local}
\KeywordTok{drwxr-xr-x}  \NormalTok{4 myuser mygroup 4096 Dec 13 18:18 .mozilla}
\KeywordTok{drwxr-xr-x}  \NormalTok{2 myuser mygroup 4096 Dec 13 18:18 Music}
\KeywordTok{drwxr-xr-x}  \NormalTok{2 myuser mygroup 4096 Dec 13 18:18 Pictures}
\KeywordTok{...and} \NormalTok{so on...}
\end{Highlighting}
\end{Shaded}

\section*{A Brief Detour Around
Parameters}\label{a-brief-detour-around-parameters}
\addcontentsline{toc}{section}{A Brief Detour Around Parameters}

In \texttt{bash} and many Linux commands in general, there are old,
"short" (terse) parameter names, like \texttt{ls\ -a}, and newer,
longer, descriptive parameter names like \texttt{ls\ -\/-all} that mean
the same thing. It is typically good to use the shorter version during
interactive sessions and testing, but I prefer long parameter names in
scripts, because when I come back and look at it in two years, I may not
remember what \texttt{rm\ -rf\ *} means (in the "UNIX" world it means
you're toast if you run it by mistake), thus
\texttt{rm\ -\/-recursive\ -\/-force\ *} seems a bit more "intuitive."
\drfnd{rm}{remove file}

\begin{quote}
\textbf{\emph{The behind you save in the future by describing things
well today may well be your own.}} - me
\end{quote}

The older style parameters are typically preceded by a single hyphen or
"switch" character:

\drcap{Short parameter}

\begin{Shaded}
\begin{Highlighting}[]
\KeywordTok{~} \NormalTok{$ }\KeywordTok{ls} \NormalTok{-r}
\end{Highlighting}
\end{Shaded}

Some commands support parameters with no "switch" character at all, as
with \texttt{xvf} (e\textbf{\emph{X}}tract, \textbf{\emph{V}}erbose,
input \textbf{\emph{F}}ile name) in the following \texttt{tar} example:
\drcmd{tar}{archive files}

\drcap{Alternate short parameter syntax}

\begin{Shaded}
\begin{Highlighting}[]
\KeywordTok{~} \NormalTok{$ }\KeywordTok{tar} \NormalTok{xvf backup.tar}
\end{Highlighting}
\end{Shaded}

The newer "GNU-style" parameters are preceded by two hyphens and usually
are quite "verbose":

\drcap{Long parameters}

\begin{Shaded}
\begin{Highlighting}[]
\KeywordTok{~} \NormalTok{$ }\KeywordTok{ls} \NormalTok{--recursive --almost-all --ignore-backups}
\end{Highlighting}
\end{Shaded}

Again, it is \textbf{\emph{highly recommended}} that you take the time
to use the GNU-style parameters in scripts as self-documenting code.

\section*{More Poking at Files}\label{more-poking-at-files}
\addcontentsline{toc}{section}{More Poking at Files}

If we suspect the file is a text file, we can echo it to the console
with the \href{http://linux.die.net/man/1/cat}{\texttt{cat}}
(\emph{concatenate}) command: \drtxt{cat}
\index{files and directories!display (\texttt{cat} command)}

\drcap{\texttt{cat} command}

\begin{Shaded}
\begin{Highlighting}[]
\KeywordTok{~} \NormalTok{$ }\KeywordTok{cat} \NormalTok{installrdp}
\CommentTok{#!/bin/bash}
\KeywordTok{sudo} \NormalTok{apt-get -y install git}
\KeywordTok{cd} \NormalTok{~}
\KeywordTok{git} \NormalTok{clone git://github.com/FreeRDP/FreeRDP.git}
\KeywordTok{cd} \NormalTok{FreeRDP}
\KeywordTok{sudo} \NormalTok{apt-get -y install build-essential git-core cmake libssl-dev libx11-dev lib}
\KeywordTok{xext-dev} \NormalTok{libxinerama-dev \textbackslash{}}
  \NormalTok{libxcursor-dev libxdamage-dev libxv-dev libxkbfile-dev libasound2-dev libcups2}
  \KeywordTok{-dev} \NormalTok{libxml2 libxml2-dev \textbackslash{}}
  \NormalTok{libxrandr-dev libgstreamer0.10-dev libgstreamer-plugins-base0.10-dev libxi-dev}
   \KeywordTok{libgstreamer-plugins-base1.0-dev}
\KeywordTok{sudo} \NormalTok{apt-get -y install libavutil-dev libavcodec-dev}
\KeywordTok{sudo} \NormalTok{apt-get -y install libcunit1-dev libdirectfb-dev xmlto doxygen libxtst-dev}
\KeywordTok{cmake} \NormalTok{-DCMAKE_BUILD_TYPE=Debug -DWITH_SSE2=ON .}
\KeywordTok{make}
\KeywordTok{sudo} \NormalTok{make install}
\KeywordTok{sudo} \NormalTok{echo }\StringTok{"/usr/local/lib/freerdp"} \KeywordTok{>} \NormalTok{/etc/ld.so.conf.d/freerdp.conf}
\KeywordTok{sudo} \NormalTok{echo }\StringTok{"/usr/local/lib64/freerdp"} \KeywordTok{>>} \NormalTok{/etc/ld.so.conf.d/freerdp.conf}
\KeywordTok{sudo} \NormalTok{echo }\StringTok{"/usr/local/lib"} \KeywordTok{>>} \NormalTok{/etc/ld.so.conf.d/freerdp.conf}
\KeywordTok{sudo} \NormalTok{ldconfig}
\KeywordTok{which} \NormalTok{xfreerdp}
\KeywordTok{xfreerdp} \NormalTok{--version}
\end{Highlighting}
\end{Shaded}

In this example when we \texttt{cat\ installrdp} we can determine it is
a \texttt{bash} shell script (because the "shebang" is pointing to
\texttt{bash}\drshl{bash}) that looks to install and configure
\href{https://github.com/FreeRDP/FreeRDP}{FreeRDP} on a Debian-style
system: \drnet{xfreerdp} \index{*@\texttt{\#"!} (shebang)}

\begin{enumerate}
\def\labelenumi{\arabic{enumi}.}
\item
  \textbf{\texttt{apt-get}} - Debian-style package manager.
  \drpkg{apt-get}
\item
  \textbf{\texttt{git\ clone}} - cloning package from
  \href{http://github.com}{GitHub}.
  \drcmd{git}{distributed version control}
\item
  \textbf{\texttt{cmake}} and \textbf{\texttt{make}} - configuring and
  building software from source. \drcmd{cmake}{configure makefiles}
  \drcmd{make}{build by recipes}
\end{enumerate}

A better way to display a longer file is to use the
\href{http://linux.die.net/man/1/less}{\texttt{less}} command (which is
a derivative of the original
\href{http://linux.die.net/man/1/more}{\texttt{more}}, hence the name).
\texttt{less} is a paginator, where the \texttt{Space},
\texttt{Page\ Down} or down arrow keys scroll down and the
\texttt{Page\ Up} or up arrow keys scrolls up. \texttt{Q} quits.
\drtxt{less} \drtxt{more}
\index{files and directories!paginate!less@\texttt{less} command}
\index{files and directories!paginate!more@\texttt{more} command}
\index{pagination!\texttt{less} command}
\index{pagination!\texttt{more} command}

\textbf{Note:} The \texttt{vi} search (\texttt{/}, \texttt{?},
\texttt{n} and \texttt{p}) and navigation (\texttt{G}, \texttt{0}) keys
work within \texttt{less}, too. In general \texttt{less} is a great
lightweight way to motor around in a text file without editing it.
\dreds{vi}

We can also look at just the end or \emph{tail} of a file (often the
most interesting when looking at log files and troubleshooting a current
problem) with the \href{http://linux.die.net/man/1/tail}{\texttt{tail}}
command. The next example shows the last 10 lines of the kernel
\texttt{dmesg} log: \drtxt{tail}
\index{files and directories!display (\texttt{tail} command)}
\index{files and directories!special!varlogdmesg@\texttt{/var/log/dmesg}}
\index{varlogdmesg@\texttt{/var/log/dmesg} (kernel log)}

\drcap{\texttt{tail} command}

\begin{Shaded}
\begin{Highlighting}[]
\KeywordTok{/var/log} \NormalTok{$ tail dmesg}
\KeywordTok{[}    \NormalTok{3.913318] Bluetooth: BNEP socket layer initialized}
\NormalTok{[    3.914888] Bluetooth: RFCOMM TTY layer initialized}
\NormalTok{[    3.914895] Bluetooth: RFCOMM socket layer initialized}
\NormalTok{[    3.914900] Bluetooth: RFCOMM ver 1.11}
\NormalTok{[    3.935772] init: failsafe main process (732) killed by TERM signal}
\NormalTok{[    4.046700] init: cups main process (896) killed by HUP signal}
\NormalTok{[    4.046710] init: cups main process ended, respawning}
\NormalTok{[    4.186239] init: samba-ad-dc main process (919) terminated with status 1}
\NormalTok{[    4.328999] r8169 0000:02:00.0 eth0: link down}
\NormalTok{[    4.329037] IPv6: ADDRCONF(NETDEV_UP): eth0: link is not ready}
\end{Highlighting}
\end{Shaded}

To show a specific number of lines use the \texttt{-n} parameter with
\texttt{tail}:

\drcap{Display last 15 lines of a file with \texttt{tail -n}}

\begin{Shaded}
\begin{Highlighting}[]
\KeywordTok{/var/log} \NormalTok{$ tail -n 15 dmesg}
\KeywordTok{[}    \NormalTok{3.899169] Bluetooth: HCI socket layer initialized}
\NormalTok{[    3.899170] Bluetooth: L2CAP socket layer initialized}
\NormalTok{[    3.899179] Bluetooth: SCO socket layer initialized}
\NormalTok{[    3.913306] Bluetooth: BNEP (Ethernet Emulation) ver 1.3}
\NormalTok{[    3.913309] Bluetooth: BNEP filters: protocol multicast}
\NormalTok{[    3.913318] Bluetooth: BNEP socket layer initialized}
\NormalTok{[    3.914888] Bluetooth: RFCOMM TTY layer initialized}
\NormalTok{[    3.914895] Bluetooth: RFCOMM socket layer initialized}
\NormalTok{[    3.914900] Bluetooth: RFCOMM ver 1.11}
\NormalTok{[    3.935772] init: failsafe main process (732) killed by TERM signal}
\NormalTok{[    4.046700] init: cups main process (896) killed by HUP signal}
\NormalTok{[    4.046710] init: cups main process ended, respawning}
\NormalTok{[    4.186239] init: samba-ad-dc main process (919) terminated with status 1}
\NormalTok{[    4.328999] r8169 0000:02:00.0 eth0: link down}
\NormalTok{[    4.329037] IPv6: ADDRCONF(NETDEV_UP): eth0: link is not ready}
\end{Highlighting}
\end{Shaded}

You can also use \texttt{tail} to with the \texttt{-f} parameter to
\emph{follow} an open file and continuously display any new output at
the end, which is useful for monitoring log files in real time:

\drcap{\texttt{tail -f} command}

\begin{Shaded}
\begin{Highlighting}[]
\KeywordTok{/var/log} \NormalTok{$ tail -f syslog}
\KeywordTok{Dec} \NormalTok{13 19:23:40 MtLindsey dhclient: DHCPACK of 192.168.0.8 from 192.168.0.1}
\KeywordTok{Dec} \NormalTok{13 19:23:40 MtLindsey dhclient: bound to 192.168.0.8 -- renewal in 1423 seco}
\KeywordTok{nds.}
\KeywordTok{Dec} \NormalTok{13 19:23:40 MtLindsey NetworkManager[960]: }\KeywordTok{<}\NormalTok{info}\KeywordTok{>} \NormalTok{(eth0)}\KeywordTok{:} \NormalTok{DHCPv4 state chang}
\KeywordTok{ed} \NormalTok{renew -}\KeywordTok{>} \NormalTok{renew}
\KeywordTok{Dec} \NormalTok{13 19:23:40 MtLindsey NetworkManager[960]: }\KeywordTok{<}\NormalTok{info}\KeywordTok{>}   \NormalTok{address 192.168.0.8}
\KeywordTok{Dec} \NormalTok{13 19:23:40 MtLindsey NetworkManager[960]: }\KeywordTok{<}\NormalTok{info}\KeywordTok{>}   \NormalTok{prefix 24 (255.255.255.0}
\NormalTok{)}
\KeywordTok{Dec} \NormalTok{13 19:23:40 MtLindsey NetworkManager[960]: }\KeywordTok{<}\NormalTok{info}\KeywordTok{>}   \NormalTok{gateway 192.168.0.1}
\KeywordTok{Dec} \NormalTok{13 19:23:40 MtLindsey NetworkManager[960]: }\KeywordTok{<}\NormalTok{info}\KeywordTok{>}   \NormalTok{nameserver }\StringTok{'97.64.168.12}
\StringTok{'}
\KeywordTok{Dec} \NormalTok{13 19:23:40 MtLindsey NetworkManager[960]: }\KeywordTok{<}\NormalTok{info}\KeywordTok{>}   \NormalTok{nameserver }\StringTok{'192.119.194.}
\StringTok{131'}
\KeywordTok{Dec} \NormalTok{13 19:23:40 MtLindsey dbus[689]: [system] Activating service name=}\StringTok{'org.freed}
\StringTok{esktop.nm_dispatcher'} \NormalTok{(using servicehelper)}
\KeywordTok{Dec} \NormalTok{13 19:23:40 MtLindsey dbus[689]: [system] Successfully activated service }\StringTok{'or}
\StringTok{g.freedesktop.nm_dispatcher'}
\end{Highlighting}
\end{Shaded}

Use \texttt{Ctrl-C} to cancel following the file.

If we know nothing about a \emph{file}, we can use the
\href{http://linux.die.net/man/1/file}{\texttt{file}} command to help us
guess: \drfnd{file}{detect file type}

\drcap{\texttt{file} command}

\begin{Shaded}
\begin{Highlighting}[]
\KeywordTok{~} \NormalTok{$ }\KeywordTok{file} \NormalTok{installrdp }
\KeywordTok{installrdp}\NormalTok{: Bourne-Again shell script, ASCII text executable}
\end{Highlighting}
\end{Shaded}

That's straightforward enough! The \texttt{file} command isn't always
100\% accurate, but it is pretty good and uses an interesting set of
heuristics and a text file "database" of
\href{http://linux.die.net/man/5/magic}{"magic" number definitions} to
define how it figures out what type of file it is examining.

\textbf{Remember:} File extensions have no real meaning per se in Linux
(although some are used especially for media and document formats), so a
file name with no extension like \texttt{installrdp} is perfectly valid.
Hence the utility of the \texttt{file} command.

\section*{Sorting Things Out}\label{sorting-things-out}
\addcontentsline{toc}{section}{Sorting Things Out}

Let's say we have three files, and want to display the contents of one
of them. We know we can do that with \texttt{cat}: \drtxt{cat}

\drcap{Show contents of one file}

\begin{Shaded}
\begin{Highlighting}[]
\KeywordTok{~} \NormalTok{$ }\KeywordTok{cd} \NormalTok{Invoices/}
\KeywordTok{~/Invoices} \NormalTok{$ ls}
\KeywordTok{ElevatorTrucks}  \NormalTok{FarmCombines  FarmTractors}
\KeywordTok{~/Invoices} \NormalTok{$ cat ElevatorTrucks }
\KeywordTok{Truck}   \NormalTok{brakes  200}
\KeywordTok{Truck}   \NormalTok{tires   400}
\KeywordTok{Truck}   \NormalTok{tires   400}
\KeywordTok{Truck}   \NormalTok{tires   400}
\KeywordTok{Truck}   \NormalTok{winch   100}
\end{Highlighting}
\end{Shaded}

But what if we wanted to process all the lines in all the files in
alphabetical order? Just directing the files into a program won't do it,
because the file names will be sorted by the shell and the lines will be
processed in file name order, not the ultimate sorted order of all the
file contents.

\drcap{Show contents of all three files}

\begin{Shaded}
\begin{Highlighting}[]
\KeywordTok{~/Invoices} \NormalTok{$ cat *}
\KeywordTok{Truck}   \NormalTok{brakes  200}
\KeywordTok{Truck}   \NormalTok{tires   400}
\KeywordTok{Truck}   \NormalTok{tires   400}
\KeywordTok{Truck}   \NormalTok{tires   400}
\KeywordTok{Truck}   \NormalTok{winch   100}
\KeywordTok{Combine} \NormalTok{motor   1500}
\KeywordTok{Combine} \NormalTok{brakes  400}
\KeywordTok{Combine} \NormalTok{tires   2500}
\KeywordTok{Tractor} \NormalTok{motor   1000}
\KeywordTok{Tractor} \NormalTok{brakes  300}
\KeywordTok{Tractor} \NormalTok{tires   2000}
\end{Highlighting}
\end{Shaded}

The \href{http://linux.die.net/man/1/sort}{\texttt{sort}} command to the
rescue! We will see that the \texttt{sort} command can be used to not
just \emph{sort} files, but also to merge them and remove duplicates.
\drtxt{sort} \index{sorting!\texttt{sort} command}

\drcap{\texttt{sort} command}

\begin{Shaded}
\begin{Highlighting}[]
\KeywordTok{~/Invoices} \NormalTok{$ sort *}
\KeywordTok{Combine} \NormalTok{brakes  400}
\KeywordTok{Combine} \NormalTok{motor   1500}
\KeywordTok{Combine} \NormalTok{tires   2500}
\KeywordTok{Tractor} \NormalTok{brakes  300}
\KeywordTok{Tractor} \NormalTok{motor   1000}
\KeywordTok{Tractor} \NormalTok{tires   2000}
\KeywordTok{Truck}   \NormalTok{brakes  200}
\KeywordTok{Truck}   \NormalTok{tires   400}
\KeywordTok{Truck}   \NormalTok{tires   400}
\KeywordTok{Truck}   \NormalTok{tires   400}
\KeywordTok{Truck}   \NormalTok{winch   100}
\end{Highlighting}
\end{Shaded}

What if we want to sort by the parts column? Well, it is the second
"key" field delimited by whitespace, so:

\drcap{Sort by the second "key" column}

\begin{Shaded}
\begin{Highlighting}[]
\KeywordTok{~/Invoices} \NormalTok{$ sort -k 2 *}
\KeywordTok{Truck}   \NormalTok{brakes  200}
\KeywordTok{Tractor} \NormalTok{brakes  300}
\KeywordTok{Combine} \NormalTok{brakes  400}
\KeywordTok{Tractor} \NormalTok{motor   1000}
\KeywordTok{Combine} \NormalTok{motor   1500}
\KeywordTok{Tractor} \NormalTok{tires   2000}
\KeywordTok{Combine} \NormalTok{tires   2500}
\KeywordTok{Truck}   \NormalTok{tires   400}
\KeywordTok{Truck}   \NormalTok{tires   400}
\KeywordTok{Truck}   \NormalTok{tires   400}
\KeywordTok{Truck}   \NormalTok{winch   100}
\end{Highlighting}
\end{Shaded}

What about by the third column, the amount?

\drcap{Sort by the third column}

\begin{Shaded}
\begin{Highlighting}[]
\KeywordTok{~/Invoices} \NormalTok{$ sort -k 3 *}
\KeywordTok{Truck}   \NormalTok{winch   100}
\KeywordTok{Tractor} \NormalTok{motor   1000}
\KeywordTok{Combine} \NormalTok{motor   1500}
\KeywordTok{Truck}   \NormalTok{brakes  200}
\KeywordTok{Tractor} \NormalTok{tires   2000}
\KeywordTok{Combine} \NormalTok{tires   2500}
\KeywordTok{Tractor} \NormalTok{brakes  300}
\KeywordTok{Combine} \NormalTok{brakes  400}
\KeywordTok{Truck}   \NormalTok{tires   400}
\KeywordTok{Truck}   \NormalTok{tires   400}
\KeywordTok{Truck}   \NormalTok{tires   400}
\end{Highlighting}
\end{Shaded}

That's not what we expected because it is sorting numbers
alphabetically. Let's fix that by telling it to sort numerically:

\drcap{Sort by third column, numerically}

\begin{Shaded}
\begin{Highlighting}[]
\KeywordTok{~/Invoices} \NormalTok{$ sort -k 3 -n *}
\KeywordTok{Truck}   \NormalTok{winch   100}
\KeywordTok{Truck}   \NormalTok{brakes  200}
\KeywordTok{Tractor} \NormalTok{brakes  300}
\KeywordTok{Combine} \NormalTok{brakes  400}
\KeywordTok{Truck}   \NormalTok{tires   400}
\KeywordTok{Truck}   \NormalTok{tires   400}
\KeywordTok{Truck}   \NormalTok{tires   400}
\KeywordTok{Tractor} \NormalTok{motor   1000}
\KeywordTok{Combine} \NormalTok{motor   1500}
\KeywordTok{Tractor} \NormalTok{tires   2000}
\KeywordTok{Combine} \NormalTok{tires   2500}
\end{Highlighting}
\end{Shaded}

Maybe we care about the top three most expensive items. We haven't
talked about \texttt{pipes} yet, but check this out:
\index{*@\texttt{"|} (pipe)}\index{files and directories!redirection}

\drcap{Top three most expensive items}

\begin{Shaded}
\begin{Highlighting}[]
\KeywordTok{~} \NormalTok{$ }\KeywordTok{sort} \NormalTok{-k 3 -n * }\KeywordTok{|} \KeywordTok{tail} \NormalTok{-n 3}
\KeywordTok{Combine} \NormalTok{motor   1500}
\KeywordTok{Tractor} \NormalTok{tires   2000}
\KeywordTok{Combine} \NormalTok{tires   2500}
\end{Highlighting}
\end{Shaded}

Finally, what if we want only unique rows?

\drcap{Sort and show only unique rows}

\begin{Shaded}
\begin{Highlighting}[]
\KeywordTok{~/Invoices} \NormalTok{$ sort -k 3 -n -u *}
\KeywordTok{Truck}   \NormalTok{winch   100}
\KeywordTok{Truck}   \NormalTok{brakes  200}
\KeywordTok{Tractor} \NormalTok{brakes  300}
\KeywordTok{Truck}   \NormalTok{tires   400}
\KeywordTok{Tractor} \NormalTok{motor   1000}
\KeywordTok{Combine} \NormalTok{motor   1500}
\KeywordTok{Tractor} \NormalTok{tires   2000}
\KeywordTok{Combine} \NormalTok{tires   2500}
\end{Highlighting}
\end{Shaded}

Just to reinforce long parameters, the last example is equivalent to:

\drcap{Sort unique rows using long parameter names}

\begin{Shaded}
\begin{Highlighting}[]
\KeywordTok{~/Invoices} \NormalTok{$ sort --key 3 --numeric-sort --unique *}
\KeywordTok{Truck}   \NormalTok{winch   100}
\KeywordTok{Truck}   \NormalTok{brakes  200}
\KeywordTok{Tractor} \NormalTok{brakes  300}
\KeywordTok{Truck}   \NormalTok{tires   400}
\KeywordTok{Tractor} \NormalTok{motor   1000}
\KeywordTok{Combine} \NormalTok{motor   1500}
\KeywordTok{Tractor} \NormalTok{tires   2000}
\KeywordTok{Combine} \NormalTok{tires   2500}
\end{Highlighting}
\end{Shaded}

If you read that command in a script file, there would be little
confusion as to what it was doing.

\section*{Rearranging Deck Chairs}\label{rearranging-deck-chairs}
\addcontentsline{toc}{section}{Rearranging Deck Chairs}

We can copy, move (or rename - same thing) and delete files and
directories. To \emph{copy}, simply use the
\href{http://linux.die.net/man/1/cp}{\texttt{cp}} command:
\drfnd{cp}{copy}

\drcap{\texttt{cp} command}

\begin{Shaded}
\begin{Highlighting}[]
\KeywordTok{~} \NormalTok{$ }\KeywordTok{cp} \NormalTok{diary.txt diary.bak}
\end{Highlighting}
\end{Shaded}

You can copy entire directories recursively:

\drcap{Copying directories recursively}

\begin{Shaded}
\begin{Highlighting}[]
\KeywordTok{~} \NormalTok{$ }\KeywordTok{cp} \NormalTok{-r thisdir thatdir}
\end{Highlighting}
\end{Shaded}

Or, if we want to be self-documenting in a script, we can use those long
parameter names again:

\drcap{\texttt{cp} command with long parameter names}

\begin{Shaded}
\begin{Highlighting}[]
\KeywordTok{~} \NormalTok{$ }\KeywordTok{cp} \NormalTok{--recursive thisdir thatdir}
\end{Highlighting}
\end{Shaded}

To \emph{move} use \href{http://linux.die.net/man/1/mv}{\texttt{mv}}:
\drfnd{mv}{move files}

\drcap{\texttt{mv} command}

\begin{Shaded}
\begin{Highlighting}[]
\KeywordTok{~} \NormalTok{$ }\KeywordTok{mv} \NormalTok{thismonth.log lastmonth.log}
\end{Highlighting}
\end{Shaded}

\textbf{Note:} There is no semantic difference between "move" and
"rename." However, there are some really cool renaming scenarios that
the \href{http://linux.die.net/man/1/rename}{\texttt{rename}} command
can take care of beyond \texttt{mv}, like renaming all file extensions
from \texttt{.htm} to \texttt{.html}. \drfnd{rename}{rename file}

\section*{Making Files Disappear}\label{making-files-disappear}
\addcontentsline{toc}{section}{Making Files Disappear}

To delete or \emph{remove} a file you use
\href{http://linux.die.net/man/1/rm}{\texttt{rm}}:
\drfnd{rm}{remove file}

\drcap{\texttt{rm} command}

\begin{Shaded}
\begin{Highlighting}[]
\KeywordTok{~} \NormalTok{$ }\KeywordTok{rm} \NormalTok{desktop.ini}
\end{Highlighting}
\end{Shaded}

\textbf{Pro Tip:} There is no "Are you sure?" prompt when removing a
single file specified with no wildcards, or even \textbf{\emph{all}}
files with a wildcard, and there is no "Recycle Bin" or "Trash Can" when
working from the command prompt, so \textbf{\emph{BE CAREFUL!}}

The following scenario can happen \textbf{\emph{way}} too often, even to
experienced system administrators. Note the accidental space between
\texttt{*} and \texttt{.bak} on the \texttt{rm} command:

\drcap{Oops!}

\begin{Shaded}
\begin{Highlighting}[]
\KeywordTok{~} \NormalTok{$ }\KeywordTok{cd} \NormalTok{MyDissertation/}
\KeywordTok{~/MyDissertation} \NormalTok{$ ls}
\KeywordTok{Bibliography.bak}  \NormalTok{Bibliography.doc  Dissertation.bak  Dissertation.doc}
\KeywordTok{~/MyDissertation} \NormalTok{$ rm * .bak}
\KeywordTok{rm}\NormalTok{: cannot remove ‘.bak’: No such file or directory}
\KeywordTok{~/MyDissertation} \NormalTok{$ ls}
\KeywordTok{~/MyDissertation} \NormalTok{$ }
\end{Highlighting}
\end{Shaded}

So, in order, our hapless user:

\begin{enumerate}
\def\labelenumi{\arabic{enumi}.}
\item
  Changed to directory \texttt{MyDissertation}.
\item
  Listed the directory contents with \texttt{ls}, saw the combination of
  \texttt{.doc} and \texttt{.bak} files.
\item
  Decided to delete the \texttt{.bak} files with \texttt{rm}, but
  accidentally typed in a space between the wildcard \texttt{*} and the
  \texttt{.bak}. Note ominous warning message.
\item
  Presto! \texttt{ls} shows \textbf{\emph{everything}} is gone, not just
  the backup files! The user's priorities just got rearranged as they go
  hunting for another copy of their dissertation.
\end{enumerate}

So be careful out there! This is an example where tab completion can be
an extra error check. Many times I use command history in these cases by
changing the \texttt{ls} to look for just the files I want to delete:

\drcap{First make sure we are dealing with the right files}

\begin{Shaded}
\begin{Highlighting}[]
\KeywordTok{~} \NormalTok{$ }\KeywordTok{ls} \NormalTok{*.bak}
\KeywordTok{Citations.bak}  \NormalTok{Dissertation.bak}
\end{Highlighting}
\end{Shaded}

Then I use the "up arrow" to bring back the \texttt{ls} command and
change \texttt{ls} to \texttt{rm} before running it. Safer that way.

\section*{\texorpdfstring{\texttt{touch} Me}{touch Me}}\label{touch-me}
\addcontentsline{toc}{section}{\texttt{touch} Me}

We just learned how to make a file disappear. We can also make a file
magically appear, just by
\href{http://linux.die.net/man/1/touch}{\texttt{touch}}:
\drfnd{touch}{change modified date or create file}

\drcap{\texttt{touch} command}

\begin{Shaded}
\begin{Highlighting}[]
\KeywordTok{~} \NormalTok{$ }\KeywordTok{touch} \NormalTok{NewEmptyDissertation.doc}
\KeywordTok{~} \NormalTok{$ }\KeywordTok{ls} \NormalTok{-l}
\KeywordTok{total} \NormalTok{0}
\KeywordTok{-rw-rwxr--+} \NormalTok{1 myuser mygroup 0 Oct 19 14:12 NewEmptyDissertation.doc}
\end{Highlighting}
\end{Shaded}

Notice the newly created file is zero bytes long.

Interestingly enough, we can also use \texttt{touch} just to update the
"last modified date" of an existing file, as you can see in time change
in the following listing after running \texttt{touch} on the same file
again: \drfnd{touch}{change modified date or create file}

\drcap{A second \texttt{touch}}

\begin{Shaded}
\begin{Highlighting}[]
\KeywordTok{~} \NormalTok{$ }\KeywordTok{touch} \NormalTok{NewEmptyDissertation.doc}
\KeywordTok{~} \NormalTok{$ }\KeywordTok{ls} \NormalTok{-l}
\KeywordTok{total} \NormalTok{0}
\KeywordTok{-rw-rwxr--+} \NormalTok{1 myuser mygroup 0 Oct 19 14:14 NewEmptyDissertation.doc}
\end{Highlighting}
\end{Shaded}

It can be useful (but also distressing from a forensics point of view)
to set the last modified date of a file to a specific date and time,
which \texttt{touch} also allows you to do, in this case to the night
before Christmas:

\drcap{Set file modified date to a specific date and time}

\begin{Shaded}
\begin{Highlighting}[]
\KeywordTok{~} \NormalTok{$ }\KeywordTok{touch} \NormalTok{-t 201412242300 NewEmptyDissertation.doc}
\KeywordTok{~} \NormalTok{$ }\KeywordTok{ls} \NormalTok{-l}
\KeywordTok{total} \NormalTok{0}
\KeywordTok{-rw-rwxr--+} \NormalTok{1 myuser mygroup 0 Dec 24  2014 NewEmptyDissertation.doc}
\end{Highlighting}
\end{Shaded}

To \emph{make a directory} you use
\href{http://linux.die.net/man/1/mkdir}{\texttt{mkdir}}:
\drfnd{mkdir}{make directory}

\drcap{\texttt{mkdir} command}

\begin{Shaded}
\begin{Highlighting}[]
\KeywordTok{~} \NormalTok{$ }\KeywordTok{cd} \NormalTok{Foo}
\KeywordTok{~/Foo} \NormalTok{$ ls -l}
\KeywordTok{total} \NormalTok{4}
\KeywordTok{-rw-r--r--} \NormalTok{1 myuser mygroup    0 Dec 14 05:49 a}
\KeywordTok{-rw-r--r--} \NormalTok{1 myuser mygroup    0 Dec 14 05:49 b}
\KeywordTok{-rw-r--r--} \NormalTok{1 myuser mygroup    0 Dec 14 05:49 c}
\KeywordTok{drwxr-xr-x} \NormalTok{2 myuser mygroup 4096 Dec 14 05:49 d}
\KeywordTok{~/Foo} \NormalTok{$ mkdir Bar}
\KeywordTok{~/Foo} \NormalTok{$ ls -l}
\KeywordTok{total} \NormalTok{8}
\KeywordTok{-rw-r--r--} \NormalTok{1 myuser mygroup    0 Dec 14 05:49 a}
\KeywordTok{-rw-r--r--} \NormalTok{1 myuser mygroup    0 Dec 14 05:49 b}
\KeywordTok{drwxr-xr-x} \NormalTok{2 myuser mygroup 4096 Dec 14 14:49 Bar}
\KeywordTok{-rw-r--r--} \NormalTok{1 myuser mygroup    0 Dec 14 05:49 c}
\KeywordTok{drwxr-xr-x} \NormalTok{2 myuser mygroup 4096 Dec 14 05:49 d}
\end{Highlighting}
\end{Shaded}

Typically you need to create all intervening directories before creating
a "child" directory:

\drcap{\texttt{mkdir} error}

\begin{Shaded}
\begin{Highlighting}[]
\KeywordTok{~} \NormalTok{$ }\KeywordTok{mkdir} \NormalTok{Xyzzy/Something}
\KeywordTok{mkdir}\NormalTok{: cannot create directory ‘Xyzzy/Something’: No such file or directory}
\end{Highlighting}
\end{Shaded}

But of course you can override that behavior:

\drcap{Make multiple intervening directories at once}

\begin{Shaded}
\begin{Highlighting}[]
\KeywordTok{~/Foo} \NormalTok{$ mkdir --parents Xyzzy/Something}
\KeywordTok{~/Foo} \NormalTok{$ ls }
\KeywordTok{a}  \NormalTok{b  Bar  c  d  Xyzzy}
\KeywordTok{~/Foo} \NormalTok{$ ls Xyzzy}
\KeywordTok{Something}
\end{Highlighting}
\end{Shaded}

\section*{Navigating Through Life}\label{navigating-through-life}
\addcontentsline{toc}{section}{Navigating Through Life}

Ever notice that "life" is an anagram for "file"? Spooky, eh?

Given that the UNIX-style file systems are hierarchical in nature they
are similar to navigate as with \texttt{CMD.EXE}. The biggest difference
is the absense of drive letters and the direction of the slashes.
\drshl{CMD.EXE}

To \emph{change directories}, simply use
\href{http://linux.die.net/man/1/cd}{\texttt{cd}} much like in Windows:
\drfnd{cd}{change directory}

\drcap{\texttt{cd} command}

\begin{Shaded}
\begin{Highlighting}[]
\KeywordTok{~} \NormalTok{$ }\KeywordTok{cd} \NormalTok{/etc}
\KeywordTok{~} \NormalTok{$ }\KeywordTok{pwd}
\KeywordTok{/etc}
\end{Highlighting}
\end{Shaded}

\href{http://linux.die.net/man/1/pwd}{\texttt{pwd}} simply \emph{prints
the working (current) directory}. If \texttt{whoami} tells you who you
are, \texttt{pwd} tells you \textbf{\emph{where}} you are.
\drfnd{pwd}{print working directory}

In Linux, users can have "home" directories (similar to Windows
profiles), typically located under
\texttt{/home/\textless{}username\textgreater{}} for normal users, and
\texttt{/root} for the "root" (admin) id. To change to a user's "home"
directory, simply use \texttt{cd} with no parameters:
\drfnd{cd}{change directory}
\index{files and directories!home (\texttt{\~{}})}

\drcap{Change to home directory}

\begin{Shaded}
\begin{Highlighting}[]
\KeywordTok{/etc} \NormalTok{$ cd}
\KeywordTok{~} \NormalTok{$ }\KeywordTok{pwd}
\KeywordTok{/home/myuser}
\end{Highlighting}
\end{Shaded}

The tilde (\texttt{\textasciitilde{}}) character is an alias for the
current user's home directory. The following example is equivalent to
above: \index{*@\texttt{\~{}} (home directory)}
\index{files and directories!home (\texttt{\~{}})}

\drcap{Alternative way to change to home directory}

\begin{Shaded}
\begin{Highlighting}[]
\KeywordTok{/etc} \NormalTok{$ cd ~}
\KeywordTok{~} \NormalTok{$ }\KeywordTok{pwd}
\KeywordTok{/home/myuser}
\end{Highlighting}
\end{Shaded}

More useful is that the tilde can be combined with a user name to
specify the home directory of \textbf{\emph{another}} user:

\drcap{Change to the home directory of another user}

\begin{Shaded}
\begin{Highlighting}[]
\KeywordTok{~} \CommentTok{# cd ~myuser}
\KeywordTok{myuser} \CommentTok{# pwd}
\KeywordTok{/home/myuser}
\end{Highlighting}
\end{Shaded}

\textbf{Note:} The above assumes you have permissions to \texttt{cd}
into \texttt{/home/myuser}. See the upcoming section on file permissions
for more info. \drfnd{cd}{change directory}

In addition, you need to know the difference between "absolute" and
"relative" paths:

\begin{itemize}
\item
  \textbf{Absolute path} - \textbf{\emph{always}} "goes through" or
  specifies the "root" (\texttt{/}) directory, e.g. regardless of the
  current working directory, \texttt{cd\ /etc} will change it to
  \texttt{/etc}. \index{*@\texttt{/} (root directory)}
  \index{files and directories!absolute path}
  \index{files and directories!root (\texttt{/})}
\item
  \textbf{Relative path} - does \textbf{\emph{not}} specify the root
  directory, and expects to start the navigation at the current
  directory with all path components traversed from there, e.g.,
  \texttt{cd\ Dissertations} changes the current directory to a
  subdirectory called \texttt{Dissertations}.
  \index{files and directories!relative path}
\end{itemize}

Windows inherited the concept of \texttt{.} for the current directory
and \texttt{..} for the parent directory directly from UNIX. Consider
the following examples that combine all of the above about relative
paths and see if it makes sense:
\index{*@\texttt{.} (current directory)}
\index{*@\texttt{..} (parent directory)}
\index{files and directories!current directory (\texttt{.})}
\index{files and directories!parent directory (\texttt{..})}

\drcap{Relative paths exercise}

\begin{Shaded}
\begin{Highlighting}[]
\KeywordTok{~/Foo} \NormalTok{$ ls}
\KeywordTok{~/Foo} \NormalTok{$ mkdir Bar Baz}
\KeywordTok{~/Foo} \NormalTok{$ ls}
\KeywordTok{Bar}  \NormalTok{Baz}
\KeywordTok{~/Foo} \NormalTok{$ cd Bar}
\KeywordTok{~/Foo/Bar} \NormalTok{$ touch a b c}
\KeywordTok{~/Foo/Bar} \NormalTok{$ ls}
\KeywordTok{a}  \NormalTok{b  c}
\KeywordTok{~/Foo/Bar} \NormalTok{$ cd ../Baz}
\KeywordTok{~/Foo/Baz} \NormalTok{$ touch d e f}
\KeywordTok{~/Foo/Baz} \NormalTok{$ ls}
\KeywordTok{d}  \NormalTok{e  f}
\KeywordTok{~/Foo/Baz} \NormalTok{$ ls ..}
\KeywordTok{Bar}  \NormalTok{Baz}
\KeywordTok{~/Foo/Baz} \NormalTok{$ ls ../Bar}
\KeywordTok{a}  \NormalTok{b  c}
\end{Highlighting}
\end{Shaded}

Did you notice how both \texttt{mkdir} and \texttt{touch} allow for
specifying multiple directory and file names in the same command?
\drfnd{mkdir}{make directory}
\drfnd{touch}{change modified date or create file}

\section*{May I?}\label{may-i}
\addcontentsline{toc}{section}{May I?}

Most "UNIX" file systems come with a set of nine permissions that can be
thought of as a "grid" of 3x3 showing "who has what?" The "who" is known
as "UGO": \index{files and directories!permissions}

\begin{itemize}
\item
  \textbf{User} - the user that is the "owner" of the file or directory.
\item
  \textbf{Group} - the group that is the "owner" of the file or
  directory.
\item
  \textbf{Other} - everyone else.
\end{itemize}

The "what" is:

\begin{itemize}
\item
  \textbf{Read}
\item
  \textbf{Write}
\item
  \textbf{Execute} - for files, for directories this means "navigate" or
  "list contents".
\end{itemize}

The combination of "who has what?" is usually shown in detailed
directory listings by a set of ten characters, with the first one
determining whether an entry is a directory (\texttt{d}) or a file
(\texttt{-}): \drbsd{FreeBSD}

\drcap{Another \texttt{ls -l} example, this time on FreeBSD}

\begin{Shaded}
\begin{Highlighting}[]
\KeywordTok{%} \NormalTok{ls -l /etc}
\KeywordTok{total} \NormalTok{1876}
\KeywordTok{drwxr-xr-x}   \NormalTok{2 root  wheel        512 Jan 15  2009 X11}
\KeywordTok{-rw-r--r--}   \NormalTok{1 root  wheel          0 Sep  3  2013 aliases}
\KeywordTok{-rw-r--r--}   \NormalTok{1 root  wheel      16384 Sep  3  2013 aliases.db}
\KeywordTok{-rw-r--r--}   \NormalTok{1 root  wheel        210 May  6  2009 amd.map}
\KeywordTok{-r--r--r--}   \NormalTok{1 root  wheel        233 Feb 15  2007 amd.map.snap}
\KeywordTok{-rw-r--r--}   \NormalTok{1 root  wheel       1234 May  6  2009 apmd.conf}
\KeywordTok{-rw-r--r--}   \NormalTok{1 root  wheel        231 May  6  2009 auth.conf}
\KeywordTok{drwxr-xr-x}   \NormalTok{2 root  wheel        512 May  6  2009 bluetooth}
\KeywordTok{-rw-r--r--}   \NormalTok{1 root  wheel        737 Mar 19  2015 crontab}
\KeywordTok{-rw-r--r--}   \NormalTok{1 root  wheel        108 May  6  2009 csh.cshrc}
\KeywordTok{-rw-r--r--}   \NormalTok{1 root  wheel        617 Apr 15  2009 csh.login}
\KeywordTok{-rw-r--r--}   \NormalTok{1 root  wheel        110 May  6  2009 csh.logout}
\KeywordTok{-rw-r--r--}   \NormalTok{1 root  wheel        565 May  6  2009 ddb.conf}
\KeywordTok{drwxr-xr-x}   \NormalTok{2 root  wheel        512 May  6  2009 defaults}
\KeywordTok{-rw-r--r--}   \NormalTok{1 root  wheel       9779 May  6  2009 devd.conf}
\KeywordTok{-rw-r--r--}   \NormalTok{1 root  wheel       2071 May  6  2009 devfs.conf}
\KeywordTok{-rw-r--r--}   \NormalTok{1 root  wheel        267 May  6  2009 dhclient.conf}
\KeywordTok{-rw-r--r--}   \NormalTok{1 root  wheel       5704 May  6  2009 disktab}
\KeywordTok{-rw-rw-r--}   \NormalTok{1 root  operator       0 Nov  3  2005 dumpdates}
\KeywordTok{drwxr-xr-x}   \NormalTok{6 root  staff        512 Nov 12  2014 fail2ban}
\KeywordTok{-rw-r--r--}   \NormalTok{1 root  wheel        142 May  6  2009 fbtab}
\KeywordTok{...and} \NormalTok{so on...}
\end{Highlighting}
\end{Shaded}

In the above, for example, we can see that the user \texttt{root} owns
the file \texttt{aliases} while the \texttt{wheel} group is the primary
group for it. \texttt{root} can both read and write the file
(\texttt{rw-}) while any user in the \texttt{wheel} group can only read
it (\texttt{r-\/-}). Any other id will also have read access
(\texttt{r-\/-}).

Similarly we see that \texttt{defaults} is a directory (\texttt{d}) that
can be read, written (new files created) and listed by \texttt{root}
(\texttt{rwx}), and read and listed by the group \texttt{wheel} and all
other ids (\texttt{r-xr-x}).

Back on Linux, if we look in \texttt{/etc/init.d} where many services
store their startup scripts we see:
\index{files and directories!special!etcinitd@\texttt{/etc/init.d/}}

\drcap{Listing the \texttt{/etc/init.d} directory}

\begin{Shaded}
\begin{Highlighting}[]
\KeywordTok{~} \NormalTok{$ }\KeywordTok{ls} \NormalTok{-l /etc/init.d}
\KeywordTok{total} \NormalTok{276}
\KeywordTok{-rwxr-xr-x} \NormalTok{1 root root 2243 Apr  3  2014 acpid}
\KeywordTok{-rwxr-xr-x} \NormalTok{1 root root 2014 Feb 19  2014 anacron}
\KeywordTok{-rwxr-xr-x} \NormalTok{1 root root 4596 Apr 24  2015 apparmor}
\KeywordTok{-rwxr-xr-x} \NormalTok{1 root root 2401 Dec 30  2013 avahi-daemon}
\KeywordTok{-rwxr-xr-x} \NormalTok{1 root root 1322 Mar 30  2014 binfmt-support}
\KeywordTok{-rwxr-xr-x} \NormalTok{1 root root 4474 Sep  4  2014 bluetooth}
\KeywordTok{-rwxr-xr-x} \NormalTok{1 root root 2125 Mar 13  2014 brltty}
\KeywordTok{-rwxr-xr-x} \NormalTok{1 root root 4651 Apr  9  2014 casper}
\KeywordTok{-rwxr-xr-x} \NormalTok{1 root root  425 Jun 26 09:11 cinnamon}
\KeywordTok{-rwxr-xr-x} \NormalTok{1 root root 1919 Jan 18  2011 console-setup}
\KeywordTok{-rwxr-xr-x} \NormalTok{1 root root 2489 May  6  2012 cpufrequtils}
\KeywordTok{lrwxrwxrwx} \NormalTok{1 root root   21 Sep  7 04:00 cron -}\KeywordTok{>} \NormalTok{/lib/init/upstart-job}
\KeywordTok{-rwxr-xr-x} \NormalTok{1 root root  938 Nov  1  2013 cryptdisks}
\KeywordTok{-rwxr-xr-x} \NormalTok{1 root root  896 Nov  1  2013 cryptdisks-early}
\KeywordTok{-rwxr-xr-x} \NormalTok{1 root root 3184 Apr  3  2014 cups}
\KeywordTok{-rwxr-xr-x} \NormalTok{1 root root 1961 Apr  7  2014 cups-browsed}
\KeywordTok{-rwxr-xr-x} \NormalTok{1 root root 2813 Nov 25  2014 dbus}
\KeywordTok{-rwxr-xr-x} \NormalTok{1 root root 1217 Mar  7  2013 dns-clean}
\KeywordTok{lrwxrwxrwx} \NormalTok{1 root root   21 Sep  7 04:00 friendly-recovery -}\KeywordTok{>} \NormalTok{/lib/init/upstart-}
\KeywordTok{job}
\KeywordTok{-rwxr-xr-x} \NormalTok{1 root root 1105 May 13  2015 grub-common}
\KeywordTok{...and} \NormalTok{so on...}
\end{Highlighting}
\end{Shaded}

In this case all the scripts are readable, writable and executable
(\texttt{rwx}) by the \texttt{root} user, and readable and executable by
the \texttt{root} group and all other users (\texttt{r-xr-x}). Later on
I will explain linked files (those that start with an \texttt{l} instead
of a \texttt{-} in the detailed listing above).

To \emph{change} the \emph{owning} user of a file or directory (assuming
you have permissions to do so), use the
\href{http://linux.die.net/man/1/chown}{\texttt{chown}} command:
\drprm{chown}

\drcap{Change file ownership}

\begin{Shaded}
\begin{Highlighting}[]
\CommentTok{# ls -l}
\KeywordTok{total} \NormalTok{4}
\KeywordTok{-rwxr--r--} \NormalTok{1 root root 17 Oct 20 10:07 foo}
\CommentTok{# chown git foo}
\CommentTok{# ls -l}
\KeywordTok{total} \NormalTok{4}
\KeywordTok{-rwxr--r--} \NormalTok{1 git root 17 Oct 20 10:07 foo}
\end{Highlighting}
\end{Shaded}

To \emph{change} the primary \emph{group}, use the
\href{http://linux.die.net/man/1/chgrp}{\texttt{chgrp}} command:
\drprm{chgrp}

\drcap{\texttt{chgrp} command}

\begin{Shaded}
\begin{Highlighting}[]
\CommentTok{# chgrp git foo}
\CommentTok{# ls -l}
\KeywordTok{total} \NormalTok{4}
\KeywordTok{-rwxr--r--} \NormalTok{1 git git 17 Oct 20 10:07 foo}
\end{Highlighting}
\end{Shaded}

To \emph{change} the various permissions or \emph{mode} bits, you use
the \href{http://linux.die.net/man/1/chmod}{\texttt{chmod}} command. It
uses mnemonics of "ugo" for user, group and "other," respectively. It
also uses mnemonics of "rwx" for read, write and execute, and \texttt{+}
to add a permission and \texttt{-} to remove it. For example, to add the
execute permission for the group and remove read permission for "other":
\drprm{chmod}

\drcap{\texttt{chmod} command}

\begin{Shaded}
\begin{Highlighting}[]
\CommentTok{# chmod g+x,o-r foo}
\CommentTok{# ls -l}
\KeywordTok{total} \NormalTok{4}
\KeywordTok{-rwxr-x---} \NormalTok{1 git git 17 Oct 20 10:07 foo}
\end{Highlighting}
\end{Shaded}

\textbf{Pro Tip:} To look like an old-hand UNIX hacker, you can also
convert any set of "rwx" permissions into an octal number from 0 (no
permissions) to 7 (all permissions). It helps to think of the three
permissions as "binary places":

\begin{itemize}
\tightlist
\item
  \textbf{\texttt{r}}
  \texttt{=\ 2}\textsuperscript{\texttt{2}}\texttt{=\ 4}
\item
  \textbf{\texttt{w}}
  \texttt{=\ 2}\textsuperscript{\texttt{1}}\texttt{=\ 2}
\item
  \textbf{\texttt{x}}
  \texttt{=\ 2}\textsuperscript{\texttt{0}}\texttt{=\ 1}
\item
  \textbf{\texttt{-}} \texttt{=\ 0}
\end{itemize}

Some examples:

\begin{itemize}
\tightlist
\item
  \textbf{\texttt{-\/-\/-}} \texttt{=\ 0\ +\ 0\ +\ 0\ =\ 0}
\item
  \textbf{\texttt{r-\/-}}
  \texttt{=\ 2}\textsuperscript{\texttt{2}}\texttt{+\ 0\ +\ 0\ =\ 4}
\item
  \textbf{\texttt{r-x}}
  \texttt{=\ 2}\textsuperscript{\texttt{2}}\texttt{+\ 0\ +\ 2}\textsuperscript{\texttt{0}}\texttt{=\ 5}
\item
  \textbf{\texttt{rw-}}
  \texttt{=\ 2}\textsuperscript{\texttt{2}}\texttt{+\ 2}\textsuperscript{\texttt{1}}\texttt{+\ 0\ =\ 6}
\item
  \textbf{\texttt{rwx}}
  \texttt{=\ 2}\textsuperscript{\texttt{2}}\texttt{+\ 2}\textsuperscript{\texttt{1}}\texttt{+\ 2}\textsuperscript{\texttt{0}}\texttt{=\ 7}
\end{itemize}

Now to use octal with \texttt{chmod}, we think of the overall result we
want for a file. For example, if we want the \texttt{foo} file to be
readable, writable and executable by both its owning user and group, and
not accessible at all by anyone else, we could use:

\drcap{\texttt{chmod} with lots of typing}

\begin{Shaded}
\begin{Highlighting}[]
\CommentTok{# chmod u+rwx,g+rwx,o- foo}
\CommentTok{# ls -l}
\KeywordTok{total} \NormalTok{4}
\KeywordTok{-rwxrwx---} \NormalTok{1 git git 17 Oct 20 10:07 foo}
\end{Highlighting}
\end{Shaded}

Or we could simply convert those permissions into octal in our head and:

\drcap{\texttt{chmod} with octal like a boss}

\begin{Shaded}
\begin{Highlighting}[]
\CommentTok{# chmod 770 foo}
\CommentTok{# ls -l}
\KeywordTok{total} \NormalTok{4}
\KeywordTok{-rwxrwx---} \NormalTok{1 git git 17 Oct 20 10:07 foo}
\end{Highlighting}
\end{Shaded}

Now you know the answer to that "How will we ever use octal in real
life?" question you asked in school!

For a script or executable file to be allowed to run, it
\textbf{\emph{must}} be marked as executable for one of the user, group
or other entries. The following should be insightful:

\drcap{Marking a file as executable}

\begin{Shaded}
\begin{Highlighting}[]
\CommentTok{# echo "echo Hello world" > foo}
\CommentTok{# ls -l}
\KeywordTok{total} \NormalTok{4}
\KeywordTok{-rw-r--r--} \NormalTok{1 root root 17 Oct 20 10:07 foo}
\CommentTok{# ./foo}
\KeywordTok{-bash}\NormalTok{: ./foo: Permission denied}
\CommentTok{# chmod u+x foo}
\CommentTok{# ls -l}
\KeywordTok{total} \NormalTok{4}
\KeywordTok{-rwxr--r--} \NormalTok{1 root root 17 Oct 20 10:07 foo}
\CommentTok{# ./foo}
\KeywordTok{Hello} \NormalTok{world}
\end{Highlighting}
\end{Shaded}

\section*{"I'll Send You a Tar Ball"}\label{ill-send-you-a-tar-ball}
\addcontentsline{toc}{section}{"I'll Send You a Tar Ball"}

In the Windows world, we are used to compressing and sending directories
around as \texttt{.zip} files. In Linux you can also deal with
\texttt{.zip} files, although they don't tend to be the most common,
using the \href{http://linux.die.net/man/1/zip}{\texttt{zip}} and
\href{http://linux.die.net/man/1/unzip}{\texttt{unzip}} commands:
\drcom{unzip} \drcom{zip}
\index{compression!zipfiles@\texttt{.zip} files}
\index{files and directories!zipfiles@\texttt{.zip} files}

\drcap{\texttt{zip} command}

\begin{Shaded}
\begin{Highlighting}[]
\KeywordTok{~} \NormalTok{$ }\KeywordTok{cd} \NormalTok{Foo}
\KeywordTok{~/Foo} \NormalTok{$ touch a b c}
\KeywordTok{~/Foo} \NormalTok{$ mkdir d}
\KeywordTok{~/Foo} \NormalTok{$ touch d/e}
\KeywordTok{~/Foo} \NormalTok{$ cd ..}
\KeywordTok{~} \NormalTok{$ }\KeywordTok{zip} \NormalTok{-r Foo Foo}
  \KeywordTok{adding}\NormalTok{: Foo/ (stored 0%)}
  \KeywordTok{adding}\NormalTok{: Foo/c (stored 0%)}
  \KeywordTok{adding}\NormalTok{: Foo/b (stored 0%)}
  \KeywordTok{adding}\NormalTok{: Foo/d/ (stored 0%)}
  \KeywordTok{adding}\NormalTok{: Foo/d/e (stored 0%)}
  \KeywordTok{adding}\NormalTok{: Foo/a (stored 0%)}
\KeywordTok{~} \NormalTok{$ }\KeywordTok{ls} \NormalTok{-l Foo.zip}
\KeywordTok{-rw-r--r--} \NormalTok{1 myuser mygroup 854 Dec 14 15:31 Foo.zip}
\end{Highlighting}
\end{Shaded}

\drcap{\texttt{unzip} command}

\begin{Shaded}
\begin{Highlighting}[]
\KeywordTok{~} \NormalTok{$ }\KeywordTok{unzip} \NormalTok{Foo}
\KeywordTok{Archive}\NormalTok{:  Foo.zip}
\KeywordTok{replace} \NormalTok{Foo/c? [y]es, [n]o, [A]ll, [N]one, [r]ename: A}
 \KeywordTok{extracting}\NormalTok{: Foo/c}
 \KeywordTok{extracting}\NormalTok{: Foo/b                   }
 \KeywordTok{extracting}\NormalTok{: Foo/d/e                 }
 \KeywordTok{extracting}\NormalTok{: Foo/a}
\end{Highlighting}
\end{Shaded}

Not too exciting, but you get the drift. There is typically support for
other compression algorithms, such as the
\href{http://linux.die.net/man/1/gzip}{\texttt{gzip}},
\href{http://linux.die.net/man/1/bzip2}{\texttt{bzip2}} and
\href{http://linux.die.net/man/1/7z}{\texttt{7z}} (7-zip) commands.
\drcom{7z} \drcom{bzip2} \drcom{gzip}

However, the "native" way to "archive" a directory's contents in "UNIX"
is with \href{http://linux.die.net/man/1/tar}{\texttt{tar}}, which is so
old that \texttt{tar} stands for "tape archive." Its purpose is to take
virtually any directory structure and create a single output "stream" or
file of it. That is then typically ran through a compression command and
the result is called a "tarball": \drcmd{tar}{archive files}

\drcap{Creating a tarball}

\begin{Shaded}
\begin{Highlighting}[]
\KeywordTok{~} \NormalTok{$ }\KeywordTok{tar} \NormalTok{cvf Foo.tar Foo/*}
\KeywordTok{Foo/a}
\KeywordTok{Foo/b}
\KeywordTok{Foo/c}
\KeywordTok{Foo/d/}
\KeywordTok{Foo/d/e}
\KeywordTok{~} \NormalTok{$ }\KeywordTok{ls} \NormalTok{-l Foo.tar }
\KeywordTok{-rw-r--r--} \NormalTok{1 myuser mygroup 10240 Dec 19 07:52 Foo.tar}
\KeywordTok{~} \NormalTok{$ }\KeywordTok{gzip} \NormalTok{Foo.tar }
\KeywordTok{~} \NormalTok{$ }\KeywordTok{ls} \NormalTok{-l Foo.tar.gz }
\KeywordTok{-rw-r--r--} \NormalTok{1 myuser mygroup 193 Dec 19 07:52 Foo.tar.gz}
\end{Highlighting}
\end{Shaded}

In the \texttt{tar} command above, the parameters are \texttt{c} (create
a new archive), \texttt{v} (turn on "verbose" output) and \texttt{f}
followed by the file name of the new \texttt{.tar} file.
\index{files and directories!tararchivefiles@\texttt{.tar} archive files}

\textbf{Note:} \texttt{tar} supports POSIX-style parameters
(\texttt{-c}), GNU-style (\texttt{-\/-create}), as well as the older
style (\texttt{c} with no hyphens at all), as shown in these examples.
So both of the following are also equivalent to the above:

\drcap{\texttt{tar} parameter styles}

\begin{Shaded}
\begin{Highlighting}[]
\KeywordTok{~} \NormalTok{$ }\KeywordTok{tar} \NormalTok{-c -v -f Foo.tar Foo/*}
\KeywordTok{~} \NormalTok{$ }\KeywordTok{tar} \NormalTok{--create --verbose --file=Foo.tar Foo/*}
\end{Highlighting}
\end{Shaded}

The use of compression commands along with \texttt{tar} is so prevalent
that they've been built into \texttt{tar} itself now as optional
parameters:

\drcap{One-step tarball}

\begin{Shaded}
\begin{Highlighting}[]
\KeywordTok{~} \NormalTok{$ }\KeywordTok{tar} \NormalTok{cvzf Foo.tgz Foo}
\KeywordTok{Foo/}
\KeywordTok{Foo/c}
\KeywordTok{Foo/b}
\KeywordTok{Foo/d/}
\KeywordTok{Foo/d/e}
\KeywordTok{Foo/a}
\KeywordTok{~} \NormalTok{$ }\KeywordTok{ls} \NormalTok{-l Foo.tgz }
\KeywordTok{-rw-r--r--} \NormalTok{1 myuser mygroup 197 Dec 19 07:54 Foo.tgz}
\end{Highlighting}
\end{Shaded}

In this case the \texttt{z} parameter says to use \texttt{gzip}
compression, and the \texttt{.tgz} file suffix means basically "tarred
and gzipped", or the equivalent to \texttt{.tar.gz} in the first
example. \drcom{gzip} \index{compression!tgzfiles@\texttt{.tgz} files}
\index{files and directories!tgzarchivefiles@\texttt{.tgz} archive files}

\texttt{tar} is used to both create and read \texttt{.tar} files. So to
extract something like the above, you can change the create (\texttt{c})
parameter to extract (\texttt{x}), like this:

\drcap{Extracting a tarball}

\begin{Shaded}
\begin{Highlighting}[]
\KeywordTok{~} \NormalTok{$ }\KeywordTok{tar} \NormalTok{xvf Foo.tgz }
\KeywordTok{Foo/}
\KeywordTok{Foo/c}
\KeywordTok{Foo/b}
\KeywordTok{Foo/d/}
\KeywordTok{Foo/d/e}
\KeywordTok{Foo/a}
\end{Highlighting}
\end{Shaded}

\section*{\texorpdfstring{Let's \texttt{link}
Up!}{Let's link Up!}}\label{lets-link-up}
\addcontentsline{toc}{section}{Let's \texttt{link} Up!}

In Windows there are "shortcuts," which are simply special files that
the OS knows to interpret as "go open this other file over there." There
are also "hard links" that allow for different directory entries
\textbf{\emph{in the same file system}} to point to the same physical
file.

UNIX file systems also have both these types of links (which isn't
surprising, given that Microsoft got the ideas from UNIX). Both are
created with the \href{http://linux.die.net/man/1/ln}{\texttt{ln}}
command. A "soft link" is equivalent to a Windows shortcut, and can
point to a file or a directory, and can point to anything on any mounted
file system: \drfnd{ln}{link} \index{files and directories!soft links}

\drcap{Soft links example}

\begin{Shaded}
\begin{Highlighting}[]
\KeywordTok{~} \NormalTok{$ }\KeywordTok{ls} \NormalTok{-l}
\KeywordTok{total} \NormalTok{4}
\KeywordTok{-rw-r--r--} \NormalTok{1 myuser mygroup    0 Oct 24 15:53 a}
\KeywordTok{-rw-r--r--} \NormalTok{1 myuser mygroup    0 Oct 24 15:53 b}
\KeywordTok{-rw-r--r--} \NormalTok{1 myuser mygroup    0 Oct 24 15:53 c}
\KeywordTok{drwxr-xr-x} \NormalTok{2 myuser mygroup 4096 Oct 24 16:00 d}
\KeywordTok{~} \NormalTok{$ }\KeywordTok{cd} \NormalTok{d}
\KeywordTok{~} \NormalTok{$ }\KeywordTok{pwd}
\KeywordTok{/tmp/foo/d}
\KeywordTok{~} \NormalTok{$ }\KeywordTok{cd} \NormalTok{..}
\KeywordTok{~} \NormalTok{$ }\KeywordTok{ln} \NormalTok{-s a MyThesis.doc}
\KeywordTok{~} \NormalTok{$ }\KeywordTok{ln} \NormalTok{-s d Dee}
\KeywordTok{~} \NormalTok{$ }\KeywordTok{ls} \NormalTok{-l}
\KeywordTok{total} \NormalTok{4}
\KeywordTok{-rw-r--r--} \NormalTok{1 myuser mygroup    0 Oct 24 15:53 a}
\KeywordTok{-rw-r--r--} \NormalTok{1 myuser mygroup    0 Oct 24 15:53 b}
\KeywordTok{-rw-r--r--} \NormalTok{1 myuser mygroup    0 Oct 24 15:53 c}
\KeywordTok{drwxr-xr-x} \NormalTok{2 myuser mygroup 4096 Oct 24 16:00 d}
\KeywordTok{lrwxrwxrwx} \NormalTok{1 myuser mygroup    1 Oct 24 16:40 Dee -}\KeywordTok{>} \NormalTok{d}
\KeywordTok{lrwxrwxrwx} \NormalTok{1 myuser mygroup    1 Oct 24 16:40 MyThesis.doc -}\KeywordTok{>} \NormalTok{a}
\KeywordTok{~} \NormalTok{$ }\KeywordTok{cd} \NormalTok{Dee}
\KeywordTok{~} \NormalTok{$ }\KeywordTok{pwd}
\KeywordTok{/tmp/foo/Dee}
\end{Highlighting}
\end{Shaded}

The things to notice about this example:

\begin{enumerate}
\def\labelenumi{\arabic{enumi}.}
\item
  The \texttt{-s} parameter indicates "create a \emph{soft} link."
\item
  Instead of a \texttt{-} or \texttt{d}, a soft link is shown in a
  \texttt{ls} listing as \texttt{l} regardless of whether the target is
  a file or directory. This is because a soft link doesn't "know"
  \textbf{\emph{what}} the target is - it is just a file in a directory
  pointing to another location. \textbf{\emph{What}} that location is
  will be determined after the link is traversed.
\end{enumerate}

A "hard link" is a bit different. It can only be made between
\emph{files} and the two files \textbf{\emph{must be on the same file
system}}. That is because hard links are actually directory entries (as
opposed to files in directories) that point to the same
\href{https://en.wikipedia.org/wiki/Inode}{"inode"} on disk. From within
a single directory it is impossible to tell if there are other
directories with pointers to the same files (inodes) on disk.
\index{files and directories!hard links}
\index{files and directories!inodes}

\drcap{Hard links example}

\begin{Shaded}
\begin{Highlighting}[]
\KeywordTok{~} \NormalTok{$ }\KeywordTok{ls}
\KeywordTok{a}  \NormalTok{b  c  d  Dee  MyThesis.doc}
\KeywordTok{~} \NormalTok{$ }\KeywordTok{ln} \NormalTok{b B}
\KeywordTok{~} \NormalTok{$ }\KeywordTok{cd} \NormalTok{d}
\KeywordTok{~} \NormalTok{$ }\KeywordTok{ln} \NormalTok{../b .}
\KeywordTok{~} \NormalTok{$ }\KeywordTok{ls} \NormalTok{-l}
\KeywordTok{total} \NormalTok{0}
\KeywordTok{-rw-r--r--} \NormalTok{3 myuser mygroup 0 Oct 24 15:53 b}
\KeywordTok{-rw-r--r--} \NormalTok{1 myuser mygroup 0 Oct 24 15:54 e}
\KeywordTok{~} \NormalTok{$ }\KeywordTok{cd} \NormalTok{..}
\KeywordTok{~} \NormalTok{$ }\KeywordTok{ls} \NormalTok{-l}
\KeywordTok{total} \NormalTok{4}
\KeywordTok{-rw-r--r--} \NormalTok{1 myuser mygroup    0 Oct 24 15:53 a}
\KeywordTok{-rw-r--r--} \NormalTok{3 myuser mygroup    0 Oct 24 15:53 b}
\KeywordTok{-rw-r--r--} \NormalTok{3 myuser mygroup    0 Oct 24 15:53 B}
\KeywordTok{-rw-r--r--} \NormalTok{1 myuser mygroup    0 Oct 24 15:53 c}
\KeywordTok{drwxr-xr-x} \NormalTok{2 myuser mygroup 4096 Oct 24 16:49 d}
\KeywordTok{lrwxrwxrwx} \NormalTok{1 myuser mygroup    1 Oct 24 16:40 Dee -}\KeywordTok{>} \NormalTok{d}
\KeywordTok{lrwxrwxrwx} \NormalTok{1 myuser mygroup    1 Oct 24 16:40 MyThesis.doc -}\KeywordTok{>} \NormalTok{a}
\end{Highlighting}
\end{Shaded}

The "net net" of all the above is that now \texttt{b}, \texttt{B} and
\texttt{d/b} all point to exactly the same inode, or disk location,
i.e., the exact same physical file.

\subsection*{I Said "Go Away!", Dammit!}\label{i-said-go-away-dammit}
\addcontentsline{toc}{subsection}{I Said "Go Away!", Dammit!}

So what can possibly go wrong with links? With soft links the answer is
easy - the "remote" location being pointed to goes away or is renamed:

\drcap{Broken soft links example}

\begin{Shaded}
\begin{Highlighting}[]
\KeywordTok{~} \NormalTok{$ }\KeywordTok{ls} \NormalTok{-l}
\KeywordTok{total} \NormalTok{4}
\KeywordTok{-rw-r--r--} \NormalTok{1 myuser mygroup    0 Oct 24 15:53 a}
\KeywordTok{-rw-r--r--} \NormalTok{3 myuser mygroup    0 Oct 24 15:53 b}
\KeywordTok{-rw-r--r--} \NormalTok{3 myuser mygroup    0 Oct 24 15:53 B}
\KeywordTok{-rw-r--r--} \NormalTok{1 myuser mygroup    0 Oct 24 15:53 c}
\KeywordTok{drwxr-xr-x} \NormalTok{2 myuser mygroup 4096 Oct 24 16:49 d}
\KeywordTok{lrwxrwxrwx} \NormalTok{1 myuser mygroup    1 Oct 24 16:40 Dee -}\KeywordTok{>} \NormalTok{d}
\KeywordTok{lrwxrwxrwx} \NormalTok{1 myuser mygroup    1 Oct 24 16:40 MyThesis.doc -}\KeywordTok{>} \NormalTok{a}
\KeywordTok{~} \NormalTok{$ }\KeywordTok{rm} \NormalTok{a}
\KeywordTok{~} \NormalTok{$ }\KeywordTok{ls} \NormalTok{-l}
\KeywordTok{total} \NormalTok{4}
\KeywordTok{-rw-r--r--} \NormalTok{3 myuser mygroup    0 Oct 24 15:53 b}
\KeywordTok{-rw-r--r--} \NormalTok{3 myuser mygroup    0 Oct 24 15:53 B}
\KeywordTok{-rw-r--r--} \NormalTok{1 myuser mygroup    0 Oct 24 15:53 c}
\KeywordTok{drwxr-xr-x} \NormalTok{2 myuser mygroup 4096 Oct 24 16:49 d}
\KeywordTok{lrwxrwxrwx} \NormalTok{1 myuser mygroup    1 Oct 24 16:40 Dee -}\KeywordTok{>} \NormalTok{d}
\KeywordTok{lrwxrwxrwx} \NormalTok{1 myuser mygroup    1 Oct 24 16:40 MyThesis.doc -}\KeywordTok{>} \NormalTok{a}
\KeywordTok{~} \NormalTok{$ }\KeywordTok{cat} \NormalTok{MyThesis.doc }
\KeywordTok{cat}\NormalTok{: MyThesis.doc: No such file or directory}
\end{Highlighting}
\end{Shaded}

So even though the soft link \texttt{MyThesis.doc} is still in the
directory, the actual underlying file \texttt{a} is now gone, and trying
to access it via the soft link leads to the somewhat confusing "No such
file or directory" error message (\emph{splutter} "I can see it!
\textbf{\emph{It's right there!}}")

With hard links, it isn't so much a problem because of the nature of the
beast. Since each hard link is a directory (metadata) entry pointing to
an inode, deleting one simply deletes that directory entry. As long as
the file has other hard links pointing to it, it "exists." Only when the
last remaining hard link is removed has it been "deleted." Let's play:

\drcap{Many hard links, one inode}

\begin{Shaded}
\begin{Highlighting}[]
\KeywordTok{~} \NormalTok{$ }\KeywordTok{echo} \StringTok{"This is b."} \KeywordTok{>} \NormalTok{b}
\KeywordTok{~} \NormalTok{$ }\KeywordTok{cat} \NormalTok{b}
\KeywordTok{This} \NormalTok{is b.}
\KeywordTok{~} \NormalTok{$ }\KeywordTok{cat} \NormalTok{B}
\KeywordTok{This} \NormalTok{is b.}
\KeywordTok{~} \NormalTok{$ }\KeywordTok{cat} \NormalTok{d/b}
\KeywordTok{This} \NormalTok{is b.}
\end{Highlighting}
\end{Shaded}

So, that makes sense. We created an original file \texttt{b} by placing
"This is b." in it, and then created two hard links to it, \texttt{B}
and \texttt{d/b}. We see that it has the same contents no matter how we
access it, because it is pointing to the same inode.

Can you guess how many \texttt{rm} commands it will take to delete the
file containing "This is b."? \drfnd{rm}{remove file}

\drcap{Deleting a file with many hard links}

\begin{Shaded}
\begin{Highlighting}[]
\KeywordTok{~} \NormalTok{$ }\KeywordTok{rm} \NormalTok{b}
\KeywordTok{~} \NormalTok{$ }\KeywordTok{cat} \NormalTok{b}
\KeywordTok{cat}\NormalTok{: b: No such file or directory}
\KeywordTok{~} \NormalTok{$ }\KeywordTok{cat} \NormalTok{B}
\KeywordTok{This} \NormalTok{is b.}
\KeywordTok{~} \NormalTok{$ }\KeywordTok{cat} \NormalTok{d/b}
\KeywordTok{This} \NormalTok{is b.}
\KeywordTok{~} \NormalTok{$ }\KeywordTok{rm} \NormalTok{B}
\KeywordTok{~} \NormalTok{$ }\KeywordTok{cat} \NormalTok{d/b}
\KeywordTok{This} \NormalTok{is b.}
\KeywordTok{~} \NormalTok{$ }\KeywordTok{rm} \NormalTok{d/b}
\end{Highlighting}
\end{Shaded}

Ultimately, it takes a \texttt{rm} for every hard link to permanently
delete a file.

\subsection*{\texorpdfstring{\texttt{mount} It? I Don't Even Know It's
Name!}{mount It? I Don't Even Know It's Name!}}\label{mount-it-i-dont-even-know-its-name}
\addcontentsline{toc}{subsection}{\texttt{mount} It? I Don't Even Know
It's Name!}

With all this talk that a hard link can only be on the same file system,
how do you know whether two directories are on the same file system? In
Windows it's easy - that's exactly what the drive letters are telling
you. But in Linux, where everything is "mounted" under a single
hierarchy starting at \texttt{/}, how do I know that
\texttt{/var/something} and \texttt{var/or/other} are on the same file
system? \index{*@\texttt{/} (root directory)}
\index{files and directories!root (\texttt{/})}

There are multiple ways to tell, actually. The easiest is with the
\href{http://linux.die.net/man/1/df}{\texttt{df}} command:
\drfnd{df}{display file system disk space}

\drcap{\texttt{df} command}

\begin{Shaded}
\begin{Highlighting}[]
\KeywordTok{~} \NormalTok{$ }\KeywordTok{df}
\KeywordTok{Filesystem}                \NormalTok{1K-blocks     Used Available Use% Mounted on}
\KeywordTok{/dev/mapper/mint--vg-root} \NormalTok{118647068 28847464  83749608  26% /}
\KeywordTok{none}                              \NormalTok{4        0         4   0% /sys/fs/cgroup}
\KeywordTok{udev}                        \NormalTok{1965068        4   1965064   1% /dev}
\KeywordTok{tmpfs}                        \NormalTok{396216     1568    394648   1% /run}
\KeywordTok{none}                           \NormalTok{5120        0      5120   0% /run/lock}
\KeywordTok{none}                        \NormalTok{1981068      840   1980228   1% /run/shm}
\KeywordTok{none}                         \NormalTok{102400       24    102376   1% /run/user}
\KeywordTok{/dev/sda1}                    \NormalTok{240972    50153    178378  22% /boot}
\end{Highlighting}
\end{Shaded}

The ones of interest are the \texttt{/dev} entries, and we see that
everything mounted under \texttt{/} is on one file system, except for
whatever happens to be on the file system mounted under \texttt{/boot}.
So outside of \texttt{/boot}, on this system we could hard link away to
our heart's content.

\texttt{df} is a good command to see the disk space utilization of each
file system. If you want to see the space used by a directory and its
subdirectories, use \href{http://linux.die.net/man/1/du}{\texttt{du}}:
\drfnd{du}{disk use by directory}

\drcap{\texttt{du} command}

\begin{Shaded}
\begin{Highlighting}[]
\KeywordTok{/tmp} \NormalTok{$ du}
\KeywordTok{4}   \NormalTok{./icedteaplugin-mdm-EMmQCt}
\KeywordTok{4}   \NormalTok{./ssh-IaP0RC1l4XCL}
\KeywordTok{4}   \NormalTok{./hsperfdata_mdm}
\KeywordTok{4}   \NormalTok{./.ICE-unix}
\KeywordTok{4}   \NormalTok{./VSCode Crashes}
\KeywordTok{4}   \NormalTok{./.X11-unix}
\KeywordTok{8}   \NormalTok{./mintUpdate}
\KeywordTok{4}   \NormalTok{./orbit-myuser}
\KeywordTok{4}   \NormalTok{./pulse-PKdhtXMmr18n}
\KeywordTok{84}  \NormalTok{.}
\end{Highlighting}
\end{Shaded}

The default size unit for \texttt{du} is 1,024 bytes, but that can be
changed. So in the above, \texttt{/tmp} and its children are taking 84KB
of disk space.

\textbf{Note:} - It is (barely) beyond the scope of this book to cover
the \href{http://linux.die.net/man/8/mount}{\texttt{mount}} command. I
wanted to, really bad, but with all the different file systems and
device types and all the options for both it can get so complex so fast
that I decided not to. Maybe if you ask, real nice...
\drsys{mount}{mount file system}

\subsection*{I'm Seeing Double}\label{im-seeing-double}
\addcontentsline{toc}{subsection}{I'm Seeing Double}

So, both hard and soft links can have some interesting side effects if
you think about them. For one, if you are backing things up, then you
may get duplicates in your backup set. In fact, with hard links you
will, by definition, unless the backup software is very smart and doing
things like de-duplication. \index{files and directories!hard links}
\index{files and directories!soft links}

But even with soft links if everything just blindly followed them you
could also get duplicates where you didn't want them, or even circular
references. Also, the pointers in the soft link files are not evaluated
until a command references them. Note that the following is perfectly
legal with soft links, but may not give the results you expect - think
about the current working directory shown by \texttt{pwd} in the
following, and the effects of the \textbf{\emph{relative paths}} as the
sample progresses: \drfnd{pwd}{print working directory}

\drcap{Soft links and relative paths}

\begin{Shaded}
\begin{Highlighting}[]
\KeywordTok{~} \NormalTok{$ }\KeywordTok{cd} \NormalTok{Foo}
\KeywordTok{~/Foo} \NormalTok{$ rm -rf *}
\KeywordTok{~/Foo} \NormalTok{$ cd ..}
\KeywordTok{~} \NormalTok{$ }\KeywordTok{cd} \NormalTok{Foo}
\KeywordTok{~/Foo} \NormalTok{$ pwd}
\KeywordTok{/home/myuser/Foo}
\KeywordTok{~/Foo} \NormalTok{$ rm -rf *}
\KeywordTok{~/Foo} \NormalTok{$ mkdir d}
\KeywordTok{~/Foo} \NormalTok{$ touch a b c d/e}
\KeywordTok{~/Foo} \NormalTok{$ ln -s . d/f}
\KeywordTok{~/Foo} \NormalTok{$ ls d/f}
\KeywordTok{e}  \NormalTok{f}
\KeywordTok{~/Foo} \NormalTok{$ ln -s .. d/g}
\KeywordTok{~/Foo} \NormalTok{$ ls d/g}
\KeywordTok{a}  \NormalTok{b  c  d}
\end{Highlighting}
\end{Shaded}

Many commands that deal with files and file systems, like \texttt{find},
have parameters specifically telling the command whether to follow soft
links or not (by default, \texttt{find} does not - see the next chapter
for more). \drfnd{find}{find files}

\section*{\texorpdfstring{What's the
\texttt{diff}?}{What's the diff?}}\label{whats-the-diff}
\addcontentsline{toc}{section}{What's the \texttt{diff}?}

Most people think of
\href{http://linux.die.net/man/1/diff}{\texttt{diff}} as a tool only
programmers find useful, but that is short-sighted. The whole purpose of
\texttt{diff} is to show differences between files. For example, I
backed up this document (which is a text file) before starting this
section, then typed this introduction to \texttt{diff}. This is what
\texttt{diff} showed after I added the new paragraph:
\drfnd{diff}{show differences between files}
\index{files and directories!compare files (\texttt{diff} command)}

\drcap{\texttt{diff} example}

\begin{Shaded}
\begin{Highlighting}[]
\KeywordTok{~} \NormalTok{$ }\KeywordTok{diff} \NormalTok{Step02.bak Step02.md}
\KeywordTok{1285a1286}\NormalTok{,1291}
\KeywordTok{>} \KeywordTok{Most} \NormalTok{people think of [}\KeywordTok{`diff`}\NormalTok{](http://linux.die.net/man/1/diff) }\KeywordTok{as} \NormalTok{a tool}
\KeywordTok{>} \KeywordTok{only} \NormalTok{programmers find useful, but that is short-sighted. The whole purpose}
\KeywordTok{>} \KeywordTok{of} \KeywordTok{`diff`} \NormalTok{is to show differences between files. For example, I backed up}
\KeywordTok{>} \KeywordTok{this} \NormalTok{document (which is a text file) }\KeywordTok{before} \NormalTok{starting this chapter, then}
\KeywordTok{>} \KeywordTok{typed} \NormalTok{this introduction to }\KeywordTok{`diff`}\NormalTok{. This is what }\KeywordTok{`diff`} \NormalTok{shows:}
\end{Highlighting}
\end{Shaded}

In other words, the "arrows" are pointing to the "new" file (by
convention the file specified on the left is the "old" file and the file
on the right is the "new" file), showing five lines were inserted,
starting at line 1285. Pretty meta, but not real exciting.

Let's look at something else, say a configuration file for an
application. We have an original file, \texttt{orig.conf}:

\drcap{\texttt{orig.conf} file}

\begin{Shaded}
\begin{Highlighting}[]
\KeywordTok{~} \NormalTok{$ }\KeywordTok{cat} \NormalTok{orig.conf}
\OtherTok{FOO=}\NormalTok{1}

\OtherTok{SOME=}\NormalTok{THINGS}
\OtherTok{STAY=}\NormalTok{THE}
\OtherTok{SAME=}\NormalTok{ALWAYS}

\OtherTok{BAR=}\NormalTok{Xyzzy}
\end{Highlighting}
\end{Shaded}

Then we have a new file, \texttt{new.conf}:

\drcap{\texttt{new.conf} file}

\begin{Shaded}
\begin{Highlighting}[]
\KeywordTok{~} \NormalTok{$ }\KeywordTok{cat} \NormalTok{new.conf}
\OtherTok{FOO=}\NormalTok{2}

\OtherTok{SOME=}\NormalTok{THINGS}
\OtherTok{STAY=}\NormalTok{THE}
\OtherTok{SAME=}\NormalTok{ALWAYS}
\end{Highlighting}
\end{Shaded}

Now if we \texttt{diff} them:

\drcap{Using \texttt{diff} on config files}

\begin{Shaded}
\begin{Highlighting}[]
\KeywordTok{~} \NormalTok{$ }\KeywordTok{diff} \NormalTok{orig.conf new.conf}
\KeywordTok{1c1}
\KeywordTok{<} \OtherTok{FOO=}\NormalTok{1}
\KeywordTok{---}
\KeywordTok{>} \OtherTok{FOO=}\NormalTok{2}
\KeywordTok{7d6}
\KeywordTok{<} \OtherTok{BAR=}\NormalTok{Xyzzy}
\end{Highlighting}
\end{Shaded}

Now we can more easily see that line \#1 changed (\texttt{1c1}) from
\texttt{FOO=1} on the "left" file to \texttt{FOO=2} on the "right," and
that line \#7 was deleted (\texttt{7d6}) from the "left" file to form
the "right." Again, not too interesting, but imagine that both files
were thousands of lines long, and there were only a few changes, and you
were trying to detect and recover an accidentally-deleted line. Now you
can see why \texttt{diff} can be handy, as long as you keep around a
prior version either in a backup file or version control system to
compare against.

\texttt{diff} is your friend. It really comes into play with a version
control system like \href{http://linux.die.net/man/1/git}{\texttt{git}},
but again, that is beyond the scope of this book.
\drcmd{git}{distributed version control}

\hypertarget{finding-meaning}{\chapter{Finding
Meaning}\label{finding-meaning}}

\textbf{\emph{The \texttt{find} command in all its glory. Probably the
single most useful command in "UNIX" (I think)}}

\begin{quote}
\emph{"If we had bacon, we could have bacon and eggs, if we had eggs."}
- old joke
\end{quote}

Different people will have different answers to "What is the single most
useful "UNIX" command?" There certainly are many to consider. But I keep
coming back to \href{http://linux.die.net/man/1/find}{\texttt{find}}. It
can be intimidating to figure out from the documentation, especially at
first, but once you start mastering it, you end up using it over and
over again. \drfnd{find}{find files}

The main concepts of \texttt{find} are simple:

\begin{enumerate}
\def\labelenumi{\arabic{enumi}.}
\item
  Starting at location \emph{X}...
\item
  Recursively find all files or directories (or "file system entries" to
  be more precise) that successfully match one or more tests...
\item
  And for each match execute one or more actions.
\end{enumerate}

The simplest example is "starting in the current directory, recursively
list all files you find":

\drcap{Simplest \texttt{find} example}

\begin{Shaded}
\begin{Highlighting}[]
\KeywordTok{~} \NormalTok{$ }\KeywordTok{find}
\KeywordTok{.}
\KeywordTok{./Agenda.md}
\KeywordTok{./Bad} \NormalTok{and Corrupted Test Files}
\KeywordTok{./Bad} \NormalTok{and Corrupted Test Files/.DS_Store}
\KeywordTok{./Bad} \NormalTok{and Corrupted Test Files/2008 Letter of Understanding.TIF}
\KeywordTok{./Bad} \NormalTok{and Corrupted Test Files/3948175.dat}
\KeywordTok{./Bad} \NormalTok{and Corrupted Test Files/3948176.dat}
\KeywordTok{./Bad} \NormalTok{and Corrupted Test Files/3948178.dat}
\KeywordTok{./Bad} \NormalTok{and Corrupted Test Files/3948180.dat}
\KeywordTok{./Bad} \NormalTok{and Corrupted Test Files/3948182.dat}
\KeywordTok{./Bad} \NormalTok{and Corrupted Test Files/3948186.dat}
\KeywordTok{./Bad} \NormalTok{and Corrupted Test Files/3948190.dat}
\KeywordTok{./Bad} \NormalTok{and Corrupted Test Files/3948193.dat}
\KeywordTok{./Bad} \NormalTok{and Corrupted Test Files/3948195.dat}
\KeywordTok{./Bad} \NormalTok{and Corrupted Test Files/3948197.dat}
\KeywordTok{./Bad} \NormalTok{and Corrupted Test Files/3948259.dat}
\KeywordTok{...and} \NormalTok{so on...}
\end{Highlighting}
\end{Shaded}

In this case \texttt{find} is just shorthand for
\texttt{find\ .\ -true\ -print}.

That's not really that interesting. Let's poke around and "find" (pun
intended) some better examples of using \texttt{find}. It is better to
show than tell in this case. Let's dive into a semi-complicated one and
pick it apart:

\drcap{More complicated \texttt{find} example}

\begin{Shaded}
\begin{Highlighting}[]
\KeywordTok{~} \NormalTok{$ }\KeywordTok{find} \NormalTok{//myserver/myshare/logs/000[4-9] -name }\DataTypeTok{\textbackslash{}*}\NormalTok{.dat -newer logchecker.csv \textbackslash{}}
    \NormalTok{-exec /home/myuser/Sandbox/FileCheckers/logchecker }\DataTypeTok{\textbackslash{}\{\textbackslash{}\}} \DataTypeTok{\textbackslash{};}
\end{Highlighting}
\end{Shaded}

How does this all work? Remembering the three steps at the beginning:

\begin{enumerate}
\def\labelenumi{\arabic{enumi}.}
\item
  \textbf{Starting at location
  \texttt{//myserver/myshare/logs/000{[}4-9{]}}} - in this case a
  CIFS/SMB share running under Cygwin\footnote{In fact, \texttt{find} is
    one of the main reasons I use Cygwin on Windows.} (this won't work
  on Linux). Note the regular expression (which we will cover later), in
  this case saying to look only in directories \texttt{0004} through
  \texttt{0009}.
\item
  \textbf{Recursively find file system entries that match one or more
  tests} - the tests in this example are:

  \begin{enumerate}
  \def\labelenumii{\alph{enumii}.}
  \item
    \textbf{\emph{All files that have a name that ends in
    \texttt{.dat}}} - the only thing to note here is the
    \texttt{\textbackslash{}} preceding the wildcard \texttt{*}. This
    prevents "shell expansion," which would allow the \texttt{bash}
    process interpreting the command to expand it to the list of files
    present in the current directory only, not recursively across all
    directories.
  \item
    \textbf{\emph{That are newer (created or modified after) the file
    \texttt{logchecker.csv}}} - presumably this file gets created by
    running \texttt{logchecker} or some related process. This is an
    optimization condition check to only look at files that have been
    updated since the last time the script ran.
  \end{enumerate}
\item
  \textbf{For each match, execute \texttt{logchecker}} - and pass in the
  name of the currently found (matching) file.
\end{enumerate}

\section*{What's With the
Backslashes?}\label{whats-with-the-backslashes}
\addcontentsline{toc}{section}{What's With the Backslashes?}

Reconsider this example:

\drcap{More complicated \texttt{find} example, explained}

\begin{Shaded}
\begin{Highlighting}[]
\KeywordTok{~} \NormalTok{$ }\KeywordTok{find} \NormalTok{//myserver/myshare/logs/000[4-9] -name }\DataTypeTok{\textbackslash{}*}\NormalTok{.dat -newer logchecker.csv \textbackslash{}}
    \NormalTok{-exec /home/myuser/Sandbox/FileCheckers/logchecker }\DataTypeTok{\textbackslash{}\{\textbackslash{}\}} \DataTypeTok{\textbackslash{};}
\end{Highlighting}
\end{Shaded}

There are five (5) backslash (\texttt{\textbackslash{}}) characters in
the above. In each case, the backslash is preventing
\href{http://www.tldp.org/LDP/Bash-Beginners-Guide/html/sect_03_04.html}{shell
expansion}: \index{*@\texttt{\textbackslash{}} (escape character)}

\begin{enumerate}
\def\labelenumi{\arabic{enumi}.}
\item
  \textbf{\texttt{\textbackslash{}*.dat}} - preserves the \texttt{*} for
  \texttt{find} to use as it recursively searches through directories,
  instead of the shell expanding it to all files that end in
  \texttt{.dat} in the current directory.
  \index{*@\texttt{*} (wildcard)}
\item
  \textbf{\texttt{\textbackslash{}}} - the \texttt{\textbackslash{}} at
  the end of the first line tells the shell that the command continues
  on the next line.
\item
  \textbf{\texttt{\textbackslash{}\{\textbackslash{}\}\ \textbackslash{};}}
  - these three prevent the shell from trying to expand the braces into
  an environment variable or the semicolon (which is meant to tell
  \texttt{find} when the command being ran via \texttt{-exec} and its
  parameters end), otherwise \texttt{;} is normally used to separate
  independent commands on the same line in the shell.
  \index{*@\texttt{;} (command separator)}
\end{enumerate}

That last point bears repeating. Any time you \texttt{-exec} in a
\texttt{find} command (which will be a lot), just get used to typing
\texttt{\textbackslash{}\{\textbackslash{}\}\ \textbackslash{};} (the
space between the ending brace and the \texttt{\textbackslash{};} is
\textbf{\emph{required}}).

\section*{\texorpdfstring{Useful \texttt{find}
Options}{Useful find Options}}\label{useful-find-options}
\addcontentsline{toc}{section}{Useful \texttt{find} Options}

The \href{http://linux.die.net/man/1/find}{\texttt{find}} documentation
gives a bewildering number of options. Here are the ones you may "find"
the most useful: \drfnd{find}{find files}

\begin{itemize}
\item
  \textbf{\texttt{-executable}} - the file is executable or the
  directory is searchable (in other words, the file or directory's
  \texttt{x} mode bit is set true for user, group or other ("ugo"), per
  the file permissions discussion above), and the user executing the
  \texttt{find} command falls into one of the categories for which it is
  set.
\item
  \textbf{\texttt{-group\ \textless{}gname\textgreater{}}} - file
  belongs to group \emph{gname}.
\item
  \textbf{\texttt{-iname\ \textless{}pattern\textgreater{}}} -
  case-insensitive name search. Any wildcard characters should be
  escaped.
\item
  \textbf{\texttt{-maxdepth\ \textless{}number\textgreater{}}} - limits
  the number of directory levels to recurse into.
\item
  \textbf{\texttt{-mindepth\ \textless{}number\textgreater{}}} - sets a
  starting directory level below the current one to recurse into.
\item
  \textbf{\texttt{-name\ \textless{}pattern\textgreater{}}} -
  case-sensitive name search. Any wildcard characters should be escaped.
\item
  \textbf{\texttt{-newer\ \textless{}file\textgreater{}}} - each file is
  tested to see if it is newer than \emph{file}.
\item
  \textbf{\texttt{-size\ \textless{}n\textgreater{}}} - file uses
  \emph{n} units of space, which can be specified in various measures
  like 512-byte blocks (\texttt{b}) through gigabytes (\texttt{G}).
\item
  \textbf{\texttt{-type\ \textless{}c\textgreater{}}} - file is of type
  \emph{c}, with the two most common being \texttt{d} (directory) or
  \texttt{f} (file).
\item
  \textbf{\texttt{-user\ \textless{}uname\textgreater{}}} - file is
  owned by \emph{uname}.
\end{itemize}

\section*{\texorpdfstring{Useful \texttt{find}
Actions}{Useful find Actions}}\label{useful-find-actions}
\addcontentsline{toc}{section}{Useful \texttt{find} Actions}

Similarly, you are going to keep coming back to just a handful of
\texttt{find} actions:

\begin{itemize}
\item
  \textbf{\texttt{-delete}} - deletes any files matched so far. Note
  that actions are also tests (predicates), so as the \texttt{find}
  documentation says, "Don't forget that the find command line is
  evaluated as an expression, so putting \texttt{-delete} first will
  make find try to delete everything below the starting points you
  specified." In other words, placing \texttt{-delete} too early in the
  expression is going to yield behavior distressingly similar to
  \texttt{rm\ -r\ *}.
\item
  \textbf{\texttt{-exec} and \texttt{-execdir}} - executes a command or
  script, typically passing in the name of the file or directory found.
  You will use this \textbf{\emph{all}} the time. The difference between
  the two is that \texttt{-execdir} changes the working directory to
  that of the item found before invoking the program or script, whereas
  \texttt{-exec} simply passes in the fully-qualified path of the found
  item.
\item
  \textbf{\texttt{-print}} - prints the full path of the found file or
  directory. This is the default action.
\item
  \textbf{\texttt{-printf}} - prints a formatted string, useful for
  reports.
\end{itemize}

The \texttt{-printf} action allows you to do some interesting things
when producing output. For example, if for some reason we wanted a
report where for each file we wanted three lines with the name, owner
and created date and time in ISO 8601 format, all followed by a blank
line, we could use the following \texttt{find} command:

\drcap{Using \texttt{find} as a simple reporting tool}

\begin{Shaded}
\begin{Highlighting}[]
\KeywordTok{~} \NormalTok{$ }\KeywordTok{touch} \NormalTok{a b c}
\KeywordTok{~} \NormalTok{$ }\KeywordTok{ls} \NormalTok{-l}
\KeywordTok{total} \NormalTok{0}
\KeywordTok{-rw-rwxr--+} \NormalTok{1 myuser mygroup 0 Oct 21 11:02 a}
\KeywordTok{-rw-rwxr--+} \NormalTok{1 myuser mygroup 0 Oct 21 11:02 b}
\KeywordTok{-rw-rwxr--+} \NormalTok{1 myuser mygroup 0 Oct 21 11:02 c}
\KeywordTok{~} \NormalTok{$ }\KeywordTok{find} \NormalTok{. -type f -printf }\StringTok{"%p\textbackslash{}n%u\textbackslash{}n%TY-%Tm-%TdT%TT\textbackslash{}n\textbackslash{}n"}
\KeywordTok{./a}
\KeywordTok{myuser}
\KeywordTok{2015-10-21T11}\NormalTok{:02:51.7014527000}

\KeywordTok{./b}
\KeywordTok{myuser}
\KeywordTok{2015-10-21T11}\NormalTok{:02:51.7035423000}

\KeywordTok{./c}
\KeywordTok{myuser}
\KeywordTok{2015-10-21T11}\NormalTok{:02:51.7048997000}
\end{Highlighting}
\end{Shaded}

That \texttt{-printf} format string
\texttt{"\%p\textbackslash{}n\%u\textbackslash{}n\%TY-\%Tm-\%TdT\%TT\textbackslash{}n\textbackslash{}n"}
breaks down into: \index{*@\texttt{\%} (format)}

\begin{itemize}
\tightlist
\item
  \textbf{\texttt{"}} - prevent shell expansion on the format string.
\item
  \textbf{\texttt{\%p}} - file name.
\item
  \textbf{\texttt{\textbackslash{}n}} - new line.
\item
  \textbf{\texttt{\%u}} - owning user name.
\item
  \textbf{\texttt{\textbackslash{}n}} - new line.
\item
  \textbf{\texttt{\%TY}} - the last modification date of the file
  expressed as a year.
\item
  \textbf{\texttt{-}} - a literal hyphen.
\item
  \textbf{\texttt{\%Tm}} - the last modification date of the file
  expressed as a month.
\item
  \textbf{\texttt{-}} - a literal hyphen.
\item
  \textbf{\texttt{\%Td}} - the last modification date of the file
  expressed as a day.
\item
  \textbf{\texttt{T}} - a literal 'T'.
\item
  \textbf{\texttt{\%TT}} - the time expressed in
  \textbf{\emph{hh:mm:ss.hhhhhh}} format.
\item
  \textbf{\texttt{\textbackslash{}n\textbackslash{}n}} - two new lines.
\item
  \textbf{\texttt{"}} - prevent shell expansion on the format string.
\end{itemize}

\hypertarget{grokking-grep}{\chapter{\texorpdfstring{Grokking
\texttt{grep}}{Grokking grep}}\label{grokking-grep}}

\textbf{\emph{And probably gawking at \texttt{awk} while we are at it,
which means regular expressions, too. Now we have two problems.}}

\begin{quote}
\emph{"Some people, when confronted with a problem, think 'I know, I'll
use regular expressions.' Now they have two problems."} - Jamie Zawinski
\end{quote}

If the \texttt{file} command is useful for finding file system entries
based on their attributes, the
\href{http://linux.die.net/man/1/grep}{\texttt{grep}} command is good
for finding files whose \textbf{\emph{contents}} match a
\href{https://en.wikipedia.org/wiki/Regular_expression}{regular
expression}. You already know at least one regular expression, the
wildcard \texttt{*} character from the \texttt{CMD.EXE} prompt and
Windows Explorer. It means "match zero or more characters." We'll cover
more on regular expressions, or "regexes," in a moment.
\drfnd{grep}{search files} \drshl{CMD.EXE}
\index{*@\texttt{*} (match zero or more characters)}
\index{regular expressions}

First, an example of \texttt{grep}, showing all files in a directory
with the pattern "is" in them:

\drcap{\texttt{grep} example}

\begin{Shaded}
\begin{Highlighting}[]
\KeywordTok{~} \NormalTok{$ }\KeywordTok{touch} \NormalTok{a b c}
\KeywordTok{~} \NormalTok{$ }\KeywordTok{echo} \NormalTok{This sequence of characters is called a }\DataTypeTok{\textbackslash{}"}\NormalTok{string}\DataTypeTok{\textbackslash{}"}\NormalTok{. }\KeywordTok{>} \NormalTok{d}
\KeywordTok{~} \NormalTok{$ }\KeywordTok{cat} \NormalTok{d}
\KeywordTok{This} \NormalTok{sequence of characters is called a }\StringTok{"string"}\NormalTok{.}
\KeywordTok{~} \NormalTok{$ }\KeywordTok{ls}
\KeywordTok{a}  \NormalTok{b  c  d}
\KeywordTok{~} \NormalTok{$ }\KeywordTok{grep} \NormalTok{is *}
\KeywordTok{d}\NormalTok{:This sequence of characters is called a }\StringTok{"string"}\NormalTok{.}
\end{Highlighting}
\end{Shaded}

\section*{Expressing Yourself
Regularly}\label{expressing-yourself-regularly}
\addcontentsline{toc}{section}{Expressing Yourself Regularly}

So what are "regular expressions?" Simply, they are patterns for
matching "strings," which are sequences of "characters," e.g.:
\index{regular expressions}

\drcap{A string}

\begin{Shaded}
\begin{Highlighting}[]
\KeywordTok{This} \NormalTok{sequence of characters is called a }\StringTok{"string"}\NormalTok{.}
\end{Highlighting}
\end{Shaded}

That is a string. So is, "That is a string." And "That" and "T" and so
on. \textbf{\emph{In general}} (with many exceptions), the UNIX world
view is that everything is composed of text (or "strings"), and that
creating, changing, finding and passing around text is the primary mode
of operation.

In the \texttt{grep} example, we can see a regular expression can be as
simple as "is". It can also be as complicated as:

\drcap{Complex regular expression}

\begin{Shaded}
\begin{Highlighting}[]
\KeywordTok{(?bhttp}\NormalTok{://[-A-Za-z0-9+}\KeywordTok{&@}\NormalTok{#/%?=~_(}\KeywordTok{)|}\NormalTok{!}\KeywordTok{:}\NormalTok{,.}\KeywordTok{;}\NormalTok{]}\KeywordTok{*}\NormalTok{[-A-Za-z0-9+}\KeywordTok{&@f}
\end{Highlighting}
\end{Shaded}

That shows at least one attempt at being
\href{http://blog.codinghorror.com/the-problem-with-urls/}{a very
complete parser of valid HTTP URLs}. Wow! What \textbf{\emph{is}} all
that? Now you see why you have two problems. Even if you get that all
figured out, or if you actually sit and create something like that from
scratch yourself (and it works!), imagine coming back six months later
and trying to decipher it again.

There are literally \href{http://www.regular-expressions.info/}{whole
web sites} and books just on regular expressions. With variations they
are used in all "UNIX" shells, Perl, Python, Javascript, Java, C\# and
more. So obviously (a) they are really useful, and (b) we're not going
to cover all of regexes here.

There are so many things you can do, the only thing to remember is
"regular expressions" when you think "I need to find things based on a
pattern" and then research what it will take to define the pattern you
want.

In the mean time, following are a few \textbf{\emph{simple}} regex
examples. Consider the file \texttt{invoices}:

\drcap{Invoices file}

\begin{Shaded}
\begin{Highlighting}[]
\KeywordTok{~} \NormalTok{$ }\KeywordTok{cat} \NormalTok{invoices}
\KeywordTok{Combine} \NormalTok{brakes  400}
\KeywordTok{Combine} \NormalTok{motor   1500}
\KeywordTok{Combine} \NormalTok{tires   2500}
\KeywordTok{Tractor} \NormalTok{brakes  300}
\KeywordTok{Tractor} \NormalTok{motor   1000}
\KeywordTok{Tractor} \NormalTok{tires   2000}
\KeywordTok{Truck}   \NormalTok{brakes  200}
\KeywordTok{Truck}   \NormalTok{tires   400}
\KeywordTok{Truck}   \NormalTok{tires   400}
\KeywordTok{Truck}   \NormalTok{tires   400}
\KeywordTok{Truck}   \NormalTok{winch   100}
\end{Highlighting}
\end{Shaded}

Let's find all lines with "tractor":

\drcap{Trying to find tractors}

\begin{Shaded}
\begin{Highlighting}[]
\KeywordTok{~} \NormalTok{$ }\KeywordTok{grep} \NormalTok{tractor invoices}
\end{Highlighting}
\end{Shaded}

Huh, nothing was found. But this is UNIX-land, so we know it is
sensitive - about case anyway:

\drcap{Trying to find tractors, part two}

\begin{Shaded}
\begin{Highlighting}[]
\KeywordTok{~} \NormalTok{$ }\KeywordTok{grep} \NormalTok{Tractor invoices}
\KeywordTok{Tractor} \NormalTok{brakes  300}
\KeywordTok{Tractor} \NormalTok{motor   1000}
\KeywordTok{Tractor} \NormalTok{tires   2000}
\end{Highlighting}
\end{Shaded}

Or we could just tell \texttt{grep} we are insensitive (to case,
anyway):

\drcap{Let's be insensitive}

\begin{Shaded}
\begin{Highlighting}[]
\KeywordTok{~} \NormalTok{$ }\KeywordTok{grep} \NormalTok{-i tractor invoices}
\KeywordTok{Tractor} \NormalTok{brakes  300}
\KeywordTok{Tractor} \NormalTok{motor   1000}
\KeywordTok{Tractor} \NormalTok{tires   2000}
\end{Highlighting}
\end{Shaded}

And just to remind you about long-style parameters:

\drcap{Spelling out our insensitivity}

\begin{Shaded}
\begin{Highlighting}[]
\KeywordTok{~} \NormalTok{$ }\KeywordTok{grep} \NormalTok{--ignore-case tractor invoices}
\KeywordTok{Tractor} \NormalTok{brakes  300}
\KeywordTok{Tractor} \NormalTok{motor   1000}
\KeywordTok{Tractor} \NormalTok{tires   2000}
\end{Highlighting}
\end{Shaded}

But what \textbf{\emph{lines}} are those on?

\drcap{Print the line numbers of matches}

\begin{Shaded}
\begin{Highlighting}[]
\KeywordTok{~} \NormalTok{$ }\KeywordTok{grep} \NormalTok{-i -n tractor invoices}
\KeywordTok{1}\NormalTok{:Tractor       motor   1000}
\KeywordTok{2}\NormalTok{:Tractor       brakes  300}
\KeywordTok{3}\NormalTok{:Tractor       tires   2000}
\end{Highlighting}
\end{Shaded}

To get more complicated, we can pass the \texttt{-E} parameter (for
\emph{extended} regular expressions) and start doing some really fun
stuff. Let's look for lines with either "Tractor" or "Truck":

\drcap{Extended regular expressions}

\begin{Shaded}
\begin{Highlighting}[]
\KeywordTok{~} \NormalTok{$ }\KeywordTok{grep} \NormalTok{-E }\StringTok{"Tractor|Truck"} \NormalTok{invoices}
\KeywordTok{Tractor} \NormalTok{brakes  300}
\KeywordTok{Tractor} \NormalTok{motor   1000}
\KeywordTok{Tractor} \NormalTok{tires   2000}
\KeywordTok{Truck}   \NormalTok{brakes  200}
\KeywordTok{Truck}   \NormalTok{tires   400}
\KeywordTok{Truck}   \NormalTok{tires   400}
\KeywordTok{Truck}   \NormalTok{tires   400}
\KeywordTok{Truck}   \NormalTok{winch   100}
\end{Highlighting}
\end{Shaded}

For me, the following keep coming up when using regular expressions:

\begin{itemize}
\item
  \textbf{\texttt{one\textbar{}other}} - find \texttt{one} pattern or
  the \texttt{other}.
  \index{*@\texttt{"|} (match zero or more characters)}
  \index{regular expressions!\texttt{"|} (or)}
\item
  \textbf{\texttt{\^{}}} - pattern for the beginning of a line.
  \index{*@\texttt{\^{}} (beginning of line)}
  \index{regular expressions!\texttt{\^{}} (beginning of line)}
\item
  \textbf{\texttt{\$}} - pattern for the end of a line.
  \index{*@\texttt{\$} (end of line)}
  \index{regular expressions!\texttt{\$} (end of line)}
\item
  \textbf{\texttt{?}} - match exactly one character.
  \index{*@\texttt{?} (match one character)}
  \index{regular expressions!\texttt{?} (match one character)}
\item
  \textbf{\texttt{*}} - match zero or more characters.
  \index{*@\texttt{*} (match zero or more characters)}
  \index{regular expressions!\texttt{*} (match zero or more characters)}
\item
  \textbf{\texttt{+}} - match one or more characters.
  \index{*@\texttt{+} (match one or more characters)}
  \index{regular expressions!\texttt{+} (match one or more characters)}
\item
  \textbf{\texttt{{[}A-Z{]}}} - match any character in a range (in this
  case any uppercase Latin alphabetic character).
  \index{*@\texttt{[A-Z]} (match a character in range)}
  \index{regular expressions!\texttt{[A-Z]} (match a character in range)}
\item
  \textbf{\texttt{{[}n\textbar{}y{]}}} - match one character or another
  (such as \texttt{n} or \texttt{y} here).
  \index{*@\texttt{[n"|y]} (match one character or other)}
  \index{regular expressions!\texttt{[n"|y]} (match one character or other)}
\end{itemize}

For example, to find the lines that end in \texttt{400}:

\drcap{Find lines ending with 400}

\begin{Shaded}
\begin{Highlighting}[]
\NormalTok{$ }\KeywordTok{grep}  \NormalTok{-E }\StringTok{"^*400$"} \NormalTok{invoices}
\KeywordTok{Combine} \NormalTok{brakes  400}
\KeywordTok{Truck}   \NormalTok{tires   400}
\KeywordTok{Truck}   \NormalTok{tires   400}
\KeywordTok{Truck}   \NormalTok{tires   400}
\end{Highlighting}
\end{Shaded}

\section{\texorpdfstring{Groveling With
\texttt{grep}}{Groveling With grep}}\label{groveling-with-grep}

To recursively find all files that contain the string "pdfinfo":

\drcap{Recursive \texttt{grep}}

\begin{Shaded}
\begin{Highlighting}[]
\KeywordTok{~} \NormalTok{$ }\KeywordTok{grep} \NormalTok{-R -i pdfinfo *}
\KeywordTok{./FileCheckers}\NormalTok{/otschecker:# }\KeywordTok{pdfinfo}\NormalTok{, too. If pdfinfo thinks it}\StringTok{'s junk, ...}
\StringTok{./FileCheckers/otschecker:        pdfinfo=`pdfinfo -opw foo "$1" 2>&1 1...}
\StringTok{./FileCheckers/otschecker:        if [ $rc != 0 -a "$pdfinfo" != "Comma...}
\StringTok{./FileCheckers/pdfchecker:        # pdfinfo, too. If pdfinfo thinks it'}\NormalTok{...}
\KeywordTok{./FileCheckers}\NormalTok{/pdfchecker:                }\OtherTok{pdfinfo=}\KeywordTok{`pdfinfo} \StringTok{"}\OtherTok{$1}\StringTok{"} \KeywordTok{>} \NormalTok{/dev/...}
\KeywordTok{./FileCheckers}\NormalTok{/pdfpwdchecker:# }\KeywordTok{pdfinfo}\NormalTok{, too. If pdfinfo thinks it}\StringTok{'s jun...}
\StringTok{./FileCheckers/pdfpwdchecker:        pdfinfo=`pdfinfo -opw foo "$1" 2>&...}
\StringTok{./FileCheckers/pdfpwdchecker:        if [ $rc != 0 -a "$pdfinfo" = "Com...}
\StringTok{./FileCheckers/README.md:* ***[pdfinfo(1)](http://linux.die.net/man/1/p...}
\end{Highlighting}
\end{Shaded}

The above is functionally equivalent but \textbf{\emph{much}} quicker
than:

\drcap{Recursive \texttt{grep} is faster than \texttt{find ... -exec grep}}

\begin{Shaded}
\begin{Highlighting}[]
\KeywordTok{~} \NormalTok{$ }\KeywordTok{find} \NormalTok{. -type f -exec grep -H -i pdfinfo }\DataTypeTok{\textbackslash{}\{\textbackslash{}\}} \DataTypeTok{\textbackslash{};} 
\end{Highlighting}
\end{Shaded}

\textbf{Note:} In general, if a command has its own "recursive" option
(such as \texttt{-R} with \texttt{grep}), it is quicker to use that
rather than to invoke the command repeatedly using \texttt{find}
instead. \drfnd{find}{find files}

However, sometimes you can use \texttt{find} to filter down files to be
checked before having \texttt{grep} read through them, and have that
result in much quicker results.

For example, if you only wanted to check files that contain "pdfinfo"
that have been created or modified since the last time you checked, it
\textbf{\emph{could be}} quicker to run something like:

\drcap{A better example of when to use \texttt{find ... -exec grep}}

\begin{Shaded}
\begin{Highlighting}[]
\KeywordTok{~} \NormalTok{$ }\KeywordTok{find} \NormalTok{. ! -name pdfinfo.log -newer pdfinfo.log -type f -exec grep -H \textbackslash{}}
    \NormalTok{-i pdfinfo }\DataTypeTok{\textbackslash{}\{\textbackslash{}\}} \DataTypeTok{\textbackslash{};} \KeywordTok{>} \NormalTok{pdfinfo.log}
\end{Highlighting}
\end{Shaded}

This says to ignore files named \texttt{pdfinfo.log}
(\texttt{!\ -name\ pdfinfo.log}) and otherwise look for files
(\texttt{-type\ f}) containing "pdfinfo"
(\texttt{-exec\ grep\ -H\ -i\ pdfinfo}) that haven't been checked since
the last time \texttt{pdfinfo.log} was modified
(\texttt{-newer\ pdfinfo.log}). In my tests the first run (which
initially creates the \texttt{pdfinfo.log} file) ran in 30 seconds but
subsequents runs took just a few seconds. This was because the number of
files to be searched through all directories was big enough it paid to
pre-filter the results with \texttt{find} before handing them to
\texttt{grep}.

\section{\texorpdfstring{Gawking at
\texttt{awk}}{Gawking at awk}}\label{gawking-at-awk}

I don't have much to say about
\href{http://linux.die.net/man/1/awk}{\texttt{awk}} other than:
\drscr{awk}

\begin{enumerate}
\def\labelenumi{\arabic{enumi}.}
\item
  It is named after its three authors,
  \href{https://en.wikipedia.org/wiki/AWK}{Aho, Weinberger and
  Kernighan}, all three of whom are computer science greats from Bell
  Labs. The GNU version is called \texttt{gawk}, of course!
\item
  It is a "data driven scripting language." That's a fancy way of saying
  it was written specifically with slicing and dicing text in mind.
\item
  It generally is broken out when the typical "UNIX" commands and shell
  features like pipes and redirection aren't enough.
\item
  Usually, if I start thinking of \texttt{awk}, I start thinking of a
  way to program the answer in another language such as Python, or
  reframe the question to get an answer not requiring \texttt{awk}.
\end{enumerate}

That said, it is a powerful knife in the tool belt, and you should be
aware it exists. If you are searching the internet and find an answer
using \texttt{awk} that you can \textbf{\emph{quickly}} adapt to your
needs, use it.

To whet your taste, here is the type of "one-liner" for which
\texttt{awk} is famous, in this case
\href{http://www.ibm.com/developerworks/library/l-awk1/}{formatting and
printing a report on user ids} from \texttt{/etc/passwd}:

\drcap{awk example}

\begin{Shaded}
\begin{Highlighting}[]
\KeywordTok{~} \NormalTok{$ }\KeywordTok{awk} \NormalTok{-F}\StringTok{":"} \StringTok{'\{ print "username: " $1 "\textbackslash{}t\textbackslash{}tuid:" $3 \}'} \NormalTok{/etc/passwd}
\KeywordTok{username}\NormalTok{: root      uid:0}
\KeywordTok{username}\NormalTok{: daemon    uid:1}
\KeywordTok{username}\NormalTok{: bin       uid:2}
\KeywordTok{username}\NormalTok{: sys       uid:3}
\KeywordTok{username}\NormalTok{: sync      uid:4}
\KeywordTok{username}\NormalTok{: games     uid:5}
\KeywordTok{username}\NormalTok{: man       uid:6}
\KeywordTok{username}\NormalTok{: lp        uid:7}
\KeywordTok{username}\NormalTok{: mail      uid:8}
\KeywordTok{username}\NormalTok{: news      uid:9}
\KeywordTok{username}\NormalTok{: uucp      uid:10}
\KeywordTok{...and} \NormalTok{so on...}
\end{Highlighting}
\end{Shaded}

\hypertarget{just-a-series-of-pipes}{\chapter{“Just a Series of
Pipes”}\label{just-a-series-of-pipes}}

\textbf{\emph{\texttt{stdin}/\texttt{stdout}/\texttt{stderr}, redirects
and piping between commands.}}

\begin{quote}
\emph{"Ceci n'est pas une pipe."} - René Magritte
\end{quote}

The \href{https://en.wikipedia.org/wiki/Unix_philosophy}{"UNIX
philosophy"} tends to be to have a bunch of small programs that each do
one thing very well, and then to combine them together in interesting
ways. The "glue" for combining them together is often the "piping" or
redirection of "streams" of data (typically text) between programs, each
doing one small change to the stream until it is finally emitted on the
console or saved to a file or sent over the Internet.

The first thing to note is there are three "file I/O streams" that are
open by default in every "UNIX" process:

\begin{itemize}
\item
  \textbf{\emph{stdin}} - input, typically from the console in an
  interactive session. In the underlying C file system APIs, this is
  file descriptor 0. \drstd{stdin}
\item
  \textbf{\emph{stdout}} - "normal" output, typically to the console in
  an interactive session. This is file descriptor 1. \drstd{stdout}
\item
  \textbf{\emph{stderr}} - "error" output, typically to the console in
  an interactive session (so it can be hard to distinguish when
  intermingled with \emph{stdout} output). This is file descriptor 2.
  \drstd{stderr}
\end{itemize}

\textbf{Note:} Those numeric file descriptors will go from being trivia
to important in just a bit.

When a program written in C calls \texttt{printf}, it is writing to
\emph{stdout}. When a \texttt{bash} script calls \texttt{echo}, it too
is writing to \emph{stdout}. When a command writes an error message, it
is writing to \emph{stderr}. If a command or program accepts input from
the console, it is reading from \emph{stdin}. \drtxt{echo}
\drstd{stdout}

In this example, \texttt{cat} is started with no file name, so it will
read from \emph{stdin} (a quite common "UNIX" command convention), and
echo each line typed by the user to \emph{stdout} until the "end of
file," which in an interactive session can be emulated with
\texttt{Ctrl-D}, shown as \texttt{\^{}D} in the example below but not
seen on the console in real life: \drtxt{cat}

\drcap{\textit{stdin} and \textit{stdout}}

\begin{Shaded}
\begin{Highlighting}[]
\KeywordTok{~} \NormalTok{$ }\KeywordTok{cat}
\KeywordTok{This} \NormalTok{shows reading from stdin}
\KeywordTok{This} \NormalTok{shows reading from stdin}
\KeywordTok{and} \NormalTok{writing to stdout.}
\KeywordTok{and} \NormalTok{writing to stdout.}
\NormalTok{^}\KeywordTok{D}
\end{Highlighting}
\end{Shaded}

In the above I typed in "This shows reading from stdin" and hit
\texttt{Enter} (which send a linefeed and hence marks the "end of the
line") and \texttt{cat} echoed that line to \emph{stdout}. Then I typed
"and writing to stdout." and hit \texttt{Enter} and that line was echoed
to \emph{stdout} as well. Finally I hit \texttt{Ctrl-D}, which ended the
process. \drstd{stdout}

\section*{All Magic is Redirection}\label{all-magic-is-redirection}
\addcontentsline{toc}{section}{All Magic is Redirection}

One way to string things together in "the UNIX way" is with file
redirection. This is a concept that also works in \texttt{CMD.EXE} and
even with the same syntax. \drshl{CMD.EXE}

Let's create a file with a single line of text in it. One way would be
to \texttt{vi\ newfilename}, edit the file, save it, and exit
\texttt{vi}. A quicker way is to simply use file redirection: \dreds{vi}

\drcap{Hello, world}

\begin{Shaded}
\begin{Highlighting}[]
\KeywordTok{~} \NormalTok{$ }\KeywordTok{echo} \NormalTok{Hello, world }\KeywordTok{>} \NormalTok{hw}
\KeywordTok{~} \NormalTok{$ }\KeywordTok{ls} \NormalTok{-l}
\KeywordTok{total} \NormalTok{1}
\KeywordTok{-rw-rwxr--+} \NormalTok{1 myuser mygroup 13 Oct 22 10:40 hw}
\KeywordTok{~} \NormalTok{$ }\KeywordTok{cat} \NormalTok{hw}
\KeywordTok{Hello}\NormalTok{, world}
\end{Highlighting}
\end{Shaded}

In this case the \texttt{\textgreater{}\ hw} tells \texttt{bash} to take
the output that \texttt{echo} sends to \emph{stdout} and send it to the
file \texttt{hw} instead. \drstd{stdout}
\index{*@\texttt{>} (output redirection)}
\index{I/O!redirection!output (\texttt{>})}

As mentioned above many "UNIX" commands are set up to take one or more
file names from the command line as parameters, and if there aren't any,
to read from \emph{stdin}. The \texttt{cat} command does that. While it
doesn't save us anything over the above example, the following example
using \texttt{\textless{}} is illustrative of redirecting a file to
\emph{stdin} for a command or program: \drstd{stdin}
\index{*@\texttt{<} (input redirection)}
\index{I/O!redirection!input (\texttt{<})}

\drcap{Redundant redirection}

\begin{Shaded}
\begin{Highlighting}[]
\KeywordTok{~} \NormalTok{$ }\KeywordTok{cat} \KeywordTok{<} \NormalTok{hw}
\KeywordTok{Hello}\NormalTok{, world}
\end{Highlighting}
\end{Shaded}

Finally, we need to deal with \emph{stderr}. By convention it is sent to
the console just like \emph{stdout}, and that can make output confusing:
\drstd{stderr}

\drcap{Default \textit{stderr} behavior}

\begin{Shaded}
\begin{Highlighting}[]
\KeywordTok{~} \NormalTok{$ }\KeywordTok{echo} \NormalTok{This is a }\KeywordTok{>} \NormalTok{a}
\KeywordTok{~} \NormalTok{$ }\KeywordTok{echo} \NormalTok{This is b }\KeywordTok{>} \NormalTok{b}
\KeywordTok{~} \NormalTok{$ }\KeywordTok{echo} \NormalTok{This is c }\KeywordTok{>} \NormalTok{c}
\KeywordTok{~} \NormalTok{$ }\KeywordTok{mkdir} \NormalTok{d}
\KeywordTok{~} \NormalTok{$ }\KeywordTok{echo} \NormalTok{This is e }\KeywordTok{>} \NormalTok{d/e}
\KeywordTok{~} \NormalTok{$ }\KeywordTok{find} \NormalTok{. -exec cat }\DataTypeTok{\textbackslash{}\{\textbackslash{}\}} \DataTypeTok{\textbackslash{};}
\KeywordTok{cat}\NormalTok{: .: Is a directory}
\KeywordTok{This} \NormalTok{is a}
\KeywordTok{This} \NormalTok{is b}
\KeywordTok{This} \NormalTok{is c}
\KeywordTok{cat}\NormalTok{: ./d: Is a directory}
\KeywordTok{This} \NormalTok{is e}
\end{Highlighting}
\end{Shaded}

In the above, between echoing the contents of the \texttt{a},
\texttt{b}, \texttt{c} and \texttt{e} files, we see two error messages
from \texttt{cat} complaining that \texttt{.} and \texttt{d} are
directories. These are being emitted on \emph{stderr}, but there is no
good way of visually telling that. One way to get rid of them would be
to change \texttt{find} to filter for only files:

\drcap{Get rid of the errors in the first place}

\begin{Shaded}
\begin{Highlighting}[]
\KeywordTok{~} \NormalTok{$ }\KeywordTok{find} \NormalTok{. -type f -exec cat }\DataTypeTok{\textbackslash{}\{\textbackslash{}\}} \DataTypeTok{\textbackslash{};}
\KeywordTok{This} \NormalTok{is a}
\KeywordTok{This} \NormalTok{is b}
\KeywordTok{This} \NormalTok{is c}
\KeywordTok{This} \NormalTok{is e}
\end{Highlighting}
\end{Shaded}

But let's say the example is not so trivial, and we want to capture and
log the error messages separately for later analysis. While we've seen
\texttt{\textless{}} used to represent redirecting \emph{stdin} and
\texttt{\textgreater{}} used for redirecting \emph{stdout}, how do we
tell the shell we want to redirect \emph{stderr}? Remember the
discussion about file handles above? That's where those esoteric numbers
come in handy! To redirect \emph{stderr} we recall it is
\textbf{\emph{always}} file descriptor 2, and then we can use:
\drstd{stderr}

\drcap{Redirecting \textit{stderr}}

\begin{Shaded}
\begin{Highlighting}[]
\KeywordTok{~} \NormalTok{$ }\KeywordTok{find} \NormalTok{. -exec cat }\DataTypeTok{\textbackslash{}\{\textbackslash{}\}} \DataTypeTok{\textbackslash{};} \KeywordTok{2>}\NormalTok{/tmp/finderrors.log}
\KeywordTok{This} \NormalTok{is a}
\KeywordTok{This} \NormalTok{is b}
\KeywordTok{This} \NormalTok{is c}
\KeywordTok{This} \NormalTok{is e}
\KeywordTok{~} \NormalTok{$ }\KeywordTok{cat} \NormalTok{/tmp/finderrors.log}
\KeywordTok{cat}\NormalTok{: .: Is a directory}
\KeywordTok{cat}\NormalTok{: ./d: Is a directory}
\end{Highlighting}
\end{Shaded}

The \texttt{2\textgreater{}/tmp/finderrors.log} is the magic that is
redirecting file descriptor 2 (\emph{stderr}) to the log file
\texttt{/tmp/finderrors.log}.
\index{*@\texttt{2>} (\textit{stderr} redirection)}
\index{I/O!redirection!error (\texttt{2>})}

A very common paradigm is to capture both \emph{stdout} and
\emph{stderr} to the same file. Here is how that is done, again using
file descriptors: \drstd{stderr} \drstd{stdout}

\drcap{Redirecting both \textit{stdout} and \textit{stderr} to a file}

\begin{Shaded}
\begin{Highlighting}[]
\KeywordTok{~} \NormalTok{$ }\KeywordTok{find} \NormalTok{. -exec cat }\DataTypeTok{\textbackslash{}\{\textbackslash{}\}} \DataTypeTok{\textbackslash{};} \KeywordTok{>}\NormalTok{/tmp/find.log }\KeywordTok{2>&1}
\KeywordTok{~} \NormalTok{$ }\KeywordTok{cat} \NormalTok{/tmp/find.log}
\KeywordTok{cat}\NormalTok{: .: Is a directory}
\KeywordTok{This} \NormalTok{is a}
\KeywordTok{This} \NormalTok{is b}
\KeywordTok{This} \NormalTok{is c}
\KeywordTok{cat}\NormalTok{: ./d: Is a directory}
\KeywordTok{This} \NormalTok{is e}
\end{Highlighting}
\end{Shaded}

Now we see \emph{stdout} being redirected to \texttt{/tmp/find.log} with
\texttt{\textgreater{}/tmp/find.log}, and \emph{stderr} (file descriptor
2) being sent to the same place as \emph{stdout} (file descriptor 1)
with \texttt{2\textgreater{}\&1}. Note that this works in
\texttt{CMD.EXE}, too! \drshl{CMD.EXE}

If we want to send \emph{stdout} to one file and \emph{stderr} to
another, you can do it like this:

\drcap{Redirecting \textit{stdout} one way \textit{stderr} another}

\begin{verbatim}
~ $ find . -exec cat \{\} \; >/tmp/find.log 2>/tmp/finderrors.log
~ $ cat /tmp/find.log
This is a
This is b
This is c
This is e
~ $ cat /tmp/finderrors.log
cat: .: Is a directory
cat: ./d: Is a directory
\end{verbatim}

One final note with redirection is the difference between creating or
re-writing a file versus appending. The following creates a new
\texttt{/tmp/find.log} file every time it runs (there is no need to
\texttt{rm} it first):

\drcap{Overwriting a file with redirection}

\begin{Shaded}
\begin{Highlighting}[]
\KeywordTok{~} \NormalTok{$ }\KeywordTok{find} \NormalTok{. -exec cat }\DataTypeTok{\textbackslash{}\{\textbackslash{}\}} \DataTypeTok{\textbackslash{};} \KeywordTok{>}\NormalTok{/tmp/find.log}
\end{Highlighting}
\end{Shaded}

However, the next sample using \texttt{\textgreater{}\textgreater{}}
creates a new \texttt{/tmp/find.log} file if it doesn't exist, but
otherwise appends to it:
\index{*@\texttt{>>} (output redirection, appending)}
\index{I/O!redirection!output, appending (\texttt{>>})}

\drcap{Appending to a file with redirection}

\begin{Shaded}
\begin{Highlighting}[]
\KeywordTok{~} \NormalTok{$ }\KeywordTok{find} \NormalTok{. -exec cat }\DataTypeTok{\textbackslash{}\{\textbackslash{}\}} \DataTypeTok{\textbackslash{};} \KeywordTok{>>}\NormalTok{/tmp/find.log}
\end{Highlighting}
\end{Shaded}

\textbf{Note:} There is also a variation on input redirection using
\texttt{\textless{}\textless{}}, but it is used mostly in scripting and
is outside the scope of this book.

\section*{Everyone Line Up}\label{everyone-line-up}
\addcontentsline{toc}{section}{Everyone Line Up}

So we can see that we could pass things between programs by redirecting
\emph{stdout} to a file and then redirecting that file to \emph{stdin}
on the next program, and so on. But "UNIX" environments take it a bit
further with the concept of a command "pipeline" that allows directly
sending \emph{stdout} from one program into \emph{stdin} of another
using the "pipe" (\texttt{\textbar{}}):
\index{*@\texttt{"|} (pipe)}\index{I/O!redirection!pipe (\texttt{"|})}
\drstd{stdin} \drstd{stdout}

\drcap{Piping output between programs}

\begin{Shaded}
\begin{Highlighting}[]
\KeywordTok{~} \NormalTok{$ }\KeywordTok{cat} \NormalTok{*.txt }\KeywordTok{|} \KeywordTok{tr} \StringTok{'\textbackslash{}\textbackslash{}'} \StringTok{'/'} \KeywordTok{|} \KeywordTok{while} \KeywordTok{read} \OtherTok{line} \NormalTok{; }\KeywordTok{do} \KeywordTok{./mycmd} \StringTok{"}\OtherTok{$line}\StringTok{"} \KeywordTok{;} \KeywordTok{done}
\end{Highlighting}
\end{Shaded}

This little one-liner starts showing off the usefulness of chaining
several small programs, each doing one thing. In this case:

\begin{enumerate}
\def\labelenumi{\arabic{enumi}.}
\item
  \texttt{cat} echos the contents of all \texttt{.txt} files in
  alphabetical order by their file name to \emph{stdout}, which is piped
  to...
\item
  \href{http://linux.die.net/man/1/tr}{\texttt{tr}} "translates"
  (replaces) any backslash characters (here "escaped" as
  \texttt{\textquotesingle{}\textbackslash{}\textbackslash{}\textquotesingle{}}
  because the backslash character is a special character) to forward
  slashes (\texttt{/}) before sending it into... \drtxt{tr}
\item
  A \texttt{while} loop that reads each line into a variable called
  \texttt{\$line} and then calls... \drscr{while}
\item
  Some custom script or program called \texttt{./mycmd} passing in the
  value of each \texttt{\$line}.
\end{enumerate}

Think about the power of that. \texttt{cat} didn't know there were
multiple \texttt{.txt} files or not - the shell expansion of the
\texttt{*.txt} wildcard did that. It read all those files and echoed
them to \emph{stdout} which in this case was a pipeline sending each
line in order to another command to transform the data, before sending
each line to the custom code in \texttt{mycmd}, that only expects a
single line or value each time it is run. It has no idea about the
\texttt{.txt} files or the transformation or the pipeline!

\textbf{\emph{That}} is the "UNIX philosophy" at work.

There are some nice performance benefits for this approach, too. In
general Linux \& Co. will overlap the processing by starting all the
commands in the pipeline, with the ones on the right getting data from
the ones further "upstream" to the left as soon as it is written,
instead of using file redirection where one program would have to finish
completely running and writing out to a file before the next program
could start and read in that file as input.

Finally, if you want to capture something to a file \textbf{\emph{and}}
see it on the console at the same time, that is where the
\href{http://linux.die.net/man/1/tee}{\texttt{tee}} command comes in:
\drtxt{tee}

\begin{Shaded}
\begin{Highlighting}[]
\KeywordTok{~} \NormalTok{$ }\KeywordTok{find} \NormalTok{. -name error.log }\KeywordTok{|} \KeywordTok{tee} \KeywordTok{>} \NormalTok{errorlogs.txt}
\end{Highlighting}
\end{Shaded}

This would write the results of finding all files names
\texttt{error.log} to the console and also to \texttt{errorlogs.txt}.
This is useful when you are manually running things and want to see the
results immediately, but also want a log of what you did.

\hypertarget{vi}{\chapter{\texorpdfstring{\texttt{vi}}{vi}}\label{vi}}

\textbf{\emph{How to stay sane for 10 minutes in \texttt{vi}.
Navigation, basic editing, find, change/change-all, cut and paste, undo,
saving and canceling. Plus easier alternatives like \texttt{nano}, and
why \texttt{vi} still matters.}}

\begin{quote}
\emph{"You're too young to know."} - Vi (\emph{Grease})
\end{quote}

\href{http://linux.die.net/man/1/vi}{\texttt{vi}} stands for
\emph{visual editor} (as well as the Roman numeral for 6, which is why
it is this chapter), and once you use it you will understand what
editing from the command line must've been like for \texttt{vi} to seem
both "visual" and a step forward. \dreds{vi}
\index{commands!vi@\texttt{vi}|seealso{\texttt{vi} Commands}}

Many Linux clones don't use \texttt{vi} proper, but a port called
\href{http://www.vim.org/}{\texttt{vim}} ("\textbf{vi}
i\textbf{m}proved"), that is then accessed via the alias \texttt{vi}.
The differences tend to be minor, with \texttt{vim} being more
customizable. \dreds{vim}

\texttt{vi} and a similar editor,
\href{http://linux.die.net/man/1/emacs}{\texttt{emacs}}, both tend to
trip up users from GUI operating systems such as Windows or OS X that
have editors like Notepad that are always ready for user input.
\dreds{emacs}

Instead, \texttt{vi} typically starts in "command mode," where
keystrokes execute various navigation and editing commands. To actually
insert text requires a keystroke such as \texttt{i} while in command
mode, which then causes \texttt{vi} to go into "insert mode." Insert
mode is what most Windows users expect from an editor, i.e., when you
type the line changes. The \texttt{ESC} key exits insert mode.
\drvic{ESC}{exit insert mode} \drvic{i}{insert mode}

It is as hard to get used to as it sounds, and you \textbf{\emph{will}}
execute text you were meaning to insert as commands, and commands that
you were meaning to execute you \textbf{\emph{will}} insert as text, and
sooner or later you \textbf{\emph{will}} enter \texttt{vi} commands into
Notepad. I guarantee it. That will be the day you know you've become
truly tainted.

We will not even begin to scratch the surface of \texttt{vi}, when there
are many books and web sites just on wielding it to its full potential.
In the hands of someone who has mastered it, \texttt{vi} can do some
really remarkable feats of editing way beyond the capability of most
modern GUI programming environments.

\section*{Command Me}\label{command-me}
\addcontentsline{toc}{section}{Command Me}

Again, when you first open \texttt{vi} it is in "command mode." That
means any keystrokes you enter will "do something." The "something" to
be done may be navigating around the file, inserting, deleting or
changing text, manipulating lines, "undoing" or "redoing," writing the
changes to disk and the like.

What are commands? Well, for example \texttt{d} means "delete." We'll
talk about how to specify \textbf{\emph{what}} to delete next.
\texttt{i} tells \texttt{vi} to enter "insert mode" at the point where
the cursor is. \texttt{0} (zero) navigates to the start of the current
line, and so on. \drvic{d}{delete} \drvic{i}{insert mode}
\drvic{0}{jump to beginning of line}

Commands can have \textbf{\emph{modifiers}} preceding and following
them. Consider the "delete" command, \texttt{d}. If we follow with
\texttt{w} as in \texttt{dw} while in command mode, it will delete a
whitespace-delimited "word" starting at where the cursor is through
(including) the next whitespace character. \drvic{dw}{delete word}

If the \texttt{\textbar{}} in the following represents the cursor:

\drcap{Deleting a "word"}

\begin{Shaded}
\begin{Highlighting}[]
\KeywordTok{This} \NormalTok{is a wo}\KeywordTok{|rd} \NormalTok{and so is this.}
\end{Highlighting}
\end{Shaded}

Then typing \texttt{dw} will delete from the cursor position the
characters \texttt{r}, \texttt{d} and the space, leaving the following:

\drcap{After deleting the "word"}

\begin{Shaded}
\begin{Highlighting}[]
\KeywordTok{This} \NormalTok{is a wo}\KeywordTok{|and} \NormalTok{so is this.}
\end{Highlighting}
\end{Shaded}

We can also specify the number of times we want to perform a command by
prefixing it to the command. So now if we wanted to delete three words
from the cursor position in the above, we'd use \texttt{3dw} and end up
with:

\drcap{Deleting multiple words}

\begin{Shaded}
\begin{Highlighting}[]
\KeywordTok{This} \NormalTok{is a wo}\KeywordTok{|this.}
\end{Highlighting}
\end{Shaded}

Again, in all these examples the \texttt{\textbar{}} represents the
cursor.

There is a little bit of nuance in using command modifiers. Consider the
\texttt{r} (\emph{replace}) command. It is typically used to change the
single \textbf{\emph{character}} under the cursor. You may be tempted to
think you can do something like \texttt{rw} for "replace word," but it
is actually going to simply replace the current character with a
\texttt{w}, whereas the real command for doing that is \texttt{cw}
("change word"). In addition, you can use repeaters as above, just be
sure you understand \texttt{r} means "replace a single character," so
\texttt{3rx} executed on: \drvic{c}{change} \drvic{cw}{change word}
\drvic{r}{replace character}

\drcap{Replace three characters with "x"}

\begin{Shaded}
\begin{Highlighting}[]
\KeywordTok{This} \NormalTok{is a wo}\KeywordTok{|this.}
\end{Highlighting}
\end{Shaded}

...results in:

\drcap{Three "x" characters}

\begin{Shaded}
\begin{Highlighting}[]
\KeywordTok{This} \NormalTok{is a woxx}\KeywordTok{|xs.}
\end{Highlighting}
\end{Shaded}

To quit without saving enter \texttt{:q}. To write any file changes to
disk use \texttt{:w}. To save \textbf{\emph{and}} quit, type
\texttt{:wq}. \drvic{:q}{quit without saving} \drvic{:w}{write to disk}
\drvic{:wq}{write to disk and quit}

\section*{Undo Me}\label{undo-me}
\addcontentsline{toc}{section}{Undo Me}

\texttt{u} is the "undo" command. It "undoes" or reverts the last
change. You can undo the last \emph{n} changes just as you'd expect,
e.g., \texttt{3u} undoes the last three changes. \drvic{u}{undo}

If you want to just cancel out of the file without writing any changes
to disk, use \texttt{:q!} (the \texttt{!} means to force the quit
without saving). \drvic{:q"!}{quit without saving (force)}

If you want to protect yourself from inadvertent changes to a file you
can always open it using
\href{http://linux.die.net/man/1/view}{\texttt{view}}, the alias for
\texttt{vi} invoked in read-only mode. \dreds{view}

\section{\texorpdfstring{Circumnavigating
\texttt{vi}}{Circumnavigating vi}}\label{circumnavigating-vi}

In modern implementations of \texttt{vi} (like \texttt{vim}) running
under modern shells the arrow and page keys will work as you expect,
\textbf{\emph{in general}}. However, you may want to be aware that when
in insert mode, while the left and right arrows may work for navigation,
often the up and down arrows can introduce "garbage" characters into the
file (since you are in insert mode). This is because the keymappings for
those keys aren't being interpreted correctly. I usually just swear,
exit insert mode, hit \texttt{u} and try again. \drvic{u}{undo}

As an example, under Cygwin I went into \texttt{vi}, went into insert
mode after the first line, typed in "This is a new line" and then hit
the up arrow five times, yielding this:

\drcap{Garbage characters}

\begin{Shaded}
\begin{Highlighting}[]
\KeywordTok{This} \NormalTok{is a word and so is this.}
\KeywordTok{A}
\KeywordTok{A}
\KeywordTok{A}
\KeywordTok{A}
\KeywordTok{A}
\KeywordTok{This} \NormalTok{is a new line}
\end{Highlighting}
\end{Shaded}

When in command mode, there are multiple ways to jump around in the file
besides using the arrow and page keys:

\begin{itemize}
\item
  \textbf{\texttt{0}} - jumps to the beginning of the current line.
  \drvic{0}{jump to beginning of line}
\item
  \textbf{\texttt{\$}} -jumps to the end of the current line.
  \drvic{\$}{jump to end of line}
  \index{*@\texttt{\$} (\texttt{vi} jump to end of line)}
\item
  \textbf{\texttt{w}} - jumps forward a whitespace-delimited "word" on
  the current line (and of course \texttt{3w} would jump forward three
  "words"). \drvic{w}{jump forward a word}
\item
  \textbf{\texttt{b}} - jumps back a whitespace-delimited "word" on the
  current line. \drvic{b}{jump back a word}
\item
  \textbf{\texttt{G}} - jumps to end of the file.
  \drvic{G}{jump to end of file}
\item
  \textbf{\texttt{:0}} - jumps to start of the file (note the preceding
  \texttt{:}). \drvic{:0}{jump to beginning of file}
\item
  \textbf{\texttt{/foo}} - find "foo" going forward toward the end of
  the file. \index{*@\texttt{/} (\texttt{vi} find forward)}
  \index{vicommands@\texttt  {vi} commands!\/@\texttt{/} (find forward)}
\item
  \textbf{\texttt{?foo}} - find "foo" going backward toward the front of
  the file. \index{*@\texttt{?} (\texttt{vi} find backward)}
  \index{vicommands@\texttt  {vi} commands!\?@\texttt{?} (find backward)}
\item
  \textbf{\texttt{n}} - find the next instance of the search text
  specified by the last \texttt{/} or \texttt{?}. \drvic{n}{find next}
\end{itemize}

\section*{Insert Tab A Into Slot B}\label{insert-tab-a-into-slot-b}
\addcontentsline{toc}{section}{Insert Tab A Into Slot B}

There are multiple ways to enter insert mode, but only one way to escape
it (pun intended - \texttt{ESC}, get it?)

\begin{itemize}
\item
  \textbf{\texttt{i}} - enters insert mode at the current cursor
  position. \drvic{i}{insert mode}
\item
  \textbf{\texttt{I}} - enters insert mode at the beginning of the
  current line. \drvic{I}{insert mode at beginning of line}
\item
  \textbf{\texttt{A}} - enters insert mode (appends) at the end of the
  current line. \drvic{A}{append at end of line}
\item
  \textbf{\texttt{o}} - inserts a new line under (lowercase \texttt{o} =
  "lower" or "below") the current line and puts the cursor on it in
  insert mode. \drvic{o}{insert new line under current line}
\item
  \textbf{\texttt{O}} - inserts a new line over (uppercase \texttt{O} =
  "upper" or "above") the current line and puts the cursor on it in
  insert mode. \drvic{O}{insert new line above current line}
\end{itemize}

\section*{Ctrl-X, Ctrl-C, Ctrl-V}\label{ctrl-x-ctrl-c-ctrl-v}
\addcontentsline{toc}{section}{Ctrl-X, Ctrl-C, Ctrl-V}

When you copy or cut/delete it, it goes into a "buffer." There are ways
to access multiple buffers, but mostly you want the very last thing to
be put in the buffer, especially for copying (or cutting) and pasting.
Note that "cutting" and "deleting" are synonymous, since deleting puts
the deleted text in the buffer.

Another thing to understand is that a command "doubled" or repeated
typically means "the whole line." So \texttt{dd} means "delete the whole
line the cursor is currently on." \drvic{dd}{delete entire line}

So if deleting is synonymous with cutting, and the cursor is on the
second line:

\drcap{Deleting a line}

\begin{Shaded}
\begin{Highlighting}[]
\KeywordTok{This} \NormalTok{is a word and so is this.}
\KeywordTok{This} \NormalTok{is a new line.}\KeywordTok{|}
\end{Highlighting}
\end{Shaded}

Then executing \texttt{dd} leaves:

\drcap{After the line is gone}

\begin{Shaded}
\begin{Highlighting}[]
\KeywordTok{|This} \NormalTok{is a word and so is this.}
\end{Highlighting}
\end{Shaded}

We know "This is a new line." went into the buffer. We can paste it back
above the current line with \texttt{P}(uppercase \texttt{P} = "upper" or
"above"), which would result in: \drvic{P}{paste above current line}

\drcap{After pasting the line above the current line}

\begin{Shaded}
\begin{Highlighting}[]
\KeywordTok{|This} \NormalTok{is a new line.}
\KeywordTok{This} \NormalTok{is a word and so is this.}
\end{Highlighting}
\end{Shaded}

Here are some more examples:

\begin{itemize}
\item
  \textbf{\texttt{p}} - paste the buffer into the current line starting
  after the cursor location. \drvic{p}{paste after cursor}
\item
  \textbf{\texttt{3dd}} - delete (cut) three lines into the buffer.
\item
  \textbf{\texttt{5yw}} - "yank" (copy) five words starting at the
  current cursor position into the buffer. \drvic{y}{copy}
\end{itemize}

\section*{Change Machine}\label{change-machine}
\addcontentsline{toc}{section}{Change Machine}

The hardest thing to get down in \texttt{vi} is the \emph{substitute}
(change or replace) command, \texttt{:s}. Its syntax is esoteric, but
once you've memorized it, it becomes intuitive. \drvic{:s}{change}

The most common scenario is the "change all" command. Given the
following file:

\drcap{Sample text file}

\begin{Shaded}
\begin{Highlighting}[]
\KeywordTok{This} \NormalTok{is a new line}
\KeywordTok{This} \NormalTok{is a word}
\KeywordTok{and} \NormalTok{so is this}
\KeywordTok{This} \NormalTok{and thus}
\KeywordTok{This} \NormalTok{and this and this}
\end{Highlighting}
\end{Shaded}

Let's change all "this" to "that" by using:

\drcap{Changing "this" to "that"}

\begin{Shaded}
\begin{Highlighting}[]
\NormalTok{:}\KeywordTok{0}\NormalTok{,}\OtherTok{$s}\NormalTok{/this/that/}
\end{Highlighting}
\end{Shaded}

We'll get into the details in a bit, but the results are interesting,
and not what we'd expect:

\drcap{What happened?}

\begin{Shaded}
\begin{Highlighting}[]
\KeywordTok{This} \NormalTok{is a new line}
\KeywordTok{This} \NormalTok{is a word}
\KeywordTok{and} \NormalTok{so is that}
\KeywordTok{This} \NormalTok{and thus}
\KeywordTok{This} \NormalTok{and that and this}
\end{Highlighting}
\end{Shaded}

It only changed the "that" at the end of the third line, and the middle
"that" on the last. Why? Two reasons:

\begin{enumerate}
\def\labelenumi{\arabic{enumi}.}
\item
  The substitute command is case sensitive, just like everything else in
  Linux, unless you tell it to be \emph{insensitive}.
\item
  The substitute command only makes one change per line unless you tell
  it to change \emph{globally}.
\end{enumerate}

So let's hit \texttt{u} to reset (undo) the change, and try again with
this: \drvic{u}{undo}

\drcap{Changing "this" to "that", redux}

\begin{Shaded}
\begin{Highlighting}[]
\NormalTok{:}\KeywordTok{0}\NormalTok{,}\OtherTok{$s}\NormalTok{/this/that/i}
\end{Highlighting}
\end{Shaded}

Results in:

\drcap{Closer, but not quite}

\begin{Shaded}
\begin{Highlighting}[]
\KeywordTok{that} \NormalTok{is a new line}
\KeywordTok{that} \NormalTok{is a word}
\KeywordTok{and} \NormalTok{so is that}
\KeywordTok{that} \NormalTok{and thus}
\KeywordTok{that} \NormalTok{and this and this}
\end{Highlighting}
\end{Shaded}

That's better. There is at least one "that" on every line that had a
"this," so passing the \texttt{i} ("insensitive") switch at the end of
the \texttt{s} (substitute) command helped with that. But we still
didn't get all the "this" words changed, as the last line shows. Hit
\texttt{u} and try one more time with this:

\drcap{Changing "this" to "that", one more time!}

\begin{Shaded}
\begin{Highlighting}[]
\NormalTok{:}\KeywordTok{0}\NormalTok{,}\OtherTok{$s}\NormalTok{/this/that/gi}
\end{Highlighting}
\end{Shaded}

Results in:

\drcap{Finally!}

\begin{Shaded}
\begin{Highlighting}[]
\KeywordTok{that} \NormalTok{is a new line}
\KeywordTok{that} \NormalTok{is a word}
\KeywordTok{and} \NormalTok{so is that}
\KeywordTok{that} \NormalTok{and thus}
\KeywordTok{that} \NormalTok{and that and that}
\end{Highlighting}
\end{Shaded}

That's what we wanted! Well, sort of. If we wanted to keep the
capitalization we'd have more work to do. See below.

In general, if you are looking for a case insensitive "change all" like
in Windows Notepad, the magic string to remember is:

\drcap{Memorize this - No, really}

\begin{Shaded}
\begin{Highlighting}[]
\NormalTok{:}\KeywordTok{0}\NormalTok{,}\OtherTok{$s}\NormalTok{/from/to/gi}
\end{Highlighting}
\end{Shaded}

Picking that apart, we have:

\begin{itemize}
\item
  \textbf{\texttt{:}} - tells \texttt{vi} a special command is coming.
\item
  \textbf{\texttt{0,\$}} - specifies a line range, in this case from the
  first (\texttt{0} - zero-relative) line to last (\texttt{\$}) line in
  the file. You can of course use other line numbers to restrict the
  range, and there are other ways to create ranges as well (see about
  marking lines, below).
\item
  \textbf{\texttt{s}} - substitute (change) command.
\item
  \textbf{\texttt{/from}} - "from" pattern (regular expression).
\item
  \textbf{\texttt{/to}} - "to" (results).
\item
  \textbf{\texttt{/gi}} - optional switches, \texttt{g} means "global"
  (change all instances on a line, not just the first one), \texttt{i}
  means (case) "insensitive."
\end{itemize}

Regular expressions you say! "Now we have two problems." But consider
where we left off: \index{regular expressions}

\drcap{But what about capitalization?}

\begin{Shaded}
\begin{Highlighting}[]
\KeywordTok{that} \NormalTok{is a new line}
\KeywordTok{that} \NormalTok{is a word}
\KeywordTok{and} \NormalTok{so is that}
\KeywordTok{that} \NormalTok{and thus}
\KeywordTok{that} \NormalTok{and that and that}
\end{Highlighting}
\end{Shaded}

First, let's capitalize all \texttt{t} characters, but only where they
are at the beginning of the line:

\drcap{Regular expression for the start of a line}

\begin{Shaded}
\begin{Highlighting}[]
\NormalTok{:}\KeywordTok{0}\NormalTok{,}\OtherTok{$s}\NormalTok{/^t/T/}
\end{Highlighting}
\end{Shaded}

Yields:

\drcap{Voila! Capitals!}

\begin{Shaded}
\begin{Highlighting}[]
\KeywordTok{That} \NormalTok{is a new line}
\KeywordTok{That} \NormalTok{is a word}
\KeywordTok{and} \NormalTok{so is that}
\KeywordTok{That} \NormalTok{and thus}
\KeywordTok{That} \NormalTok{and that and that}
\end{Highlighting}
\end{Shaded}

Now let's change all instances of "that" at the end of a line to be
"that."

\drcap{Regular expression for the end of a line}

\begin{Shaded}
\begin{Highlighting}[]
\NormalTok{:}\KeywordTok{0}\NormalTok{,}\OtherTok{$s}\NormalTok{/that$/that./}
\end{Highlighting}
\end{Shaded}

Ends up with:

\drcap{That with a full stop}

\begin{Shaded}
\begin{Highlighting}[]
\KeywordTok{That} \NormalTok{is a new line}
\KeywordTok{That} \NormalTok{is a word}
\KeywordTok{and} \NormalTok{so is that.}
\KeywordTok{That} \NormalTok{and thus}
\KeywordTok{That} \NormalTok{and that and that.}
\end{Highlighting}
\end{Shaded}

And finally as a fun exercise for the reader, using the full power of
regular expressions see if you can figure out how this is adding commas
to the end of lines that don't already have a period:

\drcap{Say what?}

\begin{Shaded}
\begin{Highlighting}[]
\NormalTok{:}\KeywordTok{0}\NormalTok{,}\OtherTok{$s}\NormalTok{/}\DataTypeTok{\textbackslash{}(}\NormalTok{[^.]$}\DataTypeTok{\textbackslash{})}\NormalTok{/\textbackslash{}1,/}
\end{Highlighting}
\end{Shaded}

Renders this:

\drcap{Nicely punctuated}

\begin{Shaded}
\begin{Highlighting}[]
\KeywordTok{That} \NormalTok{is a new line,}
\KeywordTok{That} \NormalTok{is a word,}
\KeywordTok{and} \NormalTok{so is that.}
\KeywordTok{That} \NormalTok{and thus,}
\KeywordTok{That} \NormalTok{and that and that.}
\end{Highlighting}
\end{Shaded}

\textbf{Hint:} While trying to figure that out, search the Internet for
regular expression "capturing groups."

\section*{"X" Marks the Spot}\label{x-marks-the-spot}
\addcontentsline{toc}{section}{"X" Marks the Spot}

You can "mark" lines in \texttt{vi} for use in "ranges" like the
"substitute" (change) command above. Let's say you have a file like the
following:

\drcap{Simple file}

\begin{Shaded}
\begin{Highlighting}[]
\KeywordTok{This} \NormalTok{is a line}
\KeywordTok{This} \NormalTok{is also a line}
\KeywordTok{This}\NormalTok{, too}
\KeywordTok{This} \NormalTok{is next}
\KeywordTok{This} \NormalTok{is last}
\end{Highlighting}
\end{Shaded}

Maybe we want to change the "This" on the first three lines to "That,"
but not the last two (imagine this is a much more complex example). We
could do it by hand with the \texttt{r} command, but that's tedious and
error prone. Instead, we can "mark" a range.

\begin{enumerate}
\def\labelenumi{\arabic{enumi}.}
\item
  Place the cursor on the first line and use the \texttt{m} command
  followed by a one-character "label" like \texttt{x} (I typically use
  \texttt{m} so I don't have to move my fingers, e.g., \texttt{mm}).
  \drvic{m}{mark}
\item
  Place the cursor on the third line and again use the \texttt{m}
  command, but \textbf{\emph{with a different label character}} (I
  usually use \texttt{n} so my fingers don't travel far, so
  \texttt{mn}).
\item
  Now you can use the \texttt{\textquotesingle{}} character followed by
  a label to denote the beginning and end of the range in all kinds of
  \texttt{vi} commands. In our case we want to change "This" on the
  first three lines, so: \drvic{'}{reference a mark}
\end{enumerate}

\begin{Shaded}
\begin{Highlighting}[]
\NormalTok{:}\StringTok{'m,'}\KeywordTok{ns/This/That/}
\end{Highlighting}
\end{Shaded}

Try doing that in Notepad!

\textbf{Note:} We could have done that with line ranges, too
(\texttt{:0,2s/This/That/}), but figuring out the beginning and ending
lines in a large range is a pain. It is much easier to just mark them
and go.

\section*{Executing External
Commands}\label{executing-external-commands}
\addcontentsline{toc}{section}{Executing External Commands}

Sometimes in \texttt{vi} it would be great to run the contents of the
file through an external command (\texttt{sort} is a favorite) without
saving and exiting the file, sorting it, and then re-editing it. We can
do that with \texttt{!}, which works a lot like the "substitute"
(change) command. \drtxt{sort} \drvic{"!}{invoke external command}
\index{*@\texttt{"!} (\texttt{vi} invoke external command)}

To sort the whole file in place:

\drcap{Sort a whole file in vi}

\begin{Shaded}
\begin{Highlighting}[]
\NormalTok{:}\KeywordTok{0}\NormalTok{,}\OtherTok{$!}\NormalTok{sort}
\end{Highlighting}
\end{Shaded}

To sort a marked range:

\drcap{Sorting a range}

\begin{Shaded}
\begin{Highlighting}[]
\NormalTok{:}\StringTok{'m,'}\KeywordTok{n}\NormalTok{!sort}
\end{Highlighting}
\end{Shaded}

Another handy command to check out for this kind of thing, especially
for formatting written text, is the
\href{http://linux.die.net/man/1/fmt}{\texttt{fmt}} command. \drtxt{fmt}

\section*{The Unseen World}\label{the-unseen-world}
\addcontentsline{toc}{section}{The Unseen World}

Any technical person knows that all the binary permutations of possible
values for a byte aren't mapped to visible characters. Some are
\href{https://en.wikipedia.org/wiki/Control_character}{"control
characters"} that range back to the teletype days. For example, a tab
character is hexadecimal 9 (\texttt{0x09}), but is often represented as
\texttt{\textbackslash{}t} in many programming languages, regular
expressions and the like.

Similarly, the "end of line" is marked by a control character. Or in the
case of Windows, \textbf{\emph{two}} control characters. And this causes
no end of problems when editing files that can be opened on both "UNIX"
systems and Windows.

On "UNIX," the line feed control character (\texttt{0x0a}, or
\texttt{\textbackslash{}n}) is all that marks the end of a line. For
historical reasons (CP/M), Windows ends each line with two control
characters, carriage return (\texttt{0x0d}, or
\texttt{\textbackslash{}r}) and line feed. The two together are often
referred to as "CRLF." \drios{CP/M}

This difference manifests in two ways:

\begin{enumerate}
\def\labelenumi{\arabic{enumi}.}
\item
  If you've ever opened a file on Windows in Notepad and all the lines
  "flow" even though they're supposed to be individual lines, that means
  it is probably using "UNIX" end-of-lines (\texttt{\textbackslash{}n})
  only. Use a line feed aware editor such as
  \href{https://notepad-plus-plus.org/}{Notepad++} instead.
\item
  If you open a file in \texttt{vi} and it has a \texttt{\^{}M} at the
  end of every line and/or at the bottom you see something like:

\begin{verbatim}
"Agenda.md" [dos format] 16 lines, 1692 characters
\end{verbatim}

  Either of those mean the file lines each end with "CRLF"
  (\texttt{\textbackslash{}r\textbackslash{}n}). To change it in
  \texttt{vi} you can override the \texttt{ff} (file format) setting:

\begin{verbatim}
:set ff=unix
\end{verbatim}
\end{enumerate}

Since regular expressions have syntax for expressing control codes in
either shorthand (\texttt{\textbackslash{}t}) or as hexadecimal, you can
alter control codes in \texttt{vi} easily. For example, to change all
tab characters to four spaces:

\drcap{Change all tabs to four spaces as God meant them to be}

\begin{Shaded}
\begin{Highlighting}[]
\NormalTok{:}\KeywordTok{0}\NormalTok{,}\OtherTok{$s}\NormalTok{/\textbackslash{}t/    /g}
\end{Highlighting}
\end{Shaded}

\section*{Let's Get Small}\label{lets-get-small}
\addcontentsline{toc}{section}{Let's Get Small}

So, \texttt{vi} is the best we can do? No. On many Linux systems an
alternative terminal-based editor will be installed, often several.
There may be \href{http://linux.die.net/man/1/emacs}{\texttt{emacs}},
which \textbf{\emph{will}} make you yearn for the simplicity of
\texttt{vi}. \dreds{emacs}

Here are two jokes that are only funny once you've used \texttt{emacs}:

\begin{quote}
\emph{"'emacs' stands for 'escape', 'meta', 'alt', 'control', 'shift'."}
\end{quote}

\begin{quote}
\emph{"'emacs' is a good operating system, but it could use an editor."}
\end{quote}

If those are funny to you, then you have already been infected by
\texttt{emacs}. The prognosis is grim.

But there may also be others, notably
\href{http://linux.die.net/man/1/pico}{\texttt{pico}} and its successor,
\href{http://linux.die.net/man/1/nano}{\texttt{nano}}. You can see the
difference the second you see a file open in \texttt{nano} - in this
case, the generated Github-flavored Markdown of this document:
\dreds{nano} \dreds{pico}

\drcap{Editing a file in \texttt{nano}}

\begin{Shaded}
\begin{Highlighting}[]
    \KeywordTok{GNU} \NormalTok{nano 2.2.6        File: TenStepsToLinuxSurvival.md                        }

\KeywordTok{|}\NormalTok{![}\KeywordTok{Merv} \NormalTok{sez, }\StringTok{"Don't panic."}\NormalTok{](./images/Merv.jpg }\StringTok{"Merv sez, 'Don't panic.'"}\NormalTok{)}
\KeywordTok{Merv} \NormalTok{sez, }\StringTok{"Don't panic."}

\KeywordTok{By} \NormalTok{James Lehmer}

\KeywordTok{v0.7}

\NormalTok{![]}\KeywordTok{(./images/cc-by-sa.png} \StringTok{"Creative Commons Attribution-ShareAlike 4.0 Internat$}
\StringTok{*Jim's Ten Steps to Linux Survival* by James Lehmer is licensed under a [Creati$}

\StringTok{**Dedicated to my first three technical mentors** - Jim Proffer, who taught me $}

\StringTok{Introduction}
\StringTok{============}

\StringTok{> *"}\NormalTok{And you may ask yourself, }\StringTok{'Well, how did I get here?'"* - Talking Heads (*O$}

\StringTok{This is my little "}\NormalTok{Linux and Bash in 10 steps}\StringTok{" guide. It's based around what I $}

\StringTok{                              [ Read 3627 lines ]}
\StringTok{^G Get Help  ^O WriteOut  ^R Read File ^Y Prev Page ^K Cut Text  ^C Cur Pos}
\StringTok{^X Exit      ^J Justify   ^W Where Is  ^V Next Page ^U UnCut Text^T To Spell}
\end{Highlighting}
\end{Shaded}

Two things to note about the above:

\begin{enumerate}
\def\labelenumi{\arabic{enumi}.}
\item
  The cursor (represented above by \texttt{\textbar{}}) is already in
  "insert mode" like you would expect in a "normal" editor such as
  Notepad.
\item
  Those lines at the bottom are commands that can be invoked by
  shortcuts. For example, \texttt{\^{}O} means \texttt{Ctrl-O} and
  stands for "WriteOut" or "Save." That's probably easier to remember
  than \texttt{:w} in \texttt{vi}, especially since it is reminding you
  of it right there on the screen!
\end{enumerate}

So why not always use \texttt{nano}? Why does this book harp on and on
\texttt{vi}? Why do I insist on keeping all this arcane \texttt{vi}
nonsense loaded in my head (and I do!)? Because often, like in the
nightmare scenario I posed in the \emph{Introduction}, you may not have
control over the system, no ability to install packages - you have to
take what the system has. And it's a pretty sure bet it is going to have
\texttt{vi}. So if you have \texttt{nano} (or \texttt{pico}), use it!
You can find out simply by typing in
\texttt{nano\ \textless{}filename\textgreater{}} on what you want to
edit and see if it works. But if \texttt{nano} or \texttt{pico} aren't
installed, grit your teeth, remember "insert mode" vs. "command mode",
and use \texttt{vi}.

And if you have the opportunity to use \texttt{emacs}...don't.

\section{Editing on the Command Line}\label{editing-on-the-command-line}

Sometimes you want to script an edit, typically something similar to a
"replace all" that needs to occur on a file without human intervention.
The \href{http://linux.die.net/man/1/sed}{\texttt{sed}} (stream editor)
command to the rescue! \texttt{sed} has a similar syntax to the
"substitute" commands in \texttt{vi} (in fact, the latter got its syntax
from the former). \dreds{sed}

Here is a real-world example. A \href{https://www.mysql.com/}{MySQL}
database backup is in reality a text file containing a large number of
SQL statements - the \texttt{DROP}, \texttt{CREATE} and \texttt{INSERT}
statements necessary to recreate the database from scratch. Let's say
you have two Wordpress sites, \texttt{www.mysite.com} for production,
and \texttt{dev.mysite.com} for a testing environment. When Wordpress is
configured, it puts its site address, e.g., \texttt{www.mysite.com}, in
multiple places in the database. If you want to refresh your dev site
from production, you would backup the MySQL database to a file like
\texttt{mysqlbak.sql}. But before loading it in the dev site's database,
you would like to change all those \texttt{www.mysite.com} references to
\texttt{dev.mysite.com}. \texttt{sed} to the rescue! Behold:

\drcap{Editing a file with \texttt{sed}}

\begin{Shaded}
\begin{Highlighting}[]
\KeywordTok{~} \NormalTok{$ }\KeywordTok{cat} \NormalTok{mysqlbak.sql }\KeywordTok{|} \KeywordTok{sed} \StringTok{'s/www.mysite.com/dev.mysite.com/g'} \KeywordTok{>} \NormalTok{devbak.sql}
\end{Highlighting}
\end{Shaded}

How cool is that? If you remember the "substitute" command examples for
\texttt{vi}, above, it should be perfectly clear what is going on here.

\hypertarget{the-whole-wide-world}{\chapter{The Whole Wide
World}\label{the-whole-wide-world}}

\textbf{\emph{\texttt{curl}, \texttt{wget}, \texttt{ifconfig},
\texttt{ping}, \texttt{ssh}, \texttt{telnet}, \texttt{/etc/hosts} and
email before Outlook.}}

\begin{quote}
\emph{"Gopher, Everett?"} - Delmar O'Donnell (\emph{O Brother, Where Are
Thou?})
\end{quote}

If Sun's motto "The network is the computer" is correct, then of course
Linux and similar systems must be able to access the network from the
command line and scripts.

For example, our friend
\href{http://linux.die.net/man/8/ping}{\texttt{ping}} is there:
\drnet{ping}

\drcap{ping command}

\begin{Shaded}
\begin{Highlighting}[]
\CommentTok{# ping www.yahoo.com}
\KeywordTok{PING} \NormalTok{fd-fp3.wg1.b.yahoo.com (98.138.253.109) }\KeywordTok{56}\NormalTok{(84) }\KeywordTok{bytes} \NormalTok{of data.}
\KeywordTok{64} \NormalTok{bytes from ir1.fp.vip.ne1.yahoo.com (98.138.253.109)}\KeywordTok{:} \NormalTok{icmp_req=1 ttl=...}
\KeywordTok{64} \NormalTok{bytes from ir1.fp.vip.ne1.yahoo.com (98.138.253.109)}\KeywordTok{:} \NormalTok{icmp_req=2 ttl=...}
\KeywordTok{64} \NormalTok{bytes from ir1.fp.vip.ne1.yahoo.com (98.138.253.109)}\KeywordTok{:} \NormalTok{icmp_req=3 ttl=...}
\KeywordTok{64} \NormalTok{bytes from ir1.fp.vip.ne1.yahoo.com (98.138.253.109)}\KeywordTok{:} \NormalTok{icmp_req=4 ttl=...}
\KeywordTok{64} \NormalTok{bytes from ir1.fp.vip.ne1.yahoo.com (98.138.253.109)}\KeywordTok{:} \NormalTok{icmp_req=5 ttl=...}
\KeywordTok{64} \NormalTok{bytes from ir1.fp.vip.ne1.yahoo.com (98.138.253.109)}\KeywordTok{:} \NormalTok{icmp_req=6 ttl=...}
\KeywordTok{64} \NormalTok{bytes from ir1.fp.vip.ne1.yahoo.com (98.138.253.109)}\KeywordTok{:} \NormalTok{icmp_req=7 ttl=...}
\KeywordTok{64} \NormalTok{bytes from ir1.fp.vip.ne1.yahoo.com (98.138.253.109)}\KeywordTok{:} \NormalTok{icmp_req=8 ttl=...}
\KeywordTok{64} \NormalTok{bytes from ir1.fp.vip.ne1.yahoo.com (98.138.253.109)}\KeywordTok{:} \NormalTok{icmp_req=9 ttl=...}
\KeywordTok{64} \NormalTok{bytes from ir1.fp.vip.ne1.yahoo.com (98.138.253.109)}\KeywordTok{:} \NormalTok{icmp_req=10 ttl...}
\NormalTok{^}\KeywordTok{C}
\KeywordTok{---} \NormalTok{fd-fp3.wg1.b.yahoo.com ping statistics ---}
\KeywordTok{10} \NormalTok{packets transmitted, 10 received, 0% packet loss, time 9004ms}
\KeywordTok{rtt} \NormalTok{min/avg/max/mdev = 59.933/62.581/70.935/3.191 ms}
\end{Highlighting}
\end{Shaded}

One difference with \texttt{ping} is that by default in Linux
\texttt{ping} doesn't stop until the user presses \texttt{Ctrl-C} (which
sends the
\href{https://en.wikipedia.org/wiki/Unix_signal}{\texttt{SIGINT}
interrupt} to the program). In this way it acts more like
\texttt{ping\ -t} in \texttt{CMD.EXE} Also, be aware that on Cygwin
\texttt{ping} is still the system (Windows) \texttt{ping}.
\drshl{CMD.EXE} \index{signals}

\href{http://linux.die.net/man/8/traceroute}{\texttt{traceroute}} works,
too (although for once its name is longer than the \texttt{CMD.EXE}
counterpart). \drnet{traceroute}

\drcap{traceroute command}

\begin{Shaded}
\begin{Highlighting}[]
\KeywordTok{~} \NormalTok{$ }\KeywordTok{traceroute} \NormalTok{google.com}
\KeywordTok{traceroute} \NormalTok{to google.com (216.58.216.78), }\KeywordTok{30} \NormalTok{hops max, 60 byte packets}
 \KeywordTok{1}  \NormalTok{192.168.0.1 (192.168.0.1)  }\KeywordTok{3.623} \NormalTok{ms  3.978 ms  7.231 ms}
 \KeywordTok{2}  \NormalTok{* * *}
 \KeywordTok{3}  \NormalTok{* * *}
 \KeywordTok{4}  \NormalTok{* * *}
 \KeywordTok{5}  \NormalTok{* * *}
 \KeywordTok{6}  \NormalTok{* * *}
 \KeywordTok{7}  \NormalTok{72.14.215.212 (72.14.215.212)  }\KeywordTok{26.205} \NormalTok{ms  27.502 ms  27.648 ms}
 \KeywordTok{8}  \NormalTok{209.85.242.133 (209.85.242.133)  }\KeywordTok{31.547} \NormalTok{ms  31.550 ms  31.548 ms}
 \KeywordTok{9}  \NormalTok{72.14.237.231 (72.14.237.231)  }\KeywordTok{29.516} \NormalTok{ms  29.556 ms  29.657 ms}
\KeywordTok{10}  \NormalTok{ord30s21-in-f78.1e100.net (216.58.216.78)  }\KeywordTok{30.313} \NormalTok{ms  33.138 ms  28.092 ms}
\end{Highlighting}
\end{Shaded}

You can do some digging in DNS with
\href{http://linux.die.net/man/1/dig}{\texttt{dig}}: \drnet{dig}

\drcap{dig command}

\begin{Shaded}
\begin{Highlighting}[]
\KeywordTok{~} \NormalTok{$ }\KeywordTok{dig} \NormalTok{yahoo.com}

\NormalTok{; }\KeywordTok{<<>>} \KeywordTok{DiG} \NormalTok{9.9.5-3ubuntu0.6-Ubuntu }\KeywordTok{<<>>} \NormalTok{yahoo.com}
\NormalTok{;; }\KeywordTok{global} \NormalTok{options: +cmd}
\NormalTok{;; }\KeywordTok{Got} \NormalTok{answer:}
\NormalTok{;; }\KeywordTok{->>}\NormalTok{HEADER<<- opcode: QUERY, status: NOERROR, id: 46478}
\NormalTok{;; flags: qr rd ra; QUERY: 1, ANSWER: 3, AUTHORITY: 0, ADDITIONAL: 1}

\NormalTok{;; OPT PSEUDOSECTION:}
\NormalTok{; EDNS: version: 0, flags:; udp: 4096}
\NormalTok{;; QUESTION SECTION:}
\NormalTok{;yahoo.com.         IN  A}

\NormalTok{;; ANSWER SECTION:}
\NormalTok{yahoo.com.      605 IN  A   98.138.253.109}
\NormalTok{yahoo.com.      605 IN  A   206.190.36.45}
\NormalTok{yahoo.com.      605 IN  A   98.139.183.24}

\NormalTok{;; Query time: 23 msec}
\NormalTok{;; SERVER: 127.0.1.1#53(127.0.1.1)}
\NormalTok{;; WHEN: Tue Dec 22 09:46:26 CST 2015}
\NormalTok{;; MSG SIZE  rcvd: 86}
\end{Highlighting}
\end{Shaded}

And \href{http://linux.die.net/man/1/whois}{\texttt{whois}}:
\drnet{whois}

\begin{Shaded}
\begin{Highlighting}[]
\KeywordTok{~} \NormalTok{$ }\KeywordTok{whois} \NormalTok{yahoo.com}

\KeywordTok{Whois} \NormalTok{Server Version 2.0}

\KeywordTok{Domain} \NormalTok{names in the .com and .net domains can now be registered}
\KeywordTok{with} \NormalTok{many different competing registrars. Go to http://www.internic.net}
\KeywordTok{for} \KeywordTok{detailed} \NormalTok{information.}

   \KeywordTok{Server} \NormalTok{Name: YAHOO.COM.ACCUTAXSERVICES.COM}
   \KeywordTok{IP} \NormalTok{Address: 98.136.43.32}
   \KeywordTok{IP} \NormalTok{Address: 66.196.84.168}
   \KeywordTok{Registrar}\NormalTok{: WILD WEST DOMAINS, LLC}
   \KeywordTok{Whois} \NormalTok{Server: whois.wildwestdomains.com}
   \KeywordTok{Referral} \NormalTok{URL: http://www.wildwestdomains.com}

   \KeywordTok{Server} \NormalTok{Name: YAHOO.COM.ANGRYPIRATES.COM}
   \KeywordTok{IP} \NormalTok{Address: 8.8.8.8}
   \KeywordTok{Registrar}\NormalTok{: NAME.COM, INC.}
   \KeywordTok{Whois} \NormalTok{Server: whois.name.com}
   \KeywordTok{Referral} \NormalTok{URL: http://www.name.com}

   \KeywordTok{Server} \NormalTok{Name: YAHOO.COM.AU}
   \KeywordTok{Registrar}\NormalTok{: WILD WEST DOMAINS, LLC}
\KeywordTok{...and} \NormalTok{so on...}
\end{Highlighting}
\end{Shaded}

\section*{sudo Make Me a Sandwich}\label{sudo-make-me-a-sandwich}
\addcontentsline{toc}{section}{sudo Make Me a Sandwich}

It may not be the best place to discuss it, but we've finally come to a
point where your normal user account may not have access to these tools.
On many systems network commands are considered "system" or privileged
commands and are restricted.

One way to run restricted commands is to log in as an "elevated" or
privileged user, such as \texttt{root}. But this is frowned on, and many
distros today rely on the
\href{http://linux.die.net/man/8/sudo}{\texttt{sudo}} command to act as
a way for a normal user to signal they want to escalate their privileges
temporarily, presuming they are allowed to do so, which is usually
indicated by being a member of the \texttt{sudo} group or similar.
\drsys{sudo}{execute as another user}

In a sense, \texttt{sudo} is similar to Windows User Access Control
(UAC) prompts. They ensure a human is in control, in the case of
\texttt{sudo} by prompting for the user's password. If multiple commands
are invoked by \texttt{sudo} within a short time period, you will not be
reprompted for a password each time, (unlike UAC).

Here is a really common example on Debian-based systems: \drpkg{apt-get}

\drcap{Make me a sandwich}

\begin{Shaded}
\begin{Highlighting}[]
\KeywordTok{~} \NormalTok{$ }\KeywordTok{apt-get} \NormalTok{update}
\KeywordTok{E}\NormalTok{: Could not open lock file /var/lib/apt/lists/lock - open (13: Permission denie}
\KeywordTok{d}\NormalTok{)}
\KeywordTok{E}\NormalTok{: Unable to lock directory /var/lib/apt/lists/}
\KeywordTok{E}\NormalTok{: Could not open lock file /var/lib/dpkg/lock - open (13: Permission denied)}
\KeywordTok{E}\NormalTok{: Unable to lock the administration directory (/var/lib/dpkg/), }\KeywordTok{are} \NormalTok{you root?}
\end{Highlighting}
\end{Shaded}

The error message, especially the last line, is pretty clear. Let's try
it again with \texttt{sudo}:

\drcap{sudo Make me a sandwich}

\begin{Shaded}
\begin{Highlighting}[]
\KeywordTok{~} \NormalTok{$ }\KeywordTok{sudo} \NormalTok{apt-get update}
\KeywordTok{Ign} \NormalTok{http://packages.linuxmint.com rafaela InRelease}
\KeywordTok{Ign} \NormalTok{http://extra.linuxmint.com rafaela InRelease                               }
\KeywordTok{Hit} \NormalTok{http://extra.linuxmint.com rafaela Release.gpg                             }
\KeywordTok{Hit} \NormalTok{http://packages.linuxmint.com rafaela Release.gpg                          }
\KeywordTok{Ign} \NormalTok{http://archive.ubuntu.com trusty InRelease                                 }
\KeywordTok{Hit} \NormalTok{http://security.ubuntu.com trusty-security InRelease                       }
\KeywordTok{Hit} \NormalTok{http://packages.linuxmint.com rafaela Release                              }
\KeywordTok{Hit} \NormalTok{http://extra.linuxmint.com rafaela Release                                 }
\KeywordTok{Hit} \NormalTok{http://archive.ubuntu.com trusty-updates InRelease                         }
\KeywordTok{Hit} \NormalTok{http://security.ubuntu.com trusty-security/main amd64 Packages             }
\KeywordTok{Hit} \NormalTok{http://packages.linuxmint.com rafaela/main amd64 Packages                  }
\KeywordTok{Hit} \NormalTok{http://extra.linuxmint.com rafaela/main amd64 Packages                     }
\KeywordTok{Ign} \NormalTok{http://archive.canonical.com trusty InRelease                              }
\KeywordTok{Hit} \NormalTok{http://archive.ubuntu.com trusty Release.gpg                               }
\KeywordTok{Hit} \NormalTok{http://security.ubuntu.com trusty-security/restricted amd64 Packages       }
\KeywordTok{Hit} \NormalTok{http://extra.linuxmint.com rafaela/main i386 Packages                      }
\KeywordTok{Hit} \NormalTok{http://packages.linuxmint.com rafaela/upstream amd64 Packages              }
\KeywordTok{Hit} \NormalTok{http://security.ubuntu.com trusty-security/universe amd64 Packages         }
\KeywordTok{Hit} \NormalTok{http://archive.ubuntu.com trusty-updates/main amd64 Packages               }
\KeywordTok{Hit} \NormalTok{http://packages.linuxmint.com rafaela/import amd64 Packages                }
\KeywordTok{Hit} \NormalTok{http://security.ubuntu.com trusty-security/multiverse amd64 Packages       }
\KeywordTok{Hit} \NormalTok{http://archive.canonical.com trusty Release.gpg                            }
\KeywordTok{...and} \NormalTok{so on...         }
\end{Highlighting}
\end{Shaded}

Now you should get the punchline to \href{https://xkcd.com/149/}{this
comic}, and hence the title of this section.

\textbf{NOTE:} The first time you ever run \texttt{sudo} on a machine,
you will probably see the following. They are good words to live by:

\drcap{Words to live by}

\begin{verbatim}
We trust you have received the usual lecture from the local System
Administrator. It usually boils down to these three things:

    #1) Respect the privacy of others.
    #2) Think before you type.
    #3) With great power comes great responsibility.
\end{verbatim}

\section*{Surfin' the Command Prompt}\label{surfin-the-command-prompt}
\addcontentsline{toc}{section}{Surfin' the Command Prompt}

You can browse the web from the command prompt using something like
\href{http://linux.die.net/man/1/lynx}{\texttt{lynx}}. A text-based
browser isn't too exciting, but it can have its purposes (like quickly
testing network access from a command prompt). For example,
\texttt{lynx\ http://google.com} yields: \drnet{lynx}

\drcap{Browsing like it's 1994}

\begin{Shaded}
\begin{Highlighting}[]
                                                                          \KeywordTok{Google}

   \KeywordTok{Search} \NormalTok{Images Maps Play YouTube News Gmail Drive More »}
   \KeywordTok{Web} \NormalTok{History }\KeywordTok{|} \KeywordTok{Settings} \KeywordTok{|} \KeywordTok{Sign} \NormalTok{in}

   \KeywordTok{Google}

     \KeywordTok{_______________________________________________________}
     \KeywordTok{Google} \NormalTok{Search  I}\StringTok{'m Feeling Lucky                          Advanced search}
\StringTok{                                                               Language tools}

\StringTok{   Advertising Programs     Business Solutions     +Google     About}
\StringTok{   Google}

\StringTok{                          © 2015 - Privacy - Terms}






\StringTok{(NORMAL LINK) Use right-arrow or <return> to activate.}
\StringTok{  Arrow keys: Up and Down to move.  Right to follow a link; Left to go back.}
\StringTok{ H)elp O)ptions P)rint G)o M)ain screen Q)uit /=search [delete]=history list }
\end{Highlighting}
\end{Shaded}

There are two other commands that are used to pull down web resources
and save them locally -
\href{http://linux.die.net/man/1/curl}{\texttt{curl}} and
\href{http://linux.die.net/man/1/wget}{\texttt{wget}}. Both support
HTTP(S) and FTP, but \texttt{curl} supports even more protocols and
options and tends to be the simplest to just "grab a file and go." You
see both used often in install scripts that download bits from the
internet and then execute them by piping them to \texttt{bash}:
\drscr{bash} \drnet{curl} \drnet{wget}

\drcap{\texttt{wget} in an install script}

\begin{Shaded}
\begin{Highlighting}[]
\KeywordTok{wget} \NormalTok{-O - http://foocorp.com/installs/install.sh }\KeywordTok{|} \KeywordTok{bash}
\end{Highlighting}
\end{Shaded}

Or:

\drcap{\texttt{curl} in an install script}

\begin{Shaded}
\begin{Highlighting}[]
\KeywordTok{curl} \NormalTok{http://foocorp.com/installs/install.sh }\KeywordTok{|} \KeywordTok{bash}
\end{Highlighting}
\end{Shaded}

\textbf{Note:} As always, you should be cautious when downloading and
executing arbitrary bits, and this technique doesn't lessen your
responsibility there. It is often better to use something like
\texttt{curl} to download the script but instead of piping it to
\texttt{bash} to be executed, redirect it to a file and look at what the
script is doing first:

\drcap{Check out what that script is doing first!}

\begin{Shaded}
\begin{Highlighting}[]
\KeywordTok{~} \NormalTok{$ }\KeywordTok{curl} \NormalTok{http://foocorp.com/installs/install.sh }\KeywordTok{>} \NormalTok{install.sh}
\KeywordTok{~} \NormalTok{$ }\KeywordTok{cat} \NormalTok{install.sh}
\CommentTok{#!/bin/bash}
\CommentTok{# I'm a script from a bad guy, check out the next line!}
\KeywordTok{rm} \NormalTok{-rf /*}
\end{Highlighting}
\end{Shaded}

Now, aren't you glad you didn't just execute that without checking?

But if the script looks right, then you can \texttt{chmod} it and run
it:

\drcap{Got a good script, so execute it}

\begin{Shaded}
\begin{Highlighting}[]
\KeywordTok{~} \NormalTok{$ }\KeywordTok{curl} \NormalTok{http://foocorp.com/installs/install.sh }\KeywordTok{>} \NormalTok{install.sh}
\KeywordTok{~} \NormalTok{$ }\KeywordTok{cat} \NormalTok{install.sh}
\CommentTok{#!/bin/bash}
\CommentTok{# I'm a script from a good guy.}
\CommentTok{# Do some stuff...}
\KeywordTok{~} \NormalTok{$ }\KeywordTok{chmod} \NormalTok{770 install.sh}
\KeywordTok{~} \NormalTok{$ }\KeywordTok{./install.sh}
\end{Highlighting}
\end{Shaded}

\section{It's Nice to Share}\label{its-nice-to-share}

Linux boxes can be set up to share files to Windows machines using the
SMB/CIFS protocols via a system called
\href{https://www.samba.org/}{Samba}. However, that is beyond the scope
of this book, since the intended audience will not likely need to do
that.

However, there is often a need in the scenarios I envision the reader
has been thrust into, where they have to share files back and forth
between a local *IX box and a Windows machine. Maybe it's to pull files
over to Windows to be backed up by the shop's normal backup mechanisms.
Or it could be to bring over log files for forensics. It could even be
to copy files from Windows to the Linux machine.

For example, in my company we keep copies of our Windows servers
application configuration files (such as various \texttt{web.config}
files) in Git, specifically on a
\href{https://about.gitlab.com/}{GitLab} server. There is a
\texttt{cron} job (coming up later) that copies the files every day from
the Windows servers and then commits them to GitLab if there are any
changes. It's a nice way to keep track of environment changes over time.

How do you copy files from Windows to Linux or vice versa? With the
\href{http://linux.die.net/man/1/smbclient}{\texttt{smbclient}} command.
It works somewhat similarly to the
\href{http://linux.die.net/man/1/ftp}{\texttt{ftp}} (but again, if
you're going to copy using the FTP protocol, I think \texttt{curl} or
\texttt{wget} are better). Here is an example of \texttt{smbclient} to
get you started. \drnet{ftp} \drnet{smbclient}

\drcap{Copying multiple files from a Windows machine with `mget`}

\begin{Shaded}
\begin{Highlighting}[]
\KeywordTok{~} \NormalTok{$ }\KeywordTok{smbclient} \NormalTok{//winbox/myshare -U myuser%mypassword -W mywindomain \textbackslash{}}
\NormalTok{-c }\StringTok{"prompt;cd configs;mget *.config;exit"}
\end{Highlighting}
\end{Shaded}

To understand the above:

\begin{itemize}
\item
  \textbf{\texttt{//winbox/myshare}} - is the Windows share. Note in
  this case you use forward slashes (\texttt{/}), not the typical
  backslashes (\texttt{\textbackslash{}}) that Windows uses.
\item
  \textbf{\texttt{-U\ myuser\%mypassword}} - userid and password. If you
  don't specify them, you will be prompted, but if you are using this
  command in a script you either have to specify them or have the share
  set up with guest permissions. It is a good idea to make sure the
  script file is locked down, and that the \texttt{smbclient} command is
  not recorded in your \texttt{.bash\_history} file (covered later).
\item
  \textbf{\texttt{-W\ mywindomain}} - the Active Directory domain the
  Windows machine is a member of (or often \texttt{WORKGROUP} if it is a
  standalone machine).
\item
  \textbf{\texttt{-c\ "prompt;cd\ configs;mget\ *.config;exit"}} - the
  remote commands to execute, in this case turning the
  \texttt{smbclient} prompting off (useful for scripts), changing to the
  \texttt{settings} directory on the remote Windows machine with
  \texttt{cd}, then getting multiple (\texttt{mget}) files using the
  wildcard \texttt{*.config}, and then \texttt{exit} to end the command
  sequence and disconnect.
\end{itemize}

\textbf{Note:} You can send files \textbf{\emph{to}} a Windows machine
by using \texttt{mput} rather than \texttt{mget}.

\section*{You've Got Mail}\label{youve-got-mail}
\addcontentsline{toc}{section}{You've Got Mail}

You can send and receive email from the command prompt. Reading email
will be rare, but if the system has
\href{http://linux.die.net/man/1/pine}{\texttt{pine}} installed, that's
probably the most intuitive from a non-UNIX perspective (although it is
still obviously a terminal program). Otherwise look for
\href{http://linux.die.net/man/1/mutt}{\texttt{mutt}}. \drnet{mutt}
\drnet{pine}

Sending email is more interesting, especially from shell scripts. There
are multiple ways, but
\href{http://linux.die.net/man/1/email}{\texttt{email}} is
straightforward enough: \drnet{email}

\drcap{Sending email from the command line}

\begin{Shaded}
\begin{Highlighting}[]
\KeywordTok{~} \NormalTok{$ }\KeywordTok{email} \NormalTok{--blank-mail --subject }\StringTok{"Possibly corrupted files found..."} \NormalTok{\textbackslash{}}
  \NormalTok{--smtp-server smtp --attach badfiles.csv --from-name NoReply \textbackslash{}}
  \NormalTok{--from-addr noreply@mycorp.com alert@mycorp.com}
\end{Highlighting}
\end{Shaded}

\section*{Let's Connect}\label{lets-connect}
\addcontentsline{toc}{section}{Let's Connect}

There are two primary ways to get an interactive "shell" session on a
remote machine. The first is the venerable
\href{http://linux.die.net/man/1/telnet}{\texttt{telnet}} command. It
isn't used very often for actual interactive sessions any more (for one,
because it sends credentials in plain text on the wire). However,
because you can specify the port number, it is still handy for testing
and debugging text-based protocols such as SMTP or HTTP. In the
following, after opening a \texttt{telnet} connection on port 80 to
Google, I simply entered the HTTP protocol sequence
\texttt{GET\ /\ HTTP/1.1} followed by a blank line to get Google to
return its home page: \drnet{telnet}

\drcap{Using \texttt{telnet} to diagnose HTTP}

\begin{Shaded}
\begin{Highlighting}[]
\KeywordTok{~} \NormalTok{$ }\KeywordTok{telnet} \NormalTok{google.com 80}
\KeywordTok{Trying} \NormalTok{216.58.216.78...}
\KeywordTok{Connected} \NormalTok{to google.com.}
\KeywordTok{Escape} \NormalTok{character is }\StringTok{'^]'}\NormalTok{.}
\KeywordTok{GET} \NormalTok{/ HTTP/1.1}

\KeywordTok{HTTP/1.1} \NormalTok{200 OK}
\KeywordTok{Date}\NormalTok{: Tue, 22 Dec 2015 15:58:47 GMT}
\KeywordTok{Expires}\NormalTok{: -1}
\KeywordTok{Cache-Control}\NormalTok{: private, max-age=0}
\KeywordTok{Content-Type}\NormalTok{: text/html}\KeywordTok{;} \OtherTok{charset=}\NormalTok{ISO-8859-1}
\KeywordTok{P3P}\NormalTok{: CP=}\StringTok{"This is not a P3P policy! See https://www.google.com/support/accounts/a}
\StringTok{nswer/151657?hl=en for more info."}
\KeywordTok{Server}\NormalTok{: gws}
\KeywordTok{X-XSS-Protection}\NormalTok{: 1}\KeywordTok{;} \OtherTok{mode=}\NormalTok{block}
\KeywordTok{X-Frame-Options}\NormalTok{: SAMEORIGIN}
\KeywordTok{Set-Cookie}\NormalTok{: NID=74=nqD9y_pSQudbaw6obB94Ngw6lsn4t_S8Z3NbZcUJ5HB4qUXCpu988A5QG3EQD}
\KeywordTok{kwqgOdGapsUSmsi91yHAa9_LU9JeP4pKop-1p5w7LlrdMyGrGojwoaX58ML6PSH5nGLsdZV0Z5vBqNTh}
\KeywordTok{A;} \OtherTok{expires=}\NormalTok{Wed, }\KeywordTok{22-Jun-2016} \NormalTok{15:58:47 GMT}\KeywordTok{;} \OtherTok{path=}\NormalTok{/; }\OtherTok{domain=}\NormalTok{.google.com; }\KeywordTok{HttpOnly}
\KeywordTok{Accept-Ranges}\NormalTok{: none}
\KeywordTok{Vary}\NormalTok{: Accept-Encoding}
\KeywordTok{Transfer-Encoding}\NormalTok{: chunked}

\KeywordTok{...and} \NormalTok{so on...}
\end{Highlighting}
\end{Shaded}

The \texttt{SMTP} protocol can also be diagnosed this way:

\drcap{Using \texttt{telnet} to diagnose SMTP}

\begin{Shaded}
\begin{Highlighting}[]
\KeywordTok{~} \NormalTok{$ }\KeywordTok{telnet} \NormalTok{smtp 25}
\KeywordTok{Trying} \NormalTok{10.1.1.8...}
\KeywordTok{Connected} \NormalTok{to mail.mydomain.com.}
\KeywordTok{Escape} \NormalTok{character is }\StringTok{'^]'}\NormalTok{.}
\KeywordTok{220} \NormalTok{MAIL.MYDOMAIN.COM}
\KeywordTok{HELO}
\KeywordTok{250} \NormalTok{MAIL.MYDOMAIN.COM Hello [10.1.1.8]}
\KeywordTok{MAIL} \NormalTok{FROM:}\KeywordTok{<}\NormalTok{myuser@mydomain.com}\KeywordTok{>}
\KeywordTok{250} \NormalTok{2.1.0 Sender OK}
\KeywordTok{RCPT} \NormalTok{TO:}\KeywordTok{<}\NormalTok{youruser@yourdomain.com}\KeywordTok{>}
\KeywordTok{250} \NormalTok{2.1.5 Recipient OK}
\KeywordTok{DATA}
\KeywordTok{354} \NormalTok{Start mail input}\KeywordTok{;} \KeywordTok{end} \NormalTok{with }\KeywordTok{<}\NormalTok{CRLF}\KeywordTok{>}\NormalTok{.}\KeywordTok{<}\NormalTok{CRLF}\KeywordTok{>}
\KeywordTok{This} \NormalTok{is an email. A single period terminates it.}
\KeywordTok{.}
\KeywordTok{250} \NormalTok{2.6.0 }\KeywordTok{<}\NormalTok{ea43bfd5-5f3f-4335-9c3e-e739f196c56f@MAIL.MYDOMAIN.COM}\KeywordTok{>}
\KeywordTok{Queued} \NormalTok{mail for delivery}
\end{Highlighting}
\end{Shaded}

In the above, after entering \texttt{telnet\ smtp\ 25} (our internal
email DNS CNAME is \texttt{smtp}), I entered:

\begin{itemize}
\item
  \textbf{\texttt{HELO}} - the starting command for the protocol.
\item
  \textbf{\texttt{MAIL\ FROM:\textless{}myuser@mydomain.com\textgreater{}}}
  - the "from" email address.
\item
  \textbf{\texttt{RCPT\ TO:\textless{}youruser@yourdomain.com\textgreater{}}}
  - the "to" email address.
\item
  \textbf{\texttt{DATA}} - indicating I am ready to start the email
  body, which is
  \texttt{This\ is\ an\ email,\ a\ single\ period\ terminates\ it.} And
  you will notice there is a single period on the following line, which
  tells the SMTP server the body is done and to send the email. If all
  is successful, then an email should shortly show up in the inbox of
  \texttt{youruser@yourdomain.com}.
\end{itemize}

\textbf{NOTE:} - diagnosing SMTP connectivity like this can be
\textbf{\emph{very}} handy sometimes. It is a good tool to have in your
toolchest, and you can do it under Cygwin or a PuTTY Telnet session on a
Windows box just as easily as from a Linux machine.

To get a modern, \emph{secure shell} to a remote machine over an
encrypted connection, use
\href{http://linux.die.net/man/1/ssh}{\texttt{ssh}}, passing in the
userid and server like this: \drnet{ssh}

\drcap{\texttt{ssh} command}

\begin{Shaded}
\begin{Highlighting}[]
\KeywordTok{ssh} \NormalTok{myuser@remoteserver}
\end{Highlighting}
\end{Shaded}

You will be prompted for credentials (or you can use certificates, but
that is \textbf{\emph{way}} beyond this text's goals). Once logged in,
you will be presented with a command prompt to the remote system.

You can also use the \texttt{SSH} protocol to \emph{securely copy} files
between systems with the
\href{http://linux.die.net/man/1/scp}{\texttt{scp}} command. It works
like this for a recursive directory copy: \drnet{scp}

\drcap{\texttt{scp} command}

\begin{Shaded}
\begin{Highlighting}[]
\KeywordTok{scp} \NormalTok{-r myfiles/* myuser@remoteserver:/home/myuser/.}
\end{Highlighting}
\end{Shaded}

In this case we are copying the files in \texttt{myfiles} and its
subdirectories to \texttt{/home/myuser/} on \texttt{remoteserver} logged
in as \texttt{myuser}.

\textbf{Note:} The first time you log into a remote server with
\texttt{ssh} or \texttt{scp} you will be asked to accept the remote
server's "fingerprint." You can usually just say "yes":

\drcap{Sample \texttt{ssh} session}

\begin{Shaded}
\begin{Highlighting}[]
\KeywordTok{~}\NormalTok{# }\KeywordTok{ssh} \NormalTok{myuser@remotehost}
\KeywordTok{The} \NormalTok{authenticity of host }\StringTok{'[remotehost] ([10.0.2.3]:22)'} \NormalTok{can}\StringTok{'t be established.}
\StringTok{ECDSA key fingerprint is 98:bb:17:38:ee:d0:16:ee:b2:93:08:4e:30:25:14:70.}
\StringTok{Are you sure you want to continue connecting (yes/no)? yes}
\StringTok{Warning: Permanently added '}\NormalTok{[remotehost],[10.0.2.3]:22}\StringTok{' (ECDSA) to the list}
\StringTok{of known hosts.}
\StringTok{myuser@remotehost'}\NormalTok{s password:}
\KeywordTok{Linux} \NormalTok{remotehost 3.2.0-4-amd64 }\CommentTok{#1 SMP Debian 3.2.65-1+deb7u2 x86_64}

\KeywordTok{The} \NormalTok{programs included with the Debian GNU/Linux system are free software}\KeywordTok{;}
\KeywordTok{the} \NormalTok{exact distribution terms for each program are described in the}
\KeywordTok{individual} \NormalTok{files in /usr/share/doc/*/copyright.}

\KeywordTok{Debian} \NormalTok{GNU/Linux comes with ABSOLUTELY NO WARRANTY, to the extent}
\KeywordTok{permitted} \NormalTok{by applicable law.}
\KeywordTok{Last} \NormalTok{login: Tue Oct 20 09:37:10 2015 from otherhost}
\NormalTok{$}
\end{Highlighting}
\end{Shaded}

The \texttt{scp} command does a naive copy, that is, it simply copies
everything from the source to the target. But what if you want to
"mirror" the source to the target, meaning that if you delete something
on the source side, it will get deleted on the target as well?
\texttt{scp} won't help you with that. You could delete the target
directory completely each time before copying, but that may not be safe
(what happens if the copy fails after you've deleted the target?)
Another problem with \texttt{scp} is it copies every byte of every file
every time, even if the files haven't changed. That can be really slow
and inefficient when copying a large, complex directory containing
gigabytes of files over the Internet.

This is where the
\href{http://linux.die.net/man/1/rsync}{\texttt{rsync}} command comes in
handy. \texttt{rsync} does three things well: \drnet{rsync}

\begin{enumerate}
\def\labelenumi{\arabic{enumi}.}
\item
  It copies using the \texttt{ssh} protocol by default, just like
  \texttt{scp}, so it is secure. And it uses the \texttt{scp} commands
  compression capabilities by default to increase efficiency. You can
  turn that behavior off via a parameter when copying files that are
  already compressed, such as JPEGs.
\item
  It copies only the \textbf{\emph{changed blocks}} (not even just the
  changed files), which means if there are no or few changes, it is very
  fast.
\item
  It deletes files on the target if they've been deleted from the
  source, which allows you to mirror all changes between the two servers
  or sites.
\end{enumerate}

\textbf{NOTE}: For \texttt{rsync} to work, it has to be installed on
both servers.

\drcap{Using \texttt{rsync} to mirror directories between servers}

\begin{Shaded}
\begin{Highlighting}[]
\KeywordTok{~} \NormalTok{$ }\KeywordTok{rsync} \NormalTok{--delete --progress --recursive --verbose --exclude }\StringTok{'.git'} \NormalTok{\textbackslash{}}
\NormalTok{--exclude }\StringTok{'.bak'} \NormalTok{~/dev/website/ myuser@mysite.com:public_html}
\end{Highlighting}
\end{Shaded}

This invokes \texttt{rsync} as follows:

\begin{itemize}
\item
  \textbf{\texttt{-\/-delete}} - deletes files on the target server.
\item
  \textbf{\texttt{-\/-progress}} - shows progress as it copies.
\item
  \textbf{\texttt{-\/-recursive}} - does a recursive copy.
\item
  \textbf{\texttt{-\/-verbose}} - shows verbose output.
\item
  \textbf{\texttt{-\/-exclude\ \textquotesingle{}.git\textquotesingle{}}}
  - exclude any directory or file named \texttt{.git}.
\item
  \textbf{\texttt{-\/-exclude\ \textquotesingle{}.bak\textquotesingle{}}}
  - also exclude any directory or file named \texttt{.bak}.
\item
  \textbf{\texttt{\textasciitilde{}/dev/website/}} - copy from this
  directory on the local machine, including all files under it (trailing
  slash).
\item
  \textbf{\texttt{myuser@mysite.com:public\_html}} - copy to the
  \texttt{public\_html} directory under the home directory of
  \texttt{myuser} on the \texttt{mysite.com} server.
\end{itemize}

\textbf{NOTE:} \texttt{rsync} can be used to efficiently mirror
directories even on the local server. If there are two large and complex
directories on a single server you want to keep synchronized, you can
simply use local addresses for both directories:

\drcap{Using \texttt{rsync} to mirror local directories}

\begin{Shaded}
\begin{Highlighting}[]
\KeywordTok{~} \NormalTok{$ }\KeywordTok{rsync} \NormalTok{--delete --progress --recursive --verbose --exclude }\StringTok{'.git'} \NormalTok{\textbackslash{}}
\NormalTok{--exclude }\StringTok{'.bak'} \NormalTok{~/dev/website/ /var/nginx/staging/website}
\end{Highlighting}
\end{Shaded}

\section*{Network Configuration}\label{network-configuration}
\addcontentsline{toc}{section}{Network Configuration}

We won't dive too deep into configuring a network, but there are a few
things you should know about right away. The first is the
\href{http://linux.die.net/man/8/ifconfig}{\texttt{ifconfig}} command.
In some ways is similar to \texttt{ipconfig} in \texttt{CMD.EXE}. While
you can use \texttt{ifconfig} to alter your networking settings, it is
most commonly used to get a quick display of them: \drnet{ifconfig}
\drshl{CMD.EXE}

\drcap{\texttt{ifconfig} command}

\begin{Shaded}
\begin{Highlighting}[]
\CommentTok{# ifconfig}
\KeywordTok{eth0}      \NormalTok{Link encap:Ethernet  HWaddr 00:00:56:a3:35:fe}
          \KeywordTok{inet} \NormalTok{addr:10.0.2.3  Bcast:10.0.2.255  Mask:255.255.252.0}
          \KeywordTok{inet6} \NormalTok{addr: fe80::255:56ff:fea3:35fe/64 Scope:Link}
          \KeywordTok{UP} \NormalTok{BROADCAST RUNNING MULTICAST  MTU:1500  Metric:1}
          \KeywordTok{RX} \NormalTok{packets:364565022 errors:0 dropped:386406 overruns:0 frame:0}
          \KeywordTok{TX} \NormalTok{packets:35097654 errors:0 dropped:0 overruns:0 carrier:0}
          \KeywordTok{collisions}\NormalTok{:0 txqueuelen:1000}
          \KeywordTok{RX} \NormalTok{bytes:34727642861 (32.3 GiB)  }\KeywordTok{TX} \NormalTok{bytes:195032017498 (181.6 GiB)}

\KeywordTok{lo}        \NormalTok{Link encap:Local Loopback}
          \KeywordTok{inet} \NormalTok{addr:127.0.0.1  Mask:255.0.0.0}
          \KeywordTok{inet6} \NormalTok{addr: ::1/128 Scope:Host}
          \KeywordTok{UP} \NormalTok{LOOPBACK RUNNING  MTU:16436  Metric:1}
          \KeywordTok{RX} \NormalTok{packets:111207 errors:0 dropped:0 overruns:0 frame:0}
          \KeywordTok{TX} \NormalTok{packets:111207 errors:0 dropped:0 overruns:0 carrier:0}
          \KeywordTok{collisions}\NormalTok{:0 txqueuelen:0}
          \KeywordTok{RX} \NormalTok{bytes:6839306 (6.5 MiB)  }\KeywordTok{TX} \NormalTok{bytes:6839306 (6.5 MiB)}
\end{Highlighting}
\end{Shaded}

To see what DNS servers the system is using, you can look in
\texttt{/etc/resolv.conf}:
\index{files and directories!special!etcresolvconf@\texttt{/etc/resolv.conf}}

\drcap{DNS servers in \texttt{resolv.conf}}

\begin{Shaded}
\begin{Highlighting}[]
\CommentTok{# cat /etc/resolv.conf}
\KeywordTok{domain} \NormalTok{mydomain.com}
\KeywordTok{search} \NormalTok{mydomain.com}
\KeywordTok{nameserver} \NormalTok{10.0.2.1}
\KeywordTok{nameserver} \NormalTok{10.0.2.2}
\end{Highlighting}
\end{Shaded}

And to see any local overrides of network names or aliases, look in
\texttt{/etc/hosts}:
\index{files and directories!special!etchosts@\texttt{/etc/hosts}}

\drcap{\texttt{hosts} file}

\begin{Shaded}
\begin{Highlighting}[]
\CommentTok{# cat /etc/hosts}
\KeywordTok{127.0.0.1}       \NormalTok{localhost}
\end{Highlighting}
\end{Shaded}

\textbf{Note:} The UNIX \texttt{/etc/hosts} file is the basis for the
Windows version located at
\texttt{C:\textbackslash{}Windows\textbackslash{}System32\textbackslash{}drivers\textbackslash{}etc\textbackslash{}hosts},
and has similar syntax.

\hypertarget{the-man-behind-the-curtain}{\chapter{The Man Behind the
Curtain}\label{the-man-behind-the-curtain}}

\textbf{\emph{\texttt{/proc}, \texttt{/dev}, \texttt{ps},
\texttt{/var/log}, \texttt{/tmp} and other things under the covers.}}

\begin{quote}
\emph{"As always, should any member of your team be caught or killed,
the Secretary will disavow all knowledge of your actions."} - voice on
tape (\emph{Mission: Impossible})
\end{quote}

This section will cover some "background" techniques that are valuable
for system monitoring, problem determination and the like. Depending on
your role and access levels, some of these commands may not be available
to you, or may require \texttt{sudo} or \texttt{root} access.

\section*{All Part of the Process}\label{all-part-of-the-process}
\addcontentsline{toc}{section}{All Part of the Process}

To see what \emph{processes} you are running, use
\href{http://linux.die.net/man/1/ps}{\texttt{ps}}:
\drsys{ps}{list processes}

\drcap{\texttt{ps} command}

\begin{Shaded}
\begin{Highlighting}[]
\CommentTok{# ps}
  \KeywordTok{PID} \NormalTok{TTY          TIME CMD}
\KeywordTok{14691} \NormalTok{pts/0    00:00:00 bash}
\KeywordTok{25530} \NormalTok{pts/0    00:00:00 ps}
\end{Highlighting}
\end{Shaded}

To show processes from \emph{all} users in a process \emph{hierarchy}
(child processes indented under parents) use \texttt{ps\ -AH}:

\drcap{Showing all processes}

\begin{Shaded}
\begin{Highlighting}[]
\KeywordTok{~} \NormalTok{$ }\KeywordTok{ps} \NormalTok{-AH}
  \KeywordTok{PID} \NormalTok{TTY          TIME CMD}
    \KeywordTok{2} \NormalTok{?        00:00:00 kthreadd}
    \KeywordTok{3} \NormalTok{?        00:00:00   ksoftirqd/0}
    \KeywordTok{5} \NormalTok{?        00:00:00   kworker/0:0H}
    \KeywordTok{7} \NormalTok{?        00:00:06   rcu_sched}
    \KeywordTok{8} \NormalTok{?        00:00:02   rcuos/0}
    \KeywordTok{9} \NormalTok{?        00:00:01   rcuos/1}
   \KeywordTok{10} \NormalTok{?        00:00:03   rcuos/2}
   \KeywordTok{11} \NormalTok{?        00:00:01   rcuos/3}
   \KeywordTok{12} \NormalTok{?        00:00:00   rcuos/4}
   \KeywordTok{13} \NormalTok{?        00:00:00   rcuos/5}
   \KeywordTok{14} \NormalTok{?        00:00:00   rcuos/6}
   \KeywordTok{15} \NormalTok{?        00:00:00   rcuos/7}
   \KeywordTok{16} \NormalTok{?        00:00:00   rcu_bh}
   \KeywordTok{17} \NormalTok{?        00:00:00   rcuob/0}
   \KeywordTok{18} \NormalTok{?        00:00:00   rcuob/1}
   \KeywordTok{19} \NormalTok{?        00:00:00   rcuob/2}
   \KeywordTok{20} \NormalTok{?        00:00:00   rcuob/3}
   \KeywordTok{21} \NormalTok{?        00:00:00   rcuob/4}
   \KeywordTok{22} \NormalTok{?        00:00:00   rcuob/5}
   \KeywordTok{23} \NormalTok{?        00:00:00   rcuob/6}
   \KeywordTok{24} \NormalTok{?        00:00:00   rcuob/7}
\KeywordTok{...and} \NormalTok{so on...}
\end{Highlighting}
\end{Shaded}

You can \emph{kill} a process using the
\href{http://linux.die.net/man/1/kill}{\texttt{kill}} command, which
takes a process id and optionally a "signal"\index{signals}. Here is an
example looking for any running instance of \texttt{vi} and sending it a
\texttt{kill} command: \drsys{kill}{terminate process}

\drcap{Hunting down and killing \texttt{vi} sessions}

\begin{Shaded}
\begin{Highlighting}[]
\KeywordTok{ps} \NormalTok{-A }\KeywordTok{|} \KeywordTok{grep} \NormalTok{vi }\KeywordTok{|} \KeywordTok{kill} \KeywordTok{`cut} \NormalTok{-f2 -d}\StringTok{" "}\KeywordTok{`}
\end{Highlighting}
\end{Shaded}

That's:

\begin{itemize}
\item
  \textbf{\texttt{ps\ -A}} - list all running processes.
\item
  \textbf{\texttt{\textbar{}}} - pipe \emph{stdout} from \texttt{ps} to
  next command. \drstd{stdout}
\item
  \textbf{\texttt{grep\ vi}} - find all instances of \texttt{vi} (be
  careful, because that would include \texttt{view} and anything else
  containing the string \texttt{vi}, too).
\item
  \textbf{\texttt{\textbar{}}} - pipe \emph{stdout} from \texttt{grep}
  to next command. \drfnd{grep}{search files}
\item
  \textbf{\texttt{kill}} - send a \texttt{SIGINT} signal to a process
  specified by: \index{signals}
\item
  \textbf{\texttt{`cut\ -f2\ -d"\ "`}} - execute the
  \href{http://linux.die.net/man/1/cut}{\texttt{cut}} command and take
  the second space-delimited field (in this case the process id - the
  first "field" is just leading spaces), and place the results of the
  command execution as the parameter to the \texttt{kill} command.
  \drtxt{cut}
\end{itemize}

To monitor the ongoing CPU, memory and other resource utilization of the
\emph{top} processes, you use the
\href{http://linux.die.net/man/1/top}{\texttt{top}} command, which
unlike most in this book updates dynamically every second by default:
\drsys{top}{list processes by resource use}

\drcap{\texttt{top} command}

\begin{Shaded}
\begin{Highlighting}[]
\KeywordTok{top} \NormalTok{- 14:11:26 up 106 days,  5:24,  2 users,  load average: 0.11, 0.05, ...}
\KeywordTok{Tasks}\NormalTok{:  95 total,   1 running,  94 sleeping,   0 stopped,   0 zombie}
\KeywordTok{%Cpu}\NormalTok{(s)}\KeywordTok{:}  \NormalTok{0.2 us,  0.8 sy,  0.0 ni, 99.0 id,  0.0 wa,  0.0 hi,  0.0 si, ...}
\KeywordTok{KiB} \NormalTok{Mem:   2061136 total,  1909468 used,   151668 free,   151632 buffers}
\KeywordTok{KiB} \NormalTok{Swap:  4191228 total,   287620 used,  3903608 free,   654900 cached}

  \KeywordTok{PID} \NormalTok{USER      PR  NI  VIRT  RES  SHR S  %CPU %MEM    TIME+  COMMAND}
 \KeywordTok{9715} \NormalTok{git       20   0  525m 230m 4376 S   0.7 11.4  10:11.44 ruby}
 \KeywordTok{9171} \NormalTok{git       20   0  520m 229m 4672 S   0.3 11.4  10:27.97 ruby}
\KeywordTok{22899} \NormalTok{root      20   0     0    0    0 S   0.3  0.0   0:30.16 kworker/1:0}
    \KeywordTok{1} \NormalTok{root      20   0 10648  584  560 S   0.0  0.0   1:02.60 init}
    \KeywordTok{2} \NormalTok{root      20   0     0    0    0 S   0.0  0.0   0:00.00 kthreadd}
    \KeywordTok{3} \NormalTok{root      20   0     0    0    0 S   0.0  0.0   0:38.05 ksoftirqd/0}
    \KeywordTok{5} \NormalTok{root      20   0     0    0    0 S   0.0  0.0   0:00.00 kworker/u:0}
    \KeywordTok{6} \NormalTok{root      rt   0     0    0    0 S   0.0  0.0   0:12.23 migration/0}
    \KeywordTok{7} \NormalTok{root      rt   0     0    0    0 S   0.0  0.0   0:24.83 watchdog/0}
    \KeywordTok{8} \NormalTok{root      rt   0     0    0    0 S   0.0  0.0   0:13.01 migration/1}
   \KeywordTok{10} \NormalTok{root      20   0     0    0    0 S   0.0  0.0   0:34.55 ksoftirqd/1}
   \KeywordTok{12} \NormalTok{root      rt   0     0    0    0 S   0.0  0.0   0:21.38 watchdog/1}
   \KeywordTok{13} \NormalTok{root       0 -20     0    0    0 S   0.0  0.0   0:00.00 cpuset}
\KeywordTok{...and} \NormalTok{so on...}
\end{Highlighting}
\end{Shaded}

\textbf{Note:} Use \texttt{Q} or \texttt{Ctrl-C} to exit \texttt{top}.

\section*{When All You Have is a
Hammer}\label{when-all-you-have-is-a-hammer}
\addcontentsline{toc}{section}{When All You Have is a Hammer}

Remember that one of the primary UNIX philosophies is that everything is
a file or can be made to look like a file, including network streams,
device output and the like. This is a really powerful concept, because
it allows you to access things with tools that have \textbf{\emph{no
idea}} what they are working on, as long as it "looks like" a file (or
stream of text).

One of the places this has become really handy is in the \texttt{/proc}
"file system." On modern Linux systems, there is typically a
\texttt{/proc} directory that looks like directories and files:
\index{files and directories!special!proc@\texttt{/proc/}}

\drcap{\texttt{/proc} file system}

\begin{Shaded}
\begin{Highlighting}[]
\KeywordTok{~} \NormalTok{$ }\KeywordTok{ls} \NormalTok{/proc}
\KeywordTok{1}     \NormalTok{1566  2607  299   4549  53    75         cmdline      mtrr}
\KeywordTok{10}    \NormalTok{1587  2617  3     4579  54    754        consoles     net}
\KeywordTok{100}   \NormalTok{16    2627  300   4589  55    760        cpuinfo      pagetypeinfo}
\KeywordTok{1022}  \NormalTok{17    2629  301   46    56    762        crypto       partitions}
\KeywordTok{1030}  \NormalTok{18    2699  3029  4602  575   764        devices      sched_debug}
\KeywordTok{1035}  \NormalTok{1803  27    31    4612  61    77         diskstats    schedstat}
\KeywordTok{1038}  \NormalTok{19    2712  3111  47    6146  79         dma          scsi}
\KeywordTok{11}    \NormalTok{2     2799  3112  48    6153  8          driver       self}
\KeywordTok{12}    \NormalTok{20    28    3116  49    6199  8167       execdomains  slabinfo}
\KeywordTok{1295}  \NormalTok{2073  2802  3117  4955  62    8168       fb           softirqs}
\KeywordTok{1297}  \NormalTok{2077  2811  3150  4958  6212  8200       filesystems  stat}
\KeywordTok{13}    \NormalTok{21    2815  32    4960  63    822        fs           swaps}
\KeywordTok{1304}  \NormalTok{22    2820  324   4976  640   8296       interrupts   sys}
\KeywordTok{1305}  \NormalTok{23    2823  326   5     642   9          iomem        sysrq-trigger}
\KeywordTok{1306}  \NormalTok{2324  2825  329   50    645   9266       ioports      sysvipc}
\KeywordTok{1308}  \NormalTok{2349  2829  33    5005  6463  927        irq          timer_list}
\KeywordTok{1311}  \NormalTok{2356  2831  330   5012  647   939        kallsyms     timer_stats}
\KeywordTok{14}    \NormalTok{24    2836  34    5033  649   9465       kcore        tty}
\KeywordTok{1408}  \NormalTok{2494  2846  36    5045  661   9613       keys         uptime}
\KeywordTok{1468}  \NormalTok{25    2847  37    51    665   9796       key-users    version}
\KeywordTok{147}   \NormalTok{2507  2848  3713  511   676   9850       kmsg         version_signature}
\KeywordTok{148}   \NormalTok{2518  2850  374   5122  686   99         kpagecount   vmallocinfo}
\KeywordTok{...and} \NormalTok{so on...}
\end{Highlighting}
\end{Shaded}

What is all that? Well, look a little closer:

\drcap{Detailed listing of the \texttt{/proc} file system}

\begin{Shaded}
\begin{Highlighting}[]
\KeywordTok{~} \NormalTok{$ }\KeywordTok{ls} \NormalTok{-l /proc}
\KeywordTok{total} \NormalTok{0}
\KeywordTok{dr-xr-xr-x}  \NormalTok{9 root       root                     0 Dec 22 06:06 1}
\KeywordTok{dr-xr-xr-x}  \NormalTok{9 root       root                     0 Dec 22 06:06 10}
\KeywordTok{dr-xr-xr-x}  \NormalTok{9 root       root                     0 Dec 22 06:06 100}
\KeywordTok{dr-xr-xr-x}  \NormalTok{9 myuser     mygroup                  0 Dec 22 10:17 10035}
\KeywordTok{dr-xr-xr-x}  \NormalTok{9 root       root                     0 Dec 22 06:06 1022}
\KeywordTok{dr-xr-xr-x}  \NormalTok{9 root       root                     0 Dec 22 06:06 1030}
\KeywordTok{dr-xr-xr-x}  \NormalTok{9 root       root                     0 Dec 22 06:06 1035}
\KeywordTok{dr-xr-xr-x}  \NormalTok{9 root       root                     0 Dec 22 06:06 1038}
\KeywordTok{dr-xr-xr-x}  \NormalTok{9 root       root                     0 Dec 22 06:06 11}
\KeywordTok{dr-xr-xr-x}  \NormalTok{9 root       root                     0 Dec 22 06:06 12}
\KeywordTok{dr-xr-xr-x}  \NormalTok{9 root       root                     0 Dec 22 06:06 1295}
\KeywordTok{dr-xr-xr-x}  \NormalTok{9 root       root                     0 Dec 22 06:06 1297}
\KeywordTok{dr-xr-xr-x}  \NormalTok{9 root       root                     0 Dec 22 06:06 13}
\KeywordTok{dr-xr-xr-x}  \NormalTok{9 root       root                     0 Dec 22 06:06 1304}
\KeywordTok{dr-xr-xr-x}  \NormalTok{9 root       root                     0 Dec 22 06:06 1305}
\KeywordTok{dr-xr-xr-x}  \NormalTok{9 root       root                     0 Dec 22 06:06 1306}
\KeywordTok{dr-xr-xr-x}  \NormalTok{9 root       root                     0 Dec 22 06:06 1308}
\KeywordTok{dr-xr-xr-x}  \NormalTok{9 root       root                     0 Dec 22 06:06 1311}
\KeywordTok{dr-xr-xr-x}  \NormalTok{9 root       root                     0 Dec 22 06:06 14}
\KeywordTok{dr-xr-xr-x}  \NormalTok{9 root       root                     0 Dec 22 06:06 1408}
\KeywordTok{dr-xr-xr-x}  \NormalTok{9 root       root                     0 Dec 22 06:06 1468}
\KeywordTok{...and} \NormalTok{so on...}
\end{Highlighting}
\end{Shaded}

We can see that the entries with numeric names are directories. Let's
look in one of those directories:

\drcap{Looking inside one of the \texttt{/proc} process directories}

\begin{Shaded}
\begin{Highlighting}[]
\KeywordTok{~} \CommentTok{# ls -l /proc/1}
\KeywordTok{total} \NormalTok{0}
\KeywordTok{dr-xr-xr-x} \NormalTok{2 root root 0 Dec 22 10:18 attr}
\KeywordTok{-rw-r--r--} \NormalTok{1 root root 0 Dec 22 10:18 autogroup}
\KeywordTok{-r--------} \NormalTok{1 root root 0 Dec 22 10:18 auxv}
\KeywordTok{-r--r--r--} \NormalTok{1 root root 0 Dec 22 06:06 cgroup}
\KeywordTok{--w-------} \NormalTok{1 root root 0 Dec 22 10:18 clear_refs}
\KeywordTok{-r--r--r--} \NormalTok{1 root root 0 Dec 22 06:06 cmdline}
\KeywordTok{-rw-r--r--} \NormalTok{1 root root 0 Dec 22 10:18 comm}
\KeywordTok{-rw-r--r--} \NormalTok{1 root root 0 Dec 22 10:18 coredump_filter}
\KeywordTok{-r--r--r--} \NormalTok{1 root root 0 Dec 22 10:18 cpuset}
\KeywordTok{lrwxrwxrwx} \NormalTok{1 root root 0 Dec 22 10:18 cwd -}\KeywordTok{>} \NormalTok{/}
\KeywordTok{-r--------} \NormalTok{1 root root 0 Dec 22 06:06 environ}
\KeywordTok{lrwxrwxrwx} \NormalTok{1 root root 0 Dec 22 06:06 exe -}\KeywordTok{>} \NormalTok{/sbin/init}
\KeywordTok{dr-x------} \NormalTok{2 root root 0 Dec 22 10:18 fd}
\KeywordTok{dr-x------} \NormalTok{2 root root 0 Dec 22 10:18 fdinfo}
\KeywordTok{-rw-r--r--} \NormalTok{1 root root 0 Dec 22 10:18 gid_map}
\KeywordTok{-r--------} \NormalTok{1 root root 0 Dec 22 10:18 io}
\KeywordTok{-r--r--r--} \NormalTok{1 root root 0 Dec 22 06:06 limits}
\KeywordTok{-rw-r--r--} \NormalTok{1 root root 0 Dec 22 10:18 loginuid}
\KeywordTok{dr-x------} \NormalTok{2 root root 0 Dec 22 10:18 map_files}
\KeywordTok{-r--r--r--} \NormalTok{1 root root 0 Dec 22 10:18 maps}
\KeywordTok{-rw-------} \NormalTok{1 root root 0 Dec 22 10:18 mem}
\KeywordTok{...and} \NormalTok{so on...}
\end{Highlighting}
\end{Shaded}

This contains a lot of information on the process with process id (PID)
\#1. If the directory listing shows the entry as a file, it can be
examined and holds \textbf{\emph{current}} statistics for whatever the
file name implies:

\drcap{How much I/O has process 1 done?}

\begin{Shaded}
\begin{Highlighting}[]
\KeywordTok{~} \CommentTok{# cat /proc/1/io}
\KeywordTok{rchar}\NormalTok{: 803882767}
\KeywordTok{wchar}\NormalTok{: 152731542}
\KeywordTok{syscr}\NormalTok{: 201510}
\KeywordTok{syscw}\NormalTok{: 57855}
\KeywordTok{read_bytes}\NormalTok{: 663872512}
\KeywordTok{write_bytes}\NormalTok{: 113012736}
\KeywordTok{cancelled_write_bytes}\NormalTok{: 3072000}
\end{Highlighting}
\end{Shaded}

If it is a directory it will hold other entries (files or directories)
with yet more statistics.

In addition, there are system-wide statistics, such as
\texttt{/proc/cpuinfo}:

\drcap{Looking at CPU info in \texttt{/proc/cpuinfo}}

\begin{Shaded}
\begin{Highlighting}[]
\KeywordTok{~} \CommentTok{# cat /proc/cpuinfo }
\KeywordTok{processor}   \NormalTok{: 0}
\KeywordTok{vendor_id}   \NormalTok{: GenuineIntel}
\KeywordTok{cpu} \NormalTok{family  : 6}
\KeywordTok{model}       \NormalTok{: 69}
\KeywordTok{model} \NormalTok{name  : Intel(R) }\KeywordTok{Core}\NormalTok{(TM) }\KeywordTok{i5-4200U} \NormalTok{CPU @ 1.60GHz}
\KeywordTok{stepping}    \NormalTok{: 1}
\KeywordTok{microcode}   \NormalTok{: 0x14}
\KeywordTok{cpu} \NormalTok{MHz     : 895.023}
\KeywordTok{cache} \NormalTok{size  : 3072 KB}
\KeywordTok{physical} \NormalTok{id : 0}
\KeywordTok{siblings}    \NormalTok{: 4}
\KeywordTok{core} \NormalTok{id     : 0}
\KeywordTok{cpu} \NormalTok{cores   : 2}
\KeywordTok{apicid}      \NormalTok{: 0}
\KeywordTok{initial} \NormalTok{apicid  : 0}
\KeywordTok{fpu}     \NormalTok{: yes}
\KeywordTok{fpu_exception}   \NormalTok{: yes}
\KeywordTok{cpuid} \NormalTok{level : 13}
\KeywordTok{wp}      \NormalTok{: yes}
\KeywordTok{flags}       \NormalTok{: fpu vme de pse tsc msr pae mce cx8 apic sep mtrr pge mca cmov }
\KeywordTok{pat} \NormalTok{pse36 clflush dts acpi mmx fxsr sse sse2 ss ht tm pbe syscall nx pd}
\KeywordTok{pe1gb} \NormalTok{rdtscp lm constant_tsc arch_perfmon pebs bts rep_good nopl xtopol}
\KeywordTok{ogy} \NormalTok{nonstop_tsc ap}
\KeywordTok{...and} \NormalTok{so on...}
\end{Highlighting}
\end{Shaded}

\section*{Sawing Logs}\label{sawing-logs}
\addcontentsline{toc}{section}{Sawing Logs}

Many Linux components and subsystems log to \texttt{/var/log}. Here is a
pretty standard directory listing for it on a Linux Mint system:
\index{files and directories!special!varlog@\texttt{/var/log/}}

\drcap{Looking at logs}

\begin{Shaded}
\begin{Highlighting}[]
\KeywordTok{~} \CommentTok{# ls /var/log}
\KeywordTok{alternatives.log}       \NormalTok{dmesg.4.gz             pm-suspend.log.1}
\KeywordTok{alternatives.log.1}     \NormalTok{dpkg.log               pm-suspend.log.2.gz}
\KeywordTok{alternatives.log.2.gz}  \NormalTok{dpkg.log.1             pm-suspend.log.3.gz}
\KeywordTok{alternatives.log.3.gz}  \NormalTok{dpkg.log.2.gz          pycentral.log}
\KeywordTok{apt}                    \NormalTok{dpkg.log.3.gz          samba}
\KeywordTok{aptitude}               \NormalTok{faillog                speech-dispatcher}
\KeywordTok{aptitude.1.gz}          \NormalTok{fontconfig.log         syslog}
\KeywordTok{aptitude.2.gz}          \NormalTok{fsck                   syslog.1}
\KeywordTok{auth.log}               \NormalTok{gpu-manager.log        syslog.2.gz}
\KeywordTok{auth.log.1}             \NormalTok{hp                     syslog.3.gz}
\KeywordTok{auth.log.2.gz}          \NormalTok{installer              syslog.4.gz}
\KeywordTok{auth.log.3.gz}          \NormalTok{kern.log               syslog.5.gz}
\KeywordTok{auth.log.4.gz}          \NormalTok{kern.log.1             syslog.6.gz}
\KeywordTok{boot.log}               \NormalTok{kern.log.2.gz          syslog.7.gz}
\KeywordTok{bootstrap.log}          \NormalTok{kern.log.3.gz          udev}
\KeywordTok{btmp}                   \NormalTok{kern.log.4.gz          unattended-upgrades}
\KeywordTok{btmp.1}                 \NormalTok{lastlog                upstart}
\KeywordTok{ConsoleKit}             \NormalTok{mdm                    wtmp}
\KeywordTok{cups}                   \NormalTok{mintsystem.log         wtmp.1}
\KeywordTok{dmesg}                  \NormalTok{pm-powersave.log       Xorg.0.log}
\KeywordTok{dmesg.0}                \NormalTok{pm-powersave.log.1     Xorg.0.log.old}
\KeywordTok{dmesg.1.gz}             \NormalTok{pm-powersave.log.2.gz  Xorg.20.log}
\KeywordTok{...and} \NormalTok{so on...}
\end{Highlighting}
\end{Shaded}

Some, like \texttt{samba} are their own subdirectories with log files
under that. Others are log files that get "rotated" from the most
current (no suffix) through ever older ones (increasing suffix number,
e.g., \texttt{mail.log.2}).

If you are pursuing a problem with a specific subsystem (like
\texttt{samba}), it is good to start in its log files. The two log files
of general interest are \texttt{dmesg}, which holds kernel-level debug
messages and usually is useful for debugging things like device driver
issues. It can also be displayed directly with the
\href{http://linux.die.net/man/8/dmesg}{\texttt{dmesg}} command. The
other is \texttt{messages}, which holds more general "system" messages.
\drsys{dmesg}{display kernel log}
\index{files and directories!special!varlogdmesg@\texttt{/var/log/dmesg}}
\index{files and directories!special!varlogmessages@\texttt{/var/log/messages}}
\index{varlogdmesg@\texttt{/var/log/dmesg} (kernel log)}
\index{varlogmessages@\texttt{/var/log/messages} (general log)}

Let's look for kernel errors when booting:

\drcap{Kernel errors when booting}

\begin{Shaded}
\begin{Highlighting}[]
\KeywordTok{~} \CommentTok{# cat /var/log/dmesg | grep -i error}
\KeywordTok{[}   \NormalTok{15.828463] EXT4-fs (dm-1): re-mounted. Opts: errors=remount-ro}
\end{Highlighting}
\end{Shaded}

\section*{It's All Temporary}\label{its-all-temporary}
\addcontentsline{toc}{section}{It's All Temporary}

By convention, temporary files are written to \texttt{/tmp} . You can
place your own temporary or "work" files there, too. It's a great place
to unzip install bits, for example. Just note that the temporariness is
enforced in that when the system reboots, \texttt{/tmp} is reset to
empty. \index{files and directories!special!tmp@\texttt{/tmp/}}

\hypertarget{how-do-you-know-what-you-dont-know-man}{\chapter{\texorpdfstring{How
Do You Know What You Don’t Know,
\texttt{man}?}{How Do You Know What You Don’t Know, man?}}\label{how-do-you-know-what-you-dont-know-man}}

\textbf{\emph{\texttt{man}, \texttt{info}, \texttt{apropos}, Linux
Documentation Project, Debian and Arch guides, StackOverflow and the
dangers of searching for “\texttt{man\ find}” or “\texttt{man\ touch}”
on the internet.}}

\begin{quote}
\emph{"You're soaking in it."} - Palmolive commercial
\end{quote}

The biggest issue with bootstrapping into "UNIX" is not the lack of
documentation but almost the surplus of it, coupled with a severe "RTFM"
attitude by most old-timers toward newbies. Besides the typical "Google"
and \href{http://unix.stackexchange.com/}{"StackOverflow"} answers,
there are actually lots of very reliable places to turn to for
information
\index{documentation!http://unix.stackexchange.com/@\textit{http://unix.stackexchange.com/}}

\section*{\texorpdfstring{\texttt{man}, is that \texttt{info}
\texttt{apropos}?}{man, is that info apropos?}}\label{man-is-that-info-apropos}
\addcontentsline{toc}{section}{\texttt{man}, is that \texttt{info}
\texttt{apropos}?}

There are three commands that are the basis for reading "UNIX"
documentation within "UNIX" itself -
\href{http://linux.die.net/man/1/man}{\texttt{man}},
\href{http://linux.die.net/man/1/info}{\texttt{info}} and
\href{http://linux.die.net/man/1/apropos}{\texttt{apropos}}.
\drdoc{apropos} \drdoc{info} \drdoc{man}

\texttt{man} is short for \emph{manual pages}, and is used to display
the main help for most "UNIX" commands. For example, \texttt{man\ ls}
shows:

\drcap{\texttt{man} command}

\begin{Shaded}
\begin{Highlighting}[]
\KeywordTok{LS}\NormalTok{(1)                            }\KeywordTok{User} \NormalTok{Commands                           LS(1)}

\KeywordTok{NAME}
       \KeywordTok{ls} \NormalTok{- list directory contents}

\KeywordTok{SYNOPSIS}
       \KeywordTok{ls} \NormalTok{[OPTION]... [FILE]...}

\KeywordTok{DESCRIPTION}
       \KeywordTok{List}  \NormalTok{information  about  the FILEs (the current directory by default)}\KeywordTok{.}
       \KeywordTok{Sort} \NormalTok{entries alphabetically if none of -cftuvSUX nor --sort  is  speci‐}
       \KeywordTok{fied.}

       \KeywordTok{Mandatory}  \NormalTok{arguments  to  long  options are mandatory for short options}
       \KeywordTok{too.}

       \KeywordTok{-a}\NormalTok{, --all}
              \KeywordTok{do} \KeywordTok{not} \NormalTok{ignore entries starting with .}

       \KeywordTok{-A}\NormalTok{, --almost-all}
              \KeywordTok{do} \KeywordTok{not} \NormalTok{list implied . and ..}

       \KeywordTok{--author}
 \KeywordTok{Manual} \NormalTok{page ls(1) }\KeywordTok{line} \NormalTok{1 (press h for help or q to quit)}
\end{Highlighting}
\end{Shaded}

\textbf{Note:} \texttt{man} uses \texttt{less} as a paginator, with all
that means, including the same navigation and search keys, and most
important to remember - \texttt{Q} to quit. How do I know this? Because
of course you can \texttt{man\ man}! \drtxt{less} \drdoc{man}
\index{files and directories!paginate!less@\texttt{less} command}
\index{pagination!\texttt{less} command}

Notice the \texttt{LS(1)} part. The UNIX manual was originally divided
into multiple sections by AT\&T. Section 1 is normal user commands.
Section 5 is file formats (like for config files), and section 8 is for
system administration commands. You usually don't care, and can
\texttt{man\ ls} or \texttt{man\ ifconfig} to your heart's content.

But sometimes there are duplicate names in the different sections. For
example, there is both a
\href{http://linux.die.net/man/1/passwd}{\texttt{passwd}} command and a
\texttt{passwd} file format (for \texttt{/etc/passwd}). By default,
\texttt{man\ passwd} will show you the documentation from the lowest
numbered section with a match, in this case section 1, usually referred
to as \texttt{passwd(1)} to disambiguate which thing we're talking
about: \drsys{passwd}{change password}
\index{files and directories!special!etcpasswd@\texttt{/etc/passwd}}

\drcap{Ambiguous \texttt{man passwd} command default to lowest documentation section}

\begin{Shaded}
\begin{Highlighting}[]
\KeywordTok{PASSWD}\NormalTok{(1)                        }\KeywordTok{User} \NormalTok{Commands                       PASSWD(1)}

\KeywordTok{NAME}
       \KeywordTok{passwd} \NormalTok{- change user password}

\KeywordTok{SYNOPSIS}
       \KeywordTok{passwd} \NormalTok{[options] [LOGIN]}

\KeywordTok{DESCRIPTION}
       \KeywordTok{The} \NormalTok{passwd command changes passwords for user accounts. A normal user}
       \KeywordTok{may} \NormalTok{only change the password for his/her own account, while the}
       \KeywordTok{superuser} \NormalTok{may change the password for any account.  passwd also changes}
       \KeywordTok{the} \NormalTok{account or associated password validity period.}

   \KeywordTok{Password} \NormalTok{Changes}
       \KeywordTok{The} \NormalTok{user is first prompted for his/her old password, if one is present.}
       \KeywordTok{This} \NormalTok{password is then encrypted and compared against the stored}
       \KeywordTok{password.} \NormalTok{The user has only one chance to enter the correct password.}
       \KeywordTok{The} \NormalTok{superuser is permitted to bypass this step so that forgotten}
       \KeywordTok{passwords} \NormalTok{may be changed.}

       \KeywordTok{After} \NormalTok{the password has been entered, password aging information is}
       \KeywordTok{checked} \NormalTok{to see if the user is permitted to change the password at this}
 \KeywordTok{Manual} \NormalTok{page passwd(1) }\KeywordTok{line} \NormalTok{1 (press h for help or q to quit)}
\end{Highlighting}
\end{Shaded}

To see the \texttt{man} page for the \texttt{passwd} file format, we
have to explicitly specify the section, in this case by using
\texttt{man\ 5\ passwd}:

\drcap{Specifying a specific section with \texttt{man 5 passwd}}

\begin{Shaded}
\begin{Highlighting}[]
\KeywordTok{PASSWD}\NormalTok{(5)                }\KeywordTok{File} \NormalTok{Formats and Conversions                PASSWD(5)}

\KeywordTok{NAME}
       \KeywordTok{passwd} \NormalTok{- the password file}

\KeywordTok{DESCRIPTION}
       \KeywordTok{/etc/passwd} \NormalTok{contains one line for each user account, with seven fields}
       \KeywordTok{delimited} \NormalTok{by colons (“:”)}\KeywordTok{.} \KeywordTok{These} \NormalTok{fields are:}

       \NormalTok{·   }\KeywordTok{login} \NormalTok{name}

       \NormalTok{·   }\KeywordTok{optional} \NormalTok{encrypted password}

       \NormalTok{·   }\KeywordTok{numerical} \NormalTok{user ID}

       \NormalTok{·   }\KeywordTok{numerical} \NormalTok{group ID}

       \NormalTok{·   }\KeywordTok{user} \NormalTok{name or comment field}

       \NormalTok{·   }\KeywordTok{user} \NormalTok{home directory}

       \NormalTok{·   }\KeywordTok{optional} \NormalTok{user command interpreter}

 \KeywordTok{Manual} \NormalTok{page passwd(5) }\KeywordTok{line} \NormalTok{1 (press h for help or q to quit)}
\end{Highlighting}
\end{Shaded}

\texttt{man} pages can be long and sometimes obscure to a beginner, but
it is recommended you get used to reading them, especially any warnings.
There is a great quote that explains why:

\begin{quote}
"Unix will give you enough rope to shoot yourself in the foot. If you
didn't think rope would do that, you should have read the man page." -
unknown
\end{quote}

\textbf{Note:} If you know who originally said the above and can send
proof, let me know and I will give them credit.

Besides \texttt{man}, many GNU tools come with help in \texttt{info}
format, which is originally from \texttt{emacs}. Here are the results of
\texttt{info\ find}: \drdoc{info}

\drcap{Running \texttt{info find} command}

\begin{Shaded}
\begin{Highlighting}[]
\KeywordTok{File}\NormalTok{: find.info,  Node: Invoking find,  Next: Invoking locate,  Up: Reference}

\KeywordTok{7.1} \NormalTok{Invoking }\StringTok{'find'}
\NormalTok{===================}

     \KeywordTok{find} \NormalTok{[-H] [-L] [-P] [-D DEBUGOPTIONS] [-OLEVEL] [FILE...] [EXPRESSION]}

   \StringTok{'find'} \KeywordTok{searches} \NormalTok{the directory tree rooted at each file name FILE by}
\KeywordTok{evaluating} \NormalTok{the EXPRESSION on each file it finds in the tree.}

   \KeywordTok{The} \NormalTok{command line may begin with the }\StringTok{'-H'}\NormalTok{, }\StringTok{'-L'}\NormalTok{, }\StringTok{'-P'}\NormalTok{, }\StringTok{'-D'} \NormalTok{and }\StringTok{'-O'}
\KeywordTok{options.}  \NormalTok{These are followed by a list of files or directories that}
\KeywordTok{should} \NormalTok{be searched.  If no files to search are specified, the current}
\KeywordTok{directory} \NormalTok{(}\StringTok{'.'}\NormalTok{) }\KeywordTok{is} \NormalTok{used.}

   \KeywordTok{This} \NormalTok{list of files to search is followed by a list of expressions}
\KeywordTok{describing} \NormalTok{the files we wish to search for.  The first part of the}
\KeywordTok{expression} \NormalTok{is recognised by the fact that it begins with }\StringTok{'-'} \NormalTok{followed by}
\KeywordTok{some} \NormalTok{other letters (for example }\StringTok{'-print'}\NormalTok{), }\KeywordTok{or} \NormalTok{is either }\StringTok{'('} \NormalTok{or }\StringTok{'!'}\NormalTok{.  Any}
\KeywordTok{arguments} \NormalTok{after it are the rest of the expression.}

   \KeywordTok{If} \NormalTok{no expression is given, the expression }\StringTok{'-print'} \NormalTok{is used.}
\KeywordTok{--zz-Info}\NormalTok{: (find.info.gz)}\KeywordTok{Invoking} \NormalTok{find, 44 lines --Top-------------------------}
\KeywordTok{Welcome} \NormalTok{to Info version 5.2. Type h for help, m for menu item.}
\end{Highlighting}
\end{Shaded}

While \texttt{info} is much better at enabling complex help files with
navigation I am not a fan because I tend not to hold all the keystrokes
in my head. The biggest thing to remember if you do something like
\texttt{info\ find} is that \texttt{q} quits the \texttt{info} command.

Finally, what if you don't know the name of the command? Well, each "man
page" has a title and brief description, e.g., "passwd - change user
password" in the \texttt{man\ passwd} output above. The \texttt{apropos}
command can simply search those titles and descriptions for a word or
phrase and show you all the results: \drdoc{apropos}

\drcap{\texttt{apropos} command}

\begin{Shaded}
\begin{Highlighting}[]
\KeywordTok{~} \NormalTok{$ }\KeywordTok{apropos} \NormalTok{edit}
\KeywordTok{atobm} \NormalTok{(1)            }\KeywordTok{-} \NormalTok{bitmap editor and converter utilities for the X Window...}
\KeywordTok{bitmap} \NormalTok{(1)           }\KeywordTok{-} \NormalTok{bitmap editor and converter utilities for the X Window...}
\KeywordTok{bmtoa} \NormalTok{(1)            }\KeywordTok{-} \NormalTok{bitmap editor and converter utilities for the X Window...}
\KeywordTok{cinnamon-menu-editor} \NormalTok{(1) }\KeywordTok{-} \NormalTok{Editor for the panel menu}
\KeywordTok{desktop-file-edit} \NormalTok{(1) }\KeywordTok{-} \NormalTok{Installation and edition of desktop files}
\KeywordTok{desktop-file-install} \NormalTok{(1) }\KeywordTok{-} \NormalTok{Installation and edition of desktop files}
\KeywordTok{ed} \NormalTok{(1)               }\KeywordTok{-} \NormalTok{line-oriented text editor}
\KeywordTok{edit} \NormalTok{(1)             }\KeywordTok{-} \NormalTok{execute programs via entries in the mailcap file}
\KeywordTok{editdiff} \NormalTok{(1)         }\KeywordTok{-} \NormalTok{fix offsets and counts of a hand-edited diff}
\KeywordTok{editkeep} \NormalTok{(8)         }\KeywordTok{-} \NormalTok{frontend for deborphan}
\KeywordTok{editor} \NormalTok{(1)           }\KeywordTok{-} \NormalTok{Nano}\StringTok{'s ANOther editor, an enhanced free Pico clone}
\StringTok{editres (1)          - a dynamic resource editor for X Toolkit applications}
\StringTok{elfedit (1)          - Update the ELF header of ELF files.}
\StringTok{ex (1)               - Vi IMproved, a programmers text editor}
\StringTok{fix-qdf (1)          - repair PDF files in QDF form after editing}
\StringTok{gedit (1)            - text editor for the GNOME Desktop}
\StringTok{gnome-desktop-item-edit (1) - tool to edit .desktop file}
\StringTok{gnome-text-editor (1) - text editor for the GNOME Desktop}
\StringTok{Gnome2::DateEdit (3pm) - wrapper for GnomeDateEdit}
\StringTok{grub-editenv (1)     - edit GRUB environment block}
\StringTok{jfs_debugfs (8)      - shell-type JFS file system editor}
\StringTok{mintsources (1)      - Software Sources List editor}
\StringTok{...and so on...}
\end{Highlighting}
\end{Shaded}

Note the \texttt{man} section numbers after each command name. Also note
that \texttt{apropos} is not sophisticated - it is simply searching for
the exact string you give it in the very limited "brief descriptions"
from the \texttt{man} pages. That's all. But a lot of time that's all
you need to remember, "Ah, yes, \texttt{nano} is the other editor I was
thinking about and like better than \texttt{vi}."

\textbf{Note:} \texttt{man}, \texttt{info} and \texttt{apropos} are just
normal "UNIX" commands like all the others, so while they may default to
displaying with a paginator on an interactive terminal, you can run
their output through other commands, just like any other. For example,
maybe we remember only that the command had something with "edit" and
was a system administration ("section 8") command:

\drcap{Refining output from \texttt{apropos}}

\begin{Shaded}
\begin{Highlighting}[]
\KeywordTok{~} \NormalTok{$ }\KeywordTok{apropos} \NormalTok{edit }\KeywordTok{|} \KeywordTok{grep} \StringTok{"(8)"}
\KeywordTok{editkeep} \NormalTok{(8)         }\KeywordTok{-} \NormalTok{frontend for deborphan}
\KeywordTok{jfs_debugfs} \NormalTok{(8)      }\KeywordTok{-} \NormalTok{shell-type JFS file system editor}
\KeywordTok{pdbedit} \NormalTok{(8)          }\KeywordTok{-} \NormalTok{manage the SAM database (Database of Samba Users)}
\KeywordTok{samba-regedit} \NormalTok{(8)    }\KeywordTok{-} \NormalTok{ncurses based tool to manage the Samba registry}
\KeywordTok{sudoedit} \NormalTok{(8)         }\KeywordTok{-} \NormalTok{execute a command as another user}
\KeywordTok{vigr} \NormalTok{(8)             }\KeywordTok{-} \NormalTok{edit the password, group, shadow-password or shadow-gr...}
\KeywordTok{vipw} \NormalTok{(8)             }\KeywordTok{-} \NormalTok{edit the password, group, shadow-password or shadow-gr...}
\KeywordTok{visudo} \NormalTok{(8)           }\KeywordTok{-} \NormalTok{edit the sudoers file}
\end{Highlighting}
\end{Shaded}

Or maybe you can't remember whether it's \texttt{-r}, \texttt{-R} or
\texttt{-\/-recursive} to copy subdirectories recursively with
\texttt{cp}: \drfnd{cp}{copy}

\drcap{Looking for specific parameter names in a \texttt{man} page}

\begin{Shaded}
\begin{Highlighting}[]
\NormalTok{$ }\KeywordTok{man} \NormalTok{cp }\KeywordTok{|} \KeywordTok{grep} \NormalTok{-i }\StringTok{"recurs"}
              \KeywordTok{copy} \NormalTok{contents of special files when recursive}
       \KeywordTok{-R}\NormalTok{, -r, --recursive}
              \KeywordTok{copy} \NormalTok{directories recursively}
\end{Highlighting}
\end{Shaded}

What do you know, it can be any of the three.

And yes, you can \texttt{man\ man}, \texttt{man\ info},
\texttt{info\ info} and \texttt{info\ man}, for that matter!

\section*{\texorpdfstring{How Do You Google,
\texttt{man}?}{How Do You Google, man?}}\label{how-do-you-google-man}
\addcontentsline{toc}{section}{How Do You Google, \texttt{man}?}

You can often search the internet for "UNIX" documentation, and the
\texttt{man} pages have long been online. A site I like (and link to a
lot here) is \url{http://linux.die.net/man/}. Often, though, you can
just google \href{https://www.google.com/\#q=man+ls}{"man ls"} and the
top hits will be what you want.
\index{documentation!http://linux.die.net/man/@\textit{http://linux.die.net/man/}}

\textbf{\emph{However}}, there are times you need to be careful.
Searching the internet for either \texttt{man\ touch} or
\texttt{man\ tail}, for example, will probably not give you the results
you seek and may set off filters at work, so be careful out there and
remember to bookmark a couple of actual \texttt{man} page sites so that
you can go there directly and look up a command.

\section*{Books and Stuff}\label{books-and-stuff}
\addcontentsline{toc}{section}{Books and Stuff}

There are several consistently high-quality free sources of information
on various parts of Linux and related systems on the internet.

\begin{itemize}
\item
  \href{http://www.tldp.org/guides.html}{\textbf{The Linux Documentation
  Project (LDP)}} - has fallen a bit behind over the years, but still
  has two of the best \texttt{bash} scripting books out there,
  \href{http://www.tldp.org/LDP/Bash-Beginners-Guide/html/index.html}{\emph{Bash
  Guide for Beginners}} and
  \href{http://www.tldp.org/LDP/abs/html/index.html}{\emph{Advanced
  Bash-Scripting Guide}}. I continue to use the latter all the time.
  \drshl{bash}
  \index{documentation!Advanced Bash-Scripting Guide@\textit{Advanced Bash-Scripting Guide}}
  \index{documentation!Bash Guide for Beginners@\textit{Bash Guide for Beginners}}
  \index{documentation!Linux Documentation Project}
\item
  \href{https://wiki.archlinux.org/}{\textbf{Arch Linux Wiki}} - you may
  not think this would be useful if you are running Debian or Fedora or
  something else, but remember most "UNIX" systems are all very similar,
  and often the best documentation on a package or setting something up
  in Linux is in the Arch wiki. \drdis{Arch}
  \index{documentation!Arch wiki}
\item
  \href{https://www.debian.org/doc/}{\textbf{Debian documentation}} -
  again, even if you are not running a Debian-based distro, this can be
  handy because it describes how to administer Linux in a way that often
  transcends distro specifics (and at least explains how Debian
  approaches the differences). The best books in the series are
  \href{https://www.debian.org/doc/manuals/debian-handbook/}{\emph{The
  Debian Administrator's Handbook}} and the
  \href{https://www.debian.org/doc/manuals/debian-reference/}{\emph{Debian
  Reference}}, which is a lot more formal attempt at the same type of
  territory this guide covers. \drdis{Debian}
  \index{documentation!Debian Administrator's Handbook@\textit{Debian Administrator's Handbook}}
  \index{documentation!Debian Reference@\textit{Debian Reference}}
\end{itemize}

Ubuntu, Mint and some other distros have quite active message fora, and
of course StackOverflow and its family are also very useful.

Besides the above, if you are dealing with a package that is not part of
the "core" OS, such as \href{https://www.samba.org/samba/}{Samba} for
setting up CIFS shares on Linux, you should always look at
\href{https://www.samba.org/samba/docs/}{the package site's
documentation} as well as any specific info you can find about the
distro you are running.

\hypertarget{and-so-on}{\chapter{And So On}\label{and-so-on}}

\textbf{\emph{\texttt{/etc}, starting and stopping services,
\texttt{apt-get}/\texttt{rpm}/\texttt{yum}, and more.}}

\begin{quote}
\emph{"Et cetera, et cetera, et cetera!"} - The King (\emph{The King and
I})
\end{quote}

This step is a grab bag of stuff that didn't seem to directly belong
anywhere else, but I still think needs to be known, or at least brushed
up against.

\section*{One-Stop Shopping}\label{one-stop-shopping}
\addcontentsline{toc}{section}{One-Stop Shopping}

In UNIX-like systems, most (not all) system configuration is stored in
directories and text files under \texttt{/etc}.
\index{files and directories!special!etc@\texttt{/etc/}}

\textbf{Note:} In Linux \texttt{/etc} is almost universally pronounced
"slash-et-see," \textbf{\emph{not}} "forward slash et cetera."

\drcap{\texttt{/etc} directory}

\begin{Shaded}
\begin{Highlighting}[]
\KeywordTok{~} \NormalTok{$ }\KeywordTok{ls} \NormalTok{/etc}
\KeywordTok{acpi}                    \NormalTok{hosts                 pki}
\KeywordTok{adduser.conf}            \NormalTok{hosts.allow           pm}
\KeywordTok{adjtime}                 \NormalTok{hosts.deny            pnm2ppa.conf}
\KeywordTok{alternatives}            \NormalTok{hp                    polkit-1}
\KeywordTok{anacrontab}              \NormalTok{icedtea-web           ppp}
\KeywordTok{apg.conf}                \NormalTok{ifplugd               printcap}
\KeywordTok{apm}                     \NormalTok{ImageMagick           profile}
\KeywordTok{apparmor}                \NormalTok{init                  profile.d}
\KeywordTok{apparmor.d}              \NormalTok{init.d                protocols}
\KeywordTok{apport}                  \NormalTok{initramfs-tools       pulse}
\KeywordTok{apt}                     \NormalTok{inputrc               purple}
\KeywordTok{at-spi2}                 \NormalTok{insserv               python}
\KeywordTok{avahi}                   \NormalTok{insserv.conf          python2.7}
\KeywordTok{bash.bashrc}             \NormalTok{insserv.conf.d        python3}
\KeywordTok{bash_completion}         \NormalTok{inxi.conf             python3.4}
\KeywordTok{bash_completion.d}       \NormalTok{iproute2              rc0.d}
\KeywordTok{bindresvport.blacklist}  \NormalTok{issue                 rc1.d}
\KeywordTok{blkid.conf}              \NormalTok{issue.dpkg-old        rc2.d}
\KeywordTok{blkid.tab}               \NormalTok{issue.net             rc3.d}
\KeywordTok{bluetooth}               \NormalTok{issue.net.dpkg-old    rc4.d}
\KeywordTok{bonobo-activation}       \NormalTok{java-7-openjdk        rc5.d}
\KeywordTok{brlapi.key}              \NormalTok{kbd                   rc6.d}
\KeywordTok{...and} \NormalTok{so on...}
\end{Highlighting}
\end{Shaded}

Depending on what you are trying to configure, you may need to be in one
or many files in \texttt{/etc}. This is a \textbf{\emph{very short}}
list of files and directories you may need to examine there:

\begin{itemize}
\item
  \textbf{\texttt{fstab}} - a listing of the file systems currently
  mounted and their types.
  \index{files and directories!special!etcfstab@\texttt{/etc/fstab}}
\item
  \textbf{\texttt{group}} - the security groups on the system.
  \index{files and directories!special!etcgroup@\texttt{/etc/group}}
\item
  \textbf{\texttt{hosts}} - network aliases (overrides DNS, takes effect
  immediately).
  \index{files and directories!special!etchosts@\texttt{/etc/hosts}}
\item
  \textbf{\texttt{init.d}} - startup and shutdown scripts for
  "services."
  \index{files and directories!special!etcinitd@\texttt{/etc/init.d/}}
\item
  \textbf{\texttt{mtab}} - list of current "mounts."
  \index{files and directories!special!etcmtab@\texttt{/etc/mtab}}
\item
  \textbf{\texttt{passwd}} - "shadow" file containing all the user
  accounts on the system.
  \index{files and directories!special!etcpasswd@\texttt{/etc/passwd}}
\item
  \textbf{\texttt{resolv.conf}} - DNS settings.
  \index{files and directories!special!etcresolvconf@\texttt{/etc/resolv.conf}}
\item
  \textbf{\texttt{samba}} - file sharing settings for CIFS-style shares.
  \index{files and directories!special!etcsamba@\texttt{/etc/samba/}}
\end{itemize}

There are lots of other interesting files under \texttt{/etc}, but I
keep returning to the above again and again. On most of them you can run
the \texttt{man} command against section 5 to see their format and
documentation, e.g., \texttt{man\ 5\ hosts}.

\section*{Service Station}\label{service-station}
\addcontentsline{toc}{section}{Service Station}

We are going to ignore system initialization and "stages," and assume
most of the time you are running on a well-functioning system. Even so
sometimes you want to restart a specific system service without
rebooting the whole system, often to force re-reading changed
configuration files. First check if the service has a script in
\texttt{/etc/init.d}:
\index{files and directories!special!etcinitd@\texttt{/etc/init.d/}}

\drcap{init.d directory}

\begin{Shaded}
\begin{Highlighting}[]
\KeywordTok{~} \NormalTok{$ }\KeywordTok{ls} \NormalTok{/etc/init.d}
\KeywordTok{acpid}             \NormalTok{dbus               ondemand     single}
\KeywordTok{anacron}           \NormalTok{dns-clean          pppd-dns     skeleton}
\KeywordTok{apparmor}          \NormalTok{friendly-recovery  procps       smbd}
\KeywordTok{avahi-daemon}      \NormalTok{grub-common        pulseaudio   speech-dispatcher}
\KeywordTok{binfmt-support}    \NormalTok{halt               rc           sudo}
\KeywordTok{bluetooth}         \NormalTok{hddtemp            rc.local     udev}
\KeywordTok{brltty}            \NormalTok{irqbalance         rcS          umountfs}
\KeywordTok{casper}            \NormalTok{kerneloops         README       umountnfs.sh}
\KeywordTok{cinnamon}          \NormalTok{killprocs          reboot       umountroot}
\KeywordTok{console-setup}     \NormalTok{kmod               resolvconf   unattended-upgrades}
\KeywordTok{cpufrequtils}      \NormalTok{lm-sensors         rsync        urandom}
\KeywordTok{cron}              \NormalTok{loadcpufreq        rsyslog      virtualbox-guest-utils}
\KeywordTok{cryptdisks}        \NormalTok{mdm                samba        virtualbox-guest-x11}
\KeywordTok{cryptdisks-early}  \NormalTok{mintsystem         samba-ad-dc  x11-common}
\KeywordTok{cups}              \NormalTok{networking         saned}
\KeywordTok{cups-browsed}      \NormalTok{nmbd               sendsigs}
\end{Highlighting}
\end{Shaded}

If so, then chances are it will respond to a fairly standard set of
commands, such as the following samples with \texttt{samba}:

\drcap{Stopping and starting services}

\begin{Shaded}
\begin{Highlighting}[]
\KeywordTok{~} \CommentTok{# /etc/init.d/samba stop}
\KeywordTok{[} \NormalTok{ok}\KeywordTok{ ]} \KeywordTok{Stopping} \NormalTok{Samba daemons: nmbd smbd.}
\KeywordTok{~} \CommentTok{# /etc/init.d/samba start}
\KeywordTok{[} \NormalTok{ok}\KeywordTok{ ]} \KeywordTok{Starting} \NormalTok{Samba daemons: nmbd smbd.}
\KeywordTok{~} \CommentTok{# /etc/init.d/samba restart}
\KeywordTok{[} \NormalTok{ok}\KeywordTok{ ]} \KeywordTok{Stopping} \NormalTok{Samba daemons: nmbd smbd.}
\KeywordTok{[} \NormalTok{ok}\KeywordTok{ ]} \KeywordTok{Starting} \NormalTok{Samba daemons: nmbd smbd.}
\end{Highlighting}
\end{Shaded}

\textbf{Note:} The above examples were run as \texttt{root}, otherwise
they would probably have required execution using \texttt{sudo}.

\section*{Package Management}\label{package-management}
\addcontentsline{toc}{section}{Package Management}

Almost all Linux distros have the concept of "packages" which are used
to install, update and uninstall software. There are different package
managers, including \texttt{dpkg} and \texttt{apt-get} on Debian-based
distros, \texttt{rpm} on Fedora descendants, etc. For the rest of this
section we will use Debian tools, but in general the concepts and
problems are similar for the other toolsets. \drpkg{apt-get}
\drpkg{dpkg} \drpkg{rpm}

One of the nicest things about Linux-style package managers (as opposed
to traditional Windows installers) is that they can satisfy all a
packages "dependencies" (other packages that are required for a package
to run) and automatically detect and install those, too. See
\href{https://chocolatey.org/}{Chocolately} for an attempt to build a
similar ecosystem in Windows.

One thing Linux distros do is define the "repositories" (servers and
file structures) that serve the various packages. In addition, there are
usually multiple versions of packages, typically matching different
releases of the distro. We won't go into setting up a system to point to
these here.

In Debian flavors,
\href{http://linux.die.net/man/8/apt-get}{\texttt{apt-get}} is usually
the tool of choice for package management. Another option is
\href{http://linux.die.net/man/8/aptitude}{\texttt{aptitude}}.
\drpkg{apt-get} \drpkg{aptitude}

There are three common \texttt{apt-get} commands that get used over and
over. The first downloads and \emph{updates} the local metadata cache
for the repositories:

\drcap{\texttt{apt-get update}}

\begin{Shaded}
\begin{Highlighting}[]
\KeywordTok{~} \NormalTok{$ }\KeywordTok{sudo} \NormalTok{apt-get update}
\NormalTok{[}\KeywordTok{sudo}\NormalTok{] password for myuser:}
\KeywordTok{Ign} \NormalTok{http://packages.linuxmint.com rafaela InRelease}
\KeywordTok{Ign} \NormalTok{http://extra.linuxmint.com rafaela InRelease}
\KeywordTok{Hit} \NormalTok{http://extra.linuxmint.com rafaela Release.gpg                             }
\KeywordTok{Hit} \NormalTok{http://packages.linuxmint.com rafaela Release.gpg                          }
\KeywordTok{Hit} \NormalTok{http://security.ubuntu.com trusty-security InRelease                       }
\KeywordTok{Hit} \NormalTok{http://extra.linuxmint.com rafaela Release}
\KeywordTok{Hit} \NormalTok{http://packages.linuxmint.com rafaela Release}
\KeywordTok{Hit} \NormalTok{http://security.ubuntu.com trusty-security/main amd64 Packages}
\KeywordTok{Hit} \NormalTok{http://packages.linuxmint.com rafaela/main amd64 Packages}
\KeywordTok{Hit} \NormalTok{http://extra.linuxmint.com rafaela/main amd64 Packages}
\KeywordTok{Hit} \NormalTok{http://security.ubuntu.com trusty-security/restricted amd64 Packages}
\KeywordTok{Hit} \NormalTok{http://extra.linuxmint.com rafaela/main i386 Packages}
\KeywordTok{Hit} \NormalTok{http://packages.linuxmint.com rafaela/upstream amd64 Packages}
\KeywordTok{Ign} \NormalTok{http://archive.canonical.com trusty InRelease}
\KeywordTok{Ign} \NormalTok{http://archive.ubuntu.com trusty InRelease}
\KeywordTok{Hit} \NormalTok{http://security.ubuntu.com trusty-security/universe amd64 Packages}
\KeywordTok{Hit} \NormalTok{http://packages.linuxmint.com rafaela/import amd64 Packages}
\KeywordTok{Hit} \NormalTok{http://security.ubuntu.com trusty-security/multiverse amd64 Packages}
\KeywordTok{Hit} \NormalTok{http://packages.linuxmint.com rafaela/main i386 Packages}
\KeywordTok{Hit} \NormalTok{http://archive.canonical.com trusty Release.gpg}
\KeywordTok{Hit} \NormalTok{http://archive.ubuntu.com trusty-updates InRelease}
\KeywordTok{...and} \NormalTok{so on...}
\end{Highlighting}
\end{Shaded}

\textbf{Note:} \texttt{apt-get} is an administrative command and usually
requires \texttt{sudo}.

The second common command \emph{upgrades} all the packages in the system
to the latest release in the repository (which may not be the latest and
greatest release of the package):

\drcap{Upgrading installed packages}

\begin{Shaded}
\begin{Highlighting}[]
\KeywordTok{~} \NormalTok{$ }\KeywordTok{sudo} \NormalTok{apt-get dist-upgrade}
\KeywordTok{Reading} \NormalTok{package lists... Done}
\KeywordTok{Building} \NormalTok{dependency tree       }
\KeywordTok{Reading} \NormalTok{state information... Done}
\KeywordTok{Calculating} \NormalTok{upgrade... Done}
\KeywordTok{0} \NormalTok{upgraded, 0 newly installed, 0 to remove and 0 not upgraded.}
\end{Highlighting}
\end{Shaded}

In this case there was nothing to upgrade.

\textbf{Note:} In the example above I used
\texttt{apt-get\ dist-upgrade}. There is also an
\texttt{apt-get\ upgrade} command. \texttt{apt-get\ dist-upgrade} will
resolve any new dependencies and download new packages if needed, but
\textbf{\emph{it may also remove packages it considers no longer
needed.}} \texttt{apt-get\ upgrade} simply performs an in-place upgrade
of already-installed packages and in no case will install new or remove
unneeded packages. Which is appropriate for you will depend on your
circumstances. You can use the \texttt{-\/-simulate} parameter with
either to have \texttt{apt-get} show you what it would do without
actually doing it. \drpkg{apt-get}

And the final common command is obviously to install a package:

\drcap{Installing a package}

\begin{Shaded}
\begin{Highlighting}[]
\KeywordTok{~} \NormalTok{$ }\KeywordTok{sudo} \NormalTok{apt-get install traceroute}
\KeywordTok{Reading} \NormalTok{package lists... Done}
\KeywordTok{Building} \NormalTok{dependency tree       }
\KeywordTok{Reading} \NormalTok{state information... Done}
\KeywordTok{The} \NormalTok{following NEW packages will be installed:}
  \KeywordTok{traceroute}
\KeywordTok{0} \NormalTok{upgraded, 1 newly installed, 0 to remove and 0 not upgraded.}
\KeywordTok{Need} \NormalTok{to get 0 B/45.0 kB of archives.}
\KeywordTok{After} \NormalTok{this operation, 176 kB of additional disk space will be used.}
\KeywordTok{Selecting} \NormalTok{previously unselected package traceroute.}
\KeywordTok{(Reading} \NormalTok{database ... 307895 files and directories currently installed.}\KeywordTok{)}
\KeywordTok{Preparing} \NormalTok{to unpack .../traceroute_1%3a2.0.20-0ubuntu0.1_amd64.deb ...}
\KeywordTok{Unpacking} \NormalTok{traceroute (1:2.0.20-0ubuntu0.1) }\KeywordTok{...}
\KeywordTok{Processing} \NormalTok{triggers for man-db (2.6.7.1-1ubuntu1) }\KeywordTok{...}
\KeywordTok{Setting} \NormalTok{up traceroute (1:2.0.20-0ubuntu0.1) }\KeywordTok{...}
\KeywordTok{update-alternatives}\NormalTok{: using /usr/bin/traceroute.db to provide /usr/bin/traceroute}
 \KeywordTok{(traceroute)} \KeywordTok{in} \KeywordTok{auto} \NormalTok{mode}
\KeywordTok{update-alternatives}\NormalTok{: using /usr/bin/lft.db to provide /usr/bin/lft (lft) }\KeywordTok{in} \KeywordTok{auto}
 \KeywordTok{mode}
\KeywordTok{update-alternatives}\NormalTok{: using /usr/bin/traceproto.db to provide /usr/bin/traceproto}
 \KeywordTok{(traceproto)} \KeywordTok{in} \KeywordTok{auto} \NormalTok{mode}
\KeywordTok{update-alternatives}\NormalTok{: using /usr/sbin/tcptraceroute.db to provide /usr/sbin/tcptr}
\KeywordTok{aceroute} \NormalTok{(tcptraceroute) }\KeywordTok{in} \KeywordTok{auto} \NormalTok{mode}
\end{Highlighting}
\end{Shaded}

You can also \texttt{apt-get\ remove} or \texttt{apt-get\ purge}
packages. See the \texttt{apt-get} \texttt{man} page for details.
\drpkg{apt-get}

This all looks very convenient, and it is. The problems arise because
some distros are better at tracking current versions of packages in
their repositories than others. In fact, some distros purposefully stay
behind cutting edge for system stability purposes. Debian itself is a
good example of this, as are many "LTS" (long term support) releases in
other distros.

\section*{Other Sources}\label{other-sources}
\addcontentsline{toc}{section}{Other Sources}

Besides the distribution's repositories, you can install packages and
other software from a variety of places. It may be an "official" site
for the package, GitHub, a "personal package archive" (PPA) or whatever.
The package may be in a binary installable format (\texttt{.deb} files
for Debian systems), in source format requiring it to be built, in a
zipped "tarball," and more.
\index{files and directories!debpackagefiles@\texttt{.deb} package files}

If you want the latest and greatest version of a package you often have
to go to its "official" site or GitHub repository. There, you may find a
\texttt{.deb} file, in which case you could install it with
\texttt{dpkg}: \drpkg{dpkg}

\drcap{Installing a package with \texttt{dpkg}}

\begin{Shaded}
\begin{Highlighting}[]
\KeywordTok{sudo} \NormalTok{dpkg -i somesoftware.deb}
\end{Highlighting}
\end{Shaded}

There is, however, a problem. You now have to remember that you
installed that package by hand and keep it up to date by hand (or not).
\texttt{apt-get\ upgrade} isn't going to help you here. This is true no
matter what way you get the alternative package - \texttt{.deb} file,
tarball, source code, or whatever (although \texttt{apt-get} can work
with PPAs in a more automated manner). \drpkg{apt-get}

The second problem with third-party package sources is how do you know
whether to trust them or not? If something is in an "official" distro
repository, chances are it has been vetted to a certain degree. But
otherwise, it is caveat administrator.

The final problem with package managers is that they're such a good idea
that \textbf{\emph{everything}} has them now. Not just the operating
systems like Linux, but languages like Python have
\href{https://pypi.python.org/pypi/pip/}{\texttt{pip}} and execution
environments like node have \href{https://www.npmjs.com/}{\texttt{npm}}.
So now you end up with having to keep track of what you have installed
on a system across two or three or more package managers at different
levels of abstraction. It can be a mess!

Add into this that many of these language and environment package
managers allow setting up "global" (system-wide) or "local" (current
directory) versions of a package to allow different versions of the same
package to exist on the same system, where different applications may be
relying on the different versions to work. Do you keep good notes? You'd
better!

\section*{\texorpdfstring{Which \texttt{which} is
Which?}{Which which is Which?}}\label{which-which-is-which}
\addcontentsline{toc}{section}{Which \texttt{which} is Which?}

Now that we've seen that we can have multiple versions of the same
command or executable on the system, an interesting question arises.
\emph{Which} \texttt{foo} command am I going to call if I just type
\texttt{foo} at the command prompt? In other words, after taking the
\texttt{\$PATH} variable into consideration and searching for the
program through that from left to right, which version in which
directory is going to be called? \drenv{PATH}{execution search path}

Luckily we have the
\href{http://linux.die.net/man/1/which}{\texttt{which}} command for just
that! \drcmd{which}{find program}

\drcap{\texttt{which} command}

\begin{Shaded}
\begin{Highlighting}[]
\KeywordTok{~} \NormalTok{$ }\KeywordTok{which} \NormalTok{curl}
\KeywordTok{/usr/bin/curl}
\end{Highlighting}
\end{Shaded}

How can you tell if you have multiple versions of something installed?
One way is with the
\href{http://linux.die.net/man/1/locate}{\texttt{locate}} command:
\drfnd{locate}{locate files}

\drcap{\texttt{locate} command}

\begin{Shaded}
\begin{Highlighting}[]
\KeywordTok{~} \NormalTok{$ }\KeywordTok{locate} \NormalTok{md5}
\KeywordTok{/boot/grub/i386-pc/gcry_md5.mod}
\KeywordTok{/lib/modules/3.16.0-38-generic/kernel/drivers/usb/gadget/amd5536udc.ko}
\KeywordTok{/usr/bin/md5pass}
\KeywordTok{/usr/bin/md5sum}
\KeywordTok{/usr/bin/md5sum.textutils}
\KeywordTok{/usr/include/libavutil/md5.h}
\KeywordTok{/usr/include/openssl/md5.h}
\KeywordTok{/usr/lib/casper/casper-md5check}
\KeywordTok{/usr/lib/grub/i386-pc/gcry_md5.mod}
\KeywordTok{/usr/lib/i386-linux-gnu/sasl2/libcrammd5.so}
\KeywordTok{/usr/lib/i386-linux-gnu/sasl2/libcrammd5.so.2}
\KeywordTok{/usr/lib/i386-linux-gnu/sasl2/libcrammd5.so.2.0.25}
\KeywordTok{/usr/lib/i386-linux-gnu/sasl2/libdigestmd5.so}
\KeywordTok{/usr/lib/i386-linux-gnu/sasl2/libdigestmd5.so.2}
\KeywordTok{/usr/lib/i386-linux-gnu/sasl2/libdigestmd5.so.2.0.25}
\KeywordTok{/usr/lib/python2.7/md5.py}
\KeywordTok{/usr/lib/python2.7/md5.pyc}
\KeywordTok{/usr/lib/ruby/1.9.1/x86_64-linux/digest/md5.so}
\KeywordTok{/usr/lib/x86_64-linux-gnu/sasl2/libcrammd5.so}
\KeywordTok{/usr/lib/x86_64-linux-gnu/sasl2/libcrammd5.so.2}
\KeywordTok{/usr/lib/x86_64-linux-gnu/sasl2/libcrammd5.so.2.0.25}
\KeywordTok{/usr/lib/x86_64-linux-gnu/sasl2/libdigestmd5.so}
\KeywordTok{...and} \NormalTok{so on...}
\end{Highlighting}
\end{Shaded}

The \texttt{locate} command, if installed, is basically a database of
all of the file names on the system (collected periodically - not in
real time). You are simply searching the database for a pattern. It is a
quicker way to look than
\texttt{find\ /\ -name\ \textbackslash{}*pattern*\textbackslash{}}.

One final note on which thing gets executed. Unlike in Windows, "UNIX"
environments do not consider the local directory (the current directory
you are sitting at the command prompt, i.e., what
\href{http://linux.die.net/man/1/pwd}{\texttt{pwd}} shows) as part of
the path unless \texttt{.} is explicitly listed in \texttt{\$PATH} (and
that is typically a bad idea). This is for security purposes. So it can
be a bit unnerving to try and execute \texttt{foo} in the current
directory and get: \drfnd{pwd}{print working directory}

\drcap{Command not found - but it's right there!}

\begin{Shaded}
\begin{Highlighting}[]
\KeywordTok{~} \NormalTok{$ }\KeywordTok{ls} \NormalTok{-l foo}
\KeywordTok{-rwxrwx---} \NormalTok{1 myuser mygroup 16 Oct 23 19:03 foo}
\KeywordTok{~} \NormalTok{$ }\KeywordTok{foo}
\KeywordTok{No} \NormalTok{command }\StringTok{'foo'} \NormalTok{found, did you mean:}
 \KeywordTok{Command} \StringTok{'fgo'} \NormalTok{from package }\StringTok{'fgo'} \NormalTok{(universe)}
 \KeywordTok{Command} \StringTok{'fop'} \NormalTok{from package }\StringTok{'fop'} \NormalTok{(main)}
 \KeywordTok{Command} \StringTok{'fog'} \NormalTok{from package }\StringTok{'ruby-fog'} \NormalTok{(universe)}
 \KeywordTok{Command} \StringTok{'fox'} \NormalTok{from package }\StringTok{'objcryst-fox'} \NormalTok{(universe)}
 \KeywordTok{Command} \StringTok{'fio'} \NormalTok{from package }\StringTok{'fio'} \NormalTok{(universe)}
 \KeywordTok{Command} \StringTok{'zoo'} \NormalTok{from package }\StringTok{'zoo'} \NormalTok{(universe)}
 \KeywordTok{Command} \StringTok{'xoo'} \NormalTok{from package }\StringTok{'xoo'} \NormalTok{(universe)}
 \KeywordTok{Command} \StringTok{'goo'} \NormalTok{from package }\StringTok{'goo'} \NormalTok{(universe)}
\KeywordTok{foo}\NormalTok{: command not found}
\end{Highlighting}
\end{Shaded}

Instead, to invoke \texttt{foo}, you can either fully qualify the path
as shown by \texttt{pwd}:

\drcap{Using a fully qualified path to execute a command}

\begin{Shaded}
\begin{Highlighting}[]
\KeywordTok{~} \NormalTok{$ }\KeywordTok{/home/myuser/foo}
\end{Highlighting}
\end{Shaded}

Or you can prepend the \texttt{./} relative path to it, to indicate "the
\texttt{foo} in the current directory (\texttt{.})":

\drcap{Specifying the command in the current directory}

\begin{Shaded}
\begin{Highlighting}[]
\KeywordTok{~} \NormalTok{$ }\KeywordTok{./foo}
\end{Highlighting}
\end{Shaded}

\section*{Over and Over and Over}\label{over-and-over-and-over}
\addcontentsline{toc}{section}{Over and Over and Over}

The function of scheduled tasks in Windows is performed by
\href{http://linux.die.net/man/8/cron}{\texttt{cron}} in Linux. It reads
in the various
\href{http://linux.die.net/man/5/crontab}{\texttt{crontab(5)}} files on
the system and executes the commands in them at the specified times. You
use the \href{http://linux.die.net/man/1/crontab}{\texttt{crontab(1)}}
command to view and edit the \texttt{crontab} files for your user (and
other users if you have admin privileges).
\drsys{cron}{run scheduled jobs} \drsys{crontab}{edit scheduled jobs}
\index{crontab@\texttt{crontab} (file)}
\index{files and directories!special!crontab@\texttt{crontab}}

The sample given in the comments of the \texttt{crontab} when initially
opened using \texttt{crontab\ -e} give a fine example of the syntax of
the \texttt{crontab} file:

\drcap{Looking at default \texttt{crontab} file}

\begin{Shaded}
\begin{Highlighting}[]
\CommentTok{# Edit this file to introduce tasks to be run by cron.}
\CommentTok{#}
\CommentTok{# Each task to run has to be defined through a single line}
\CommentTok{# indicating with different fields when the task will be run}
\CommentTok{# and what command to run for the task}
\CommentTok{#}
\CommentTok{# To define the time you can provide concrete values for}
\CommentTok{# minute (m), hour (h), day of month (dom), month (mon),}
\CommentTok{# and day of week (dow) or use '*' in these fields (for 'any').#}
\CommentTok{# Notice that tasks will be started based on the cron's system}
\CommentTok{# daemon's notion of time and timezones.}
\CommentTok{#}
\CommentTok{# Output of the crontab jobs (including errors) is sent through}
\CommentTok{# email to the user the crontab file belongs to (unless redirected).}
\CommentTok{#}
\CommentTok{# For example, you can run a backup of all your user accounts}
\CommentTok{# at 5 a.m every week with:}
\CommentTok{# 0 5 * * 1 tar -zcf /var/backups/home.tgz /home/}
\CommentTok{#}
\CommentTok{# For more information see the manual pages of crontab(5) and cron(8)}
\CommentTok{#}
\CommentTok{# m h  dom mon dow   command}
\end{Highlighting}
\end{Shaded}

If you have \texttt{sudo} privileges you can edit the \texttt{crontab}
file for another user with:

\drcap{Editing another user's \texttt{crontab} file}

\begin{Shaded}
\begin{Highlighting}[]
\NormalTok{$ }\KeywordTok{sudo} \NormalTok{crontab -e -u otheruser}
\end{Highlighting}
\end{Shaded}

This can be useful to do things like run backup jobs as the user that is
running the web server, say, so it has access rights to all the
necessary files to back up the web server installation by definition.

The only other thing I have to add about \texttt{cron} is when it runs
the commands from each \texttt{crontab}, they are typically not invoked
with that particular user's environment settings, so it is best to fully
specify the paths to files both in the \texttt{crontab} file itself and
in any scripts or parameters to scripts it calls. Depending on the
system and whether \texttt{\$PATH} is set at all when a "\texttt{cron}
job" runs, you may have to specify the full paths to binaries in
installed packages or even what you would consider "system" libraries!
The \texttt{which} command comes in handy here for finding out where
each command is installed. \drenv{PATH}{execution search path}

\section*{Start Me Up}\label{start-me-up}
\addcontentsline{toc}{section}{Start Me Up}

If you need to reboot the system the quickest way is with the
\href{http://linux.die.net/man/8/reboot}{\texttt{reboot}} command:
\drsys{reboot}{reboot system}

\drcap{\texttt{reboot} command}

\begin{Shaded}
\begin{Highlighting}[]
\NormalTok{$ }\KeywordTok{sudo} \NormalTok{reboot}
\end{Highlighting}
\end{Shaded}

You can also use the
\href{http://linux.die.net/man/8/shutdown}{\texttt{shutdown}} command
with the \texttt{-r} option, but why? The handier use for
\texttt{shutdown} is to tell a system to halt and power off after
shutting down: \drsys{shutdown}{shutdown or reboot system}

\drcap{Shutdown and power off}

\begin{Shaded}
\begin{Highlighting}[]
\NormalTok{$ }\KeywordTok{sudo} \NormalTok{shutdown -h now}
\end{Highlighting}
\end{Shaded}

\section*{Turn on Your Signals}\label{turn-on-your-signals}
\addcontentsline{toc}{section}{Turn on Your Signals}

One of the basic concepts in UNIX program is that of
\href{https://en.wikipedia.org/wiki/Unix_signal}{"signals"}. You are
probably already familiar with one way to send signals to a program,
which is via \texttt{Ctrl-C} at the command prompt, which sends the
\texttt{SIGINT} ("interrupt") signal to the program. Typically this will
cause a program to terminate. \index{signals}

However, most signals can be "caught" by a program and coded around.
There is one "uninterruptable" signal, however - \texttt{SIGKILL}. We
can send \texttt{SIGKILL} to a process and cause it to terminate
immediately with:

\drcap{Terminating a process with extreme prejudice}

\begin{Shaded}
\begin{Highlighting}[]
\KeywordTok{kill} \NormalTok{-s 9 14302}
\end{Highlighting}
\end{Shaded}

The \texttt{-s\ 9} is for signal \#9, which is the \texttt{SIGKILL}
signal (it is the tenth signal in the signal list, which is 0-relative,
hence \#9).

You can also use the following "shorthand" for \texttt{SIGKILL}:

\drcap{Even shorter way to \texttt{kill} the process}

\begin{Shaded}
\begin{Highlighting}[]
\KeywordTok{kill} \NormalTok{-9 14302}
\end{Highlighting}
\end{Shaded}

Or if you want to get all verbose:

\drcap{A more verbose killer}

\begin{Shaded}
\begin{Highlighting}[]
\KeywordTok{kill} \NormalTok{-s SIGKILL 14302}
\end{Highlighting}
\end{Shaded}

\textbf{Note:} \texttt{SIGKILL} should be used as a last resort, because
a program is not allowed to catch it or be notified of it and hence can
perform no closing logic or cleanup and that may lead to data
corruption. It is for getting rid of "hung" processes when nothing else
will work. Always try to stop a program with a more "normal" method,
which can include sending \texttt{SIGINT} to it first.

\section*{Exit, Smiling}\label{exit-smiling}
\addcontentsline{toc}{section}{Exit, Smiling}

Sometimes a command runs and there isn't a good way to tell if it worked
or not. "UNIX" programs are supposed to set an "exit status" when they
end that by convention is \texttt{0} if the program exited successfully
and a non-zero, (typically) positive number if there was an error. The
exit status for the last executed command or program can be shown at the
command line using the \texttt{\$?} environment variable. Consider if
the file \texttt{foo} exists and \texttt{bar} does not:
\index{*@\texttt{?} (exit status environment variable)}
\index{environment variables!\?@\texttt{\$}\texttt{?} (exit status code)}

\drcap{Examining exit codes}

\begin{Shaded}
\begin{Highlighting}[]
\KeywordTok{~} \NormalTok{$ }\KeywordTok{ls} \NormalTok{foo}
\KeywordTok{foo}
\KeywordTok{~} \NormalTok{$ }\KeywordTok{echo} \OtherTok{$?}
\KeywordTok{0}
\KeywordTok{~} \NormalTok{$ }\KeywordTok{ls} \NormalTok{bar}
\KeywordTok{ls}\NormalTok{: cannot access bar: No such file or directory}
\KeywordTok{~} \NormalTok{$ }\KeywordTok{echo} \OtherTok{$?}
\KeywordTok{2}
\end{Highlighting}
\end{Shaded}

\textbf{Note:} In many cases the exit codes come from the ANSI Standard
C library's \href{http://mazack.org/unix/errno.php}{\texttt{errno.h}
file}. All of this is much handier when handling errors in scripts, but
we're not going to go into script logic here.

However, sometimes even at the command line we want to be able to
conditionally control a sequence of commands, and continue (or not)
based on the success (or failure) of a previous command. In
\texttt{bash} we have \texttt{\&\&} and \texttt{\textbar{}\textbar{}} to
the rescue! \index{*@\texttt{\&}\texttt{\&} (logical "and" operator)}
\index{*@\texttt{"|"|} (logical "or" operator)}

\begin{itemize}
\item
  \textbf{\texttt{a\ \&\&\ b}} - execute \texttt{a} \textbf{\emph{and}}
  \texttt{b}, i.e., execute \texttt{b} only if \texttt{a} is successful.
\item
  \textbf{\texttt{a\ \textbar{}\textbar{}\ b}} - execute \texttt{a}
  \textbf{\emph{or}} \texttt{b}, that is execute \texttt{b} whether
  \textbf{\emph{or}} not \texttt{a} is successful.
\end{itemize}

Our example of file \texttt{foo} (which exists) and file \texttt{bar}
(which does not) and the effect on the exit code of \texttt{ls} can be
illustrative here, too:

\drcap{Using \texttt{\&}\texttt{\&} to chain commands together}

\begin{Shaded}
\begin{Highlighting}[]
\KeywordTok{~} \NormalTok{$ }\KeywordTok{ls} \NormalTok{foo }\KeywordTok{&&} \KeywordTok{ls} \NormalTok{bar}
\KeywordTok{foo}
\KeywordTok{ls}\NormalTok{: cannot access bar: No such file or directory}
\KeywordTok{~} \NormalTok{$ }\KeywordTok{echo} \OtherTok{$?}
\KeywordTok{2}
\end{Highlighting}
\end{Shaded}

Both \texttt{ls} commands execute because the first successfully found
\texttt{foo}, but the second emits its error and sets the exit code to
\texttt{2} (failure).

\drcap{Using \texttt{||} to execute the first and possibly the second command}

\begin{Shaded}
\begin{Highlighting}[]
\KeywordTok{~} \NormalTok{$ }\KeywordTok{ls} \NormalTok{foo }\KeywordTok{||} \KeywordTok{ls} \NormalTok{bar}
\KeywordTok{foo}
\KeywordTok{~} \NormalTok{$ }\KeywordTok{echo} \OtherTok{$?}
\KeywordTok{0}
\end{Highlighting}
\end{Shaded}

Note in this case the \texttt{ls\ bar} command \textbf{\emph{did not}}
execute because the logical "or" condition was already satisfied by the
successful execution of the first \texttt{ls}. The exit code is
obviously \texttt{0} (success).

\drcap{With \texttt{\&}\texttt{\&} the second command won't execute if the first fails}

\begin{Shaded}
\begin{Highlighting}[]
\KeywordTok{~} \NormalTok{$ }\KeywordTok{ls} \NormalTok{bar }\KeywordTok{&&} \KeywordTok{ls} \NormalTok{foo}
\KeywordTok{ls}\NormalTok{: cannot access bar: No such file or directory}
\KeywordTok{~} \NormalTok{$ }\KeywordTok{echo} \OtherTok{$?}
\KeywordTok{2}
\end{Highlighting}
\end{Shaded}

Obviously if the first command fails, the "and" condition as a whole
fails and the expression exits with a code of \texttt{2}. And finally,
while the first command failed the second still can execute because of
the "or", and the whole expression returns \texttt{0}.

\drcap{One more example with \texttt{||}}

\begin{Shaded}
\begin{Highlighting}[]
\KeywordTok{~} \NormalTok{$ }\KeywordTok{ls} \NormalTok{bar }\KeywordTok{||} \KeywordTok{ls} \NormalTok{foo}
\KeywordTok{ls}\NormalTok{: cannot access bar: No such file or directory}
\KeywordTok{foo}
\KeywordTok{~} \NormalTok{$ }\KeywordTok{echo} \OtherTok{$?}
\KeywordTok{0}
\end{Highlighting}
\end{Shaded}

\textbf{Note:} There is actually a
\href{http://linux.die.net/man/1/true}{\texttt{true}} command whose
purpose is to "do nothing, successfully." All it does is return a
\texttt{0} (success) exit code. This can be useful in scripting and also
sometimes when building "and" and "or" clauses like above. \drscr{true}

And yes, of course, that means there is also a
\href{http://linux.die.net/man/1/false}{\texttt{false}} command to "do
nothing, unsuccessfully!" \drscr{false}

\drcap{\texttt{true} and \texttt{false} commands}

\begin{Shaded}
\begin{Highlighting}[]
\KeywordTok{~} \NormalTok{$ }\KeywordTok{true}
\KeywordTok{~} \NormalTok{$ }\KeywordTok{echo} \OtherTok{$?}
\KeywordTok{0}
\KeywordTok{~} \NormalTok{$ }\KeywordTok{false}
\KeywordTok{~} \NormalTok{$ }\KeywordTok{echo} \OtherTok{$?}
\KeywordTok{1}
\end{Highlighting}
\end{Shaded}

\section*{The End}\label{the-end}
\addcontentsline{toc}{section}{The End}

Now you know what I know. Or at least what I keep loaded in my head vs.
what I simply search for when I need to know it, and you know how to do
that searching, too.

Good luck, citizen!

\renewcommand{\chaptername}{}
\renewcommand{\thechapter}{A}

\hypertarget{appendices}{\chapter{Appendices}\label{appendices}}

\begin{quote}
\emph{"That rug really tied the room together, did it not?"} - Walter
Sobchak (\emph{The Big Lebowski})
\end{quote}

\section*{Cheat Sheet}\label{cheat-sheet}
\addcontentsline{toc}{section}{Cheat Sheet}

This list outlines all the commands, files and other UNIX items of
interest brought up in this book. Use \texttt{man} or other methods
outlined in the book to find more information on them.

\subsection*{Environment Variables}\label{environment-variables}
\addcontentsline{toc}{subsection}{Environment Variables}

\begin{itemize}
\item
  \href{http://linux.die.net/abs-guide/exit-status.html}{\textbf{\texttt{\$?}}}
  - the exit code for the last command or program executed.
  \index{*@\texttt{?} (exit status environment variable)}
  \index{environment variables!\?@\texttt{\$}\texttt{?} (exit status code)}
\item
  \href{http://linux.die.net/man/1/sh}{\textbf{\texttt{\$HISTIGNORE}}} -
  a colon-separated list of patterns to keep from being recorded in the
  command history file.
  \drenv{HISTIGNORE}{commands to ignore in command history}
\item
  \href{http://linux.die.net/Bash-Beginners-Guide/sect_03_02.html}{\textbf{\texttt{\$PATH}}}
  - the execution search path. \drenv{PATH}{execution search path}
\end{itemize}

\subsection*{Conditional Execution}\label{conditional-execution}
\addcontentsline{toc}{subsection}{Conditional Execution}

See \href{http://linux.die.net/abs-guide/ops.html}{"logical operators"}.

\begin{itemize}
\item
  \textbf{\texttt{\&\&}} - execute the second command only if the first
  command succeeds.
  \index{*@\texttt{\&}\texttt{\&} (logical "and" operator)}
\item
  \textbf{\texttt{\textbar{}\textbar{}}} - execute the second command
  even if the first command fails.
  \index{*@\texttt{"|"|} (logical "or" operator)}
\end{itemize}

\subsection*{Redirection}\label{redirection}
\addcontentsline{toc}{subsection}{Redirection}

See \href{http://linux.die.net/abs-guide/io-redirection.html}{"I/O
Redirection"}.

\begin{itemize}
\item
  \textbf{\emph{stderr}} - file descriptor 2, always open for writing
  from a process, defaults to the screen on a terminal session.
  \drstd{stderr}
\item
  \textbf{\emph{stdin}} - file descriptor 0, always open for reading in
  a process, defaults to the keyboard input on a terminal session.
  \drstd{stdin}
\item
  \textbf{\emph{stdout}} - file descriptor 1, always open for writing
  from a process, defaults to the screen on a terminal session.
  \drstd{stdout}
\item
  \textbf{\texttt{\textless{}}} - redirect a file to \emph{stdin}.
  \drstd{stdin} \index{*@\texttt{<} (input redirection)}
  \index{I/O!redirection!input (\texttt{<})}
\item
  \textbf{\texttt{\textgreater{}}} - redirect \emph{stdout} to a file.
  \drstd{stdout} \index{*@\texttt{>} (output redirection)}
  \index{I/O!redirection!output (\texttt{>})}
\item
  \textbf{\texttt{2\textgreater{}}} - redirect \emph{stderr} to a file.
  \drstd{stderr} \index{*@\texttt{2>} (\textit{stderr} redirection)}
  \index{I/O!redirection!error (\texttt{2>})}
\item
  \textbf{\texttt{\textbar{}}} - pipe \emph{stdout} from one process
  into \emph{stdin} in another process. \drstd{stdin} \drstd{stdout}
  \index{*@\texttt{"|} (pipe)}
  \index{I/O!redirection!pipe (\texttt{"|})}
\end{itemize}

\subsection*{Special Files and
Directories}\label{special-files-and-directories}
\addcontentsline{toc}{subsection}{Special Files and Directories}

\begin{itemize}
\item
  \href{http://linux.die.net/Bash-Beginners-Guide/sect_03_04.html}{\textbf{\texttt{\textasciitilde{}}}}
  - shortcut for current user's home directory.
  \index{files and directories!home (\texttt{\~{}})}
\item
  \textbf{\texttt{.bash\_history}} - history of commands entered at the
  command prompt (also a nice example of a hidden "dotfile").
  \index{files and directories!special!bashhistory@\texttt{.bash\_history}}
\end{itemize}

\subsection*{System Directories}\label{system-directories}
\addcontentsline{toc}{subsection}{System Directories}

See \href{http://linux.die.net/abs-guide/systemdirs.html}{Important
System Directories}.

\begin{itemize}
\item
  \textbf{\texttt{/etc}} - configuration files location.
  \index{files and directories!special!etc@\texttt{/etc/}}
\item
  \textbf{\texttt{/home}} - "home" or user profile directories.
  \index{files and directories!home (\texttt{/home/})}
\item
  \textbf{\texttt{/proc}} - system run-time information.
  \index{files and directories!special!proc@\texttt{/proc/}}
\item
  \textbf{\texttt{/root}} - "home" directory for "root" user (system
  admin). \index{files and directories!root "home" (\texttt{/root/})}
\item
  \textbf{\texttt{/tmp}} - temporary files location.
  \index{files and directories!special!tmp@\texttt{/tmp/}}
\item
  \textbf{\texttt{/var/log}} - log files location.
  \index{files and directories!special!varlog@\texttt{/var/log/}}
\end{itemize}

\subsection*{Commands}\label{commands}
\addcontentsline{toc}{subsection}{Commands}

These are "section 1" commands, i.e., normal user commands that
typically don't require any special privileges beyond permissions to
access files and the like.

\begin{itemize}
\item
  \href{http://linux.die.net/man/1/7z}{\textbf{\texttt{7z}}} - compress
  and uncompress files and directories using the 7-zip algorithm.
  \drcom{7z}
\item
  \href{http://linux.die.net/man/1/apropos}{\textbf{\texttt{apropos}}} -
  search for help on commands by pattern. \drdoc{apropos}
\item
  \href{http://linux.die.net/man/1/awk}{\textbf{\texttt{awk}}} -
  language for processing streams of data. \drscr{awk}
\item
  \href{http://linux.die.net/man/1/bash}{\textbf{\texttt{bash}}} - the
  Bourne-again shell. \drscr{bash}
\item
  \href{http://linux.die.net/man/1/bzip2}{\textbf{\texttt{bzip2}}} -
  compress and uncompress files using the \texttt{bzip2} algorithm.
  \drcom{bzip2}
\item
  \href{http://linux.die.net/man/1/cat}{\textbf{\texttt{cat}}} -
  concatenate the input files to \emph{stdout}. \drtxt{cat}
  \drstd{stdout}
\item
  \href{http://linux.die.net/man/1/cd}{\textbf{\texttt{cd}}} - change
  the current directory. \drfnd{cd}{change directory}
\item
  \href{http://linux.die.net/man/1/chgrp}{\textbf{\texttt{chgrp}}} -
  change the primary group of a file or directory. \drprm{chgrp}
\item
  \href{http://linux.die.net/man/1/chmod}{\textbf{\texttt{chmod}}} -
  change the permissions (mode bits) of a file or directory.
  \drprm{chmod}
\item
  \href{http://linux.die.net/man/1/chown}{\textbf{\texttt{chown}}} -
  change the owner of a file or directory. \drprm{chown}
\item
  \href{http://linux.die.net/man/1/cmake}{\textbf{\texttt{cmake}}} -
  configure makefiles. \drcmd{cmake}{configure makefiles}
\item
  \href{http://linux.die.net/man/1/cp}{\textbf{\texttt{cp}}} - copy
  files or directories. \drfnd{cp}{copy}
\item
  \href{http://linux.die.net/man/1/crontab}{\textbf{\texttt{crontab}}} -
  display or edit tasks to be run by \texttt{cron}.
  \drsys{crontab}{edit scheduled jobs}
\item
  \href{http://linux.die.net/man/1/curl}{\textbf{\texttt{curl}}} -
  download files from the internet. \drnet{curl}
\item
  \href{http://linux.die.net/man/1/cut}{\textbf{\texttt{cut}}} - remove
  (cut) sections from lines. \drtxt{cut}
\item
  \href{http://linux.die.net/man/1/df}{\textbf{\texttt{df}}} - show
  space utilization by file system.
  \drfnd{df}{display file system disk space}
\item
  \href{http://linux.die.net/man/1/diff}{\textbf{\texttt{diff}}} - show
  the differences between files.
  \drfnd{diff}{show differences between files}
\item
  \href{http://linux.die.net/man/1/dig}{\textbf{\texttt{dig}}} - look up
  DNS info on an address. \drnet{dig}
\item
  \href{http://linux.die.net/man/1/dpkg}{\textbf{\texttt{dpkg}}} -
  package manager for Debian flavors. \drpkg{dpkg}
\item
  \href{http://linux.die.net/man/1/du}{\textbf{\texttt{du}}} - estimate
  disk usage. \drfnd{du}{disk use by directory}
\item
  \href{http://linux.die.net/man/1/echo}{\textbf{\texttt{echo}}} -
  display passed parameters to \emph{stdout}. \drtxt{echo}
  \drstd{stdout}
\item
  \href{http://linux.die.net/man/1/emacs}{\textbf{\texttt{emacs}}} -
  great operating system, but it could use an editor. \dreds{emacs}
\item
  \href{http://linux.die.net/man/1/email}{\textbf{\texttt{email}}} -
  send email. \drnet{email}
\item
  \href{http://linux.die.net/man/1/false}{\textbf{\texttt{false}}} - do
  nothing, unsuccessfully. \drscr{false}
\item
  \href{http://linux.die.net/man/1/file}{\textbf{\texttt{file}}} - give
  best guess as to type of file. \drfnd{file}{detect file type}
\item
  \href{http://linux.die.net/man/1/find}{\textbf{\texttt{find}}} - find
  files based on various conditions and execute actions against the
  results. \drfnd{find}{find files}
\item
  \href{http://linux.die.net/man/1/fmt}{\textbf{\texttt{fmt}}} - simple
  text formatter. \drtxt{fmt}
\item
  \href{http://linux.die.net/man/1/grep}{\textbf{\texttt{grep}}} -
  search for a pattern (regular expression) in files.
  \drfnd{grep}{search files}
\item
  \href{http://linux.die.net/man/1/gzip}{\textbf{\texttt{gzip}}} -
  compression program. \drcom{gzip}
\item
  \href{http://linux.die.net/man/1/help}{\textbf{\texttt{help}}} - help
  for built-in commands in \texttt{bash}. \drdoc{help}
\item
  \href{http://linux.die.net/man/1/if}{\textbf{\texttt{if}}} -
  conditionally execute a program. \drscr{if}
\item
  \href{http://linux.die.net/man/1/info}{\textbf{\texttt{info}}} - an
  alternative for \texttt{man}, especially for GNU programs. Remember
  \texttt{q} quits. \drdoc{info}
\item
  \href{http://linux.die.net/man/1/latex}{\textbf{\texttt{latex}}} -
  process LaTeX document markup. \drtxt{latex}
\item
  \href{http://linux.die.net/man/1/less}{\textbf{\texttt{less}}} -
  display the file one page at a time on \emph{stdout}. \drtxt{less}
  \drstd{stdout}
  \index{files and directories!paginate!less@\texttt{less} command}
  \index{pagination!\texttt{less} command}
\item
  \href{http://linux.die.net/man/1/ln}{\textbf{\texttt{ln}}} - create
  hard or soft (shortcut) links. \drfnd{ln}{link}
\item
  \href{http://linux.die.net/man/1/locate}{\textbf{\texttt{locate}}} -
  locate files by name. \drfnd{locate}{locate files}
\item
  \href{http://linux.die.net/man/1/ls}{\textbf{\texttt{ls}}} - list
  directory contents. \drfnd{ls}{list directory contents}
\item
  \href{http://linux.die.net/man/1/lynx}{\textbf{\texttt{lynx}}} -
  command line web browser. \drnet{lynx}
\item
  \href{http://linux.die.net/man/1/make}{\textbf{\texttt{make}}} - run
  programs according to "recipes" in makefiles.
  \drcmd{make}{build by recipes}
\item
  \href{http://linux.die.net/man/1/man}{\textbf{\texttt{man}}} - display
  manual pages. Remember \texttt{q} quits. \drdoc{man}
\item
  \href{http://linux.die.net/man/1/mkdir}{\textbf{\texttt{mkdir}}} -
  make a new directory. \drfnd{mkdir}{make directory}
\item
  \href{http://linux.die.net/man/1/more}{\textbf{\texttt{more}}} -
  display the file one page at a time on \emph{stdout}. \drtxt{more}
  \drstd{stdout}
  \index{files and directories!paginate!more@\texttt{more} command}
  \index{pagination!\texttt{more} command}
\item
  \href{http://linux.die.net/man/1/mutt}{\textbf{\texttt{mutt}}} - email
  client. \drnet{mutt}
\item
  \href{http://linux.die.net/man/1/mv}{\textbf{\texttt{mv}}} - move
  files or directories. \drfnd{mv}{move files}
\item
  \href{http://linux.die.net/man/1/nano}{\textbf{\texttt{nano}}} -
  small, intuitive text editor. \dreds{nano}
\item
  \href{http://pandoc.org/README.html}{\textbf{\texttt{pandoc}}} -
  markup converter. The primary tool used to create this book in
  multiple formats including PDF, EPUB, HTML and Markdown.
  \drcmd{pandoc}{markup converter}
\item
  \href{http://linux.die.net/man/1/pdflatex}{\textbf{\texttt{pdflatex}}}
  - create PDF files. \drcmd{pdflatex}{create PDF files}
\item
  \href{http://linux.die.net/man/1/pico}{\textbf{\texttt{pico}}} -
  small, intuitive text editor. \dreds{pico}
\item
  \href{http://linux.die.net/man/1/pine}{\textbf{\texttt{pine}}} - email
  client. \drnet{pine}
\item
  \href{http://linux.die.net/man/1/ps}{\textbf{\texttt{ps}}} - list
  running processes. \drsys{ps}{list processes}
\item
  \href{http://linux.die.net/man/1/pwd}{\textbf{\texttt{pwd}}} - print
  the current (working) directory name.
  \drfnd{pwd}{print working directory}
\item
  \href{http://linux.die.net/man/1/rename}{\textbf{\texttt{rename}}} -
  rename files in more complex ways than \texttt{mv} can.
  \drfnd{rename}{rename file}
\item
  \href{http://linux.die.net/man/1/rm}{\textbf{\texttt{rm}}} - delete
  (remove) files or directories. \drfnd{rm}{remove file}
\item
  \href{http://linux.die.net/man/1/rsync}{\textbf{\texttt{rsync}}} -
  efficiently and securely "mirror" directories between local and remote
  locations. \drnet{rsync}
\item
  \href{http://linux.die.net/man/1/scp}{\textbf{\texttt{scp}}} - file
  copy over secure shell protocol. \drnet{scp}
\item
  \href{http://linux.die.net/man/1/sed}{\textbf{\texttt{sed}}} - stream
  editor for editing from the command line. \dreds{sed}
\item
  \href{http://linux.die.net/man/1/set}{\textbf{\texttt{set}}} - set an
  environment variable, or display all environment variables.
  \drcmd{set}{set shell options}
\item
  \href{http://linux.die.net/man/1/smbclient}{\textbf{\texttt{smbclient}}}
  - copy files to and from Windows using the SMB/CIFS (Windows file
  share) protocol. \drnet{smbclient}
\item
  \href{http://linux.die.net/man/1/sort}{\textbf{\texttt{sort}}} - sort
  \emph{stdin} or a file to \emph{stdout}. \drtxt{sort} \drstd{stdin}
  \drstd{stdout}
\item
  \href{http://linux.die.net/man/1/ssh}{\textbf{\texttt{ssh}}} - secure
  shell terminal progam and protocol. \drnet{ssh}
\item
  \href{http://linux.die.net/man/1/tail}{\textbf{\texttt{tail}}} -
  display the last lines of a file. \drtxt{tail}
\item
  \href{http://linux.die.net/man/1/tar}{\textbf{\texttt{tar}}} - "tape
  archive", a way to combine directories into a single flat file.
  \drcmd{tar}{archive files}
\item
  \href{http://linux.die.net/man/1/tee}{\textbf{\texttt{tee}}} - write
  to a file and \emph{stdout} at the same time. \drtxt{tee}
  \drstd{stdout}
\item
  \href{http://linux.die.net/man/1/telnet}{\textbf{\texttt{telnet}}} -
  ancient terminal program and protocol. \drnet{telnet}
\item
  \href{http://linux.die.net/man/1/top}{\textbf{\texttt{top}}} - list
  processes by resource utilization (CPU).
  \drsys{top}{list processes by resource use}
\item
  \href{http://linux.die.net/man/1/touch}{\textbf{\texttt{touch}}} -
  create an empty file or change the last-modified time of an existing
  file. \drfnd{touch}{change modified date or create file}
\item
  \href{http://linux.die.net/man/1/tr}{\textbf{\texttt{tr}}} - translate
  (map, convert) characters. \drtxt{tr}
\item
  \href{http://linux.die.net/man/1/true}{\textbf{\texttt{true}}} - do
  nothing, successfully. \drscr{true}
\item
  \href{http://linux.die.net/man/1/uname}{\textbf{\texttt{uname}}} -
  print system info. \drsys{uname}{system info}
\item
  \href{http://linux.die.net/man/1/unzip}{\textbf{\texttt{unzip}}} -
  uncompress \texttt{.zip} files. \drcom{unzip}
\item
  \href{http://linux.die.net/man/1/vi}{\textbf{\texttt{vi}}} - "visual"
  editor, a file editor. \dreds{vi}
\item
  \href{http://linux.die.net/man/1/view}{\textbf{\texttt{view}}} -
  read-only version of \texttt{vim}. \dreds{view}
\item
  \href{http://linux.die.net/man/1/vim}{\textbf{\texttt{vim}}} -
  \texttt{vi} Improved, another implementation of \texttt{vi} allowing
  more customization. \dreds{vim}
\item
  \href{http://linux.die.net/man/1/wget}{\textbf{\texttt{wget}}} -
  download files from the internet. \drnet{wget}
\item
  \href{http://linux.die.net/man/1/which}{\textbf{\texttt{which}}} -
  determine the path of a program. \drcmd{which}{find program}
\item
  \href{http://linux.die.net/man/1/while}{\textbf{\texttt{while}}} -
  perform a command multiple times. \drscr{while}
\item
  \href{http://linux.die.net/man/1/whoami}{\textbf{\texttt{whoami}}} -
  the answer to life's most existential question.
  \drcmd{whoami}{existential question}
\item
  \href{http://linux.die.net/man/1/whois}{\textbf{\texttt{whois}}} -
  look up DNS ownership info on an address. \drnet{whois}
\item
  \href{http://linux.die.net/man/1/xfreerdp}{\textbf{\texttt{xfreerdp}}}
  - RDP protocol client. \drnet{xfreerdp}
\item
  \href{http://linux.die.net/man/1/zip}{\textbf{\texttt{zip}}} -
  compress files and directories using the PKZip algorithm. \drcom{zip}
\end{itemize}

\subsection*{System Commands}\label{system-commands}
\addcontentsline{toc}{subsection}{System Commands}

Most of these are "section 8" commands, and \textbf{\emph{may}} require
special privileges such as \texttt{sudo} to run, depending on the
system. Yes, some systems restrict the use of \texttt{ping}!

\begin{itemize}
\item
  \href{http://linux.die.net/man/8/apt-get}{\textbf{\texttt{apt-get}}} -
  package manager for Debian flavors. \drpkg{apt-get}
\item
  \href{http://linux.die.net/man/8/aptitude}{\textbf{\texttt{aptitude}}}
  - package manager for Debian flavors. \drpkg{aptitude}
\item
  \href{http://linux.die.net/man/8/cron}{\textbf{\texttt{cron}}} -
  system for running "scheduled tasks." \drsys{cron}{run scheduled jobs}
\item
  \href{http://linux.die.net/man/8/dmesg}{\textbf{\texttt{dmesg}}} -
  display kernel log messages. \drsys{dmesg}{display kernel log}
\item
  \href{http://linux.die.net/man/8/ifconfig}{\textbf{\texttt{ifconfig}}}
  - display network (interface) configuration. \drnet{ifconfig}
\item
  \href{http://linux.die.net/man/1/kill}{\textbf{\texttt{kill}}} -
  terminate a process. \drsys{kill}{terminate process}
\item
  \href{http://linux.die.net/man/8/mount}{\textbf{\texttt{mount}}} -
  mount a file system to a specific location.
  \drsys{mount}{mount file system}
\item
  \href{https://www.archlinux.org/pacman/pacman.8.html}{\textbf{\texttt{pacman}}}
  - package manager for Arch Linux. \footnote{Not to be confused with
    the game. \url{http://linux.die.net/man/1/pacman}} \drpkg{pacman}
\item
  \href{http://linux.die.net/man/1/passwd}{\textbf{\texttt{passwd}}} -
  change password. \drsys{passwd}{change password}
\item
  \href{http://linux.die.net/man/8/ping}{\textbf{\texttt{ping}}} - test
  for network connectivity to an IP address. \drnet{ping}
\item
  \href{http://linux.die.net/man/8/reboot}{\textbf{\texttt{reboot}}} -
  restart the system. \drsys{reboot}{reboot system}
\item
  \href{http://linux.die.net/man/8/rpm}{\textbf{\texttt{rpm}}} - package
  manager for Fedora flavors. \drdis{Fedora} \drdis{Red Hat} \drpkg{rpm}
\item
  \href{http://linux.die.net/man/8/shutdown}{\textbf{\texttt{shutdown}}}
  - shutdown or restart the system.
  \drsys{shutdown}{shutdown or reboot system}
\item
  \href{http://linux.die.net/man/8/sudo}{\textbf{\texttt{sudo}}} -
  execute a command with elevated privileges.
  \drsys{sudo}{execute as another user}
\item
  \href{http://linux.die.net/man/8/traceroute}{\textbf{\texttt{traceroute}}}
  - trace the route to an IP address. \drnet{traceroute}
\item
  \href{https://www.centos.org/docs/4/html/yum/}{\textbf{\texttt{yum}}}
  - package manager for CentOS, originally created for Yellow Dog Linux
  ("Yellow dog Updater, Modified"). \drdis{CentOS} \drdis{Yellow Dog}
  \drpkg{yum}
\end{itemize}

\section*{Examples}\label{examples}
\addcontentsline{toc}{section}{Examples}

The following are meant to be simple, mostly "one-liner" type samples to
reinforce the concepts here and continue to show you "the UNIX
philosophy" of approaching solutions in multiple small steps.

\subsection*{Keep It Simple, Stupid}\label{keep-it-simple-stupid}
\addcontentsline{toc}{subsection}{Keep It Simple, Stupid}

Here's a good example. During the debugging of this book I kept having
problems with internal links to other parts of the generated PDF not
working. Some did, some didn't. And they \textbf{\emph{all}} worked in
epub and HTML outputs.

I had a suspicion it was because I was wrapping links from one line to
the next in Markdown (trying to keep below a certain column count), so I
wanted to find all lines that had an opening square bracket but
\textbf{\emph{not}} a closing one, e.g., I wanted to catch the first
line in the following:

\drcap{Some Markdown}

\begin{Shaded}
\begin{Highlighting}[]
\KeywordTok{See} \NormalTok{[Important System}
\KeywordTok{Directories.}\NormalTok{](http://linux.die.net/abs-guide/systemdirs.html)}
\end{Highlighting}
\end{Shaded}

Now you could spend a long time with regular expressions trying to
figure out how to do negative matching on that closing \texttt{{]}}.
Good luck!

Or you could do something as simple as this:

\drcap{Searching through the Markdown for mismatched brackets}

\begin{Shaded}
\begin{Highlighting}[]
\NormalTok{$ }\KeywordTok{grep} \StringTok{'\textbackslash{}['} \NormalTok{*.md }\KeywordTok{|} \KeywordTok{grep} \NormalTok{-v }\StringTok{']'}
\KeywordTok{Step01.md}\NormalTok{: (( expression ))           }\KeywordTok{if} \KeywordTok{COMMANDS;} \KeywordTok{then} \KeywordTok{COMMANDS; [} \NormalTok{elif C>}
\NormalTok{Step01.md: :                          kill [}\OtherTok{-s} \NormalTok{sigspec | }\OtherTok{-n} \NormalTok{signum | -sigs>}
\NormalTok{Step04.md:./FileCheckers/otschecker:  [ }\OtherTok{$rc} \OtherTok{!=} \NormalTok{0 }\OtherTok{-a} \StringTok{"}\OtherTok{$pdfinfo}\StringTok{"} \OtherTok{!=} \StringTok{"Comma...}
\StringTok{Step04.md:./FileCheckers/pdfpwdchecker: if [ }\OtherTok{$rc}\StringTok{ != 0 -a "}\OtherTok{$pdfinfo}\StringTok{" = "}\NormalTok{C...}
\end{Highlighting}
\end{Shaded}

What makes this simple? Finding \texttt{{[}} with the first
\texttt{grep} and then simply piping it to a second \texttt{grep} and
inverting the match logic (\texttt{-v}) on \texttt{{]}}.

\subsection*{Chain Gangs}\label{chain-gangs}
\addcontentsline{toc}{subsection}{Chain Gangs}

Remembering that \texttt{\&\&} only executes the next command if the
prior one is successful, we can do things like set up a sample directory
and (empty) files for playing around with files and directories in one
fell swoop: \index{*@\texttt{\&}\texttt{\&} (logical "and" operator)}

\drcap{Make a bunch of files and directories at once}

\begin{Shaded}
\begin{Highlighting}[]
\KeywordTok{~} \NormalTok{$ }\KeywordTok{mkdir} \NormalTok{-p /tmp/foo/d }\KeywordTok{&&} \KeywordTok{cd} \NormalTok{/tmp/foo }\KeywordTok{&&} \KeywordTok{touch} \NormalTok{a b c d/e}
\KeywordTok{~} \NormalTok{$ }\KeywordTok{ls}
\KeywordTok{a}  \NormalTok{b  c  d}
\end{Highlighting}
\end{Shaded}

That is roughly equivalent to:

\drcap{Make a bunch of files the long way}

\begin{Shaded}
\begin{Highlighting}[]
\KeywordTok{~} \NormalTok{$ }\KeywordTok{cd} \NormalTok{/tmp}
\KeywordTok{~} \NormalTok{$ }\KeywordTok{mkdir} \NormalTok{-p foo}
\KeywordTok{~} \NormalTok{$ }\KeywordTok{cd} \NormalTok{foo}
\KeywordTok{~} \NormalTok{$ }\KeywordTok{mkdir} \NormalTok{-p d}
\KeywordTok{~} \NormalTok{$ }\KeywordTok{touch} \NormalTok{a b c d/e}
\KeywordTok{~} \NormalTok{$ }\KeywordTok{ls}
\KeywordTok{a}  \NormalTok{b  c  d}
\end{Highlighting}
\end{Shaded}

\subsection*{Simple Scripts}\label{simple-scripts}
\addcontentsline{toc}{subsection}{Simple Scripts}

I said I wasn't going to cover scripting, especially logical constructs
like \texttt{if}/\texttt{fi}. But simple scripts that just "do things"
in a certain order are within scope, and the following, which installs
\href{https://github.com/freerdp/freerdp}{\texttt{freerdp}}, is a good
example of simply taking the guesswork out of doing something repetitive
across multiple machines. I keep this \texttt{installrdp} script in
Dropbox so I can run it quickly and easily on any new machine I set up
(once I get Dropbox set up on the machine!) \drscr{if} \drnet{xfreerdp}

\drcap{A simple install script}

\begin{Shaded}
\begin{Highlighting}[]
\CommentTok{#!/bin/bash}
\KeywordTok{sudo} \NormalTok{apt-get -y install git}
\KeywordTok{cd} \NormalTok{~}
\KeywordTok{git} \NormalTok{clone git://github.com/FreeRDP/FreeRDP.git}
\KeywordTok{cd} \NormalTok{FreeRDP}
\KeywordTok{sudo} \NormalTok{apt-get -y install build-essential git-core cmake libssl-dev \textbackslash{}}
  \NormalTok{libx11-dev libxext-dev libxinerama-dev libxcursor-dev libxdamage-dev \textbackslash{}}
  \NormalTok{libxv-dev libxkbfile-dev libasound2-dev libcups2-dev   libxml2 \textbackslash{}}
  \NormalTok{libxml2-dev libxrandr-dev libgstreamer0.10-dev \textbackslash{}}
  \NormalTok{libgstreamer-plugins-base0.10-dev libxi-dev \textbackslash{}}
  \NormalTok{libgstreamer-plugins-base1.0-dev libavutil-dev libavcodec-dev \textbackslash{}}
  \NormalTok{libcunit1-dev libdirectfb-dev xmlto doxygen libxtst-dev}
\KeywordTok{cmake} \NormalTok{-DCMAKE_BUILD_TYPE=Debug -DWITH_SSE2=ON .}
\KeywordTok{make}
\KeywordTok{sudo} \NormalTok{make install}
\KeywordTok{sudo} \NormalTok{echo }\StringTok{"/usr/local/lib/freerdp"} \KeywordTok{>} \NormalTok{/etc/ld.so.conf.d/freerdp.conf}
\KeywordTok{sudo} \NormalTok{echo }\StringTok{"/usr/local/lib64/freerdp"} \KeywordTok{>>} \NormalTok{/etc/ld.so.conf.d/freerdp.conf}
\KeywordTok{sudo} \NormalTok{echo }\StringTok{"/usr/local/lib"} \KeywordTok{>>} \NormalTok{/etc/ld.so.conf.d/freerdp.conf}
\KeywordTok{sudo} \NormalTok{ldconfig}
\KeywordTok{which} \NormalTok{xfreerdp}
\KeywordTok{xfreerdp} \NormalTok{--version}
\end{Highlighting}
\end{Shaded}

You should be able to understand all of the above now, or know where to
look to figure it out. The only nuance we may not have covered is that
at the shell prompt and in scripts both you can put a
\texttt{\textbackslash{}} at the end of a line and it will "escape" the
newline (\texttt{\textbackslash{}r}) so you can continue the same
command on the next line. This is useful because some interactive
terminals don't "wrap" well, and it makes more readable script files,
too.

And yes, in the section on package management I talked about the dangers
of installing packages directly from source. In this case, though,
\texttt{freerdp} in the Mint repositories is lagging far enough behind
the new RDP protocol version 8 support that I want to use the latest and
greatest \texttt{freerdp} from GitHub for performance reasons. But now
it's up to me to track and update the software (if I care).

\renewcommand{\chaptername}{}
\renewcommand{\thechapter}{C}

\chapter{Colophon}\label{colophon}

\begin{quote}
\emph{"I can't come back, I don't know how it works! Good-bye, folks!"}
- The Wizard of Oz
\end{quote}

This document was produced in the environments it discusses, including
(with their \href{http://linux.die.net/man/1/uname}{\texttt{uname\ -rv}}
results): \drsys{uname}{system info}

\begin{itemize}
\item
  \href{https://cygwin.com/}{\textbf{Cygwin}} -
  \texttt{2.2.1(0.289/5/3)\ 2015-08-20\ 11:42} \drunx{Cygwin}
\item
  \href{http://www.debian.org/}{\textbf{Debian}} -
  \texttt{3.2.0-4-amd64\ \#1\ SMP\ Debian\ 3.2.65-1+deb7u2}
  \drdis{Debian}
\item
  \href{http://www.freebsd.org/}{\textbf{FreeBSD}} -
  \texttt{7.3-RELEASE-p2\ FreeBSD\ 7.3-RELEASE-p2\ \#0:\ Tue\ Nov\ \ 4\ 22:08:52\ EST\ 2014}
  \drbsd{FreeBSD}
\item
  \href{http://linuxmint.com/}{\textbf{Linux Mint}} -
  \texttt{3.16.0-38-generic\ \#52\textasciitilde{}14.04.1-Ubuntu\ SMP\ Fri\ May\ 8\ 09:43:57\ UTC\ 2015}
  \drdis{Mint}
\end{itemize}

I could have done something with my Raspberry Pi, too, but that would
just be showing off.

Written in
\href{http://pandoc.org/README.html\#pandocs-markdown}{\texttt{pandoc}-flavored
Markdown} using \href{http://linux.die.net/man/1/vi}{\texttt{vi}} and
\href{https://github.com/Microsoft/vscode}{Visual Studio Code}, among
others. \dreds{vi}

Output produced using \href{http://pandoc.org/}{\texttt{pandoc}},
\href{http://www.tug.org/texlive/}{TeX Live},
\href{http://linux.die.net/man/1/pdflatex}{\texttt{pdflatex}},
\href{http://linux.die.net/man/1/make}{\texttt{make}}, originally based
on the
\href{https://github.com/evangoer/pandoc-ebook-template}{@evangoer's
pandoc ebook template} but long since modified so don't blame him.
\drcmd{make}{build by recipes} \drcmd{pandoc}{markup converter}
\drcmd{pdflatex}{create PDF files} \drtxt{latex}

Source code control is provided by
\href{http://linux.die.net/man/1/git}{\texttt{git}}. You can view
\href{https://github.com/dullroar/ten-steps-to-linux-survival}{the files
used to create this book} on GitHub.
\drcmd{git}{distributed version control}

The fonts used are
\href{https://en.wikipedia.org/wiki/DejaVu_fonts}{DejaVu} Serif for the
body text, DejaVu Sans for headers, and
\href{https://en.wikipedia.org/wiki/Ubuntu_\%28typeface\%29}{Ubuntu
Mono} for code (because it is nicely condensed).

The cover photo is of our dog, Merv, who is reminding you, "Don't
panic!" Photo by Gloria Anderson, used with permission.

\section*{About the Author}\label{about-the-author}
\addcontentsline{toc}{section}{About the Author}

Jim is son to Barb and Lou; husband to Leslie; father to Meghann (and
Jeremy), Morgann, Erin, Gloria and Jon; grandfather to Ryan, Lindsay,
Logan and Hannah; and alpha wolf to Merv. He has been "in computers"
since 1980. His hobbies include reading, running, hiking, climbing and
apparently writing books.

% Add index.
\cleardoublepage
\phantomsection
\printindex
% End of index.

\end{document}
